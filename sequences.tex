\section{Sequences}\index{Sequences}
\fancyhead[R]{\large Sequences}

\begin{tagblock}{Sequences, Challenge, WarmUp, W11}
\begin{question}

A \textit{sequence} is an ordered list of terms. Here are some examples with just numbers, see if you can find all the terms in the pattern:

\bigskip

a) $1,2,5,10,20,\dots$ (7 terms)
 
\bigskip

b) $31,28,31,30,31,30,31,\dots$ (12 terms)

\bigskip

c) $1,1,1,3,1,\dots$ (26 terms)

	
\index{Sequences}
	
\begin{tags}
	    Sequences, Challenge, WarmUp, W11
\end{tags}
	
\begin{diary}
	   
\end{diary}
	
\begin{solution}	

\end{solution}
	
\end{question}

\end{tagblock}

%-------------------------------------------------------------------------------------------------------------



\begin{tagblock}{Sequences, Challenge, WarmUp, W11}
\begin{question}

A sequence doesn't just have to be numbers. You can use pretty much anything that you can dream up an order on. The second example is finite and the other two are infinite. 

\bigskip

a) $\triangle,\square,\dots$

\bigskip

b) $AZ,GT,MN,\dots$ (26 terms)

\bigskip

c) $O,T,T,F,F,S,S,\dots$

	
\index{Sequences}
	
\begin{tags}
	    Sequences, Challenge, WarmUp, W11
\end{tags}
	
\begin{diary}
	   
\end{diary}
	
\begin{solution}	

\end{solution}
	
\end{question}

\end{tagblock}

%-------------------------------------------------------------------------------------------------------------



\begin{tagblock}{Sequences, Challenge, WarmUp, W11}
\begin{question}

Our focus in this class will be on infinite sequences of numbers. See if you can find the pattern:

\bigskip

a) $1,0,1,0,1,\dots$

\bigskip

b) $1,3,7,15,31,\dots$

\bigskip

c) $3,8,13,18,\dots$ 

\bigskip 

d) $1/4,-1/9,1/16,-1/25,\dots$

\bigskip

e) $0,6,24,60,120,\dots$

\bigskip

f) $1,1,2,3,5,8,13,\dots$

\bigskip

g) $2,3,5,7,11,13,\dots$

\bigskip

h) $23,21,24,19,26,15,\dots$ 

\bigskip

i) $1,11,21,1211,111221,312211,\dots$
	
\index{Sequences}
	
\begin{tags}
	    Sequences, Challenge, WarmUp, W11
\end{tags}
	
\begin{diary}
	   
\end{diary}
	
\begin{solution}	

\end{solution}
	
\end{question}

\end{tagblock}

%-------------------------------------------------------------------------------------------------------------


\begin{tagblock}{Sequences, Challenge, W11}
\begin{question}

Find the ones digit of the 63rd term of $7,7^2,7^3,7^4,7^5,\dots$
	
\index{Sequences}
	
\begin{tags}
	    Sequences, Challenge, W11
\end{tags}
	
\begin{diary}
	   
\end{diary}
	
\begin{solution}	

\end{solution}
	
\end{question}

\end{tagblock}

%-------------------------------------------------------------------------------------------------------------


\begin{tagblock}{Sequences, GeometricSequences, Definition, W12}
\begin{question}

Let's start with a very basic kind of sequence called a \textit{geometric sequence}. For such a sequence, any given term is a fixed multiple of the previous term; so if the $n^{th}$ term is $a_n$ and the $(n+1)^{st}$ term is $a_{n+1}$, there is a number $x$ called the \textit{ratio} with
\[
\frac{a_{n+1}}{a_n}=x.
\]
So for example, the sequence $(-1)^n$ that we saw in the last worksheet has
\[
\frac{a_{n+1}}{a_n}=-1
\]
and so is a geometric sequence.

This may look pretty simple, but it will eventually be the basis by which we can ensure that functions constructed with sequences make sense. 

Check whether the following sequences are geometric or not, and if so, find the ratio $x$.

\bigskip

a) $\left(4^{2n}\right)$
 
\bigskip

b) $\left(\dfrac{(-2)^n} n\right)$

\bigskip

c) $\left(\dfrac{(-5)^{n+2}} {12^n}\right)$

	
\index{Sequences}
	
\begin{tags}
	    Sequences, GeometricSequences, Definition, W12
\end{tags}
	
\begin{diary}
	   
\end{diary}

\begin{solution}	

\end{solution}
	
\end{question}

\end{tagblock}

%-------------------------------------------------------------------------------------------------------------



\begin{tagblock}{Sequences, GeometricSequences, L'Hopital, W12}
\begin{question}

Recall the following result:

\bigskip

\begin{itemize} 



\item (Representation Theorem) If $f$ is a real valued function of a real variable $x$ and $f(n)=a_n$ for some sequence $\{a_n\}_{n=1}^{\infty}$, then if $\displaystyle\lim_{x\to\infty}f(x)=L$, then $\displaystyle\lim_{n\to\infty}a_n=L$.

\end{itemize}

This theorem opens up l'Hopital's Rule for you if you need it! Use the Representation Theorem to check whether the following geometric sequences converge or diverge, and if they converge, find the limit.

\bigskip

a) $a_n=1$

\bigskip

b) $b_n=\left(\dfrac{-1}{3}\right)^n$

\bigskip

c) $d_n=\left(\dfrac{15}{14}\right)^n$

	
\index{Sequences}
	
\begin{tags}
	    Sequences, GeometricSequences, L'Hopital, W12
\end{tags}
	
\begin{diary}
	   
\end{diary}
	
\begin{solution}	

\end{solution}
	
\end{question}

\end{tagblock}

%-------------------------------------------------------------------------------------------------------------



\begin{tagblock}{Sequences, Factorial, SqueezeTheorem, W12}
\begin{question}

Recall the Representation Theorem, which we used to examine geometric sequences:

\bigskip

\begin{itemize} 



\item (Representation Theorem) If $f$ is a real valued function of a real variable $x$ and $f(n)=a_n$ for some sequence $\{a_n\}_{n=1}^{\infty}$, then if $\displaystyle\lim_{x\to\infty}f(x)=L$, then $\displaystyle\lim_{n\to\infty}a_n=L$.

\end{itemize} 



Not every sequence is geometric, and for those, there are no hard and fast rules, so we need more tricks. Another consequence of the definition is 

\begin{itemize} 

\item (Squeeze Theorem for Sequences:) If there is an $N$ with $a_n\leq b_n\leq c_n$ for all $n\geq N$, then $\displaystyle\lim_{n\to\infty}a_n=\displaystyle\lim_{n\to\infty}c_n=L$, implies $\displaystyle\lim_{n\to\infty}b_n=L$. 

\end{itemize}

Use either the Squeeze or Representation Theorem to check whether the following sequences converge or diverge, and if they converge, find the limit.

\bigskip

a) $a_n=\dfrac {(-1)^n}{n+1}$

\bigskip

b) $b_n=\dfrac{2n+1}{n+3}$

\bigskip

c) $d_n=\dfrac{3^n}{n!}$ (this one takes some thought.)

	
\index{Sequences}
	
\begin{tags}
	    Sequences, Factorial, SqueezeTheorem, W12
\end{tags}
	
\begin{diary}
	   
\end{diary}
	
\begin{solution}	

\end{solution}
	
\end{question}

\end{tagblock}

%-------------------------------------------------------------------------------------------------------------



\begin{tagblock}{Sequences, Theory, Challenge, W12}
\begin{question}

Finally, we look at some special sequences, called \textit{continued fractions}. Every real number $x$ has a continued fraction expansion. Such an expansion is just a sequence that converges to $x$. What follows are some continued fraction expansions; try to find the number $x$ that the sequence converges to (you don't have to worry about whether there is actually a limit, you may assume there is). The terms are purposefully written in a rather silly way. Parts a) and b) are infinite sequences. 

\bigskip 

a) $1, 1+\frac 1{1+1},1+\frac 1 {1+\frac 1 {1+1}},1+\frac 1 {1+\frac 1 {1+\frac 1 {1+1}}},\dots$ 

\bigskip

b) $2, 2+\frac 1 {2+2},2+\frac 1 {2+\frac 1 {2+2}},2+\frac 1 {2+\frac 1 {2+\frac 1 {2+2}}},\dots$ (If you've gotten a), this should be easy.)

\bigskip

c) Notice what kind of numbers you're getting in a) and b). If you have a \textit{finite} continued fraction expansion, what kind of number results? Can you determine a continued fraction expansion for such a number?

	
\index{Sequences}
	
\begin{tags}
	    Sequences, Theory, Challenge, W12
\end{tags}
	
\begin{diary}
	   
\end{diary}
	
\begin{solution}	

\end{solution}
	
\end{question}

\end{tagblock}

%-------------------------------------------------------------------------------------------------------------

