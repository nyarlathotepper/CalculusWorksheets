\section{Parametric}\index{Parametric}
\fancyhead[R]{\large Parametric}

\begin{tagblock}{Parametric, Geometry, Graph, Trigonometry, Challenge, W21}
\begin{question}
Parameterization is a way of bringing graphs that don't play by the rules of regular functions into the fold. 

\bigskip

a) Draw a graph that is NOT the graph of a function $y=f(x)$. What important geometric test does the graph fail?

\bigskip

If you couldn't come up with an example for 1a), the easiest thing to draw is a circle. Let's take the unit circle defined by the equation $x^2+y^2=1$. 

A \textit{parameterization} of the circle is a choice of functions $x$ and $y$ depending on the real variable $t$ that together satisfy the equation $x^2+y^2=1$. 

\bigskip

b) Check that $x(t)=\cos(t)$, $y(t)=\sin(t)$ parameterizes the unit circle. 

\bigskip

c) Find another parameterization for the unit circle (so parameterizations aren't unique).

\bigskip

d) Find infinitely many parameterizations of the unit circle (so parameterizations REALLY aren't unique)!
	
\index{Parametric}
	
\begin{tags}
	    Parametric, Geometry, Graph, Trigonometry, Challenge, W21
\end{tags}
	
\begin{diary}
	   % S2016-HW1-Q1
\end{diary}
	
\begin{solution}
	   
\end{solution}
	
\end{question}

\end{tagblock}

%-------------------------------------------------------------------------------------------------------------
\begin{tagblock}{Parametric, Graph, Challenge, Definition, Example, Theory, W21}
\begin{question}
A \textit{parametric curve} is just the collection of all $(x,y)$ values that satisfy a given parameterization $x=x(t)$ and $y=y(t)$. The dependence on $t$ is (almost) completely suppressed, but when you draw the curve, you usually write an arrow in the direction the $(x,y)$ values move as $t$ increases.

\bigskip

a) Try this for the parameterizations of the unit circle from 1b) and 1c).

\bigskip

One of the great features of parameterizations is that they completely subsume regular Cartesian functions! For example, if $y=f(x)=x^2$, a parameterization can be almost cheatingly obtained by 
\[
x(t)=t \ \textrm{and} \ y(t)=t^2.
\] 

\bigskip
 
b) Find a parameterization for $y=\cos(x)$. Draw the direction you move as $t$ increases. 

\bigskip

c) Find a parameterization for an arbitrary Cartesian function $y=f(x)$.
	
\index{Parametric}
	
\begin{tags}
	    Parametric, Graph, Challenge, Definition, Example, Theory, W21
\end{tags}
	
\begin{diary}
	   % S2016-HW1-Q1
\end{diary}
	
\begin{solution}
	   
\end{solution}
	
\end{question}

\end{tagblock}

%-------------------------------------------------------------------------------------------------------------
\begin{tagblock}{Parametric, Graph, Challenge, Exponentials, Logarithms, Trigonometry, W21}
\begin{question}

Find Cartesian equations for the following parametric curves in a) through d). 

\bigskip

a) $x(t)= \ln(7t)$, $y(t)=\ln(5t)$

\bigskip

b) $x(t)= e^t$, $y(t)=e^{-t}$

\bigskip

c) $x(t)= 2\cos(t)$, $y(t)=3\sin(t)$.

\bigskip

d) $x(t)=\dfrac{\cos(t)}{4}$, $y(t)=\dfrac{\sin(t)}9$.
	
\index{Parametric}
	
\begin{tags}
	    Parametric, Graph, Challenge, Exponentials, Logarithms, Trigonometry, W21
\end{tags}
	
\begin{diary}
	   % S2016-HW1-Q1
\end{diary}
	
\begin{solution}
	   
\end{solution}
	
\end{question}

\end{tagblock}

%-------------------------------------------------------------------------------------------------------------
\begin{tagblock}{Parametric, Graph, Theory, TangentLines, Derivatives, W22}
\begin{question}

 Now that we have introduced parameterizations, what else would we do but figure out a way to do some calculus with them? The first thing on our list is to recover derivatives and tangent lines. 

\bigskip

a) If you parameterize $y=f(x)$ as $x(t)=t$, $y(t)=f(t)$, what is the slope of the tangent line to the graph of $f$ at the point $t=1$?

\bigskip

b) Look at the coordinates from the previous example and see if you can express the slope of the tangent line in terms of the $x$ and $y$ derivatives.

\bigskip

c) Generalize your answer in b) to arbitrary parametric curves. What happens when $y'(1)=0$?
	
\index{Parametric}
	
\begin{tags}
	    Parametric, Graph, Theory, TangentLines, Derivatives, W22
\end{tags}
	
\begin{diary}
	   % S2016-HW1-Q1
\end{diary}
	
\begin{solution}
	   
\end{solution}
	
\end{question}

\end{tagblock}

%-------------------------------------------------------------------------------------------------------------
\begin{tagblock}{Parametric, TangentLines, Derivatives, Exponentials, Logarithms, Trigonometry, W22}
\begin{question}

Find an equation for the tangent line for the following parametric curves at the given point, then parameterize the tangent lines.

\bigskip

a) $x(t)=\ln(t^2), \ y(t)=16^{\frac t 2}$, $t=1$

\bigskip

b) $x(t)=(e\cos(t))^{\sin(t)}, \ y(t)=\arctan(t^2+1)$, $(x,y)=(1,\pi/4)$
	
\index{Parametric}
	
\begin{tags}
	    Parametric, TangentLines, Derivatives, Exponentials, Logarithms, Trigonometry, W22
\end{tags}
	
\begin{diary}
	   % S2016-HW1-Q1
\end{diary}
	
\begin{solution}
	   
\end{solution}
	
\end{question}

\end{tagblock}

%-------------------------------------------------------------------------------------------------------------
\begin{tagblock}{Parametric, ArcLength, Derivatives, Integration, Definition, Trigonometry, Logarithms, Challenge, W22}
\begin{question}

Now for another ``geometric" application of parametric curves: arc length!

\bigskip

If you draw a sufficiently nice curve, you can approximate the length of the curve by drawing line segments between points on the curve.

\bigskip

a) Draw a curve and try doing this. 

\bigskip

The distance between points on the curve is given by the \textit{distance formula}, which will also give you the lengths of the line segments used to approximate the length of the curve. If we take the limit as the number of segments goes to infinity (and the length goes to zero), we arrive at the following formula for the arc length $L$ of a parametric curve from $t=a$ to $t=b$:

\bigskip
\[
L=\int_a^b\sqrt{(x'(t))^2+(y'(t))^2} \ dt.
\]

Calculate the arc length for the following curves: 

\bigskip

b) Take your favorite line, write down a parameterization, choose two $t$ values, and check that this formula actually works. 

\bigskip

c) $x(t)=\sin(\pi\sin(t))$, $y(t)=\cos(\pi\sin(t))$ from $t=0$ to $t=\pi/6$

\bigskip

d) $x(t)=\ln(t)$, $y(t)=t$ from $t=1$ to $t=e^2$

\bigskip

e) $y=x^2$ from $x=0$ to $x=1$ (this is actually horrendously hard- contrast it with part b)!)
	
\index{Parametric}
	
\begin{tags}
	    Parametric, ArcLength, Derivatives, Integration, Definition, Trigonometry, Logarithms, Challenge, W22
\end{tags}
	
\begin{diary}
	   % S2016-HW1-Q1
\end{diary}
	
\begin{solution}
	   
\end{solution}
	
\end{question}

\end{tagblock}

%-------------------------------------------------------------------------------------------------------------
\begin{tagblock}{Parametric, ArcLength, Graph, Derivatives, Integration, Theory, Challenge, W22}
\begin{question}

Let $C$ be the circle $x^2+y^2=1$.  

\bigskip

a) Find a parameterization for $C$. 

\bigskip

b) Let $0\leq \theta_0<2\pi$ be any angle in \textit{radians}. Show that the length of the portion of the graph of $C$ from the angle zero to the angle $\theta_0$ is precisely equal to $\theta_0$. 

\bigskip

c) Does the result from part b) help you understand what radian measure actually means (if you didn't already know it)?
	
\index{Parametric}
	
\begin{tags}
	    Parametric, ArcLength, Derivatives, Integration, Theory, Challenge, W22
\end{tags}
	
\begin{diary}
	   % S2016-HW1-Q1
\end{diary}
	
\begin{solution}
	   
\end{solution}
	
\end{question}

\end{tagblock}

%-------------------------------------------------------------------------------------------------------------