\section{Derivatives}\index{Derivatives}
\fancyhead[R]{\large Derivatives}



\begin{tagblock}{TangentLines, Derivatives, Graph, Definition}
\begin{question}
	

\bigskip

\textbf{The Tangent Problem}

Suppose we have a curve given by the function $f(x)$.  For a given value $x=a$, we have a point on the curve, namely $(a, f(a))$.  A line is \emph{tangent} to the curve, if it ``just touches'' the curve.  For example:
\begin{figure}[h]
\centering
\includegraphics[width=3cm,height=3cm]{tangent2.jpg}
\end{figure}
%graph of x^2 with tangent line at (1,1)

\textbf{Goal:} Given a curve and a point on the curve, find the equation of the tangent line to the curve at that point.

\bigskip
 Consider the function $f(x)=x^3$, and the point $P=(1,1)$ on the curve.  
\begin{figure}[h]
\centering
\includegraphics[width=8cm,height=8cm]{tangent3.jpg}
\end{figure}
%graph of x^3 with point (1,1)

a)  Try your best to draw a tangent line to the curve at the point $P=(1,1)$ 
 
 \bigskip
 
To find the slope of the line you just drew, you need two points on the line.  We have one point, namely $(1,1)$, and certainly we could guess another point from our drawing.  However, we would like to be more precise.  To do so, we will compute slopes of secant lines.  A \emph{secant line} of a curve is a line between any \emph{two} points on a curve.  

\bigskip
b) Plot and label the points on the graph of $f(x) = x^3$ (It will also help to compute the $y$-value)
\[ Q_0 = (0, f(0)), \,Q_{1/2} = (\frac{1}{2}, f(\frac{1}{2})) , \, Q_{3/4} = (\frac{3}{4}, f(\frac{3}{4}) ),  \, Q_{7/8} = (\frac{7}{8}, f(\frac{7}{8}))  \]

\vspace{.5in}
c) Draw a secant line to the curve at the points $P = (1,1)$ and $Q_0 = (0,f(0))$ (that is draw a line connecting $(1,1)$ and $(0,f(0)$ ). Note that we have two points on the line.... Compute the slope of this line.

\vspace{.75in}

d) Draw a secant line to the curve at the points $P$ and $Q_{1/2} =  (\frac{1}{2}, f(\frac{1}{2}))$.  Compute the slope of the line.  Notice that the point $Q_{1/2}$ is closer to $P$ than $Q_0$.

\vspace{.75in}

e)  Compute the slope of the secant line through the points $P$ and $Q_{3/4}$. 


\vspace{.75in}

f)  Compute the slope of the secant line through the points $P$ and $Q_{7/8}$. 

\vspace{.75in}

g) In our example $P=(1,1)$, and the point $(1 + h, f(1+ h))$ is ``close to'' $P$ (in part c) $h=-1$, in part d) $h=-0.5$, in part e) $h=-0.25$).   Compute the slope of the secant line between $P$ and $(1 + h, f(1+ h))$.   Be sure to fully simplify your answer.  (Note this is the \textbf{difference quotient} we looked at on the first day)
\vspace{1.5in}

 h) The tangent line to our curve at $(1,1)$ can be estimated by the secant line between the point $(1,1)$ and a point ``close to $(1,1)$,'' that is a point $Q=(1+h, f(1+h) )$, where $h$ is small, either positive or negative.    Using the language we have learned thus far, 
 \[ \text{slope of the tangent line to $f(x)$ at }(1,1) = \lim_{h \to 0} \{ \text{slope of a secant line between the points $P=(1,1)$ and $(1+h, f(1+h) )$} \} \]
 Using your answer to g), compute the slope of the tangent line to $f(x) = x^3$ at the point $P=(1,1)$.


 
 \vspace{1in}
 i) Now write an equation for the tangent line to $f(x)=x^3$ at the point $P=(1,1)$
  
 \vspace{.5in}

\bigskip


	
\index{Derivatives}
	
\begin{tags}
	    TangentLines, Derivatives, Graph, Definition
\end{tags}
	
\begin{diary}
	    %S2016-HW1-Q1
\end{diary}
	
\begin{solution}
	   
\end{solution}
	
\end{question}

\end{tagblock}

%-------------------------------------------------------------------------------------------------------------

\begin{tagblock}{TangentLines, Velocity, Derivatives, Graph}
\begin{question}
	

\bigskip

\textbf{Computing Velocity}

As a second application we will look at velocity.  We define the \emph{average velocity} of an object as
\[ \text{average velocity} = \frac{ \text{change in position}}{\text{time elapsed}} \]

Suppose an object moves so that it's position after $t$ seconds is $s(t)=3t^2$ feet from where it started.
\bigskip

a) Find the average velocity from $t=2$ to $t=3$ seconds.
\vspace{.5in}

b) Find the average velocity from $t=2$ to $t=2.5$ seconds.
\vspace{.5in}

c) Find the average velocity from $t=2$ to $t=2+h$ seconds (where $h>0$).  Simplify your answer.
\vspace{1in}

d) We define the \emph{instantaneous velocity} at $t=2$ to be the limit of the average velocities (from part c)) as $h$ approaches $0$.   Find the instantaneous velocity at $t=2$.

\vspace{.8in}

e)  Below is a graph of the particles distance $s(t)$.  How can you see the average velocity from parts a) and b) and the instantaneous velocity at $t=2$ from the graph. (Hint: what did we do in the first part?)
\begin{figure}[h]
\centering
\includegraphics[width=8cm,height=8cm]{velocity1.jpg}
\end{figure}


\bigskip


	
\index{Derivatives}
	
\begin{tags}
	    TangentLines, Velocity, Derivatives, Graph
\end{tags}
	
\begin{diary}
	    %S2016-HW1-Q1
\end{diary}
	
\begin{solution}
	   
\end{solution}
	
\end{question}

\end{tagblock}

%-------------------------------------------------------------------------------------------------------------

\begin{tagblock}{Derivatives, Definition}
\begin{question}
	

\bigskip

\textbf{Notation:}  If $y=f(x)$ we have many notations for the derivative:  $f'(x)$, $y'$, $\frac{dy}{dx}$ and $\frac{d}{dx}[f(x)]$ all mean the derivative of $f$.  



So given a function $f(x)$, to compute the derivative $f'(x)$, we form the difference quotient $\displaystyle \frac{f(x+h)-f(x)}{h}$ and then do some algebra (like we did earlier) until we take the limit as $h \to 0$.  





\begin{enumerate}

\item Let $f(x) = x^2$.  \textbf{Using the definition of the derivative} we will compute $f'(x)$.  I've started this off for you, finish up the computation.  

\[f'(x) = \lim_{h \to 0} \frac{f(x+h) - f(x)}{h} =  \lim_{h \to 0} \frac{(x+h)^2 - x^2}{h} \]


\vspace{5in}



\item Let $f(x) = \sqrt{3x+2}$.  Use the definition of the derivative to get an equation for $f'(x)$.  \emph{Hint: multiply and divide by the conjugate}

\vfill  As we see, using the definition of the derivative can be lengthy computation.  Our next goal will be to develop tools to compute derivatives efficiently.  

\end{enumerate}

\bigskip


	
\index{Derivatives}
	
\begin{tags}
	    Derivatives, Definition
\end{tags}
	
\begin{diary}
	    %S2016-HW1-Q1
\end{diary}
	
\begin{solution}
	   
\end{solution}
	
\end{question}

\end{tagblock}


%-------------------------------------------------------------------------------------------------------------

\begin{tagblock}{Derivatives, Definition, Graph}
\begin{question}
	

\bigskip







Let's look at two easy functions, the constant function, $f(x) = C$ and the linear function $g(x) =x$.  

\begin{figure}[h]
\centering
\includegraphics[width=5cm]{constant.png} \hspace{.05in} \includegraphics[width=5cm]{line.png}
\end{figure}
Both are lines, so if we draw a tangent line at any value $a$ we'll just get the same line. 
\begin{enumerate}
\item What is the slope of the constant line $f(x) = C$?  
\item What is the slope of $g(x) = x$?
\end{enumerate}

\bigskip


This tells us that $\frac{d}{dx}[C] = 0$ for any constant and $\frac{d}{dx}[x] = 1$.


	
\index{Derivatives}
	
\begin{tags}
	    Derivatives, Definition, Graph
\end{tags}
	
\begin{diary}
	    %S2016-HW1-Q1
\end{diary}
	
\begin{solution}
	   
\end{solution}
	
\end{question}

\end{tagblock}

%-------------------------------------------------------------------------------------------------------------

\begin{tagblock}{Derivatives, Definition, PowerRule}
\begin{question}
	

\bigskip



Next we'll consider functions of the form $f(x) = x^n$.  
\begin{enumerate}
\item From the previous problem we know $\frac{d}{dx}[x]=1$.
\item In Problem 1 you computed $\frac{d}{dx}[x^2] = $


\item Using the definition of the derivative, find $f'(x)$ for $f(x) = x^3$. \\
(Hint $(x+h)^3 = x^3+3x^2h+3xh^2 + h^3$).  

%\item Based on your work in (a), (b), and (c), (d) what do you conjecture is the derivative of $f(x) = x^{11}$ is?
\vfill

\item Conjecture a formula for the derivative of $f(x)=x^n$ that holds for any positive integer $n$. That is, given $f(x)=x^n$ where $n$ is a positive integer, what do you think is the formula for $f'(x)$?
\end{enumerate}
\newpage

Hopefully you conjectured the:

\begin{center}\textbf{Power Rule}: If $f(x) = x^n$, then $f'(x) = nx^{n-1}$. \end{center}

In fact, the Power Rule works for any $n$, not just positive integers:

\textbf{Example}: If $g(x) = x^{-3}$, then $g'(x) = -3x^{-3-1} = -3x^{-4}$ \\
\bigskip

\textbf{Example}: If $h(t) = t^{3/5}$, then $h'(t) = \frac{3}{5}t^{3/5-1} =  \frac{3}{5}t^{-2/5}$ \\






	
\index{Derivatives}
	
\begin{tags}
	    Derivatives, Definition, PowerRule
\end{tags}
	
\begin{diary}
	    %S2016-HW1-Q1
\end{diary}
	
\begin{solution}
	   
\end{solution}
	
\end{question}

\end{tagblock}

%-------------------------------------------------------------------------------------------------------------

\begin{tagblock}{Derivatives, Definition, PowerRule}
\begin{question}
	

\bigskip



Compute the following derivatives:
\begin{enumerate}
\item $f(x) = \sqrt{x}$ (Remember $\sqrt{x} = x^{1/2}$)
\vspace{.7in}

\item $g(w) = w^{4/3}$
\vspace{.7in}

\item $\displaystyle s(t) = \frac{1}{t^5}$
\vspace{.7in}

\item $h(x) = \sqrt{2}$
\end{enumerate}






	
\index{Derivatives}
	
\begin{tags}
	    Derivatives, Definition, PowerRule
\end{tags}
	
\begin{diary}
	    %S2016-HW1-Q1
\end{diary}
	
\begin{solution}
	   
\end{solution}
	
\end{question}

\end{tagblock}

%-------------------------------------------------------------------------------------------------------------

\begin{tagblock}{Derivatives, Example, PowerRule, SumRule}
\begin{question}
	

\bigskip



Most functions we encounter are a bit more complicated than simply a constant or a power.  For example, we might want to compute the derivative of the polynomial $p(x) = 5x^7 + 6x^3 -3x +2$, which is a function made up of constant multiples and sums of powers of $x$.  For this we'll need the \textbf{Constant Multiple Rule} and the \textbf{Sum Rule}.

%Suppose we have a function $y=f(x)$ whose derivative formula is known. How is the derivative of $y=cf(x)$ related to the derivative of the original function? Recall that when we multiply a function by a constant $c$, we vertically stretch the graph by a factor of $|c|$ (and reflect the graph across $y=0$ if $c<0$). This vertical stretch affects the slope of the graph, making the slope of the function $y=cf(x)$ be $c$ times as steep as the slope of $y=f(x)$. In terms of the derivative, this is essentially saying that when we multiply a function by a factor of $c$, we change the value of its derivative by a factor of $c$ as well.

\begin{center}\textbf{Constant Multiple Rule}: For any constant $c$, and any differentiable function $f(x)$, $$\frac{d}{dx}[cf(x)] = c \frac{d}{dx}[f(x)].$$\end{center}

In words, this rule says that ``the derivative of a constant times a function is the constant times the derivative of the function.''

\textbf{Example:} $\frac{d}{dx} [5x^4] = 5 \frac{d}{dx}[x^4] = 5 \cdot 4 x^3 = 20 x^3$.  

\newpage
Next we see what happens when we take a sum of two functions. %If we have $y=f(x)$ and $y=g(x)$, we can compute a new function $y=(f+g)(x)$ by adding the outputs of the two functions: $(f+g)(x)=f(x)+g(x)$. Not only does this result in the value of the new function being the sum of the values of the two known functions, but also the slope of the new function is the sum of the slopes of the known functions. Therefore, we arrive at the following Sum Rule for derivatives

\begin{center}\textbf{Sum Rule}: If $f(x)$ and $g(x)$ are differentiable functions with derivatives $f'(x)$ and $g'(x)$, then $\frac{d}{dx}[f(x)+g(x)] =f'(x)+g'(x)$. \end{center}

In words, the Sum Rule tells us that ``the derivative of a sum is the sum of the derivatives.''

We now can find the derivative of any polynomial.  \emph{In the below example I'll show every step; but as you get comfortable with the rules, you don't need to show this much detail}

\textbf{Example:} We'll find the derivative of $p(x) = 5x^7 + 6x^3 -3x +2$.  We'll start by using our Sum Rule
\[ p'(x) = \frac{d}{dx} [5x^7] +  \frac{d}{dx} [6x^3] +  \frac{d}{dx} [-3x] + \frac{d}{dx} [2] \]
Then apply the Constant Multiple Rule
\[p'(x) =5 \frac{d}{dx} [x^7] + 6 \frac{d}{dx} [x^3] + (-3) \frac{d}{dx} [x] + \frac{d}{dx} [2]\]
And finally the Power Rule and Constant Rule
\[p'(x) =5 \cdot 7x^6+ 6 \cdot 3x^2 + (-3)\cdot 1+ 0 = 35x^6 + 18x^2 -3 \]



\item Use only the rules for constant, and power functions, together with the Constant Multiple and Sum Rules, to compute the derivative of each function below with respect to the given independent variable. Note that we do not yet know any rules for how to differentiate the product or quotient of functions. This means that you may have to do some algebra first on the functions below before you can actually use existing rules to compute the desired derivative formula.
\begin{enumerate}
\item $f(x) = x^{5/3} + x^7 - 3$
\vspace{.7in}

\item $\displaystyle h(z) = \sqrt{z} + \frac{1}{z^4} + \sqrt{7}z^3$
\vspace{.7in}

\item $s(t) = (t^2+1)(t^2-1) $
\vspace{.7in}

\item $\displaystyle q(x) = \frac{x^3-x+1}{x}$.
\end{enumerate} 






	
\index{Derivatives}
	
\begin{tags}
	    Derivatives, Example, PowerRule, SumRule
\end{tags}
	
\begin{diary}
	    %S2016-HW1-Q1
\end{diary}
	
\begin{solution}
	   
\end{solution}
	
\end{question}

\end{tagblock}

%-------------------------------------------------------------------------------------------------------------

\begin{tagblock}{Derivatives, TangentLine, Graph}
\begin{question}
	

\bigskip



Let $\displaystyle f(x) = x^3 +2x^2-3x^{\frac{1}{3}}-2$.  The graph of $f(x)$ is given below:
\begin{figure}[h]
\centering
\includegraphics[width=8cm]{recaptangent.png}  \end{figure}

\begin{enumerate}
\item Draw the tangent line to $f(x)$ at $(1,-2)$.
\item Compute $f'(x)$ 

\vspace{1in}

\item Find the slope of the tangent line to $f(x)$ at $(1,-2)$, and then write the equation of the tangent line to $f(x)$ at $(1,-2)$.
\vspace{1in}

\item Draw the tangent line to $f(x)$ at $(0,-2)$.  What kind of line do you get?
\vspace{.5in}


\item Using the formula for $f'(x)$, can you calculate the slope of the tangent line to $f(x)$ at $(0,-2)$?  Explain.
\vspace{1in}
\end{enumerate}







	
\index{Derivatives}
	
\begin{tags}
	    Derivatives, TangentLine, Graph
\end{tags}
	
\begin{diary}
	    %S2016-HW1-Q1
\end{diary}
	
\begin{solution}
	   
\end{solution}
	
\end{question}

\end{tagblock}

%-------------------------------------------------------------------------------------------------------------

\begin{tagblock}{Derivatives, SecondDerivative, HigherDerivatives, Velocity}
\begin{question}
	




\textbf{Higher Derivatives}: Starting with an equation of a function $f(x)$, we now have tools to compute the derivative $f'(x)$.  We then could differentiate $f'(x)$ and get a new function $f''(x)$: the \emph{second derivative of $f(x)$}.  We could continue differentiating to get the third derivative $f'''(x)$, the fourth derivative $f'''(x)$, \ldots.  After a while we'd get tired of writing so many primes, and instead write $f^{[n]}(x)$ for the $n^{th}$ derivative  
\begin{enumerate}
\item  Find the first and second derivative of $f(x) = \sqrt{x} + 5e^x$ 
\vspace{1.5in}
\item A useful application of the second derivative is as follows: given a function $s(t)$ that gives the position of a function at time $t$, the velocity function, $v(t)$, is the derivative of position: $v(t) = s'(t)$.  If we differentiate again, we will get the acceleration of the object at time $t$, $a(t)$; that is $a(t) = v'(t) = s''(t)$.  \\
\smallskip

If the motion of a particle is given by $s(t) = t^4 - 2t^3 +t^2 -t$, where $s$ is in meters and $t$ is in seconds, find
\begin{enumerate}
\item the velocity and acceleration as functions of $t$
\vspace{2in}

\item the acceleration after $1$ sec (include units in your answer)
\end{enumerate}


\end{enumerate}








	
\index{Derivatives}
	
\begin{tags}
	    Derivatives, SecondDerivative, HigherDerivatives, Velocity
\end{tags}
	
\begin{diary}
	    %S2016-HW1-Q1
\end{diary}
	
\begin{solution}
	   
\end{solution}
	
\end{question}

\end{tagblock}

%-------------------------------------------------------------------------------------------------------------

\begin{tagblock}{Derivatives, HigherDerivatives, Polynomial}
\begin{question}
	




If $g(x) = 5x^4 + 3x^2 - 7x + \sqrt{2}$, how many derivatives would we need to take until we get to $0$? Explain. 











	
\index{Derivatives}
	
\begin{tags}
	    Derivatives, HigherDerivatives, Polynomial
\end{tags}
	
\begin{diary}
	    %S2016-HW1-Q1
\end{diary}
	
\begin{solution}
	   
\end{solution}
	
\end{question}

\end{tagblock}

%-------------------------------------------------------------------------------------------------------------

\begin{tagblock}{Derivatives, HigherDerivatives, Polynomial, Theory}
\begin{question}
	




If $p(x) = a_nx^n + a_{n-1}x^{n-1} + \cdots + a_1x + a_0$ is any polynomial, how many derivatives would we need to take until we get to $0$?  Explain.











	
\index{Derivatives}
	
\begin{tags}
	    Derivatives, HigherDerivatives, Polynomial, Theory
\end{tags}
	
\begin{diary}
	    %S2016-HW1-Q1
\end{diary}
	
\begin{solution}
	   
\end{solution}
	
\end{question}

\end{tagblock}

%-------------------------------------------------------------------------------------------------------------

\begin{tagblock}{Derivatives, TangentLine, ProductRule, Graph}
\begin{question}
	




Using the definition of the derivative, we just saw how to take the derivative of a product:  
\bigskip



\noindent\fbox{%
    \parbox{\textwidth}{
\textbf{Product Rule:} If $P(x)=f(x) \cdot g(x)$, then $P'(x) = f'(x)g(x) + f(x) g'(x)$.  




\bigskip


In words what the product rule says: if $P$ is the product of two functions $f$ (the first function) and $g$ (the second), then ``the derivative of $P$ is the first times the derivative of the second, plus the second times the derivative of the first.''}}


\bigskip

Let $P(x) = (x^5+3x^2 - \frac{1}{x}) (\sqrt{x} + \frac{x}{3})$, which is graphed on the right.

 
\begin{minipage}{.6\textwidth}
\begin{enumerate}
\item  Use the product rule to compute $P'(x)$.  (It is not necessary to algebraically simplify) %(Note that one could multiply out $P(x)$ and then use our previous rules to compute $P'(x)$. ) 
\item Draw the tangent line to $P(x)$ at $x=1$.  
\item    Find slope of the tangent line to $P(x)$ at $x=1$.  
\item Give the equation of the tangent line to $P(x)$ at $x=1$.  
%\item  Determine the slope of the tangent line to the curve $y=f(x)$ at the point where $a=\pi$ if $f$ is given by the rule $f(x)=x^3\sin(x)$.
\end{enumerate}
\end{minipage}% This must go next to `\end{minipage}`
\begin{minipage}{.4\textwidth}
\includegraphics[width=\textwidth]{product1.png}
\end{minipage}
%\begin{figure}[h]
%\centering
%\end{figure}











	
\index{Derivatives}
	
\begin{tags}
	    Derivatives, TangentLine, ProductRule, Graph
\end{tags}
	
\begin{diary}
	    %S2016-HW1-Q1
\end{diary}
	
\begin{solution}
	   
\end{solution}
	
\end{question}

\end{tagblock}

%-------------------------------------------------------------------------------------------------------------

\begin{tagblock}{Derivatives, TangentLine, QuotientRule}
\begin{question}
	









Similarly, we have a rule for differentiating quotients

\noindent\fbox{%
    \parbox{\textwidth}{
\textbf{Quotient Rule:} If $\displaystyle Q(x) = \frac{f(x)}{g(x)}$ then $\displaystyle Q'(x) = \frac{g(x)f'(x)-f(x)g'(x)}{(g(x))^2} $.  

\bigskip

In words, if a function $Q$ is the quotient of a top function $f$ and a bottom function $g$, then $Q'$ is given by ``the bottom times the derivative of the top, minus the top times the derivative of the bottom, all over the bottom squared.''  }}

\bigskip

Use the quotient rule to answer each of the questions below.  It is not necessary to algebraically simplify any of the derivatives you compute.
\begin{enumerate}
\item  Let $\displaystyle f(z)=\frac{z^3}{z^4+1}.$  Find $f'(z)$.
\vspace{1in}
%\item Let $\displaystyle v(t)=\frac{\sin(t)}{\cos(t)+t^2}$.  Find $v'(t).$
\item Determine the slope of the tangent line to the curve $\displaystyle R(x)=\frac{x^2-2x-8}{x^2-9}$ at the point where $x=0$.  
\end{enumerate}





	
\index{Derivatives}
	
\begin{tags}
	    Derivatives, TangentLine, QuotientRule
\end{tags}
	
\begin{diary}
	    %S2016-HW1-Q1
\end{diary}
	
\begin{solution}
	   
\end{solution}
	
\end{question}

\end{tagblock}

%-------------------------------------------------------------------------------------------------------------

\begin{tagblock}{Derivatives, Graphs, Differentiable}
\begin{question}
	


So far we have been given a function via a formula and found a formula for the derivative.  We next will take a different perspective and ask  given a graph of $y=f(x)$, how does this graph lead to the graph of the derivative function $y=f'(x)$? 

\bigskip

Below on the left is a graph of $g(x) = |x|$.  On the right sketch a graph of $g'(x)$. \emph{Remember that the derivative gives slopes of tangent lines}  


\begin{figure}[h]
\centering
\includegraphics[width=5cm]{absolutevalue.png} \hspace{.2in} \includegraphics[width=5cm]{derivblank.png} \end{figure}

Write 2-3 sentences explaining how you know that you have determined the correct graph. 

\vspace{2in}
Notice that something funny is happening at $x=0$.  At $x=0$ we can't draw the tangent line to $g(x)$.  We say that $g$ is \emph{not differentiable at $x=0$}.

In general:  we say that a function $f(x)$ is \emph{differentiable at $x=a$} if $f'(a)$ exists.

\index{Derivatives}
	
\begin{tags}
	    Derivatives, Graphs, Differentiable
\end{tags}
	
\begin{diary}
	    %S2016-HW1-Q1
\end{diary}
	
\begin{solution}
	   
\end{solution}
	
\end{question}

\end{tagblock}


%-------------------------------------------------------------------------------------------------------------

\begin{tagblock}{Derivatives, Graphs, Differentiable, Theory, Continuity}
\begin{question}
	
{How can a function fail to be differentiable at $a$?}

\begin{enumerate}
\item \textbf{If we can't draw a tangent line to $f(x)$ at $x=a$:}  
Consider the function $f(x)$ given by the graph below:
\begin{minipage}{.4\textwidth}
\includegraphics[width=5cm]{notcts.png} 
\end{minipage}% This must go next to `\end{minipage}`
\begin{minipage}{.6\textwidth}
Is $f$ continuous at $x=1$? \\ \vspace{.5in}

 Can you draw a tangent line at $x=1$?  
\end{minipage}


The sharp corner seen in the previous problem also prevented us from drawing a tangent line.  
\bigskip


\item \textbf{$f$ has a vertical tangent line at $x=a$}: Consider the function $g(x)$ given by the graph below:
\begin{minipage}{.4\textwidth}
\includegraphics[width=5cm]{vertical.png} 
\end{minipage}% This must go next to `\end{minipage}`
\begin{minipage}{.6\textwidth}
Draw the tangent line to $g(x)$ at $x=1$.  Can you compute it's slope?  Why or why not?  

\end{minipage}




\item  Determine if the following statements are TRUE or FALSE:

If a function $h(x)$ is \textbf{continuous} at $x=a$, then it is \textbf{differentiable} at $x=a$. \hfill TRUE \hspace{.2in} FALSE

\bigskip


If a function $h(x)$ is \textbf{differentiable} at $x=a$, then it is \textbf{continuous} at $x=a$. \hfill TRUE \hspace{.2in} FALSE



\end{enumerate}













	









	
\index{Derivatives}
	
\begin{tags}
	    Derivatives, Graphs, Differentiable, Theory, Continuity
\end{tags}
	
\begin{diary}
	    %S2016-HW1-Q1
\end{diary}
	
\begin{solution}
	   
\end{solution}
	
\end{question}

\end{tagblock}

%-------------------------------------------------------------------------------------------------------------

\begin{tagblock}{Derivatives, QuotientRule, Theory}
\begin{question}
	
One can prove the Quotient Rule, but using the product rule.  Let $\displaystyle Q(x) = \frac{f(x)}{g(x)}$, where $f$ and $g$ are differentiable functions.   We begin by observing that we can multiply by $g(x)$ to get
\[ f(x) = Q(x) \cdot g(x) \]
If we differentiate  both sides, using the product rule on the right hand side we get
\[ f'(x) =   \rule[-0.1cm]{8cm}{0.01cm} \] 
Now solve the above equation for $Q'(x)$ (your answer will involve $f'(x), g'(x)$ and $Q(x)$.)

\vspace{1.5in}
Replacing $\displaystyle Q(x) = \frac{f(x)}{g(x)}$ and multiplying by a clever choice of $\displaystyle 1 = \frac{g(x)}{g(x)}$, we get
\begin{eqnarray*}
Q'(x) &=& \frac{f'(x)-Q(x)g'(x)}{g(x)}\\
&=&  \frac{f'(x)-\frac{f(x)}{g(x)}g'(x)}{g(x)} \cdot \frac{g(x)}{g(x)} \\
&=&  \frac{g(x)f'(x)-f(x)g'(x)}{(g(x))^2}  \\
\end{eqnarray*}
















	










	
\index{Derivatives}
	
\begin{tags}
	    Derivatives, QuotientRule, Theory
\end{tags}
	
\begin{diary}
	    %S2016-HW1-Q1
\end{diary}
	
\begin{solution}
	   
\end{solution}
	
\end{question}

\end{tagblock}

%-------------------------------------------------------------------------------------------------------------

\begin{tagblock}{Derivatives, QuotientRule, ProductRule}
\begin{question}
	



Let $f(x)$ and $g(x)$ be continuous functions with $f(-3) = 4$, $g(-3)=2$, $f'(-3) = \frac{1}{3}$ and $g'(-3) =-4$. 
\begin{enumerate}
\item  Let $P(x) = f(x)g(x) + x^2f(x)$.  
\begin{enumerate} 
\item  Determine a formula for $P'(x)$ in terms of $f(x), g(x), f'(x)$ and $g'(x)$.

\vspace{2in}

\item Determine $P'(-3)$.

\end{enumerate} 

\vspace{1.5in}

\item  Let $\displaystyle Q(x) = \frac{3f(x) + e^x}{g(x)}$.  Determine $Q'(-3)$


\end{enumerate} 














	










	
\index{Derivatives}
	
\begin{tags}
	    Derivatives, QuotientRule, ProductRule
\end{tags}
	
\begin{diary}
	    %S2016-HW1-Q1
\end{diary}
	
\begin{solution}
	   
\end{solution}
	
\end{question}

\end{tagblock}

%-------------------------------------------------------------------------------------------------------------

\begin{tagblock}{Derivatives, Trigonometry, Graphs, TangentLines}
\begin{question}
	






 In this worksheet we'll look at trig functions and their derivatives.  

\begin{enumerate}

\item Consider the function $f(x)=\sin(x)$, which is graphed in below. Note carefully that the grid in the diagram does not have boxes that are $1\times 1$, but rather approximately $1.57 \times 1$, as the horizontal scale of the grid is $\pi/2$ units per box. 

\begin{figure}[h]
%\centering
\includegraphics[width=8cm]{sinxwithtangent.png} \hfill \includegraphics[width=8cm]{blank.png}

\end{figure}
\begin{enumerate}
\item At each of $x=-2 \pi, -\frac{3\pi}{2}, - \pi, -\frac{\pi}{2}, 0, \frac{\pi}{2}, \pi, \frac{3\pi}{2}, 2\pi$ use a straightedge to sketch an accurate tangent line to $y=f(x)$.  (I've already drawn in the tangent line at $x=0$) 

\item Use the provided grid to estimate the slope of the tangent line you drew at $x= 0$, $x=\frac{\pi}{2}$, $x=\pi$ and $x=\frac{3 \pi}{2}$. Pay careful attention to the scale of the grid.  Notice that the tangent lines at $x=0$ and $x=2\pi$ are parallel, and hence will have the same slope.  Fill in the table below.  

\bigskip

\begin{tabular}{|c|c|c|c|c|c|c|c|c|c|} \hline
$x$-value & $-2 \pi$ & $-\frac{3\pi}{2}$ & $- \pi$ &  $-\frac{\pi}{2}$ &$ 0$ & $\frac{\pi}{2}$ & $\pi$ & $\frac{3\pi}{2}$& $2\pi$ \\ 
& \hspace{.3in} &  \hspace{.3in} & \hspace{.3in} & \hspace{.3in} & \hspace{.3in} & \hspace{.3in} & \hspace{.3in} & \hspace{.3in} & \hspace{.3in} \\ \hline 
slope of tangent & \hspace{.3in} &  \hspace{.3in} & \hspace{.3in} & \hspace{.3in} & \hspace{.3in} & \hspace{.3in} & \hspace{.3in} & \hspace{.3in} & \hspace{.3in}  \\
& \hspace{.3in} &  \hspace{.3in} & \hspace{.3in} & \hspace{.3in} & \hspace{.3in} & \hspace{.3in} & \hspace{.3in} & \hspace{.3in} & \hspace{.3in}\\
& \hspace{.3in} &  \hspace{.3in} & \hspace{.3in} & \hspace{.3in} & \hspace{.3in} & \hspace{.3in} & \hspace{.3in} & \hspace{.3in} & \hspace{.3in}\\ \hline
\end{tabular}
\bigskip

%\item Use the limit definition of the derivative to estimate $f'(0)$ by using small values of $h$, and compare the result to your visual estimate for the slope of the tangent line to $y=f(x)$ at $x=0$ in (b). Using periodicity, what does this result suggest about $f'(2\pi)$? about $f'(-2\pi)$?
\item Based on your work in (a) and (b), sketch an accurate graph of $y=f'(x)$ on the axes adjacent to the graph of $y=f(x)$.  

\item What familiar function do you think is the derivative of $f(x)=\sin(x)$?

\end{enumerate}
\newpage

One can perform a similar process looking at the slopes of tangent lines to the graph of $f(x) = \cos(x)$.  On the left is a graph of $f(x) = \cos(x)$ and on the right is a graph of $f'(x)$.  
\begin{figure}[h]
%\centering
\includegraphics[width=8cm]{cosx.png} \hfill \includegraphics[width=8cm]{negsinx.png}
\[f(x) = \cos(x) \hspace{2.5in} f'(x)\]

\end{figure}


\item What familiar function do you think is the derivative of $f(x)=\cos(x)$?


\end{enumerate}








	









	
\index{Derivatives}
	
\begin{tags}
	    Derivatives, Trigonometry, Graphs, TangentLines
\end{tags}
	
\begin{diary}
	    %S2016-HW1-Q1
\end{diary}
	
\begin{solution}
	   
\end{solution}
	
\end{question}

\end{tagblock}

%-------------------------------------------------------------------------------------------------------------

\begin{tagblock}{Derivatives, Trigonometry, Exponentials, TangentLines}
\begin{question}
	




In this worksheet we'll look at trig functions and their derivatives.  On the Preview Worksheet, you hopefully  found that the sine and cosine functions are  even further linked through calculus, as the derivative of each involves the other. The following rules summarize our findings:
 
\[ \frac{d}{dx}[ \sin(x)] = \cos(x) \, \text{ and }\, \frac{d}{dx}[ \cos(x)] = -\sin(x)\]





%The results of the two preceding activities suggest that the sine and cosine functions not only have the beautiful interrelationships that are learned in a course in trigonometry -- connections such as the identities $\sin^2(x)+\cos^2(x)=1$ and $\cos(x-\frac{\pi}{ 2} )=\sin(x)$ -- but that they are even further linked through calculus, as the derivative of each involves the other. The following rules summarize the results of the above two problems:
%\[ \frac{d}{dx}[ \sin(x)] = \cos(x) \, \text{ and }\, \frac{d}{dx}[ \cos(x)] = -\sin(x)\]

One can formally show these by going back to the definition of the derivative (like we did with the product rule), and using some trig identities and limits.  


\bigskip

We can now add these two new rules to our previous rules, and compute more derivatives efficiently.  

\bigskip

Answer each of the following questions. Where a derivative is requested, be sure to label the derivative function with its name using proper notation.

\begin{enumerate}
\item    Determine the derivative of $h(t)=3\cos(t)-4\sin(t)$.  

\vspace{.5in}

%\item Find the exact slope of the tangent line to $y=f(x)=2x+\frac{\sin(x)}{2}$ at the point where $x=\frac{\pi}{6}$.
\item Determine the derivative of $f(x)=x^3e^x\sin(x)$

\vspace{1.3in}


\item Find the equation of the tangent line to $y=g(x)=x^2+2\cos(x)$ at the point $(\frac{\pi}{2}, (\frac{\pi}{2})^2)$.

\vspace{2in}


%\item Let $p(t)=(\sin(t)+\cos(t))(t^4+3t^2)$. Find $p'(t)$. (No need to simplify your final answer) 
 \end{enumerate}









	









	
\index{Derivatives}
	
\begin{tags}
	    Derivatives, Trigonometry, Exponentials, TangentLines
\end{tags}
	
\begin{diary}
	    %S2016-HW1-Q1
\end{diary}
	
\begin{solution}
	   
\end{solution}
	
\end{question}

\end{tagblock}

%-------------------------------------------------------------------------------------------------------------

\begin{tagblock}{Derivatives, Trigonometry, Exponentials, QuotientRule}
\begin{question}
	






Recall that the trig functions $\tan(x)$, $\sec(x)$, $\csc(x)$ and $\cot(x)$ can all be expressed in terms of $\sin(x)$ and $\cos(x)$:
\[ \tan(x) = \frac{\sin(x)}{\cos(x)}, \hspace{.1in} \sec (x) = \frac{1}{\cos(x)}, \, \csc (x) = \frac{1}{\sin(x)}, \, \cot (x) = \frac{\cos(x)}{\sin(x)}\]



   We'll first develop a formula for the derivative of $f(x) = \tan(x) = \frac{\sin(x)}{\cos(x)}$.
\begin{enumerate}
%\item  What is the domain of $f(x)$?

%\vspace{.5in}

\item Using the quotient rule, we have that
\[f'(x) = \frac{\cos(x) \cos(x) + \sin(x) \sin(x)}{\cos^2(x)}\]

\vspace{1in}

\item What is our favorite trig identity? How can this identity be used to simplify the above expression for $f'(x)$?

\vspace{1.2in} 
\item How can we express $f'(x)$ in terms of the secant function?
\end{enumerate}







	









	
\index{Derivatives}
	
\begin{tags}
	    Derivatives, Trigonometry, Exponentials, QuotientRule
\end{tags}
	
\begin{diary}
	    %S2016-HW1-Q1
\end{diary}
	
\begin{solution}
	   
\end{solution}
	
\end{question}

\end{tagblock}

%-------------------------------------------------------------------------------------------------------------

\begin{tagblock}{Derivatives, Trigonometry, QuotientRule}
\begin{question}
	






Let $g(x) = \cot(x) = \frac{\cos(x)}{\sin(x)}$.
\begin{enumerate}
%\item  What is the domain of $g(x)$?
%\vspace{.5in}


\item Use the quotient rule to develop a formula for $g'(x)$ that is expressed completely in terms of $\sin(x)$ and $\cos(x).$

\vspace{1.5in}


\item How can you use other relationships among trigonometric functions to write $g'(x)$ only in terms of  the cosecant function.

\vspace{1.2in}


%\item What is the domain of $g'$? How does this compare to the domain of $g$?
\end{enumerate}







	









	
\index{Derivatives}
	
\begin{tags}
	    Derivatives, Trigonometry, QuotientRule
\end{tags}
	
\begin{diary}
	    %S2016-HW1-Q1
\end{diary}
	
\begin{solution}
	   
\end{solution}
	
\end{question}

\end{tagblock}

%-------------------------------------------------------------------------------------------------------------

\begin{tagblock}{Derivatives, Trigonometry, QuotientRule}
\begin{question}
	






Let $h= \sec(x) =  \frac{1}{\cos(x)}$
\begin{enumerate}
%\item What is the domain of $h$?
%\vspace{.5in}


\item Use the quotient rule to develop a formula for $h'(x)$ that is expressed completely in terms of $\sin(x)$ and $\cos(x).$
\vspace{1.5in}


\item How can you use other relationships among trigonometric functions to write $h'(x)$ only in terms of $\tan(x)$ and $\sec(x)$?

\vspace{1.2in}


%\item What is the domain of $h'$? How does this compare to the domain of $h$?
\end{enumerate}







	









	
\index{Derivatives}
	
\begin{tags}
	    Derivatives, Trigonometry, QuotientRule
\end{tags}
	
\begin{diary}
	    %S2016-HW1-Q1
\end{diary}
	
\begin{solution}
	   
\end{solution}
	
\end{question}

\end{tagblock}

%-------------------------------------------------------------------------------------------------------------

\begin{tagblock}{Derivatives, Trigonometry}
\begin{question}
	






Similarly, one can determine the derivative of $\csc(x)$ (It's practice exercise \# 17 ).  The quotient rule has thus enabled us to determine the derivatives of the tangent, cotangent, and secant expanding our overall library of basic functions we can differentiate. Fill in the table to summarize our new rules:
\bigskip

\begin{tabular}{ll}
$\frac{d}{dx}[\sin(x)] = \rule[-0.1cm]{5cm}{0.01cm}$  & $\frac{d}{dx}[\cos(x)] = \rule[-0.1cm]{5cm}{0.01cm} $ \\ \\
$\frac{d}{dx}[\tan(x)] = \rule[-0.1cm]{5cm}{0.01cm}$  & $\frac{d}{dx}[\cot(x)] = \rule[-0.1cm]{5cm}{0.01cm} $ \\ \\
$\frac{d}{dx}[\sec(x)] =  \rule[-0.1cm]{5cm}{0.01cm}$  & $\frac{d}{dx}[\csc(x)] = -\csc(x) \cot(x) $ \\
\end{tabular}
%\newpage

\bigskip
\textbf{In practice, you should remember the derivative of $\sin(x)$, $\cos(x)$ and $\tan(x)$.  }








	









	
\index{Derivatives}
	
\begin{tags}
	    Derivatives, Trigonometry
\end{tags}
	
\begin{diary}
	    %S2016-HW1-Q1
\end{diary}
	
\begin{solution}
	   
\end{solution}
	
\end{question}

\end{tagblock}

%-------------------------------------------------------------------------------------------------------------

\begin{tagblock}{Derivatives, Trigonometry, HigherDerivatives}
\begin{question}
	






Let $g(x) = \cos(x)$, find the $33^{rd}$ derivative of $g(x)$, $g^{[33]}(x)$.   \\ \emph{Hint: You really don't want to compute all $33$ derivatives.  Differentiate a few times and see if you detect a pattern}



\index{Derivatives}
	
\begin{tags}
	    Derivatives, Trigonometry, HigherDerivatives
\end{tags}
	
\begin{diary}
	    %S2016-HW1-Q1
\end{diary}
	
\begin{solution}
	   
\end{solution}
	
\end{question}

\end{tagblock}

\begin{tagblock}{Derivatives, ChainRule, WarmUp}
\begin{question}
	






We've developed many rules for computing derivatives.  For example we can compute the derivative of $f(x)=\sin(x)$ and $g(x) = x^2$, as well as combinations of the two.

\begin{enumerate}
\item Warm-up: Compute the derivative of 
\begin{enumerate}
\item $s(x) = 3x^2 - 5\sin(x)$
\vspace{.5in}

\item $p(x) = x^2\sin(x)$
\vspace{.75in}
\item $\displaystyle q(x) = \frac{\sin(x)}{x^2}$
\end{enumerate}
\vspace{.75in}
\end{enumerate}

Recall another way of making functions is by composing them.  For example, consider $C(x) = \sin(x^2)$.  Here we see that $C(x)$ is a composition:
\[x \to x^2 \to \sin(x^2) \]
We can write $C$ as $C(x) = f(g(x))$, where $g(x) = x^2$ and $f(x) = \sin(x)$.  We will call  $g$, the function that is first applied to $x$,  the \emph{inner function}, and $f$, the function that is applied to the result, the \emph{outer function.}

\smallskip
\textbf{Our question in this worksheet is the following: given a composite function $C(x)=f(g(x))$, how can we compute the derivative of $C(x)$?  We expect that it will involve $f, g, f'$ and $g'$. }

\bigskip

 We first need to be comfortable with decomposing a function.
\begin{enumerate}
\item[2.] For each function, decompose it as $f(g(x))$, that is determine the inner function $g(x)$ and the outer function $f(x)$.

\begin{tabular}{|l l | c| c| } \hline
& &  inner function $g(x)$ \hspace{.2in} & outer function $f(x)$ \hspace{.2in} \\ \hline
&&&\\

(a) & $\cos(x^4)$ && \\ 
&&&\\
&&&\\ \hline 
&&&\\

(b) & $\cos^4(x)$ && \\
&remember $\cos^4(x) = (\cos (x))^4$ &&\\
&&&\\ \hline
&&&\\
(c) & $\sqrt{3x-1}$  &
& \\ 
&&&\\
&&&\\ \hline
&&&\\

(d) & $(\tan(x) + x^3)^5$ &&\\ 
&&&\\  
&&&\\ \hline
\end{tabular} 

\end{enumerate}


\index{Derivatives}
	
\begin{tags}
	    Derivatives, ChainRule, WarmUp
\end{tags}
	
\begin{diary}
	    %S2016-HW1-Q1
\end{diary}
	
\begin{solution}
	   
\end{solution}
	
\end{question}

\end{tagblock}

%-------------------------------------------------------------------------------------------------------------

\begin{tagblock}{Derivatives, Trigonometry, HigherDerivatives}
\begin{question}
	






Let $g(x) = \cos(x)$, find the $33^{rd}$ derivative of $g(x)$, $g^{[33]}(x)$.   \\ \emph{Hint: You really don't want to compute all $33$ derivatives.  Differentiate a few times and see if you detect a pattern}



\index{Derivatives}
	
\begin{tags}
	    Derivatives, Trigonometry, HigherDerivatives
\end{tags}
	
\begin{diary}
	    %S2016-HW1-Q1
\end{diary}
	
\begin{solution}
	   
\end{solution}
	
\end{question}

\end{tagblock}

%-------------------------------------------------------------------------------------------------------------


\begin{tagblock}{Derivatives, ChainRule, Example, Trigonometry}
\begin{question}
	






As a motivating example, consider $C(x) = \sin(2x)$.  Using the double-angle trig identity we can rewrite 
\[C(x) = \sin(2x) = 2\sin(x)\cos(x) \]
Note that we have now expressed $C(x)$ as a product. Using the product rule to compute the derivative of $C(x) = 2\sin(x) \cos(x)$ gives
\[C'(x) = (2\cos(x))(\cos(x)) + (2\sin(x))(-\sin(x)) = 2(\cos^2(x) - \sin^2(x))\]
Recalling the other double-angle trig identity $\cos(2x) = \cos^2(x) - \sin^2(x) $, we determine that 
\[C'(x) = 2\cos(2x)\]

Next let's approach $C(x) = \sin(2x)$ slightly differently.

\begin{enumerate}
\item  Decompose $C(x)$ as$f(g(x))$, that is determine the inner function $g(x)$ and the outer function $f(x)$.

\bigskip

$g(x) = $

\bigskip
$f(x) = $

\bigskip

\item Compute $g'(x)$ and $f'(x)$.

\vspace{.75in}

\item For this example, can you express $C'(x)$ in terms of $f, g, f',$ and $g'$?

\end{enumerate}
\newpage
It turns out this holds in general:

\textbf{Chain Rule}  If $g$ is differentiable at $x$ and $f$ is differentiable at $g(x)$, then the composite function $C(x)=f(g(x))$ is differentiable at $x$ and
\[C'(x)=f'(g(x))g'(x).\]

\emph{In words the Chain Rule says the derivative of a composition is the ''derivative of the outer function, evaluated at the inner function, times the derivative of the inner function.'' }


\index{Derivatives}
	
\begin{tags}
	    Derivatives, ChainRule, Example, Trigonometry
\end{tags}
	
\begin{diary}
	    %S2016-HW1-Q1
\end{diary}
	
\begin{solution}
	   
\end{solution}
	
\end{question}

\end{tagblock}

%-------------------------------------------------------------------------------------------------------------


	








\begin{tagblock}{Derivatives, ChainRule, Trigonometry}
\begin{question}
	






Returning to the functions you decomposed in Problem 1, determine $g'(x)$, $f'(x)$, and $f'(g(x))$, and finally apply the chain rule to determine the derivative of the given function. 
\bigskip


\begin{tabular}{| l | c| c| c| c| c| c| } \hline
 $C(x)$ &  inner $g(x)$  & outer $f(x)$  & $g'(x)$ \hspace{.2in} & $f'(x)$ \hspace{.2in} & $f'(g(x))$ \hspace{.5in} & $C'(x)= f'(g(x))g'(x)$ \hspace{.3in}  \\ \hline
 &&&&&&\\

 $\cos(x^4)$ &&&&&& \\ 
&&&&&&\\
&&&&&&\\
&&&&&&\\ \hline 
&&&&&&\\

 $\cos^4(x)$ &&&&&& \\
&&&&&&\\
&&&&&&\\
&&&&&&\\ \hline 
&&&&&&\\

 $\sqrt{3x-1}$  &&&&&& \\ 
&&&&&&\\
&&&&&&\\
&&&&&&\\ \hline 
&&&&&&\\

 $(\tan(x) + x^3)^5$ &&&&&&\\ 
&&&&&&\\
&&&&&&\\
&&&&&&\\ \hline 
\end{tabular} 


In the last two parts of the previous problem, our outer function was of the form $x^n$.  This special case of the chain rule is also called the \textbf{Generalized Power Rule:}  \[\frac{d}{dx}[(g(x))^n]= n(g(x))^{n-1} g'(x) \]


\index{Derivatives}
	
\begin{tags}
	    Derivatives, ChainRule, Trigonometry
\end{tags}
	
\begin{diary}
	    %S2016-HW1-Q1
\end{diary}
	
\begin{solution}
	   
\end{solution}
	
\end{question}

\end{tagblock}

%-------------------------------------------------------------------------------------------------------------


	
\begin{tagblock}{Derivatives, ChainRule, Trigonometry}
\begin{question}
	

In the previous worksheet  we computed the derivative of $\displaystyle \sec(x) = \frac{1}{\cos(x)}$ by using the quotient rule.  Note that we could re-write $\sec(x) = (\cos(x))^{-1}$.  Compute the derivative of  $\sec(x) = (\cos(x))^{-1}$ using the Chain Rule/Generalized Power Rule.  Which way do you find easier? 


\index{Derivatives}
	
\begin{tags}
	    Derivatives, ChainRule, Trigonometry
\end{tags}
	
\begin{diary}
	    %S2016-HW1-Q1
\end{diary}
	
\begin{solution}
	   
\end{solution}
	
\end{question}

\end{tagblock}

%-------------------------------------------------------------------------------------------------------------


	
\begin{tagblock}{Derivatives, ChainRule}
\begin{question}
	

Compute the derivative of $\displaystyle C(x) = \frac{4}{\sqrt{5x^3+7x}}$.  (Note: You can rewrite this first, like in the previous problem). 


\index{Derivatives}
	
\begin{tags}
	    Derivatives, ChainRule
\end{tags}
	
\begin{diary}
	    %S2016-HW1-Q1
\end{diary}
	
\begin{solution}
	   
\end{solution}
	
\end{question}

\end{tagblock}

%-------------------------------------------------------------------------------------------------------------


	
\begin{tagblock}{Derivatives, ChainRule, QuotientRule, Trigonometry}
\begin{question}
	

Compute the derivative of $\displaystyle H(x) =  \cos \left( \frac{1-x^2}{1+3x^2} \right)$.


\index{Derivatives}
	
\begin{tags}
	    Derivatives, ChainRule, QuotientRule, Trigonometry
\end{tags}
	
\begin{diary}
	    %S2016-HW1-Q1
\end{diary}
	
\begin{solution}
	   
\end{solution}
	
\end{question}

\end{tagblock}

%-------------------------------------------------------------------------------------------------------------


	
\begin{tagblock}{Derivatives, ChainRule, ProductRule, Trigonometry}
\begin{question}
	

Compute the derivative of $G(x) = \tan(5x)(6x+13)^{33}$  Note: we need to start with the product rule


\index{Derivatives}
	
\begin{tags}
	    Derivatives, ChainRule, ProductRule, Trigonometry
\end{tags}
	
\begin{diary}
	    %S2016-HW1-Q1
\end{diary}
	
\begin{solution}
	   
\end{solution}
	
\end{question}

\end{tagblock}

%-------------------------------------------------------------------------------------------------------------


	
\begin{tagblock}{Derivatives, ChainRule, Trigonometry}
\begin{question}
	

Compute the derivative of $F(x) = \sec^2(4x) = (\sec(4x))^2$.  Note: Here we need to use the Chain Rule twice! 


\index{Derivatives}
	
\begin{tags}
	    Derivatives, ChainRule, Trigonometry
\end{tags}
	
\begin{diary}
	    %S2016-HW1-Q1
\end{diary}
	
\begin{solution}
	   
\end{solution}
	
\end{question}

\end{tagblock}

%-------------------------------------------------------------------------------------------------------------


	
\begin{tagblock}{Derivatives, ChainRule, Exponentials}
\begin{question}
	

We have seen that $\frac{d}{dx}[e^x] = e^x$.  As an application of the Chain Rule we can now determine the derivative of $f(x) = a^x$, where $a>0$.  Recall that 
\[f(x) = a^x = e^{\ln(a^x)} = e^{x \ln(a)} \]

Now use the Chain Rule!


\index{Derivatives}
	
\begin{tags}
	    Derivatives, ChainRule, Exponentials
\end{tags}
	
\begin{diary}
	    %S2016-HW1-Q1
\end{diary}
	
\begin{solution}
	   
\end{solution}
	
\end{question}

\end{tagblock}

%-------------------------------------------------------------------------------------------------------------


	
\begin{tagblock}{Derivatives, ChainRule, Exponentials, ProductRule, Trigonometry}
\begin{question}
	

Compute the derivative of $G(x) = \tan(5x) e^{7x^2}$. 


\index{Derivatives}
	
\begin{tags}
	    Derivatives, ChainRule, Exponentials, ProductRule, Trigonometry
\end{tags}
	
\begin{diary}
	    %S2016-HW1-Q1
\end{diary}
	
\begin{solution}
	   
\end{solution}
	
\end{question}

\end{tagblock}

%-------------------------------------------------------------------------------------------------------------


	
\begin{tagblock}{Derivatives, ChainRule, Exponentials, ProductRule, Trigonometry, QuotientRule, Challenge}
\begin{question}
	

For each of the following functions, find the function's derivative. State the rule(s) you use, label relevant derivatives appropriately, and be sure to clearly identify your overall answer.
 \begin{enumerate}

\item $ f(x) = \sqrt{5^x + \sqrt{5x}}$

\vspace{1in}
\item $g(x) = \sin(x^2)\cos(x^3)$

\vspace{1in}

\item $ \displaystyle Q(x) = \frac{\sec(5x)}{3x^2 + \sqrt{7}}$
\vspace{1in}


\item $ P(x) = 3x^2 \tan^4(x)$
\vspace{1in}
\item $ y = \sin(\sin (\sin (x)))$
\vspace{1in}


\end{enumerate}


\index{Derivatives}
	
\begin{tags}
	    Derivatives, ChainRule, Exponentials, ProductRule, Trigonometry, QuotientRule, Challenge
\end{tags}
	
\begin{diary}
	    %S2016-HW1-Q1
\end{diary}
	
\begin{solution}
	   
\end{solution}
	
\end{question}

\end{tagblock}

%-------------------------------------------------------------------------------------------------------------


	
\begin{tagblock}{Derivatives, ChainRule, Trigonometry, Challenge, HigherDerivatives}
\begin{question}
	

 Let $g(x) = \cos(6x)$, determine $g^{[33]}(x)$.


\index{Derivatives}
	
\begin{tags}
	    Derivatives, ChainRule, Trigonometry, Challenge, HigherDerivatives
\end{tags}
	
\begin{diary}
	    %S2016-HW1-Q1
\end{diary}
	
\begin{solution}
	   
\end{solution}
	
\end{question}

\end{tagblock}

%-------------------------------------------------------------------------------------------------------------


	
\begin{tagblock}{Derivatives, ChainRule, TangentLines, ImplicitDifferentiation, Graph, Example}
\begin{question}
	

 In our study of derivatives, we've learned
\begin{itemize}
\item How to efficiently take derivatives of functions of the form $y=f(x)$, and
\item Given a function $y=f(x)$, the slope of the the tangent line of $f(x)$ at the point $(a,f(a))$ is given by $f'(a)$.
\end{itemize}


In this worksheet we'll look at other types of curves.  

\bigskip

Warm-up: Let $f(x)$ be a differentiable function.  Using our derivative rules, compute the derivative of the following functions, which are built from $f(x)$.  Your answer may involve $f(x)$ and it's derivative $f'(x)$.  
\begin{enumerate}
\item $\displaystyle \frac{d}{dx}[x^2 + f(x) ] = $
\vspace{.25in}

\item $\displaystyle \frac{d}{dx}[xf(x) ] = $
\vspace{.25in}

\item $\displaystyle \frac{d}{dx}[(f(x))^2 ] = $
\vspace{.25in}

\item $\displaystyle \frac{d}{dx}[\sin(x + f(x) )] = $
%\vspace{.25in}

%\item $\displaystyle \frac{d}{dx}[(x^2+f(x))^2]$
\end{enumerate}
\vspace{.3in}

\bigskip

\begin{enumerate}

\item Consider the following curves: the circle centered at the origin of radius $3$:  $x^2+y^2 = 9$; and the \emph{lemniscate} curve $x^3-y^3=6xy$

\begin{figure}[h]
%\centering
\includegraphics[width=6cm]{implicitcircle.png} \hfill \includegraphics[width=6cm]{implicitl.png}

\end{figure}
\begin{enumerate}
\item For either of these two curves, can you solve for $y$ as a function of $x$?
\vspace{.5in}

\item On the circle, draw a tangent line at the point $(\sqrt{5}, 2)$, and draw a tangent line at the point $(0,-3)$
\item On the lemniscate, draw a tangent line at the point $(-3,3)$.  
\end{enumerate}

\bigskip

Curves like the circle and the lemniscate that are \textbf{not} of the form $y=f(x)$ for a function $f$ are called \emph{implicit curves}.  

As you saw above, we can still draw tangent lines to implicit curves, so it makes sense to ask: 

\textbf{Question:} How can we find the slope of a tangent line to an implicit curve?  

\smallskip
\textbf{Answer:  Use Implicit Differentiation}  

We will  interested in finding an equation for $\frac{dy}{dx}$ that tells us the slope of the tangent line to the curve at a point $(x,y)$. To do so, it will be necessary for us to work with $y$ while thinking of $y$ as a function of $x$, but without being able to write an explicit formula for $y$ in terms of $x.$  This process is called \emph{implicit differentiation}.  

\newpage

\item Returning to the circle $x^2+y^2 = 9$.  We'll start by taking the derivative with respect to $x$ of both sides of the equation:
\[\frac{d}{dx} [x^2+y^2] = \frac{d}{dx}[9] \]
Using the Sum Rule and the fact that the derivative of a constant is $0$ gives
\[ \frac{d}{dx} [x^2]+\frac{d}{dx}[y^2] = 0\]
Since $x$ is the independent variable, it is the variable with respect to which we are differentiating, and thus $\frac{d}{dx} [x^2] = 2x$.  
But $y$ is the dependent variable and $y$ is an implicit function of $x$. Thus, to compute $\frac{d}{dx} [ y^2]$ we need the Chain Rule, just like you did in 1(c):  $\frac{d}{dx} [ y^2] = 2y^1 \frac{dy}{dx} = 2y \frac{dy}{dx}$.  Hence 
\[2x+2y \frac{dy}{dx} = 0 \]
Now solve the equation for $\frac{dy}{dx}$.
\vspace{1in}

Note that $\frac{dy}{dx}$ depends on both $x$ and $y$.  To determine the slope of the tangent line at a given point, we will need to evaluate $\frac{dy}{dx}$ at both the $x$ and $y$ coordinate of the point.
\begin{enumerate}
\item Find the slope of the tangent line to the circle at $(\sqrt{5},2)$, by evaluating $\frac{dy}{dx}$ at $x=\sqrt{5}$ and $y=2$.  (We often write this as $\frac{dy}{dx}|_{(\sqrt{5},2)}$).  Does this agree with the your line that you drew in 3(b)?
\vspace{.5in}


\item Find the slope of the tangent line to the circle at $(0,-3)$?  Does this agree with the your line that you drew in 3(b)?
\vspace{.5in}

\item Using the equation for $\frac{dy}{dx}$, find all points (both $x$ and $y$-coordinates) on the circle where the tangent line is a horizontal line.  
\vspace{1in}


\item Using the equation for $\frac{dy}{dx}$, find all points on the circle where the tangent line is a vertical line.
\end{enumerate}

\vfill
Note for (c) and (d) you can see this fairly clearly on the graph of the circle, but make sure you can also find the points using the equation for $\frac{dy}{dx}$.  

\end{enumerate}


\index{Derivatives}
	
\begin{tags}
	    Derivatives, ChainRule, TangentLines, ImplicitDifferentiation, Graph, Example
\end{tags}
	
\begin{diary}
	    %S2016-HW1-Q1
\end{diary}
	
\begin{solution}
	   
\end{solution}
	
\end{question}

\end{tagblock}

%-------------------------------------------------------------------------------------------------------------


	
\begin{tagblock}{Derivatives, ChainRule, TangentLines, ImplicitDifferentiation, Graph}
\begin{question}
	

Returning to the lemniscate curve $x^3-y^3=6xy$, find an equation for $\frac{dy}{dx}$.   Then find the slope of the tangent line at the point $(-3,3)$.  

\emph{Hints:}
\begin{itemize}
\item \emph{On the right hand side you will need to use the Product Rule like you did in 1(b).}
\item \emph{After differentiating you will have a $\frac{dy}{dx}$ term on both sides: get all the terms with a $\frac{dy}{dx}$ on one side of the equality and those without on the other side; then we can factor out the $\frac{dy}{dx}$ and divide.  This is the same process you did in 2. when you solved for $z$. }  
\end{itemize}



\index{Derivatives}
	
\begin{tags}
	    Derivatives, ChainRule, TangentLines, ImplicitDifferentiation, Graph
\end{tags}
	
\begin{diary}
	    %S2016-HW1-Q1
\end{diary}
	
\begin{solution}
	   
\end{solution}
	
\end{question}

\end{tagblock}

%-------------------------------------------------------------------------------------------------------------


	
\begin{tagblock}{Derivatives, ChainRule, TangentLines, ImplicitDifferentiation, Graph, Trigonometry}
\begin{question}
	

For the curve $\sin(x+y)+\cos(x-y) = 1$, use implicit differentiation to find an equation for $\frac{dy}{dx}$, and then find the slope of the tangent line to the curve  at $(\frac{\pi}{2},\frac{\pi}{2})$.  Draw the tangent line at $(\frac{\pi}{2},\frac{\pi}{2})$ and make sure it agrees with your answer.  

\begin{figure}[h]
\centering
\includegraphics[width=6cm]{implicitsin.png}
\end{figure}



\index{Derivatives}
	
\begin{tags}
	    Derivatives, ChainRule, TangentLines, ImplicitDifferentiation, Graph, Trigonometry
\end{tags}
	
\begin{diary}
	    %S2016-HW1-Q1
\end{diary}
	
\begin{solution}
	   
\end{solution}
	
\end{question}

\end{tagblock}

%-------------------------------------------------------------------------------------------------------------


	
\begin{tagblock}{Derivatives, ChainRule, InverseTrig}
\begin{question}
	

Compute the derivative of $g(x) = \sin^{-1}( x^7)$.



\index{Derivatives}
	
\begin{tags}
	    Derivatives, ChainRule, InverseTrig
\end{tags}
	
\begin{diary}
	    %S2016-HW1-Q1
\end{diary}
	
\begin{solution}
	   
\end{solution}
	
\end{question}

\end{tagblock}
%-------------------------------------------------------------------------------------------------------------


	
\begin{tagblock}{Derivatives, ChainRule, ProductRule, InverseTrig}
\begin{question}
	

Compute the derivative of $y=\sqrt{x} \cos^{-1}(\pi x)$. 



\index{Derivatives}
	
\begin{tags}
	    Derivatives, ChainRule, ProductRule, InverseTrig
\end{tags}
	
\begin{diary}
	    %S2016-HW1-Q1
\end{diary}
	
\begin{solution}
	   
\end{solution}
	
\end{question}

\end{tagblock}

%-------------------------------------------------------------------------------------------------------------


	
\begin{tagblock}{Derivatives, ChainRule, ProductRule, Exponentials}
\begin{question}
	

Compute the derivative of $g(x) = a^x e^{-bx+12}$, where $a$ and  $b$ are constants.



\index{Derivatives}
	
\begin{tags}
	    Derivatives, ChainRule, ProductRule, Exponentials
\end{tags}
	
\begin{diary}
	    %S2016-HW1-Q1
\end{diary}
	
\begin{solution}
	   
\end{solution}
	
\end{question}

\end{tagblock}

%-------------------------------------------------------------------------------------------------------------


	
\begin{tagblock}{Derivatives, ChainRule, ProductRule, ImplicitDifferentiation, TangentLines}
\begin{question}
	

Find $\frac{dy}{dx}$ for the implicit curve $e^y\sin(x)= y+xy$, and then find the slope of the tangent  line at the point $(0,0)$.



\index{Derivatives}
	
\begin{tags}
	    Derivatives, ChainRule, ProductRule, ImplicitDifferentiation, TangentLines
\end{tags}
	
\begin{diary}
	    %S2016-HW1-Q1
\end{diary}
	
\begin{solution}
	   
\end{solution}
	
\end{question}

\end{tagblock}

%-------------------------------------------------------------------------------------------------------------


	
\begin{tagblock}{Derivatives, ChainRule, ProductRule, Exponentials, QuotientRule, Theory}
\begin{question}
	

For each of the following, find a formula for the derivative of $f$, $f'(x)$, in terms of $g(x)$ and $g'(x)$.  If we additionally know that $g(2)=-1$ and $g'(2)=10$, determine $f'(2)$.  

\begin{enumerate}
\item $f(x) = x^2g(x)$ 

\vspace{1.25in}
\item $f(x) = (g(x))^2$

\vspace{1.25in}

%\item $f(x) = g(e^x)$

%\vspace{1.25in}

\item $f(x) = e^{g(x)}$ 

\vspace{1.25in}

\item $f(x) = g(g(x))$

\vspace{1.25in}

\item $\displaystyle f(x) = \frac{g(x)}{x^2+1}$
\end{enumerate}



\index{Derivatives}
	
\begin{tags}
	    Derivatives, ChainRule, ProductRule, Exponentials, QuotientRule, Theory
\end{tags}
	
\begin{diary}
	    %S2016-HW1-Q1
\end{diary}
	
\begin{solution}
	   
\end{solution}
	
\end{question}

\end{tagblock}

%-------------------------------------------------------------------------------------------------------------


	
\begin{tagblock}{Derivatives, RelatedRates, WarmUp}
\begin{question}
	

\textbf{Motivating Question:} If two quantities that are related, such as the radius and volume of a spherical balloon, are both changing as implicit functions of time, how are their rates of change related? That is, how does the relationship between the values of the quantities affect the relationship between their respective derivatives with respect to time?

\bigskip

In the last worksheet we introduced implicit differentiation and saw that for implicit curves $\frac{dy}{dx}$ evaluated at a point gave us the slope of the tangent line to the curve at that point.  

\bigskip

We are next going to consider situations where multiple quantities are related to one another and changing, but where each quantity can be considered an implicit function of the variable $t$, which represents time. Through knowing how the quantities are related, we will be interested in determining how their respective rates of change with respect to time are related. 

\bigskip

\textbf{Problem:}  Suppose that air is being pumped into a spherical balloon in such a way that its volume increases at a constant rate of $20$ cubic inches per second. It makes sense that since the balloon's volume and radius are related, by knowing how fast the volume is changing, we ought to be able to relate this rate to how fast the radius is changing. More specifically, can we find how fast the radius of the balloon is increasing at the moment the balloon's radius is 6 inches?

\bigskip
We'll do this in a number of steps and use this same strategy for other related rates problems.  Perhaps the challenge to these problems is translating the ``words'' into ``equations.''

\bigskip

\begin{enumerate}
\item {\bf Diagram: } Draw several spheres with different radii, and observe that as volume changes, the radius, diameter, and surface area of the balloon also change.  Label the radius, $r$, in each of your pictures.  

\vspace{2in}



\item {\bf Rates:} Identify rates given in the problem and the rate you need to compute. You may use units to help decide which numbers are rates.\\

\smallskip

Note that in the setting of this problem, both the volume, $V$ and the radius $r$ are changing as time $t$ changes, and thus both $V$ and $r$ may be viewed as implicit functions of $t$, with respective derivatives $\frac{dV}{dt}$ and $\frac{dr}{dt}$.  Recall that we are given in the problem that the balloon is being inflated at a constant rate of $20$ cubic inches per second. Is this rate the value of $\frac{dr}{dt}$ or $\frac{dV}{dt}$? Why?  What is the rate that we want to find?  

\vspace{1in}

\item {\bf Equation:} Write an equation relating the variables you identified. \\

\smallskip

Recall that the volume of a sphere of radius r is   $V=\frac{4}{3} \pi r^3$

\vspace{.5in}

\item {\bf Differentiate:} It wouldn't be calculus without this step! Go ahead and differentiate the equation that relates your variables.


Differentiate both sides of the equation $V=\frac{4}{3} \pi r^3$ with respect to $t$ (using implicit differentiation) to find a formula for $\frac{dV}{dt}$ that depends on both $r$ and $\frac{dr}{dt}$.

\vspace{2in}

At this point in the problem, by differentiating we have ``related the rates'' of change of $V$ and $r$, hence the name \emph{related rates}.   

\item {\bf Substitute and solve:} Plug in all known quantities into the equation from the last step. Solve for the desired rate and answer the question!  Don't forget your units on the final answer.  


\end{enumerate}
\vspace{3in}
We'll look at more complicated examples in the next worksheet, but the strategy we outlined will be the same.  



\index{Derivatives}
	
\begin{tags}
	    Derivatives, RelatedRates, WarmUp
\end{tags}
	
\begin{diary}
	    %S2016-HW1-Q1
\end{diary}
	
\begin{solution}
	   
\end{solution}
	
\end{question}

\end{tagblock}

%-------------------------------------------------------------------------------------------------------------


	
\begin{tagblock}{Derivatives, RelatedRates}
\begin{question}
	

Car A is traveling west at $50$ $mi/hr$ and car B is traveling north at $60$ $mi/hr$. Both are headed toward the intersection of the two roads. At what rate are the cars approaching each other when car A is $0.3$ $mi$ and car B is $0.4$ $mi$ from the intersection?
 \begin{enumerate}
 \item Diagram with labels:
 \vspace{1.5in}
 \item Rates: (Note that each car is getting closer to the intersection, so their rates will be negative)
 \vspace{1in}
 \item Equation:
 \vspace{1in}
 \item Differentiate:
 \vspace{1.5in}
 \item Substitute and solve:
 \vspace{2in}
\end{enumerate}




\index{Derivatives}
	
\begin{tags}
	    Derivatives, RelatedRates
\end{tags}
	
\begin{diary}
	    %S2016-HW1-Q1
\end{diary}
	
\begin{solution}
	   
\end{solution}
	
\end{question}

\end{tagblock}

%-------------------------------------------------------------------------------------------------------------


	
\begin{tagblock}{Derivatives, RelatedRates}
\begin{question}
	

Given an equation relating variables and knowing how one variable changes, we can use calculus to detect the rate of change of another variable. The problems in this worksheet ask you to compute rates of change by using the steps outlined in our Introduction to Related Rates Worksheet: 
\begin{enumerate}
\item {\bf Diagram: }Draw a picture of problem. Label important variables.
\item {\bf Rates:} Identify rates given in the problem and the rate you need to compute. You may use units to help decide which numbers are rates.
\item {\bf Equation:} Write an equation relating the variables you identified.
\item {\bf Differentiate:} It wouldn't be calculus without this step! Go ahead and differentiate the equation that relates your variables.
\item {\bf Substitute and solve:} Plug in all known quantities into the equation from the last step. Solve for the desired rate and answer the question.
\end{enumerate}

\bigskip



A $5$ foot ladder is leaning against a vertical wall and starts to fall.  If the bottom of the ladder moves away from the wall at $2$ ft/sec, how fast is the top of the ladder sliding when the bottom of the ladder is $4$ feet from the wall?
\begin{enumerate}
\item Diagram: Draw a picture of the ladder, wall and floor. Label all variables of interest: the distance from the floor to the top of the ladder and the distance from the wall to the bottom of the ladder.  
\vspace{1.5in}
\item Rates: Which rate of change is given? Which rate of change do you need?
\vspace{1in}
\item Equation: Note that the ladder, wall and floor make a right triangle.  Do you remember the Pythagorean Theorem?  
\vspace{1in}
\item Differentiate: Since the ladder is moving the distance from the floor to the top of the ladder and the distance from the wall to the bottom of the ladder are really functions of time. Differentiate your formula from the last part with respect to time, $t$.
\vspace{1.5in}
\item Substitute and solve: Substitute the numbers given in the problem into your last equation and solve for the desired rate.  Don't forget your units!
\vspace{3in}
\item Your final rate should be negative, why does that make sense?  
\end{enumerate}




\index{Derivatives}
	
\begin{tags}
	    Derivatives, RelatedRates
\end{tags}
	
\begin{diary}
	    %S2016-HW1-Q1
\end{diary}
	
\begin{solution}
	   
\end{solution}
	
\end{question}

\end{tagblock}

%-------------------------------------------------------------------------------------------------------------


	
\begin{tagblock}{Derivatives, RelatedRates}
\begin{question}
	

A water tank has the shape of an inverted right circular cone with height $4$ $m$. If water is being pumped into the tank at a rate of $2$ $m^3/min$, find the rate at which the water level is rising when the water is $3$ $m$ deep. (Recall the volume of a right cone is $V=\frac{1}{3}\pi r^2 h$, and the height will always be twice the radius.)
\begin{enumerate}
 \item Diagram with labels:
 \vspace{1.in}
 \item Rates:
 \vspace{1in}
 \item Equation:  
 \vspace{1in}
 \item Differentiate:
 \vspace{1.5in}
 \item Substitute and solve:
 \vspace{1in}
\end{enumerate}




\index{Derivatives}
	
\begin{tags}
	    Derivatives, RelatedRates
\end{tags}
	
\begin{diary}
	    %S2016-HW1-Q1
\end{diary}
	
\begin{solution}
	   
\end{solution}
	
\end{question}

\end{tagblock}

%-------------------------------------------------------------------------------------------------------------


	
\begin{tagblock}{Derivatives, RelatedRates, Challenge}
\begin{question}
	

A boy on a skateboard rolls away from a $15$ $ft$ lamppost at a speed of $3$ $ft/s$. The boy's height on the skateboard is $6$ feet. Find the rate at which his shadow is increasing in length. \\ See \url{http://webspace.ship.edu/msrenault/GeoGebraCalculus/derivative_app_rr_streetlamp.html}
 for some helpful animation.




\index{Derivatives}
	
\begin{tags}
	    Derivatives, RelatedRates, Challenge
\end{tags}
	
\begin{diary}
	    %S2016-HW1-Q1
\end{diary}
	
\begin{solution}
	   
\end{solution}
	
\end{question}

\end{tagblock}

%-------------------------------------------------------------------------------------------------------------


	
\begin{tagblock}{Derivatives, RelatedRates}
\begin{question}
	

The base of a triangle is shrinking at a rate of 1 cm/min and the height of the triangle is increasing at a rate of 5 cm/min. Find the rate at which the area of the triangle changes when the height is 22 cm and the base is 10 cm.



\index{Derivatives}
	
\begin{tags}
	    Derivatives, RelatedRates
\end{tags}
	
\begin{diary}
	    %S2016-HW1-Q1
\end{diary}
	
\begin{solution}
	   
\end{solution}
	
\end{question}

\end{tagblock}

%-------------------------------------------------------------------------------------------------------------


	
\begin{tagblock}{Derivatives, MaxMin, Graph, TangentLines, Differentiable, Absolute}
\begin{question}
	

Let $f(x)$ be given by the graph below, defined on the interval $[-3,3]$.  Use the graph to answer each of the following questions.
\begin{figure}[h]
\centering
\includegraphics[width=8cm]{minmax1.png} 
\end{figure}

\begin{enumerate}
\item Identify all $x$-value(s) $c$ at which $f(x)$ has an \textbf{absolute maximum}.

\vspace{.25in}
\item Identify all  $x$-value(s) $c$ at which $f(x)$ has an \textbf{absolute minimum}.
\vspace{.25in}

\item Identify all $x$-value(s) $c$ at which $f(x)$ has an \textbf{local maximum}.
\vspace{.25in}

\item Identify all  $x$-value(s) $c$ at which $f(x)$ has an \textbf{local minimum}.
\vspace{.25in}

\item Identify all values of $c$ for which $f'(c)$ does not exist.  (Remember the derivative won't exist at $c$ if we have a discontinuity at $c$, a corner at $c$, or a vertical tangent line at $c$).

\vspace{.25in}

\item Identify all values of $c$ for which $f'(c) = 0$.

\vspace{.25in}




\end{enumerate}


\index{Derivatives}
	
\begin{tags}
	   Derivatives, MaxMin, Graph, TangentLines, Differentiable, Absolute
\end{tags}
	
\begin{diary}
	    %S2016-HW1-Q1
\end{diary}
	
\begin{solution}
	   
\end{solution}
	
\end{question}

\end{tagblock}

%-------------------------------------------------------------------------------------------------------------


	
\begin{tagblock}{Derivatives, MaxMin, Graph, Continuity, Differentiable, Absolute}
\begin{question}
	

As you saw in the previous problem, determining absolute minimums and maximums from a \emph{graph} of a function isn't too difficult.  Our next goal is to use the \emph{equation} of a function to determine absolute and local minimums and maximums.  

\bigskip

In this problem we'll look at absolute minimum and maximums.  
\begin{enumerate}
\item Consider the function $f(x) = x^3$.  Does $f(x)$ have an absolute maximum?  Does $f(x)$ have an absolute minimum?  Why or why not?  (Hint: Think of the graph of $f(x)$)

\vspace{1.5in}
\item Now consider $f(x)=x^3$ on the closed interval $[-2,1]$.  Does $f(x)$ now have an absolute maximum on $[-2,1]$? Does $f(x)$ have an absolute minimum on $[-2,1]$?  If so, what is the absolute maximum and absolute minimum?  

\vspace{1in}
\item Consider the function $g(x)$ given by the graph below, which has a vertical asymptote at $x=3$.  

\begin{minipage}{.4\textwidth}
\includegraphics[width=6cm]{minmax2.png}\end{minipage}% This must go next to `\end{minipage}`
\begin{minipage}{.6\textwidth}
\begin{enumerate}
\item Is $g(x)$ continuous on $[1,3]$? \\ If not, where are the discontinuties?

\vspace{.5in}

\item Does $g(x)$ have an absolute minimum on $[1,3]$?  \\
Does $g(x)$ have an absolute maximum on $[1,3]$?  

\end{enumerate}

\end{minipage}

\end{enumerate}

\vfill

This problem gives us an example of the \\

\bigskip
\noindent\fbox{%
    \parbox{\textwidth}{
\textbf{Extreme Value Theorem}:  If $f$ is a continuous function on a closed interval $[a,b]$, then $f$ has both an absolute minimum and absolute maximum on $[a,b]$.  }}






\index{Derivatives}
	
\begin{tags}
	   Derivatives, MaxMin, Graph, Continuity, Differentiable, Absolute
\end{tags}
	
\begin{diary}
	    %S2016-HW1-Q1
\end{diary}
	
\begin{solution}
	   
\end{solution}
	
\end{question}

\end{tagblock}

%-------------------------------------------------------------------------------------------------------------


	
\begin{tagblock}{Derivatives, MaxMin, Graph, Differentiable, Definition, Critical}
\begin{question}
	

Returning to the function $f(x)$ given by

\bigskip

\begin{figure}[h]
\centering
\includegraphics[width=8cm]{minmax1.png} 
\end{figure}



\begin{enumerate}
\item True or false: every local maximum and minimum of $f$ occurs at a point where $f'(c)$ is either zero or does not exist.

\vspace{.25in}

\item True or false: at every point where $f'(c)$ is zero or does not exist, $f$ has a local maximum or minimum.

\vspace{.25in}


\end{enumerate}

In fact, the general statement is true and is known as \textbf{Fermat's Theorem}

\bigskip

\noindent\fbox{%
    \parbox{\textwidth}{
\textbf{Fermat's Theorem}  If $f(x)$ has a local minimum or maximum at $x=c$, and if $f'(c)$ exists, then $f'(c) =0$.}}

\bigskip

The upshot of Fermat's Theorem, is that we can make a list of candidate points where a local minimum or maximum may occur.  These points are so important, that we call them \emph{critical numbers}.

\bigskip

\textbf{Definition}  A \emph{critical number} of a function $f(x)$ is a number $x=c$ where $f'(c) =0$ or $f'(c)$ does not exist.




\index{Derivatives}
	
\begin{tags}
	   Derivatives, MaxMin, Graph, Differentiable, Definition, Critical
\end{tags}
	
\begin{diary}
	    %S2016-HW1-Q1
\end{diary}
	
\begin{solution}
	   
\end{solution}
	
\end{question}

\end{tagblock}

%-------------------------------------------------------------------------------------------------------------


	
\begin{tagblock}{Derivatives, MaxMin, Absolute}
\begin{question}
	


Consider the function $\displaystyle f(x) = \frac{1}{4} x^4 - \frac{2}{3} x^3 - 4x^2$.
\begin{enumerate}
\item Compute the derivative $f'(x)$.
\vspace{1in}
\item Are there any $x$ values for which $f'(x)$ is undefined?
\vspace{1in}
\item Find all the $x$ values where $f'(x) = 0$.
\vspace{1in}
\item What are the critical numbers of $f(x)$?
\end{enumerate}



\newpage
\textbf{How to Find Absolute Minimum and Maximum of a function $f$ on a closed interval $[a,b]$.}
\begin{itemize}
\item Find all the critical numbers $c$ of $f(x)$.
\item For each critical number $c$ in the interval $[a,b]$, compute the $y$-value $f(c)$.
\item Compute the $y$-value of the endpoints $f(a)$ and $f(b)$.
\item Compare all the $y$-values: the largest $y$-value will give you the absolute maximum, the smallest $y$-value will give you the absolute minimum.  
\end{itemize}

\bigskip

Returning to the function $\displaystyle f(x) = \frac{1}{4} x^4 - \frac{2}{3} x^3 - 4x^2$ from the previous problem.   Find the absolute maximum and absolute minimum on $[-6,6]$. 




\index{Derivatives}
	
\begin{tags}
	   Derivatives, MaxMin, Absolute
\end{tags}
	
\begin{diary}
	    %S2016-HW1-Q1
\end{diary}
	
\begin{solution}
	   
\end{solution}
	
\end{question}

\end{tagblock}

%-------------------------------------------------------------------------------------------------------------


	
\begin{tagblock}{Derivatives, MaxMin, Absolute}
\begin{question}
	


 Let $g(x) = x + \frac{1}{x}$.
\begin{enumerate}
\item Find all the critical numbers of $g(x)$ 

\vspace{1.5in}
\item Find the absolute maximum and absolute minimum on $[.2,4]$.  
\end{enumerate}




\index{Derivatives}
	
\begin{tags}
	   Derivatives, MaxMin, Absolute
\end{tags}
	
\begin{diary}
	    %S2016-HW1-Q1
\end{diary}
	
\begin{solution}
	   
\end{solution}
	
\end{question}

\end{tagblock}

%-------------------------------------------------------------------------------------------------------------


	
\begin{tagblock}{Derivatives, MaxMin, Absolute, Graph, Continuity}
\begin{question}
	


Sketch the graph of a function $f$ that is continuous on $[1, 6]$ and has the given properties:  Absolute maximum at $x=2$, absolute minimum at $x=6$, and $f'(4)=0$, but $x=4$ is neither a local minimum or maximum.




\index{Derivatives}
	
\begin{tags}
	   Derivatives, MaxMin, Absolute, Graph, Continuity
\end{tags}
	
\begin{diary}
	    %S2016-HW1-Q1
\end{diary}
	
\begin{solution}
	   
\end{solution}
	
\end{question}

\end{tagblock}

\begin{tagblock}{Derivatives, InverseTrig, ImplicitDifferentiation}
\begin{question}
	


As an application of implicit differentiation, we can find the derivatives of inverse trig functions.


\bigskip

  We'll start with the inverse sine function or arcsine function:  $y=\sin^{-1}(x)$ for $-1<x<-1$.  Recall that since $\sin(x)$ and $\sin^{-1}(x)$ are inverses of each other: 
\[\sin(\sin^{-1}(x)) = x\]
Our goal is to determine the derivative of $y=\sin^{-1}(x)$.  

\bigskip
We'll start by applying $\sin$ to both sides of the equation $y=\sin^{-1}(x)$:
\begin{eqnarray*} \sin(y) &= &\sin(\sin ^{-1}(x)) \\
 \sin(y) &= & x \\
 \end{eqnarray*}
 
 Then we'll use implicit differentiation to find $\frac{dy}{dx}$!
 

 \begin{eqnarray*}
  \frac{d}{dx} [\sin(y)] &= & \frac{d}{dx}[x] \\
  \cos(y) \frac{dy}{dx} & = & 1 \\
    \frac{dy}{dx} &=&  \frac{1}{\cos(y)}
 \end{eqnarray*}
 
 Recall that $y=\sin^{-1}(x)$, so we have found that the derivative of $\sin^{-1}(x)$ is $ \frac{dy}{dx} =  \frac{1}{\cos(\sin^{-1}(x))}$.  
 
\begin{minipage}{.7\textwidth}
This is not a very pretty expression, but we can simplify it using trigonometry:  Let $\theta = \sin^{-1}(x)$, so that $\theta$ is the angle whose sine is $x$.  We can picture $\theta$ as an angle in a right triangle with hypotenuse $1$ and opposite side of $x$.  By the Pythagorean Theorem, the third side has length $\sqrt{1-x^2}$.
\end{minipage}% This must go next to `\end{minipage}`
\begin{minipage}{.3\textwidth} 
\includegraphics[width=3cm]{triangle.png}
\end{minipage}

Using this triangle, find an expression for $\cos(\theta)$ (which will involve an $x$):

\[ \cos (\theta) = \rule[-0.1cm]{2.5cm}{0.01cm}\]
 
 
 \bigskip
 
 Putting this all together then tells us 
 \[ \frac{d}{dx} [\sin^{-1}(x)] = \frac{1}{\cos(\sin^{-1}(x))} = \frac{1}{\cos(\theta)} = \frac{1}{ \hspace{1in}} \]
 
 





\bigskip

Find the derivative of $y= \tan^{-1}(x)$, using a similar process as we did with the derivative of $\sin^{-1}(x)$.  




\index{Derivatives}
	
\begin{tags}
	   Derivatives, InverseTrig, ImplicitDifferentiation
\end{tags}
	
\begin{diary}
	    %S2016-HW1-Q1
\end{diary}
	
\begin{solution}
	   
\end{solution}
	
\end{question}

\end{tagblock}

%-------------------------------------------------------------------------------------------------------------


	
\begin{tagblock}{Derivatives, MaxMin, Absolute, Graph, Continuity}
\begin{question}
	


Sketch the graph of a function $f$ that is continuous on $[1, 6]$ and has the given properties:  Absolute maximum at $x=2$, absolute minimum at $x=6$, and $f'(4)=0$, but $x=4$ is neither a local minimum or maximum.



\index{Derivatives}
	
\begin{tags}
	   Derivatives, MaxMin, Absolute, Graph, Continuity
\end{tags}
	
\begin{diary}
	    %S2016-HW1-Q1
\end{diary}
	
\begin{solution}
	   
\end{solution}
	
\end{question}

\end{tagblock}

%-------------------------------------------------------------------------------------------------------------


\begin{tagblock}{Derivatives, Logarithm, WarmUp}
\begin{question}
	


Today we'll look at logarithmic functions.  A few reminders about logarithms (that we looked at in the first worksheet on Functions)



\begin{tcolorbox}

\textbf{Logarithmic Rules} 
    \begin{itemize}
    \item     $\log_b(b^x)=x$  (in particular, $\ln(e^x) = x$ )
    \item $b^{\log_b(x)} =x$  (in particular, $e^{\ln(x)} = x$ )
    \item  $\log_b(xy) = \log_b(x) +  \log_b(y)$
     \item  $\log_b(\frac{x}{y}) = \log_b(x) -  \log_b(y)$
   \item $\log_b(x^r) = r \log_b(x)$
   \end{itemize}    

\end{tcolorbox}

\textbf{Warning:} We don't have a rule for $\log_b(x+y)$.  


\bigskip


Using our log rules, we can rewrite a single logarithm $g(x) = \log_2((14x^9-10x)^4(x^5+x^3-1))$ into multiple ones:
\begin{eqnarray*} g(x) &= &\log_2((14x^9-10x)^4(x^5+x^3-1)) =   \log_2((14x^9-10x)^4) + \log_2(x^5+x^3-1) \\
&=& 4 \log_2(14x^9-10x)+ + \log_2(x^5+x^3-1) \end{eqnarray*}

\begin{enumerate}
\item \textbf{Warm Up}.  
\begin{enumerate}
\item Rewrite the following functions 
\begin{enumerate}

\item $f(x) = \log_5 (x(x^2+2)^2)$

\vspace{1in}
\item $\displaystyle h(x) = \ln \left(\frac{1+e^x}{1-e^x} \right)$

\vspace{1.5in}

\end{enumerate}  

\item Recall our derivatives of exponential functions:
\begin{enumerate}

\item $\frac{d}{dx} [e^x] = $ \rule[-0.1cm]{2.5cm}{0.01cm} 

\vspace{.5in}

\item $\frac{d}{dx} [b^x] =$ \rule[-0.1cm]{2.5cm}{0.01cm}


\end{enumerate}  

\end{enumerate}

\end{enumerate}



\index{Derivatives}
	
\begin{tags}
	   Derivatives, Logarithm, WarmUp
\end{tags}
	
\begin{diary}
	    %S2016-HW1-Q1
\end{diary}
	
\begin{solution}
	   
\end{solution}
	
\end{question}

\end{tagblock}

%-------------------------------------------------------------------------------------------------------------


	
\begin{tagblock}{Derivatives, Logarithm, Example, ImplicitDifferentiation}
\begin{question}
	


\textbf{Our First Goal:} Determine the derivative of $y = \ln(x)$.  
 
 Since $e^x$ is the inverse of $\ln(x)$, we have
 \begin{eqnarray*} y &=& \ln (x) \\
 e^y & = & e^(\ln (x)) \\
 e^y & = & x \end{eqnarray*}
 
 Now use implicit differentiation to find $\frac{dy}{dx}$
 
  \begin{eqnarray*} 
\frac{d}{dx}[ e^y ]& = & \frac{d}{dx}[x] \end{eqnarray*}
 
 \vspace{2in}
 
 This then tells us $\frac{d}{dx}[\ln(x)] = $ \rule[-0.1cm]{2.5cm}{0.01cm}

\index{Derivatives}
	
\begin{tags}
	   Derivatives, Logarithm, Example, ImplicitDifferentiation
\end{tags}
	
\begin{diary}
	    %S2016-HW1-Q1
\end{diary}
	
\begin{solution}
	   
\end{solution}
	
\end{question}

\end{tagblock}

%-------------------------------------------------------------------------------------------------------------

\begin{tagblock}{Derivatives, Logarithm, Example, ImplicitDifferentiation}
\begin{question}

 We can employ the same process to determine the derivative of  $y = \log_b(x)$.  
 
 Since $b^x$ and $\log_b(x)$ are inverses of each other we have:
 
 \begin{eqnarray*} y &=& \log_b(x) \\
 b^y & = & b^(\log_b(x)) \\
 b^y & = & x \end{eqnarray*}
 
 Now use implicit differentiation as in the previous problem.

 \vspace{2in}
 
 This then tells us $\frac{d}{dx}[\log_b(x)] = $ \rule[-0.1cm]{2.5cm}{0.01cm}


\index{Derivatives}
	
\begin{tags}
	   Derivatives, Logarithm, Example, ImplicitDifferentiation
\end{tags}
	
\begin{diary}
	    %S2016-HW1-Q1
\end{diary}
	
\begin{solution}
	   
\end{solution}
	
\end{question}

\end{tagblock}

%-------------------------------------------------------------------------------------------------------------

\begin{tagblock}{Derivatives, Logarithm, Example, ImplicitDifferentiation}
\begin{question}

 We can employ the same process to determine the derivative of  $y = \log_b(x)$.  
 
 Since $b^x$ and $\log_b(x)$ are inverses of each other we have:
 
 \begin{eqnarray*} y &=& \log_b(x) \\
 b^y & = & b^(\log_b(x)) \\
 b^y & = & x \end{eqnarray*}
 
 Now use implicit differentiation as in the previous problem.

 \vspace{2in}
 
 This then tells us $\frac{d}{dx}[\log_b(x)] = $ \rule[-0.1cm]{2.5cm}{0.01cm}


\index{Derivatives}
	
\begin{tags}
	   Derivatives, Logarithm, Example, ImplicitDifferentiation
\end{tags}
	
\begin{diary}
	    %S2016-HW1-Q1
\end{diary}
	
\begin{solution}
	   
\end{solution}
	
\end{question}

\end{tagblock}

%-------------------------------------------------------------------------------------------------------------

\begin{tagblock}{Derivatives, Logarithm, ChainRule}
\begin{question}

 Compute the derivative of $f(x) = \sqrt{\ln(x)}$


\index{Derivatives}
	
\begin{tags}
	   Derivatives, Logarithm, ChainRule
\end{tags}
	
\begin{diary}
	    %S2016-HW1-Q1
\end{diary}
	
\begin{solution}
	   
\end{solution}
	
\end{question}

\end{tagblock}

%-------------------------------------------------------------------------------------------------------------

\begin{tagblock}{Derivatives, Logarithm, ChainRule}
\begin{question}

 Compute the derivative of $g(x) = \ln(ax)$, where $a$ is any constant.


\index{Derivatives}
	
\begin{tags}
	   Derivatives, Logarithm, ChainRule
\end{tags}
	
\begin{diary}
	    %S2016-HW1-Q1
\end{diary}
	
\begin{solution}
	   
\end{solution}
	
\end{question}

\end{tagblock}

%-------------------------------------------------------------------------------------------------------------

\begin{tagblock}{Derivatives, Logarithm, ChainRule}
\begin{question}

 Compute the derivative of $G(x) = \ln(x^2+25x)$.  (\emph{Hint:  You will need the Chain Rule})


\index{Derivatives}
	
\begin{tags}
	   Derivatives, Logarithm, ChainRule
\end{tags}
	
\begin{diary}
	    %S2016-HW1-Q1
\end{diary}
	
\begin{solution}
	   
\end{solution}
	
\end{question}

\end{tagblock}

%-------------------------------------------------------------------------------------------------------------

\begin{tagblock}{Derivatives, IncreasingDecreasing, Graph, TangentLine}
\begin{question}

 Recall that we say a function is \emph{increasing} on an interval $(a,b)$ provided that for all $x_1$ and $x_2$ in the interval, if $x_1 < x_2$, then $f(x_1) < f(x_2)$.  A function is \emph{decreasing} on an interval $(a,b)$ provided that for all $x_1$ and $x_2$ in the interval, if $x_1 < x_2$, then $f(x_1) > f(x_2)$.

\bigskip

Simply put, an increasing function is one that is rising as we move from left to right along the graph, and a decreasing function is one that falls as the value of the input increases.


 

  
Let $f(x)$ be given by the graph below.  Use the graph to answer each of the following questions. (You may need to estimate answers based on the graph)
\begin{figure}[h]
\centering
\includegraphics[width=8cm]{incdec1.png} 
\end{figure}

\begin{enumerate}
\item Find the intervals where $f(x)$ is increasing.  

\vspace{.25in}
\item Find the intervals where $f(x)$ is decreasing.  
\vspace{.25in}

\item Draw the tangent line to $f(x)$ at $x=-3$.  Is $f'(-3)$ positive, negative or zero?  
\vspace{.25in}

\item Draw the tangent line to $f(x)$ at $x=0$.  Is $f'(0)$ positive, negative or zero?  
\vspace{.25in}

\item Draw the tangent line to $f(x)$ at $x=4$.  Is $f'(4)$ positive, negative or zero?  

\vspace{.25in}

\item Fill in the blank with ``increasing'' or ``decreasing":  If the derivative is positive on an interval, that is if $f'(x) >0$ for all $x$ in $(a,b)$, then $f(x)$ is \rule{4cm}{0.1mm}
\vspace{.25in}

\item Fill in the blank with ``increasing'' or ``decreasing":  If the derivative is negative on an interval, that is if $f'(x) <0$ for all $x$ in $(a,b)$, then $f(x)$ is \rule{4cm}{0.1mm}




\end{enumerate}


\bigskip

So the first derivative will allow us to determine if a function is increasing or decreasing! 


\index{Derivatives}
	
\begin{tags}
	   Derivatives, IncreasingDecreasing, Graph, TangentLine
\end{tags}
	
\begin{diary}
	    %S2016-HW1-Q1
\end{diary}
	
\begin{solution}
	   
\end{solution}
	
\end{question}

\end{tagblock}

%-------------------------------------------------------------------------------------------------------------

\begin{tagblock}{Derivatives, IncreasingDecreasing, Critical, MaxMin, FirstDerivativeTest}
\begin{question}

Our next goal is to use the first derivative to determine where we have local minimums and local maximums.  Recall from our previous worksheet the \emph{critical numbers} are our candidates for local minimums and local maximums.  

 Let $f(x)$ be given by the graph below.  Use the graph to answer each of the following questions.
\begin{figure}[h]
\centering
\includegraphics[width=7cm]{minmax1.png} 
\end{figure}

\begin{enumerate}

\item Find the critical numbers of $f(c)$: identify all values of $c$ for which $f'(c)$ does not exist; identify all values of $c$ for which $f'(c) = 0$.

\vspace{.4in}



\item Which critical numbers $c$ give local minimums? For each $c$, does the function change from increasing to decreasing at $c$, or decreasing to increasing at $c$?

\vspace{.4in}



\item Which critical numbers $c$ give local maximums? For each $c$, does the function change from increasing to decreasing at $c$, or decreasing to increasing at $c$?

\vspace{.4in}

\item Which critical numbers $c$ are neither local minimums nor local maximums?  What can you say about the increasing and decreasing?  







\end{enumerate}


\bigskip

The results of the previous problem is exactly the \textbf{First Derivative Test} for determining local minimums and maximums.

\bigskip

\textbf{First Derivative Test:}  Suppose $c$ is a critical number of a function $f(x)$
\begin{itemize}
\item If $f'(x)$ changes sign from \emph{positive to negative} at $c$ (that is $f$ changes from increasing to decreasing) then $f(x)$ has a \emph{local maximum} at $x=c$.
\item If $f'(x)$ changes sign from \emph{negative to positive} at $c$ (that is $f$ changes from decreasing to increasing ) then $f(x)$ has a \emph{local minimum} at $x=c$.
\item If $f'(x)$ doesn't change sign at $c$, then $c$ is neither a local min nor a local max.  
\end{itemize}

\vspace{.5in}






\index{Derivatives}
	
\begin{tags}
	   Derivatives, IncreasingDecreasing, Critical, MaxMin, FirstDerivativeTest
\end{tags}
	
\begin{diary}
	    %S2016-HW1-Q1
\end{diary}
	
\begin{solution}
	   
\end{solution}
	
\end{question}

\end{tagblock}

%-------------------------------------------------------------------------------------------------------------

\begin{tagblock}{Derivatives, Graph, MaxMin, FirstDerivativeTest}
\begin{question}

Returning to $\displaystyle g(x) = \frac{7}{4} x^4 + 7x^3 - 14x^2$,

\begin{enumerate}
\item Find the local minimum(s) and local maximum(s) of $g(x)$.  Explain how you know each is a local min or max.  

\vspace{2in}
\item For each $x$ value that gives you a local min or max, compute the $y$-value $g(x)$.  Use these points together with the fact that each is a local min or max to get a rough sketch of the graph of $g$. 
\vspace{.5in}


\end{enumerate}



\index{Derivatives}
	
\begin{tags}
	   Derivatives, Logarithm, ChainRule
\end{tags}
	
\begin{diary}
	    %S2016-HW1-Q1
\end{diary}
	
\begin{solution}
	   
\end{solution}
	
\end{question}

\end{tagblock}

%-------------------------------------------------------------------------------------------------------------

\begin{tagblock}{Derivatives, IncreasingDecreasing, Concavity, Definition, Graph, MaxMin, FirstDerivativeTest, InflectionPoints}
\begin{question}

Recap from the last worksheet:  Let $f(x)$ be a function 
\begin{enumerate}
\item $c$ is a critical number of $f(x)$ if $f '(c)$  \rule{8cm}{0.1mm}

\bigskip
\item If $f '(x) >0$ for all $x$ in the interval $(a,b)$, then $f$ is (circle one) INCREASING  or DECREASING on $(a,b)$.  

\bigskip
\item If $f '(x) <0$ for all $x$ in the interval $(a,b)$, then $f$ is (circle one) INCREASING  or DECREASING on $(a,b)$.  

\bigskip
\item If $c$ is a critical number of $f(x)$  and $f'(x)$ changes sign from positive to negative at $c$, then $c$ is (circle one) a LOCAL MIN of $f$, a LOCAL MAX or $f$, NEITHER A LOCAL MIN NOR LOCAL MAX.

\end{enumerate}


In this worksheet we'll see how the second derivative can give us information about the graph of $f$.  Applying what we did in the last worksheet to $f '(x)$ we get

\begin{itemize}
\item If $f''(x) >0$ on $(a,b)$, then $f'(x)$ is increasing on $(a,b)$, in which case we say $f(x)$ is \emph{concave up}


\item If $f''(x) <0$ on $(a,b)$, then $f'(x)$ is decreasing on $(a,b)$, in which case we say $f(x)$ is \emph{concave down}
\begin{figure}[h]
\centering
\includegraphics[width=3cm]{concaveup.png}  \hspace{1in}
\includegraphics[width=3cm]{concavedown.png} 
\end{figure}

\centering Concave up \hspace{1in} Concave down

\end{itemize}

Graphically, concave up means that $f(x)$ looks like a ``bowl'', and concave down means that $f(x)$ looks like a ``hill.''

We will call a point $c$ in the domain of $f$ a \emph{point of inflection} if $f''(x)$ changes from positive to negative at $c$ or negative to positive at $c$, that is $f$ changes concavity at $c$.  In the graph below, $f$ has an inflection point at $x=1$.  
\begin{figure}[h]
\centering
\includegraphics[width=4cm]{inflection.png}
\end{figure}





\index{Derivatives}
	
\begin{tags}
	   Derivatives, IncreasingDecreasing, Concavity, Definition, Graph, MaxMin, FirstDerivativeTest, InflectionPoints
\end{tags}
	
\begin{diary}
	    %S2016-HW1-Q1
\end{diary}
	
\begin{solution}
	   
\end{solution}
	
\end{question}

\end{tagblock}

%-------------------------------------------------------------------------------------------------------------

\begin{tagblock}{Derivatives, InflectionPoints, Concavity}
\begin{question}

Let $f(x)$ be given by the graph below.  Use the graph to answer each of the following questions. (You may need to estimate answers based on the graph)
\begin{figure}[h]
\centering
\includegraphics[width=8cm]{incdec1.png} 
\end{figure}

\begin{enumerate}
\item Find the intervals where $f(x)$ concave up.  

\vspace{.75in}
\item Find the intervals where $f(x)$ concave down.
\vspace{.75in}

\item Find all the inflection points of $f(x)$.   
\vspace{.75in}

\end{enumerate}



Given an equation of a function $f$, we can determine the intervals of concavity similarly to how we determined the intervals of increasing/decreasing, but using the \emph{second derivative} instead of the first derivative.



\index{Derivatives}
	
\begin{tags}
	   Derivatives, InflectionPoints, Concavity
\end{tags}
	
\begin{diary}
	    %S2016-HW1-Q1
\end{diary}
	
\begin{solution}
	   
\end{solution}
	
\end{question}

\end{tagblock}

%-------------------------------------------------------------------------------------------------------------

\begin{tagblock}{Derivatives, InflectionPoints, Concavity}
\begin{question}

Let $\displaystyle f(x) = \frac {1}{30}x^6 - \frac{1}{10}x^5 - \frac{1}{4}x^4 + 2x$.  We will determine the intervals where $f(x)$ is concave up and concave down and the inflection points.
\begin{enumerate}
\item Compute the second derivative of $f(x)$ and find all values $c$ where $f''(c)$ is undefined or $f''(c) = 0$.

\vspace{2in}
\item Plot your $c$ values from (a) on a number line.  How many intervals do you have?

\vspace{.7in}
\item For each interval, choose a test value and determine if the second derivative at that test value is positive or negative.  

\vspace{3in}
\item Summarizing your work from part (c):  \\
$f$ is concave up on the interval(s) :  \rule{8cm}{0.1mm} \\
\bigskip
$f$ is concave down on the interval(s) :  \rule{8cm}{0.1mm} \\

\bigskip
$f$ has inflection point(s) at $x = $   \rule{8cm}{0.1mm} \\

\bigskip
\item In fact the graph from Problem 2. is the graph of this function.  How do your intervals of concavity compare?  
\end{enumerate}




\index{Derivatives}
	
\begin{tags}
	   Derivatives, InflectionPoints, Concavity
\end{tags}
	
\begin{diary}
	    %S2016-HW1-Q1
\end{diary}
	
\begin{solution}
	   
\end{solution}
	
\end{question}

\end{tagblock}

%-------------------------------------------------------------------------------------------------------------

\begin{tagblock}{Derivatives, InflectionPoints, Concavity, MaxMin, SecondDerivativeTest }
\begin{question}

In the last worksheet we saw that the \textbf{First Derivative Test} could be used to determine if a critical number is a local min or local max.  Similarly we have a test involving the second derivative.

\bigskip

\begin{enumerate}
\item Consider the graph of $f(x)$ given below. 
\begin{minipage}{.4\textwidth}
\includegraphics[width=4cm]{concavemax.png}\end{minipage}% This must go next to `\end{minipage}`
\begin{minipage}{.6\textwidth}
\begin{enumerate}
\item Compute $f '(c) = $

\vspace{.5in}

\item Is $f ''(c)$ positive or negative?  

\vspace{.5in}
\item Is $c$ a local min or local max of $f$?
\vspace{.5in}

\end{enumerate}

\end{minipage}

\item Consider the graph of $f(x)$ given below. 
\begin{minipage}{.4\textwidth}
\includegraphics[width=4cm]{concavemin.png}\end{minipage}% This must go next to `\end{minipage}`
\begin{minipage}{.6\textwidth}
\begin{enumerate}
\item Compute $f '(c) = $

\vspace{.5in}

\item Is $f ''(c)$ positive or negative?  

\vspace{.5in}
\item Is $c$ a local min or local max of $f$?

\vspace{.5in}

\end{enumerate}

\end{minipage}


\end{enumerate}



This leads us to the \textbf{Second Derivative Test}:  Let $c$ be a critical number of $f(x)$
\begin{itemize}
\item If $f''(c) >0$, then $c$ is a local min;
\item If $f''(c) <0$, then $c$ is a local max; 
\item If $f''(c) =0$, we can't conclude anything, and we need to instead use the First Derivative Test.
\end{itemize}




\index{Derivatives}
	
\begin{tags}
	   Derivatives, InflectionPoints, Concavity, MaxMin, SecondDerivativeTest
\end{tags}
	
\begin{diary}
	    %S2016-HW1-Q1
\end{diary}
	
\begin{solution}
	   
\end{solution}
	
\end{question}

\end{tagblock}

%-------------------------------------------------------------------------------------------------------------

\begin{tagblock}{Derivatives, InflectionPoints, Concavity, MaxMin, SecondDerivativeTest }
\begin{question}

Let $\displaystyle f(x) = \frac{1}{4}x^4 + \frac{1}{3}x^3 - x^2 +3$.  Find the critical numbers of $f$ and then use the \textbf{Second Derivative Test} to determine if they are local mins or maxs.


\index{Derivatives}
	
\begin{tags}
	   Derivatives, InflectionPoints, Concavity, MaxMin, SecondDerivativeTest
\end{tags}
	
\begin{diary}
	    %S2016-HW1-Q1
\end{diary}
	
\begin{solution}
	   
\end{solution}
	
\end{question}

\end{tagblock}

%-------------------------------------------------------------------------------------------------------------

\begin{tagblock}{Derivatives, InflectionPoints, Concavity, MaxMin, InflectionPoints, IncreasingDecreasing, Graph}
\begin{question}

Summarizing what we learned in the last two worksheets.
\begin{enumerate}
\item If $f '(x) >0$ on $(a,b)$, then $f$ is  \rule{8cm}{0.1mm} 
\bigskip

\item If $f '(x) <0$ on $(a,b)$, then $f$ is  \rule{8cm}{0.1mm}
\bigskip
\item If $f ''(x) >0$ on $(a,b)$, then $f$ is  \rule{8cm}{0.1mm} 
\bigskip

\item If $f ''(x) <0$ on $(a,b)$, then $f$ is  \rule{8cm}{0.1mm} 
\bigskip 
\item This then gives us 4 different possibilities:  increasing and concave up, increasing and concave down, decreasing and concave up, decreasing and concave down.  On the axes below draw the general shape of the graph.
\end{enumerate}

\hangindent=-1in \includegraphics[width=3.6cm]{incup.png}\, \includegraphics[width=3.6cm]{incdown.png} \, \includegraphics[width=3.6cm]{decup.png} \, \includegraphics[width=3.6cm]{decdown.png} 
 
 \begin{enumerate}
 \item[(f)] Give a sketch of a graph $f$ that satisfies the following properties:  $f(0)=2$;  $f'(2) = 0 = f'(3)$; \\
 $f'(x) >0$ on the intervals $(-\infty, 2)$ and $(3, \infty)$, and $f'(x)<0$ on the interval $(2,3)$; \\
 $f''(x)>0$ on the intervals $(0,1)$ and $(2.5,\infty)$, $f''(x)<0$ on the intervals $(-\infty, 0)$ and $(1,2.5)$.
 

\end{enumerate}


\index{Derivatives}
	
\begin{tags}
	   Derivatives, InflectionPoints, Concavity, MaxMin, InflectionPoints, IncreasingDecreasing, Graph
\end{tags}
	
\begin{diary}
	    %S2016-HW1-Q1
\end{diary}
	
\begin{solution}
	   
\end{solution}
	
\end{question}

\end{tagblock}

%-------------------------------------------------------------------------------------------------------------

\begin{tagblock}{Derivatives, HorizontalAsymptotes, VerticalAsymptotes, Graph, WarmUp, RationalFunction  }
\begin{question}

\textbf{Goal:  Given an equation of a function $f(x)$, tie together our work on derivatives and limits to sketch a graph of $f(x)$.}

\bigskip

We will use following technique to sketch a curve given an equation
\begin{itemize}
\item Find the intercepts (both $x$ and $y$-intercepts)
\item Find any asymptotes (horizontal, vertical, slant)
\item Find the intervals where the function is increasing and decreasing
\item Find the intervals where the function is concave up and concave down
\item Find both the $x$ and $y$ coordinates of any local maximums, local minimums, and inflection points
\end{itemize}


\bigskip


\textbf{Quick Review of Intercepts and Asymptotes:}

\bigskip

\textbf{Intercepts:}
\begin{itemize}
\item Recall to find the $x$-intercepts, we want to find when the function crosses the $x$- axis.  This means we will set our equation equal to $0$ and solve for $x$.
\item To find the $y$-intercepts, we want to find when the function crosses the $y$- axis.  This means we will set $x=0$ and compute $f(0)$.
\end{itemize}

\begin{enumerate}  
\item Intercepts:
\begin{enumerate}
\item Find both the $x$ and $y$-intercepts of $g(x) = x^2-4x -21$.  
\begin{itemize}
\item $x$-intercept(s):  
\vspace{.5in}




\item $y$-intercepts(s):

\vspace{.5in}

\end{itemize}
\item Can a function have more than one $x$-intercept?  Can a function have more than one $y$-intercept?  

\vspace{.5in}

\item Given an example of a function $f$ that does not have a $y$-intercept.  (Either give a graph or an equation of  $f$ )


\vspace{.5in}


\end{enumerate}
\end{enumerate}

\newpage
\textbf{Asymptotes:}


 \textbf{Vertical Asymptotes}  Recall we get a vertical asymptote at $x=a$, when $\lim_{x \to a^+} f(x) = \pm \infty$ or $\lim_{x \to a^-} f(x) = \pm \infty$.  More informally, if $f(x) = \frac{g(x)}{h(x)}$, with $h(a)=0$, but $g(a) \neq 0$, then we will get a vertical asymptote at $x=a$.

\bigskip

Find the vertical asymptotes of $\displaystyle f(x) = \frac{x^2+3}{x+1}$,  $\displaystyle g(x) = \frac{x^2+3}{2x^2 - 1}$ and $\displaystyle h(x) = \frac{x^2+3}{x^3-1}$.  

\vspace{1in}

\bigskip

 \textbf{Horizontal Asymptotes}  Recall we get a horizontal asymptote at $y=M$, when $\lim_{x \to \infty} f(x) = M$ or $\lim_{x \to -\infty} f(x) = M$, and to compute limits as $x \to \infty$ we ``divided by the highest power of $x$''.  

We also developed a short cut for determining horizontal asymptotes of \emph{rational functions}:

\[\text{ Let } f(x) = \frac{p(x)}{q(x)} = \frac{a_nx^n + a_{n-1}x^{n-1} + \cdots + a_1x+a_0}{b_mx^m + b_{m-1}x^{m-1} + \cdots + b_1x+b_0}\]
\bigskip

\begin{itemize}
\item If $n=m$, then $f(x)$ has a horizontal asymptote at $y =$ \rule{5cm}{0.1mm} 

\bigskip

\item If $n<m$, then $f(x)$ has a horizontal asymptote at $y = $ \rule{5cm}{0.1mm} 

\bigskip
\item If $n>m$, then $f(x)$ has \rule{8cm}{0.1mm} 
\bigskip

\end{itemize}



Find the horizontal asymptotes (if any) of $\displaystyle f(x) = \frac{x^2+3}{x+1}$,   $\displaystyle g(x) = \frac{x^2+3}{2x^2 - 1}$ and $\displaystyle h(x) = \frac{x^2+3}{x^3-1}$.



\index{Derivatives}
	
\begin{tags}
	   Derivatives, HorizontalAsymptotes, VerticalAsymptotes, Graph, WarmUp, RationalFunction
\end{tags}
	
\begin{diary}
	    %S2016-HW1-Q1
\end{diary}
	
\begin{solution}
	   
\end{solution}
	
\end{question}

\end{tagblock}

%-------------------------------------------------------------------------------------------------------------

\begin{tagblock}{Derivatives, HorizontalAsymptotes, VerticalAsymptotes, SlantAsymptotes, Graph, RationalFunction  }
\begin{question}


Consider the function $f(x)$ given by the graph below. 
\begin{figure}[h]
\centering
\includegraphics[width=8cm]{slant.png} 
\end{figure}
\begin{enumerate}
\item Does $f(x)$ have any vertical and/or horizontal asymptotes?  If so what are they.  

\vspace{1in}
\item Graph the line $y=x$ on the graph of $f(x)$ above.  As $x$ gets large, what do you notice about the graph of $f$ compared to the graph of $y=x$?

\vspace{1in}
\end{enumerate}


We call the line $y=x$ in the above example a \emph{slant asymptote}.  

A rational function 
\[ f(x) = \frac{p(x)}{q(x)} = \frac{a_nx^n + a_{n-1}x^{n-1} + \cdots + a_1x+a_0}{b_mx^m + b_{m-1}x^{m-1} + \cdots + b_1x+b_0}\]
will have a slant asymptote when $n=m+1$, that is the highest power of $x$ in the numerator is one more than the highest power of $x$ in the denominator.  

\newpage

\textbf{Computing Slant Asymptotes}  To compute slant asymptotes, we will do polynomial long division, which will leave us with a linear equation and a remainder.


Let $\displaystyle f(x) = \frac{x^2+3}{x+1}$

\begin{enumerate}
\item Use polynomial long division to divide $x^2+3$ by $x+1$.

\vspace{.5in}

\begin{center}
\Mydiv{x+1} {x^2+3 \hspace{.5in}} 
\end{center}

\vspace{2in}
\item We have a remainder of $4$, or in other words a term $\frac{4}{x+1}$.  What happens to $\frac{4}{x+1}$ as $x \to \infty$?

\vspace{1in}

\item This means we can ignore the remainder, and so we have a slant asymptote at $y = $

\end{enumerate}

\index{Derivatives}
	
\begin{tags}
	   Derivatives, HorizontalAsymptotes, VerticalAsymptotes, SlantAsymptotes, Graph, RationalFunction
\end{tags}
	
\begin{diary}
	    %S2016-HW1-Q1
\end{diary}
	
\begin{solution}
	   
\end{solution}
	
\end{question}

\end{tagblock}

%-------------------------------------------------------------------------------------------------------------

\begin{tagblock}{Derivatives, HorizontalAsymptotes, VerticalAsymptotes, SlantAsymptotes, Graph, RationalFunction, Critical, MaxMin, Concavity, InflectionPoints, IncreasingDecreasing  }
\begin{question}


We'll now put this all together to get a good graph of $\displaystyle f(x) = \frac{x^2+3}{x+1}$
\begin{enumerate}
\item Find the $x$ and $y$-intercepts of $f(x)$.

\vspace{.75in}
\item Find all the asymptotes of $f(x)$
\begin{enumerate}
\item Vertical Asymptotes:
\item Horizontal Asymptotes:
\item Slant Asymptotes:
\end{enumerate}
At this point we might want to start plotting our asymptotes and our intercepts on our graph.

\begin{figure}[h]
\centering
\includegraphics[width=8cm]{blank.png} 
\end{figure}




\item Find the critical numbers of $f(x)$.





\newpage
\item Find the intervals where $f$ is increasing and decreasing.
\vspace{2in}

\item Using the quotient rule again, and simplifying we find that the second derivative is $f''(x) = \frac{8}{(x+1)^3}$.  Use this find when $f$ is concave up and concave down.  Does $f$ have any inflection points?

\vspace{1.5in}

\item Find both the $x$ and $y$ coordinate of all the local minimums, local maximums and inflection points.  

\vspace{1.5in}
\item Combining parts (d) and (e):  
Interval(s) where $f$ is \\
increasing and concave down:   \rule{2cm}{0.1mm} \, decreasing and concave down:   \rule{2cm}{0.1mm} \\
\bigskip

decreasing and concave up:   \rule{2cm}{0.1mm} \,   increasing and concave up:   \rule{2cm}{0.1mm}

\item Finish the sketch of the graph, labeling all asymptotes, local mins, local maxs and inflection points.  
\end{enumerate}


\index{Derivatives}
	
\begin{tags}
	   Derivatives, HorizontalAsymptotes, VerticalAsymptotes, SlantAsymptotes, Graph, RationalFunction, Critical, MaxMin, Concavity, InflectionPoints, IncreasingDecreasing
\end{tags}
	
\begin{diary}
	    %S2016-HW1-Q1
\end{diary}
	
\begin{solution}
	   
\end{solution}
	
\end{question}

\end{tagblock}

%-------------------------------------------------------------------------------------------------------------

\begin{tagblock}{Derivatives, HorizontalAsymptotes, VerticalAsymptotes, SlantAsymptotes, Graph, RationalFunction, Critical, MaxMin, Concavity, InflectionPoints, IncreasingDecreasing  }

\begin{question}

We have followed the subsequent technique to sketch a curve given an equation
\begin{itemize}
\item Find the intercepts (both $x$ and $y$-intercepts)
\item Find any asymptotes (horizontal, vertical, slant)
\item Find when the function is increasing or decreasing
\item Find when the function is concave up or concave down
\item Find both the $x$ and $y$ coordinates of any local maximums, local minimums, and inflection points
\end{itemize}

Below are four functions  $f(x)$, $g(x)$, $h(x)$ and $j(x)$ along with their first and second derivatives (so you do not have to compute them).  Using the technique we outlined above you should sketch each function.  \textbf{Be sure to include any asymptotes on your graph and label the local maximums, local minimums, and inflection points (both the $x$ and $y$ coordinates).}  



\begin{enumerate}  
\item $\displaystyle f(x) = \frac{2x}{x^2+1} \hspace{.5in} f'(x) = \frac{-2(x^2-1)}{(x^2+1)^2} \hspace{.5in} f''(x) = \frac{4x(x^2-3)}{(x^2+1)^3} $
\item  $\displaystyle  g(x) = \frac{x^3}{x^2-1} \hspace{.5in} g'(x) = \frac{x^2(x^2-3)}{(x^2-1)^2} \hspace{.5in} g''(x) = \frac{2x(x^2+3)}{(x^2-1)^3} $
\item $\displaystyle h(x) =  \frac{x^3}{x^2+2x+1} \hspace{.2in} h'(x) = \frac{x^2(x+3)}{(x+1)^3} \hspace{.5in}  h''(x) = \frac{6x}{(x+1)^4} $
\item $\displaystyle  j(x) = \frac{3x^2-3}{x^2-4} \hspace{.5in} j'(x) = \frac{-18x}{(x^2-4)^2} \hspace{.5in} j''(x) = \frac{18(3x^2+4)}{(x^2-4)^3} $


\end{enumerate}


\index{Derivatives}
	
\begin{tags}
	   Derivatives, HorizontalAsymptotes, VerticalAsymptotes, SlantAsymptotes, Graph, RationalFunction, Critical, MaxMin, Concavity, InflectionPoints, IncreasingDecreasing

\end{tags}
	
\begin{diary}
	    %S2016-HW1-Q1
\end{diary}
	
\begin{solution}
	   
\end{solution}
	
\end{question}

\end{tagblock}

%-------------------------------------------------------------------------------------------------------------

\begin{tagblock}{Derivatives, WarmUp, L'Hopital, Limits }

\begin{question}

\textbf{Warm Up}

\begin{enumerate}
\item Earlier in the semester we investigated the function $\displaystyle f(x) = \frac{x^2-4}{x-2}$ and the $\lim_{x \to 2} f(x)$.  
What happens when we try to evaluate $f(x)$ at $x=2$?  
\vspace{.5in}

 How can we use algebraic tools to determine  the $\displaystyle \lim_{x \to 2} f(x)$?  
 \vspace{1in}
 
 \item Consider the function $\displaystyle H(x) = \frac{f(x)}{g(x)}$, where $f(x) = {e^x}$ and $g(x) = {x}$.  Compute the $\displaystyle \lim_{x \to \infty} f(x)$ and $\displaystyle \lim_{x \to \infty} g(x)$.  Can we use this information to determine $\displaystyle \lim_{x \to \infty} H(x)$?  
 
 
 \vspace{1in}
 
 \end{enumerate}
 
 If we are in the situation as in the previous examples where we get either $\frac{0}{0}$ or $\frac{\pm \infty}{\pm \infty}$ we say that we have an \textbf{indeterminate form}.  In this worksheet we will learn a new technique for evaluating limits of this type.  
 
 
 


\begin{tcolorbox}
 \textbf{l'Hospital's Rule:} Suppose $f(x)$ and $g(x)$ and differentiable and $g'(x) \neq 0$ on an open interval containing $a$.  If 
\[\frac{\lim_{x \to a}f(x)} { \lim_{x \to a}g(x)} = \frac{0}{0} \text{ or } \frac{\lim_{x \to a}f(x)} { \lim_{x \to a}g(x)} = \frac{\infty}{\infty}, \]\text{ then} \[ \lim_{x \to a} \frac{f(x)}{g(x)} = \lim_{x \to a} \frac{f'(x)}{g'(x)}\]

\end{tcolorbox}

\bigskip

\textbf{Warning!} We can only use l'Hospital's Rule if we start with an indeterminate form, that is $\frac{0}{0}$ or $\frac{\infty}{\infty}$.


\bigskip


Returning to the function $\displaystyle H(x) = \frac{e^x}{x}$.  To compute the $\displaystyle \lim_{x \to \infty}  \frac{e^x}{x}$ we saw we have an indeterminate form of the type $\frac{\infty}{\infty}$.  Use  l'Hospital's Rule to determine the limit.


\index{Derivatives}
	
\begin{tags}
	   Derivatives, WarmUp, L'Hopital, Limits

\end{tags}
	
\begin{diary}
	    %S2016-HW1-Q1
\end{diary}
	
\begin{solution}
	   
\end{solution}
	
\end{question}

\end{tagblock}

%-------------------------------------------------------------------------------------------------------------

\begin{tagblock}{Derivatives, L'Hopital, Limits }

\begin{question}

Calculate $\displaystyle  \lim_{x \to 0} \frac{e^{2x}-1}{x}$ using l'Hospital's Rule.  Make sure you justify why you can apply  l'Hospital's Rule.


\index{Derivatives}
	
\begin{tags}
	   Derivatives, L'Hopital, Limits

\end{tags}
	
\begin{diary}
	    %S2016-HW1-Q1
\end{diary}
	
\begin{solution}
	   
\end{solution}
	
\end{question}

\end{tagblock}

%-------------------------------------------------------------------------------------------------------------

\begin{tagblock}{Derivatives, L'Hopital, Limits, HorizontalAsymptotes }

\begin{question}

Calculate  $\displaystyle  \lim_{x \to \infty} \frac{\ln(x)}{x}$ using l'Hospital's Rule.   Make sure you justify why you can apply  l'Hospital's Rule.   Determine the horizontal asymptotes (if any) of $ \displaystyle f(x) =   \frac{\ln(x)}{x}$.


\index{Derivatives}
	
\begin{tags}
	   Derivatives, L'Hopital, Limits, HorizontalAsymptotes 

\end{tags}
	
\begin{diary}
	    %S2016-HW1-Q1
\end{diary}
	
\begin{solution}
	   
\end{solution}
	
\end{question}

\end{tagblock}

%-------------------------------------------------------------------------------------------------------------

\begin{tagblock}{Derivatives, L'Hopital, Limits}

\begin{question}

Note that it is possible that after applying l'Hospital's Rule you still have an indeterminate form.  In that case you may need to use l'Hospital's rule a second time!  \\
 Calculate $\displaystyle \lim_{x\to 0}\frac{x-\sin(x)}{x^2}$ using l'Hospital's Rule.


\index{Derivatives}
	
\begin{tags}
	   Derivatives, L'Hopital, Limits

\end{tags}
	
\begin{diary}
	    %S2016-HW1-Q1
\end{diary}
	
\begin{solution}
	   
\end{solution}
	
\end{question}

\end{tagblock}

%-------------------------------------------------------------------------------------------------------------

\begin{tagblock}{Derivatives, L'Hopital, Limits, Example}

\begin{question}

If we are trying to compute a limit that is not already a quotient, we may be able to do some algebra first, so that we can apply l'Hospital's Rule. 
\bigskip

  Consider 
 $\displaystyle \lim_{x\to 0^+} {x^2} \ln(x)$ 
 
 Note that we can rewrite $x^2 \ln(x)$ as $\displaystyle \frac{\ln(x)}{x^{-2}}$; and further we know $\displaystyle \lim_{x \to 0^+} \ln(x) = - \infty$, and $\displaystyle \lim_{x \to 0^+} x^{-2} = \lim_{x \to 0^+} \frac{1}{x^2} = \infty$.  Putting this together, we have 
 \[  \lim_{x\to 0^+} {x^2} \ln(x) =  \lim_{x\to 0^+} \frac{\ln(x)}{x^{-2}} \]
 which is an indeterminate form.  Now finish computing the limit using  l'Hospital's Rule. 


\index{Derivatives}
	
\begin{tags}
	   Derivatives, L'Hopital, Limits, Example

\end{tags}
	
\begin{diary}
	    %S2016-HW1-Q1
\end{diary}
	
\begin{solution}
	   
\end{solution}
	
\end{question}

\end{tagblock}

%-------------------------------------------------------------------------------------------------------------

\begin{tagblock}{Derivatives, L'Hopital, Limits}

\begin{question}

Calculate $\displaystyle\lim_{x\to \infty} {x} \sin(\frac{1}{x})$.  Make sure you justify why you can apply  l'Hospital's Rule.  


\index{Derivatives}
	
\begin{tags}
	   Derivatives, L'Hopital, Limits

\end{tags}
	
\begin{diary}
	    %S2016-HW1-Q1
\end{diary}
	
\begin{solution}
	   
\end{solution}
	
\end{question}

\end{tagblock}

%-------------------------------------------------------------------------------------------------------------

\begin{tagblock}{Derivatives, L'Hopital, Limits, Theory}

\begin{question}

Let $n>0$ be any integer.  Determine  $\displaystyle\lim_{x\to\infty}\frac {x^n}{e^{x}}$.   


\index{Derivatives}
	
\begin{tags}
	   Derivatives, L'Hopital, Limits, Theory

\end{tags}
	
\begin{diary}
	    %S2016-HW1-Q1
\end{diary}
	
\begin{solution}
	   
\end{solution}
	
\end{question}

\end{tagblock}

%-------------------------------------------------------------------------------------------------------------

\begin{tagblock}{Derivatives, L'Hopital, Limits}

\begin{question}

Consider $\displaystyle \lim_{x\to 0^+} {x}^{x^2}$.  Note that we don't have a quotient, but we can manipulate our function so that l'Hospital's Rule can be applied.

In general we can re-write $f(x)^{g(x)}$ as follows:
\[f(x)^{g(x)} = e^{\ln (f(x)^{g(x)})} = e^{g(x) \ln (f(x))}  \]

So for our function $ {x}^{x^2}$, we have
\[  {x}^{x^2} = e^{x^2 \ln(x)} \]
Then to compute 
\[ \lim_{x\to 0^+} {x}^{x^2} =  \lim_{x\to 0^+}  e^{x^2 \ln(x)} \]
we simply need to determine $ \lim_{x\to 0^+} {x^2 \ln(x)}$.  Recall we did this in an earlier problem, and found that $ \lim_{x\to 0^+} {x^2 \ln(x)} = 0$.  This tells us then that
\[ \lim_{x\to 0^+} {x}^{x^2} =  \lim_{x\to 0^+}  e^{x^2 \ln(x)}   = e^0 = 1\]

\bigskip

Calculate $\displaystyle\lim_{x\to \infty} (1+ \frac{3}{x})^{5x}$.  


\index{Derivatives}
	
\begin{tags}
	   Derivatives, L'Hopital, Limits

\end{tags}
	
\begin{diary}
	    %S2016-HW1-Q1
\end{diary}
	
\begin{solution}
	   
\end{solution}
	
\end{question}

\end{tagblock}

%-------------------------------------------------------------------------------------------------------------

\begin{tagblock}{Derivatives, Optimization, WarmUp}

\begin{question}

\begin{enumerate}
\item \textbf{A first example:}  In this problem we will find two numbers whose difference is $20$ and whose product is as small as possible. 
\begin{enumerate}
\item Let $x$ and $y$ be two numbers, whose difference is $20$, and suppose $y>x$.    For example, when $x=5$, $y=\underline{\hspace{1in}}$ and when $x=-5$, $y=\underline{\hspace{1in}}$.  Give a general equation for $y$ in terms of $x$.
\vspace{.5in}

\item Write a function in terms of $x$ and $y$ that gives their product, $P$.  This is the function we want to minimize.  Note though that it has two variables.
\vspace{.5in}
\item  Using part (a),  rewrite your function in (b) so that it only involves the variable $x$.
\vspace{.5in}
\item Now we have a function, $P(x)$ that gives the product in terms of one variable $x$.  Now we can use either the First or Second Derivative Test to find the minimum of our function.  Find the minimum using our calculus techniques!
\vspace{2in}
\item Finally we want both $x$ and $y$.  What is $y$?


\end{enumerate}
\end{enumerate}

\newpage

Our general strategy will be as follows:
\begin{itemize}
\item Draw a useful picture. You may need to draw more than one to understand what is going on.
\item Use variables to name your unknowns and identify the variable you want to optimize. 
\item Write an equation for the variable you want to optimize. Sometimes this is done for you.
\item If your equation from the last step is not a function of one variable, you may need to find an extra constraint relating the variables.
\item Once your equation is a function of just one variable, go ahead and optimize it by finding critical points and using first or second derivative test. \emph{Make sure you check that your critical number is indeed a min or max}
\end{itemize}



\index{Derivatives}
	
\begin{tags}
	   Derivatives, Optimization, WarmUp

\end{tags}
	
\begin{diary}
	    %S2016-HW1-Q1
\end{diary}
	
\begin{solution}
	   
\end{solution}
	
\end{question}

\end{tagblock}

%-------------------------------------------------------------------------------------------------------------

\begin{tagblock}{Derivatives, Optimization}

\begin{question}

Find the dimensions of a rectangle with maximum area and perimeter equal to $800$.
\begin{enumerate}
\item Draw a picture of a rectangle and label the sides (here you will need to choose variables)

\vspace{1in}
\item The problem asks us to maximize their area, so the equation to be optimized is\\
\bigskip
 $$A=\underline{\hspace{2in}} .$$
 \item What constraints do we have?  How can we use this to write the area in terms of one variable?
 
 \vspace{1.5in} 
 \item Once your area is a function of one variable, find the maximum.  Make sure you check that your critical number indeed gives a max!
 
\vfill
 \item Going back to our original question, we want the dimensions of the rectangle.  You found one dimension in (d), now find the other one.
 
 \end{enumerate}



\index{Derivatives}
	
\begin{tags}
	   Derivatives, Optimization

\end{tags}
	
\begin{diary}
	    %S2016-HW1-Q1
\end{diary}
	
\begin{solution}
	   
\end{solution}
	
\end{question}

\end{tagblock}

%-------------------------------------------------------------------------------------------------------------

\begin{tagblock}{Derivatives, Optimization}

\begin{question}

A box will have a square base, an open top and a volume of 4000 $\text{cm}^2$.  What are the dimensions of the box to minimize the surface area?
 \begin{enumerate}
 \item Draw a picture of your box and label the sides.  Note that you will have two dimensions: one that gives the length of the square base and one that gives the height.
 
 \vspace{1.5in}
\item The problem asks us to minimize the surface area.  The surface area will consist of the area of the square base and the area of the four sides (no top!)  so the equation to be optimized is\\
\bigskip
 $$SA=\underline{\hspace{2in}} .$$
 
  \item What constraints do we have?  How can we use this to write the surface area in terms of one variable?
 
 \vspace{1in} 
  \item Once your area is a function of one variable, find the minimum.  Make sure you check that your critical number indeed gives a min!
 
\vfill
 \item Going back to our original question, we want the dimensions of the box.  You found one dimension in (d), now find the other one.
 
 \end{enumerate}



\index{Derivatives}
	
\begin{tags}
	   Derivatives, Optimization

\end{tags}
	
\begin{diary}
	    %S2016-HW1-Q1
\end{diary}
	
\begin{solution}
	   
\end{solution}
	
\end{question}

\end{tagblock}

%-------------------------------------------------------------------------------------------------------------

\begin{tagblock}{Derivatives, Optimization}

\begin{question}

We'll modify the previous problem.  We still want a box with a square base, an open top and a volume of 4000 $\text{cm}^2$.  Suppose material for the base costs $\$.10$ per $\text{cm}^2$ and the material for the sides cost $\$.02$  per $\text{cm}^2$.  Find the cheapest cost of such a box.  
\begin{enumerate}
 \item Notice we have the same picture as in 1.
\item The problem asks us to minimize the cost, so the equation to be optimized is \\
 \bigskip
 $$C=\underline{\hspace{3in}} .$$
 \item We have the same constraints as in 1.   How can we use this to write the cost area in terms of one variable?
 \vspace{1in} 
  \item Once your area is a function of one variable, find the minimum.  Make sure you check that your critical number indeed gives a min!
 
\vspace{3in}
 \item Going back to our original question, we want the cheapest cost.  Find the cheapest cost.
 \end{enumerate}



\index{Derivatives}
	
\begin{tags}
	   Derivatives, Optimization

\end{tags}
	
\begin{diary}
	    %S2016-HW1-Q1
\end{diary}
	
\begin{solution}
	   
\end{solution}
	
\end{question}

\end{tagblock}

%-------------------------------------------------------------------------------------------------------------

\begin{tagblock}{Derivatives, Optimization}

\begin{question}

A farmer wants to start raising cows, horses, goats, and sheep, and desires to have a rectangular pasture for the animals to graze in. However, no two different kinds of animals can graze together. In order to minimize the amount of fencing she will need, she has decided to enclose a large rectangular area and then divide it into four equally sized pens by adding three segments of fence inside the large rectangle that are parallel to two existing sides. She has decided to purchase 7500 ft of fencing. What is the maximum possible area that the four pens will enclose?


\index{Derivatives}
	
\begin{tags}
	   Derivatives, Optimization

\end{tags}
	
\begin{diary}
	    %S2016-HW1-Q1
\end{diary}
	
\begin{solution}
	   
\end{solution}
	
\end{question}

\end{tagblock}

%-------------------------------------------------------------------------------------------------------------

\begin{tagblock}{Derivatives, Optimization}

\begin{question}

A farmer wants to start raising cows, horses, goats, and sheep, and desires to have a rectangular pasture for the animals to graze in. However, no two different kinds of animals can graze together. In order to minimize the amount of fencing she will need, she has decided to enclose a large rectangular area and then divide it into four equally sized pens by adding three segments of fence inside the large rectangle that are parallel to two existing sides. She has decided to purchase 7500 ft of fencing. What is the maximum possible area that the four pens will enclose?


\index{Derivatives}
	
\begin{tags}
	   Derivatives, Optimization

\end{tags}
	
\begin{diary}
	    %S2016-HW1-Q1
\end{diary}
	
\begin{solution}
	   
\end{solution}
	
\end{question}

\end{tagblock}

%-------------------------------------------------------------------------------------------------------------

\begin{tagblock}{Derivatives, Optimization}

\begin{question}

Consider a rectangle with bottom left vertex at the origin $(0,0)$ and top right vertex on the graph of $y=-x^2 +4$.  What is the largest possible area of such a rectangle?  As always, justify your answer using calculus.  
\begin{figure}[h]
%\centering
\includegraphics[width=5cm]{optimization6.png} 
\end{figure}


\index{Derivatives}
	
\begin{tags}
	   Derivatives, Optimization

\end{tags}
	
\begin{diary}
	    %S2016-HW1-Q1
\end{diary}
	
\begin{solution}
	   
\end{solution}
	
\end{question}

\end{tagblock}

%-------------------------------------------------------------------------------------------------------------

\begin{tagblock}{Derivatives, Optimization, Challenge}

\begin{question}

A piece of wire $10$ cm long is cut into two pieces. One piece is bent into a square. The second piece is bent into a rectangle whose width is twice its length. Where should the wire be cut in order to \textbf{minimize} the total area of the square and the rectangle? Where should the wire be cut in order to \textbf{maximize} the total area of the square and the rectangle?


\index{Derivatives}
	
\begin{tags}
	   Derivatives, Optimization, Challenge

\end{tags}
	
\begin{diary}
	    %S2016-HW1-Q1
\end{diary}
	
\begin{solution}
	   
\end{solution}
	
\end{question}

\end{tagblock}
\begin{tagblock}{Differentiation, W1}
\begin{question}
	Find the first derivative of $f(x)=x^2\tan(3x+1)$.
	
\index{Derivatives}
	
\begin{tags}
	    Differentiation, ProductRule, ChainRule, Trig
\end{tags}
	
\begin{diary}
	    %S2016-HW1-Q1
\end{diary}
	
\begin{solution}
	   
\end{solution}
	
\end{question}

\end{tagblock}

%-------------------------------------------------------------------------------------------------------------

\begin{tagblock}{Differentiation, ChainRule, FundamentalTheoremI, SquareRoots, W1}
\begin{question}
	Compute $\displaystyle\frac {dg}{dx}$ where $\displaystyle g(x)=\int_3^{x^5}\sqrt{1+t^6} \ dt$.
	
\index{Derivatives}
	
\begin{tags}
	    Differentiation, FundamentalTheoremI, ChainRule, SquareRoots
\end{tags}
	
\begin{diary}
	    %S2016-HW1-Q1
\end{diary}
	
\begin{solution}
	   
\end{solution}
	
\end{question}

\end{tagblock}

%-------------------------------------------------------------------------------------------------------------

\begin{tagblock}{Differentiation, W3, Exponentials, Inverse, Theory}
\begin{question}
	Last class, we defined $\ln(x)=\displaystyle\int^x_1 \frac 1 t \ dt$. We defined $e$ as the number with $\ln(e)=1$. Our point all along has been to solve a certain differential equation, which we will do at the end of this worksheet. 

\bigskip
a) Remind yourselves what $\displaystyle\frac d {dx}\ln(x)$ is equal to for $x>0$. 

\bigskip

b) Remind yourselves that if a function is always increasing, it is invertible with respect to function composition. What condition on the derivative guarantees that a function is increasing?

\bigskip
                
c) Combine parts a) and b) to show that $\ln$ is invertible for $x>0$.


	
\index{Derivatives}
	
\begin{tags}
	    Differentiation, W3, Exponentials, Inverse, Theory
\end{tags}
	
\begin{diary}DFV            
	    %S2016-HW1-Q1
\end{diary}
	
\begin{solution}
	   
\end{solution}
	
\end{question}

\end{tagblock}

%-------------------------------------------------------------------------------------------------------------

\begin{tagblock}{Differentiation, W3, Exponentials, Inverse, Theory}
\begin{question}
	Now we will try to describe the inverse function to $\ln$ fully. 

\bigskip

a) Using the fact that $\ln(x^r)=r\ln(x)$ for all $x>0$ and all $r$, show that $f(x)=e^x$ is the inverse function to $\ln$.  

\bigskip


b) Write 
\[
x=\ln(e^x)=\ln(f(x))
\]
and differentiate both sides, using the chain rule on the right hand side, then solve for $f'(x)$.

\bigskip

c) Plug $e^x$ back in for $f(x)$ to find the derivative of $e^x$. 
	
\index{Derivatives}
	
\begin{tags}
	    Differentiation, W3, Exponentials, Inverse, Theory
\end{tags}
	
\begin{diary}
	    %S2016-HW1-Q1
\end{diary}
	
\begin{solution}
	   
\end{solution}
	
\end{question}

\end{tagblock}

%-------------------------------------------------------------------------------------------------------------

\begin{tagblock}{Differentiation, W1, TangentLines, PowerRule}
\begin{question}
	

\bigskip

Find the equation of the tangent line to $q(x)=x^{3/2}+4x^2+15$ at the point $a=1$.

\bigskip


	
\index{Derivatives}
	
\begin{tags}
	    Differentiation, W1, TangentLines, PowerRule
\end{tags}
	
\begin{diary}
	    %S2016-HW1-Q1
\end{diary}
	
\begin{solution}
	   
\end{solution}
	
\end{question}

\end{tagblock}

%-------------------------------------------------------------------------------------------------------------

\begin{tagblock}{Differentiation, W1, PowerRule, ProductRule, ChainRule, Trigonometry}
\begin{question}
	

\bigskip

Find the first derivative of $f(x)=x^2\tan(3x+1)$.

\bigskip


	
\index{Derivatives}
	
\begin{tags}
	    Differentiation, W1, PowerRule, ProductRule, ChainRule, Trigonometry
\end{tags}
	
\begin{diary}
	    %S2016-HW1-Q1
\end{diary}
	
\begin{solution}
	   
\end{solution}
	
\end{question}

\end{tagblock}

%-------------------------------------------------------------------------------------------------------------

\begin{tagblock}{W1, Integration, Substitution, Integration, Trigonometry, DefiniteIntegral}
\begin{question}
	

\bigskip

Determine the value of $\displaystyle\int_0^{\sqrt{\frac {\pi} 4}} 2x\cos(x^2) \ dx$.

\bigskip


	
\index{Derivatives}
	
\begin{tags}
	    W1, Integration, Substitution, Integration, Trigonometry, DefiniteIntegral
\end{tags}
	
\begin{diary}
	    %S2016-HW1-Q1
\end{diary}
	
\begin{solution}
	   
\end{solution}
	
\end{question}

\end{tagblock}
%-------------------------------------------------------------------------------------------------------------