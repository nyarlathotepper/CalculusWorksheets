\section{Polar}\index{Polar}
\fancyhead[R]{\large Polar}

\begin{tagblock}{Polar, Geometry, WarmUp, W19}
\begin{question}
	Remember your first experience with vectors in physics (if you have never seen vectors, let me know). Polar coordinates are really the proper way to describe vectors: a magnitude and a direction. 

\bigskip

a) Draw a right triangle with legs of length $3$ and $4$- you may pick your own units. Find the length of the hypotenuse.

\bigskip

b) Again draw a right triangle with legs of length $3$ and $3\sqrt{3}$. Find the angle opposite either side. 
	
\index{Polar}
	
\begin{tags}
	    Polar, Geometry, WarmUp, W19
\end{tags}
	
\begin{diary}
	    %S2016-HW1-Q1
\end{diary}
	
\begin{solution}
	   
\end{solution}
	
\end{question}

\end{tagblock}

%-------------------------------------------------------------------------------------------------------------


\begin{tagblock}{Polar, Geometry, Definition, W19}
\begin{question}
	If you have a point $(x,y)$ in rectangular coordinates, the polar coordinates are (almost) given by 
\begin{equation}\label{cartpolar}
r=\sqrt{x^2+y^2}, \quad \theta=\arctan(y/x).
\end{equation}
The quantities $r$ and $\theta$ represent the distance to the origin and the angle with the $x$-axis, respectively, for the vector from $(0,0)$ to $(x,y)$. There isn't much of a problem with $r$ in general, but remember that the range of arctangent is the restricted domain of tangent, $(-\pi/2,\pi/2)$- not quite enough angle!

\bigskip

a) Mindlessly apply equation \eqref{cartpolar} to the point $(-2,3)$. Is what you get correct? How do you fix it?

\bigskip

b) Same question as a) for the point $(-2,-3)$.

\bigskip

c) Try to apply equation \eqref{cartpolar} to the point $(0,2)$. What goes wrong? What should the polar coordinates of $(0,2)$ be, interpreted geometrically?

\bigskip

d) Same question as c) for the point $(0,-2)$. You should now have covered all the bases. 
	
\index{Polar}
	
\begin{tags}
	    Polar, Geometry, Definition, W19
\end{tags}
	
\begin{diary}
	    %S2016-HW1-Q1
\end{diary}
	
\begin{solution}
	   
\end{solution}
	
\end{question}

\end{tagblock}

%-------------------------------------------------------------------------------------------------------------


\begin{tagblock}{Polar, Geometry, W19}
\begin{question}

Going from polar coordinates to rectangular coordinates is even easier. 

\bigskip

a) Draw a right triangle with base on the $x$-axis and hypotenuse of length $r$ making an angle $\theta$ with the $x$-axis in the first quadrant. What are the $x$ and $y$ coordinates of the highest point on the triangle?

\bigskip

b) Find the rectangular coordinates of the polar point $(16,-\pi/3)$ using the formulas $x=r\cos(\theta)$, $y=r\sin(\theta)$. 
	
\index{Polar}
	
\begin{tags}
	    Polar, Geometry, W19
\end{tags}
	
\begin{diary}
	    %S2016-HW1-Q1
\end{diary}
	
\begin{solution}
	   
\end{solution}
	
\end{question}

\end{tagblock}

%-------------------------------------------------------------------------------------------------------------


\begin{tagblock}{Polar, Geometry, Definition, Example, W20}
\begin{question}

The neat thing about points in polar coordinates is that there is more than one representation:
\[
(2,\pi/4)=(2,9\pi/4)=(2,17\pi/4),\cdots
\]

There are a couple of things to cover before we leave the basics. Since people really, really want freedom of expression when it comes to writing down points, we'd best be able to handle \textit{negative} radius and angle. Negative angles are easy: just rotate clockwise intead of counterclockwise.

\bigskip

a) Find two representations in polar coordinates for $(4,-7\pi/8)$ that have positive angle. 

\bigskip

Negative radius is handled by the idea that $-r$ should be in the opposite direction as $r$, so 
\[
(-r,\theta)=(r,\theta+\pi).
\]

b) Find two representations in polar coordinates, both with positive radius but one with positive angle and the other negative, for $(-7,13\pi/6)$. 

\bigskip

c) Find two representations of $(-4,-3\pi/17)$. b) Find the rectangular coordinates of the polar point $(16,-\pi/3)$ using the formulas $x=r\cos(\theta)$, $y=r\sin(\theta)$. 
	
\index{Polar}
	
\begin{tags}
	    Polar, Geometry, Definition, Example, W20
\end{tags}
	
\begin{diary}
	    %S2016-HW1-Q1
\end{diary}
	
\begin{solution}
	   
\end{solution}
	
\end{question}

\end{tagblock}

%-------------------------------------------------------------------------------------------------------------
\begin{tagblock}{Polar, Geometry, Graph, W20}
\begin{question}

Let's take a look at how to express curves given in Cartesian coordinates via polar. Try to find a polar equation for the following curves. Your equation should contain at least one of the variables $r$ and $\theta$, though not necessarily both.

\bigskip

a) $x^2+y^2=16$

\bigskip

b) $y=x$

\bigskip

c) $\dfrac{x^2}9+\dfrac{y^2}{4}=1$

\bigskip

d) $(x-1/2)^2+y^2=1/4$
	
\index{Polar}
	
\begin{tags}
	    Polar, Geometry, Graph, W20
\end{tags}
	
\begin{diary}
	    %S2016-HW1-Q1
\end{diary}
	
\begin{solution}
	   
\end{solution}
	
\end{question}

\end{tagblock}

%-------------------------------------------------------------------------------------------------------------
\begin{tagblock}{Polar, Geometry, Graph, W20}
\begin{question}

Now we'll try the opposite direction: try to find a rectangular equation for the given polar curve. All angle measures are in radians. 

\bigskip

a) $r=7$

\bigskip

b) $\theta=\pi/3$

\bigskip

c) $r=2\sin(\theta)+\cos(\theta)$

\bigskip

d) $r=\theta$ (try to graph this curve!)
	
\index{Polar}
	
\begin{tags}
	    Polar, Geometry, Graph, W20
\end{tags}
	
\begin{diary}
	    %S2016-HW1-Q1
\end{diary}
	
\begin{solution}
	   
\end{solution}
	
\end{question}

\end{tagblock}

%-------------------------------------------------------------------------------------------------------------
