\documentclass[12pt]{article}

%%% Indexing package
\usepackage{makeidx}
\makeindex

\usepackage{comment}

\usepackage{tcolorbox}

%%% Tagging package
\usepackage{etoolbox,verbatim}
%%%

\newenvironment{tagse}{\color{orange}\textbf{Tags.} }{}
\newenvironment{diarys}{\color{magenta}\textbf{Diary.} }{}

\newenvironment{soln}
{\let\oldqedsymbol=\qedsymbol
	\renewcommand{\qedsymbol}{$ $}
	\begin{proof}[\bfseries\upshape \color{blue}Solution]\color{blue}}
	{\end{proof}
	\renewcommand{\qedsymbol}{\oldqedsymbol}}
	
\newenvironment{ssoln}
{\let\oldqedsymbol=\qedsymbol
	\renewcommand{\qedsymbol}{$ $}
	\begin{proof}[\bfseries\upshape \color{red}Student Solution]\color{blue}}
	{\end{proof}
	\renewcommand{\qedsymbol}{\oldqedsymbol}}
	
	
%%%%%%%%%%%%%%%%% Editing of tagging package %%%%%%%%%%%%%%%%%%%%%%%%%
% A counter used to count the number of tags defined in this file. If
% using matchsometags, this counter should be 0.
\newcounter{tagnum}

% A counter used to count the number of tags belonging to a question
% that matches defined tags in this file. If tagnum = matchnum, the
% question is shown. Only for matchalltags.
\newcounter{matchnum}


% These are mutually exclusive tag commands %
% Show questions that match ALL given tags
\newcommand{\matchalltags}[1]{%
  \def\do##1{\csdef{tagged@##1}{}\stepcounter{tagnum}}%
  \docsvlist{#1}}%
  
% Show questions that match at least ONE given tag %
\newcommand{\matchsometags}[1]{%
  \setcounter{tagnum}{0}%
  \def\do##1{\csdef{tagged@##1}{}}%
  \docsvlist{#1}}%
  
% Show all questions %
\newcommand{\showalltags}{%
  \csdef{tagged@ALL}{}}
 
 
  
% This resets a counter for number of matched tags to 0,
% then loops through each tag in the current question and
% counts the number of matches. If the number of matches
% is the same as the total number of defined tags, show
% the question.
%%% tagged@flag is the name of the csname defined in this file for the tag "flag". 
%%%%% This is how matching is done, by checking the csnames of all of the desired tags and counting them.
%
\newenvironment{tagblock}[1]%
    {
    \ifcsname tagged@ALL\endcsname%
    	\let\comment\relax%
    	\let\endcomment\relax%
    \else%
    	\ifnum \value{tagnum}=0%
    		\def\do##1{%
       		 	\ifcsname tagged@##1\endcsname%
       	 			\let\comment\relax%
        			\let\endcomment\relax%
    			\fi}%
   	 		\docsvlist{#1}%
   		\else%
   			\setcounter{matchnum}{0}%
    		\def%
    		\do##1{%
       			\ifcsname tagged@##1\endcsname%
       				\stepcounter{matchnum}%
       			\fi}% 
			\docsvlist{#1}%
   		 	\ifnum \value{matchnum}=\value{tagnum}%
   			 	\let\comment\relax%
   		    	\let\endcomment\relax%
   			\fi%
   		\fi%
   	\fi%
    \comment}%
   {\endcomment \vspace{-\baselineskip} \leavevmode}%
   
%%%%%%%%%%%%%%%%%%%%%%%%%%%%%%%%%%%%%%%%%%%%%%%%%%%%%%%%%%%%%%%%%%%%%%%%%

\specialcomment{tags}{\begin{tagse}}{\end{tagse}}
\specialcomment{diary}{\begin{diarys}}{\end{diarys}}
\specialcomment{solution}{\begin{soln}}{\end{soln}}
\specialcomment{instructor}{\begin{mdframed}{\color{red}\textbf{COMMENTS FOR INSTRUCTORS: } } }{\end{mdframed}}
\specialcomment{scratch}{\begin{mdframed}{\color{blue}\textbf{Scratch Work: } } }{\end{mdframed}}


%%% These are some packages that are useful
\usepackage{amsfonts, lipsum}
\usepackage{amsmath,amssymb, amscd,amsbsy, bbm, amsthm, enumerate}
\usepackage{mdframed, titlesec, setspace,verbatim, multicol}
\usepackage[top=1in, bottom=1in, left=.45in, right=.45in]{geometry}
\usepackage[unicode]{hyperref}
\usepackage{tikz, pgfplots, xcolor, fancyhdr}
\usepackage{graphicx, subcaption}
\usepackage{multirow}

%%% Page formatting
%\setlength{\headsep}{30pt}
\setlength{\parindent}{0pt}
\setlength{\textheight}{9in}

%%% Header and Footer Info
\pagestyle{fancy}
\fancyhead[L]{{\large Calculus - \textbf{Master Document}}}
\fancyhead[C]{}
\fancyhead[R]{}
\fancyfoot[L]{UM Dearborn}
\fancyfoot[C]{\thepage}
\fancyfoot[R]{}

%%% These define our question environment and help number things correctly
\theoremstyle{definition}
\newtheorem{thm}{Theorem}
\newtheorem{question}[thm]{Question}
\newtheorem{prop}[thm]{Proposition}
\newtheorem{lem}[thm]{Lemma}
\newtheorem{DEF}[thm]{Definition}
\newtheorem{rem}[thm]{Remark}

%%% These are some shortcuts that are handy
\def\real{{\mathbb R}}
\def\Natural{\mathbb{N}}
\def\dx{\textnormal{dx}}
\def\dy{\textnormal{dy}}
\def\dz{\textnormal{dz}}
\def\dt{\textnormal{dt}}
\def\ds{\textnormal{ds}}
\def\dw{\textnormal{dw}}
\def\Re{\textnormal{Re}}
\def\Im{\textnormal{Im}}
\def\exp{\textnormal{exp}}
\def\interior{\textnormal{interior}}
\def\al{\alpha}
\def\del{\delta}
\def\Del{\Delta}
\def\gam{\gamma}
\def\Gam{\Gamma}
\def\Om{\Omega}
\def\ep{\varepsilon}
\def\lam{\lambda}
\def\rational{{\mathbb Q}}
\def\integer{{\mathbb Z}}
\def\Q{{\mathbb Q}}
\def\Z{{\mathbb Z}}
\def\N{{\mathbb N}}
\def\R{{\mathbb R}}
\def\grad{\nabla}
\def\C{\mathcal C}
\def\P{\mathcal P}
\def\T{\mathcal T}
\def\I{\mathcal I}
\newcommand{\eL}{\mathcal{L}}
\newcommand{\abs}[1]{\left| #1 \right|}
\newcommand{\inner}[1]{\langle #1 \rangle}
\newcommand{\norm}[1]{\left\lVert#1\right\rVert}
\newcommand{\spanvect}{\textnormal{span}}
\newcommand{\union}{\cup}
\newcommand{\Union}{\bigcup}
\newcommand\Mydiv[2]{%
$\strut#1$\kern.25em\smash{\raise.3ex\hbox{$\big)$}}$\mkern-8mu
        \overline{\enspace\strut#2}$}
\def\intersect{\cap}
\def\Intersect{\bigcap}

%%%%%%%%%%%%%%%%%%%%%%%%%%%%%%%%%%%%%%%%%%%  NOTE!!! %%%%%%%%%%%%%%%%%%%%%%%%%%%%%%%%%%%%%%%%%%

% There cannot be any whitespace before a \begin{solution} or \end{solution},
% or else the document will likely fail to compile.

%%%%%%%%%%%%%%%%%%%%%%%%%%%%%%%%%%%%%%%%%%%%%%%%%%%%%%%%%%%%%%%%%%%%%%%%%%%%%%%%%%%%%%%%%%%%%%%

%%% Current list of tags in tags.tex as of August 19, 2021

% Absolute AlgebraicLimits AlternatingSeriesTest AntiDerivative Applications ArcLength Area AverageValue Bijective Bounded Cases ChainRule Challenge      
% Comparison Cosine Concavity Continuity Convergent Critical CurveSketching DefiniteIntegral Definition Derivatives Differentiable Differential DiskWasher Domain Example Exponentials         
% Factorial FirstDerivativeTest Fractions FundamentalTheoremI FundamentalTheoremII Geometry GeometricSequences GeometricSeries Graph HigherDerivatives HorizontalAsymptotes 
% HyperbolicTrig ImplicitDifferentiation ImproperIntegralFinite ImproperIntegralInfinite IncreasingDecreasing IndefiniteIntegral Inequality InfiniteDiscontinuity      
% InflectionPoints IntegralTest Integration IntegrationByParts IntermediateValueTheorem IntervalOfConvergence InverseTrig        
% JumpDiscontinuity L'Hopital Limits Logarithms MacLaurinSeries MaxMin MeanValueTheorem Miscellaneous Non-Convergent    
% Optimization Other p-Series Parametric PartialFractions PartialSums Piecewise Polar Polynomial PowerRule PowerSeries PreCalc ProductRule QuotientRule  
% RadiusOfConvergence Range RationalFunctions RatioTest RelatedRates RemovableDiscontinuity Review RiemannSum RootTest SecondDerivative SecondDerivativeTest          
% Series Sequences Shells SlantAsymptotes SquareRoots SqueezeTheorem Substitution SurfaceArea SumRule Table TangentLines TaylorSeries Theory Trigonometry 
% TrigIntegral TrigSubstitution Unbounded VerticalAsymptotes Velocity Volume WarmUp
% W1 W2 W3 W4 W5 W6 W7 W8 W9 W10 W11 W12 W13 W14 W15 W16 W17 W18 W19 W20 W21 W22 W23    (These are from worksheets)
% HW1 HW2 HW3 HW4 HW5 HW6 HW7 HW8          (These are from homework)
% CQ                                                                     (These are from the challenge questions documents)
% EX1 EX2 EX3 Final                                                      (These are from exams)




%%% Use these to turn off tags, diarys, and solutions
%\excludecomment{tags}
\excludecomment{diary}
\excludecomment{solution}
%%%%%%%%%%%%%%%%%%%%%%%%%%%%%%%%%%%%%%%%%%%%%%%%%%%%%%%%%%%%%%%%%%%%%%%%%%%%%%%%%%%%%%%%%%%%%%%%%%%%


%%%% Use this to include tags/diaries to show questions.
%%%% \matchalltags{tag1, tag2, tag3, ...} shows questions that have all of these tags.
%%%% \matchsometag{tag1, tag2, tag3, ...} shows questions that match at least one of these tags.
%%%% \showalltags shows all questions in the entire document.
%%%% These are mutually exclusive commands, using more than one will likely cause compile errors.

%\matchalltags{Review}
\matchsometags{AntiDerivative, Area}
%\showalltags

%%%%%%%%%%%%%%%%%%%%%%%%%%%%%%%%%%%%%%%%%%%%%%%%%%%%%%%%%%%%%%%%%%%%%%%%%%%%%%%%%%%%%%%%%%%%%%%%%%%%

%%% Document Starts now
\begin{document}


 

\section{Applications}\index{Applications}
\fancyhead[R]{\large Applications}


\begin{tagblock}{Applications, Integration, DefiniteIntegral, Substitution, Area, Graph}
\begin{question}
	


In this worksheet we will look more in depth at \textbf{areas}.

Compute the area given by the definite integral and shade the area on the graph.  Note that you may need to use a $u$-substitution.
\begin{enumerate}
\item $\displaystyle \int_0^1 \sin(\pi x) \, dx$

\includegraphics[width=5cm]{areasinpix.png}

\vspace{1in}


\item $\displaystyle \int_1^2 \frac{x+1}{(x^2+2x)^2} \, dx$

\includegraphics[width=5cm]{area2.png}

\end{enumerate}

\index{Applications}
    
\begin{tags}
       Applications, Integration, DefiniteIntegral, Substitution, Area, Graph
\end{tags}
    
\begin{diary}
        %S2016-HW8-Q5
\end{diary}
	
\begin{solution}

\end{solution}
	
\end{question}

\end{tagblock}

%------------------------------------------------------------------------------------------------------------
\begin{tagblock}{Applications, Integration, DefiniteIntegral, Area, Graph}
\begin{question}
	


So far we have looked at areas under \emph{one} curve, next we'll consider more than one curve.  Recall that the definite integral gives us the area under the graph of $f(x)$ from $x=a$ to $x=b$.  

\[Area = \int_a^b f(x) \, dx\]

%\begin{enumerate}


\item Suppose $f(x) \geq g(x)$ and $f(x)$ and $g(x)$ intersect at the points $(a,f(a)) = (a,g(a))$ and $(b,f(b)) = (b,g(b))$.  

\begin{enumerate}
\item On the graph below on the left, shade the area under $f(x)$ from $x=a$ to $x=b$ and set up an integral which computes the area.

\vspace{.75in}

\item On the graph below in the center, shade the area under $g(x)$ from $x=a$ to $x=b$ and set up an integral which computes the area.

\vspace{.75in}

\item On the graph below on the right, shade the area under enclosed by the two curves and set up an integral which computes the area.
\vspace{1in}

\end{enumerate}


\includegraphics[width=5cm]{area3a.png} \, \includegraphics[width=5cm]{area3a.png} \, \includegraphics[width=5cm]{area3a.png} 

In general, if $f(x) \geq g(x)$ on the interval $[a,b]$, then the area bounded by the curves $f(x)$, $g(x)$ and $x=a$ and $x=b$ is given by the definite integral
\[ \int _a^b f(x) - g(x) \, dx \]
or in other words if we think of $f(x)$ as the ``upper function'' and $g(x)$ as the ``lower function'' then the area is given by 
\[ \int _a^b \text{upper} - \text{lower} \, dx \]

A slightly different perspective is also helpful here: if we take the region between two curves and slice it up into thin vertical rectangles (in the same spirit as we originally sliced the region between a single curve and the $x$-axis), then we see that the height of a typical rectangle is given by the difference between the two functions. For example, the rectangle shown at below has height $f(x) - g(x)$.  


\[\includegraphics[width=5cm]{area3rectangle.png} \]

\index{Applications}
    
\begin{tags}
       Applications, Integration, DefiniteIntegral, Area, Graph
\end{tags}
    
\begin{diary}
        %S2016-HW8-Q5
\end{diary}
	
\begin{solution}

\end{solution}
	
\end{question}

\end{tagblock}

%------------------------------------------------------------------------------------------------------------
\begin{tagblock}{Applications, Integration, DefiniteIntegral, Area, Graph}
\begin{question}
	


Find the intersection points of the curves  $y=x^2$ and $y= \sqrt{x}$, label them on the graph, and compute the area between $y=x^2$ and $y= \sqrt{x}$, shaded below.  

\includegraphics[width=5cm]{area4.png} 

\index{Applications}
    
\begin{tags}
       Applications, Integration, DefiniteIntegral, Area, Graph
\end{tags}
    
\begin{diary}
        %S2016-HW8-Q5
\end{diary}
	
\begin{solution}

\end{solution}
	
\end{question}

\end{tagblock}

%------------------------------------------------------------------------------------------------------------
\begin{tagblock}{Applications, Integration, DefiniteIntegral, Area, Graph}
\begin{question}
	


Consider the curves  $x=y^2-4y$ and $x=2y-y^2$, graphed below.  
\begin{enumerate}
\item Find the intersection points, both the $x$ and $y$ coordinate, of the two curves, and label the points on the graph.

\includegraphics[width=6cm]{areahorizslice.png} 


\item Is there a clear upper and lower function?

\item Instead of working in $x$ and slicing up the area into thin \emph{vertical} rectangles, we will work in $y$ and slice up the area into thin \emph{horizontal} rectangles.  For such a horizontal rectangle, note that its width depends on $y$, the height at which the rectangle is constructed.  What is the width of the rectangle?

\includegraphics[width=6cm]{areahorizslicerectangle.png} 
 
  In general, if there is a clear ``right'' and ``left" function, then we can compute the area between the curves by working in $y$
\[Area = \int_{y=c}^d \text{right} - \text{left} \, dy \]
Note that our endpoints are $y$ values, and are ``right'' and ``left" functions are functions in $y$.

\item Set up an integral that computes the area bounded by our curves and evaluate it to find the area.  

\end{enumerate}

\index{Applications}
    
\begin{tags}
       Applications, Integration, DefiniteIntegral, Area, Graph
\end{tags}
    
\begin{diary}
        %S2016-HW8-Q5
\end{diary}
	
\begin{solution}

\end{solution}
	
\end{question}

\end{tagblock}

%------------------------------------------------------------------------------------------------------------
\begin{tagblock}{Applications, Integration, DefiniteIntegral, Area, Graph}
\begin{question}
	


Compute the area between the curves $x=y^2-1$ and $y=x-1$.  Start by finding the intersection points and determining if you want to work in $x$ or $y$.  

\includegraphics[width=5cm]{area5.png} 

\index{Applications}
    
\begin{tags}
       Applications, Integration, DefiniteIntegral, Area, Graph
\end{tags}
    
\begin{diary}
        %S2016-HW8-Q5
\end{diary}
	
\begin{solution}

\end{solution}
	
\end{question}

\end{tagblock}

%------------------------------------------------------------------------------------------------------------
\begin{tagblock}{Applications, Integration, DefiniteIntegral, Area, Graph}
\begin{question}
	


Compute the area between the curves $y=|x|$ and $y=x^2-2$.   Start by finding the intersection points and determining if you want to work in $x$ or $y$.  Note the graph is symmetric about the $y$-axis.

\includegraphics[width=5cm]{area6.png} 

\index{Applications}
    
\begin{tags}
       Applications, Integration, DefiniteIntegral, Area, Graph
\end{tags}
    
\begin{diary}
        %S2016-HW8-Q5
\end{diary}
	
\begin{solution}

\end{solution}
	
\end{question}

\end{tagblock}

%------------------------------------------------------------------------------------------------------------
\begin{tagblock}{Applications, Integration, DefiniteIntegral, Area, Graph}
\begin{question}
	


Compute the area between the curves $y=|x|$ and $y=x^2-2$.   Start by finding the intersection points and determining if you want to work in $x$ or $y$.  Note the graph is symmetric about the $y$-axis.

\includegraphics[width=5cm]{area6.png} 

\index{Applications}
    
\begin{tags}
       Applications, Integration, DefiniteIntegral, Area, Graph
\end{tags}
    
\begin{diary}
        %S2016-HW8-Q5
\end{diary}
	
\begin{solution}

\end{solution}
	
\end{question}

\end{tagblock}

%------------------------------------------------------------------------------------------------------------
\begin{tagblock}{Applications, Integration, DefiniteIntegral, Volume, Graph, Definition, DiskWasher}
\begin{question}
	


\textbf{The Disk Method:} If $y=r(x)$ is a nonnegative continuous function on $[a,b]$, then the volume of the solid of revolution generated by revolving the curve about the 
$x$-axis over this interval is given by
\[ \int_a^b \pi r(x)^2 \, dx \]

Consider the curve $y=4-x^2$.  We'll find the volume of the solid of revolution generated when the region $R$ bounded by $y=4-x^2$ and the $x$-axis is revolved around the $x$-axis.  
\begin{enumerate}
\item Below is a graph of $y=4-x^2$.  Shade the region which is $R$.  Where does the curve $y=4-x^2$ intersect the $x$-axis?

\includegraphics[width=4cm]{diskR.png}

\item Set up the integral which computes the volume of our solid of revolution.   \textbf{Make sure you have all the proper notation: integral with endpoints, $dx$, etc.}


\includegraphics[width=5cm]{diskex.png}

\item Evaluate the integral to compute the volume.


\end{enumerate}

\index{Applications}
    
\begin{tags}
       Applications, Integration, DefiniteIntegral, Volume, Graph, Definition, DiskWasher
\end{tags}
    
\begin{diary}
        %COPYRIGHTISSUE?
\end{diary}
	
\begin{solution}

\end{solution}
	
\end{question}

\end{tagblock}

%------------------------------------------------------------------------------------------------------------
\begin{tagblock}{Applications, Integration, DefiniteIntegral, Volume, Graph, Definition, DiskWasher}
\begin{question}
	


We will find the volume of the solid of revolution generated when the finite region $R$ that lies between $y=4-x^2$ and $y=x+2$ is revolved about the $x$-axis.
\begin{enumerate}
\item Below is a graph of our two curves.  Shade the region which is $R$ and find the intersection points of the two curves.

\includegraphics[width=4cm]{washerR.png}

\item When we take the region $R$ that lies between the curves and revolve it about the $x$-axis, we get the three-dimensional solid pictured below.  Notice that we have a cone cut out of the solid from the previous problem.  What shape is our slice in this example?  Is it a circular disk or something else?

\includegraphics[width=5cm]{washerex.png}

\item We call this shape a \emph{washer}.  Notice it is one circle cut out of another, so we have an \emph{outer radius}, $R(x)$, and an \emph{inner radius}, $r(x)$.  In our example which curve will give the outer radius and which curve will give the inner radius? Use this to find the area of our washer.  

\includegraphics[width=4cm]{washerradius.png}


\end{enumerate}

Adding up all these slices as before we get

\textbf{The Washer Method for rotating about the $x$-axis:}
If $R(x)$ gives the outer radius and $r(x)$ gives the inner radius for all $x$ in $[a,b]$, then the volume of the solid of revolution generated by revolving the region between them about the $x$-axis over this interval is given by
\[V = \int_a^b \pi(R(x))^2 - \pi (r(x))^2 \, dx\]
\begin{enumerate}
\item[(d)]  Set up an integral which computes the volume of our  solid of revolution.   \textbf{Make sure you have all the proper notation: endpoints, $dx$, etc.}   (No need to evaluate it this time)
\end{enumerate}



\index{Applications}
    
\begin{tags}
       Applications, Integration, DefiniteIntegral, Volume, Graph, Definition, DiskWasher
\end{tags}
    
\begin{diary}
        %COPYRIGHTISSUE?
\end{diary}
	
\begin{solution}

\end{solution}
	
\end{question}

\end{tagblock}

%------------------------------------------------------------------------------------------------------------
\begin{tagblock}{Applications, Integration, DefiniteIntegral, Volume, Graph, Definition, DiskWasher}
\begin{question}
	


We will find the volume of the solid of revolution generated when the finite region $R$ that lies between $y=4-x^2$ and $y=x+2$ is revolved about the $x$-axis.
\begin{enumerate}
\item Below is a graph of our two curves.  Shade the region which is $R$ and find the intersection points of the two curves.

\includegraphics[width=4cm]{washerR.png}

\item When we take the region $R$ that lies between the curves and revolve it about the $x$-axis, we get the three-dimensional solid pictured below.  Notice that we have a cone cut out of the solid from the previous problem.  What shape is our slice in this example?  Is it a circular disk or something else?

\includegraphics[width=5cm]{washerex.png}

\item We call this shape a \emph{washer}.  Notice it is one circle cut out of another, so we have an \emph{outer radius}, $R(x)$, and an \emph{inner radius}, $r(x)$.  In our example which curve will give the outer radius and which curve will give the inner radius? Use this to find the area of our washer.  

\includegraphics[width=4cm]{washerradius.png}


\end{enumerate}

Adding up all these slices as before we get

\textbf{The Washer Method for rotating about the $x$-axis:}
If $R(x)$ gives the outer radius and $r(x)$ gives the inner radius for all $x$ in $[a,b]$, then the volume of the solid of revolution generated by revolving the region between them about the $x$-axis over this interval is given by
\[V = \int_a^b \pi(R(x))^2 - \pi (r(x))^2 \, dx\]
\begin{enumerate}
\item[(d)]  Set up an integral which computes the volume of our  solid of revolution.   \textbf{Make sure you have all the proper notation: endpoints, $dx$, etc.}   (No need to evaluate it this time)
\end{enumerate}



\index{Applications}
    
\begin{tags}
       Applications, Integration, DefiniteIntegral, Volume, Graph, Definition, DiskWasher
\end{tags}
    
\begin{diary}
        %COPYRIGHTISSUE?
\end{diary}
	
\begin{solution}

\end{solution}
	
\end{question}

\end{tagblock}

%------------------------------------------------------------------------------------------------------------
\begin{tagblock}{Applications, Integration, DefiniteIntegral, Volume, Graph, Definition, DiskWasher}
\begin{question}
	


Let  $R$ be the region bounded by $y=\sqrt{x}$ and $y=x^4$.  We will find the volume of the  solid of revolution generated when $R$ is revolved about the $y$-axis.
\begin{enumerate}
\item Below is a graph of our two curves.  Shade the region which is $R$ and find the intersection points of the two curves.

\includegraphics[width=4cm]{washerinyR.png}

\item When we take the region $R$ that lies between the curves and revolve it about the $y$-axis, we get the three-dimensional solid pictured below.   Instead of slicing vertically, we will slice horizontally, so that we get a washer shape.  This means we will need to work in $y$.  Find equations $R(y)$ and $r(y)$.  


\[\includegraphics[width=4cm]{washerinyex.png} \hspace{1in} \includegraphics[width=4cm]{washerinyradius.png} \]

\end{enumerate}

\bigskip

Adding up all these slices as before we get

\textbf{The Washer Method for rotating about the $y$-axis :}  If $R(y)$ gives the outer radius and $r(y)$ gives the inner radius, then the  volume of the solid of revolution generated by revolving the region between them about the $y$-axis is given by
\[V = \int_{y=c}^d \pi(R(y))^2 - \pi (r(y))^2 \, dy\]
\begin{enumerate}

\item[(c)] Set up an integral which  computes the volume of our  solid of revolution.  \textbf{Make sure you have all the proper notation: endpoints, $dx$ or $dy$, etc.}  Evaluate the integral to compute the volume.  
\end{enumerate}



\index{Applications}
    
\begin{tags}
       Applications, Integration, DefiniteIntegral, Volume, Graph, Definition, DiskWasher
\end{tags}
    
\begin{diary}
        %COPYRIGHTISSUE?
\end{diary}
	
\begin{solution}

\end{solution}
	
\end{question}

\end{tagblock}

%------------------------------------------------------------------------------------------------------------
\begin{tagblock}{Applications, Integration, DefiniteIntegral, Volume, Graph, Definition, DiskWasher}
\begin{question}
	


So far we've looked at rotating about the $x$-axis ($y=0$ - a horizontal line) and the $y$-axis ($x=0$ - a vertical line).  What if we want to rotate about another horizontal or vertical line?


\bigskip



Let   $R$ be the region bounded by $y=x^2$ and $y=x$.  We will find the volume of the  solid of revolution generated when $R$ is revolved about the horizontal line $y=-1$.
\begin{enumerate}
\item  Below is a graph of our two curves.  Shade the region which is $R$ and find the intersection points of the two curves.  The dotted line is $y=-1$, which curve is further from the line $y=-1$?

\includegraphics[width=4cm]{washerhorizR.png}

\item  When we take the region $R$ that lies between the curves and revolve it about the line $y=-1$, we get the three-dimensional solid pictured below.  We again will have a washer with an outer radius $R(x)$ and an inner radius $r(x)$.  Find $R(x)$ and $r(x)$.  (Note your radius must go all the way to the line $y=-1$).

\includegraphics[width=5cm]{washerhorizex.png}
\item Set up an integral which  computes the volume of our  solid of revolution.   \textbf{Make sure you have all the proper notation.} (No need to evaluate it this time)


\end{enumerate}



\index{Applications}
    
\begin{tags}
       Applications, Integration, DefiniteIntegral, Volume, Graph, Definition, DiskWasher
\end{tags}
    
\begin{diary}
        %COPYRIGHTISSUE?
\end{diary}
	
\begin{solution}

\end{solution}
	
\end{question}

\end{tagblock}

%------------------------------------------------------------------------------------------------------------
\begin{tagblock}{Applications, Integration, DefiniteIntegral, Volume, Graph, DiskWasher}
\begin{question}
	


Let  $R$ be the region bounded by $y=2x^2$ and $y=2x$.  We will find the volume of the  solid of revolution generated when $R$ is revolved about the vertical line $x=2$.
\begin{enumerate}
\item Below is a graph of our two curves and the line $x=2$.  Note the curves intersect at $(0,0)$ and $(1,2)$.  Do we need to work in $x$ or in $y$?  If we need to work in $y$, rewrite the equations of our curves in terms of $x=g(y)$.  

\includegraphics[width=6cm]{washerinywarmup.png}

\item  Which curve is further from the line $x=2$?  Use this to determine the outer radius $R(y)$ and the inner radius $r(y)$:
\[ R(y) = \hspace{2in} r(y) = \]

\item Set up the integral which gives the volume:

\end{enumerate}



\index{Applications}
    
\begin{tags}
       Applications, Integration, DefiniteIntegral, Volume, Graph, DiskWasher
\end{tags}
    
\begin{diary}
        %COPYRIGHTISSUE?
\end{diary}
	
\begin{solution}

\end{solution}
	
\end{question}

\end{tagblock}

%------------------------------------------------------------------------------------------------------------
\begin{tagblock}{Applications, Integration, DefiniteIntegral, Volume, Graph, Shell}
\begin{question}
	


Consider the region $R$ bounded by $f(x) = -12+8x-x^2$ and the $x$-axis.  We want to find the volume of the  solid of revolution generated when $R$ is revolved about the $y$-axis.   We have a vertical rotation, so to use the disk washer method we need to work in $y$.  Can you rewrite our curve $y = -12+8x-x^2$ as $x=g(y)$?  

\includegraphics[width=6cm]{ShellintroR.png}
 

\emph{We would like then a different method, in which we can have a vertical rotation, but still work in $x$.  This is what we will call the \textbf{Shell Method.}}




\index{Applications}
    
\begin{tags}
       Applications, Integration, DefiniteIntegral, Volume, Graph, Shell
\end{tags}
    
\begin{diary}
        %COPYRIGHTISSUE?
\end{diary}
	
\begin{solution}

\end{solution}
	
\end{question}

\end{tagblock}

%------------------------------------------------------------------------------------------------------------
\begin{tagblock}{Applications, Integration, DefiniteIntegral, Volume, Graph, Shell, Definition}
\begin{question}
	


\textbf{The Shell Method:} If $y=f(x)$ is a nonnegative continuous function on $[a,b]$, then the volume of the solid of revolution generated by revolving the curve about the 
$y$-axis over this interval is given by
\[ V = \int_a^b 2\pi x f(x) \, dx \]
where $2\pi x = \text{circumference}, f(x) = \text{ height, and } dx = \text{thickness}. $

Let  $R$ be the region bounded by $y=3x^2 - x^3$ and the $x$-axis.   We will find the  the volume of the  solid of revolution generated when $R$ is revolved about the $y$-axis. 

\includegraphics[width=5cm]{ShellR1.png}

\begin{enumerate}
\item Using the Shell Method set up an integral that computes the volume.  On the graph above draw a typical slab.  

\vspace{1in}

\item Evaluate the integral to compute the volume.  
\vspace{2in}

\end{enumerate}




\index{Applications}
    
\begin{tags}
       Applications, Integration, DefiniteIntegral, Volume, Graph, Shell, Definition
\end{tags}
    
\begin{diary}
        
\end{diary}
	
\begin{solution}

\end{solution}
	
\end{question}

\end{tagblock}

%------------------------------------------------------------------------------------------------------------
\begin{tagblock}{Applications, Integration, DefiniteIntegral, Volume, Graph, Shell}
\begin{question}
	


As we did earlier, we can consider regions $R$, where $R$ is bounded by two curves.  Let $R$ be the region bounded by  $y=2x^2$ and $y=2x$.  We will find the volume of the  solid of revolution generated when $R$ is revolved about the $y$-axis. 

\begin{minipage}{.4\textwidth} \includegraphics[width=5cm]{ShellR2.png} \end{minipage}%
 \begin{minipage}{.6\textwidth}
\begin{enumerate}
\item Draw a slab on the graph above.  What is the height of that slab?  

\vspace{.5in}
\item Using the Shell Method set up an integral that computes the volume. 
\end{enumerate}

\end{minipage}




\index{Applications}
    
\begin{tags}
       Applications, Integration, DefiniteIntegral, Volume, Graph, Shell
\end{tags}
    
\begin{diary}
        
\end{diary}
	
\begin{solution}

\end{solution}
	
\end{question}

\end{tagblock}

%------------------------------------------------------------------------------------------------------------
\begin{tagblock}{Applications, Integration, DefiniteIntegral, Volume, Graph, Shell}
\begin{question}
	


We also could rotate about a different vertical line.  This will cause the \emph{radius} to change, so our circumference will no longer be just $2\pi x$.  Let $R$ be as in the previous problem.  We will find the volume of the  solid of revolution generated when $R$ is revolved about the vertical line $x=2$.  

\begin{minipage}{.4\textwidth} \includegraphics[width=5cm]{ShellR3.png} \end{minipage}%
 \begin{minipage}{.6\textwidth}
\begin{enumerate}
\item Draw a slab on the graph above.  What is the height of that slab?  

\vspace{.5in}
\item If we rotate that slab around the line $x=2$ what will be the radius?  Use this to find the circumference of the shell.  
\end{enumerate}
\end{minipage}

\begin{enumerate}
 \item[(c)] Using the Shell Method set up an integral that computes the volume.   
\end{enumerate}



\index{Applications}
    
\begin{tags}
       Applications, Integration, DefiniteIntegral, Volume, Graph, Shell
\end{tags}
    
\begin{diary}
        
\end{diary}
	
\begin{solution}

\end{solution}
	
\end{question}

\end{tagblock}

%------------------------------------------------------------------------------------------------------------
\begin{tagblock}{Applications, Integration, DefiniteIntegral, Volume, Graph, Shell}
\begin{question}
	


We can also use the Shell Method if we have a \emph{horizontal} rotation.  This means that we will have to work in $y$.
 
 \item Let $R$ be the region bounded by $y=2-\sqrt{x}$, the $x$-axis and the $y$-axis.  We will find the volume of the  solid of revolution generated when $R$ is revolved about the $x$-axis. 

\begin{minipage}{.4\textwidth} \includegraphics[width=5cm]{ShellR4.png} \end{minipage}%
 \begin{minipage}{.6\textwidth}
\begin{enumerate}
\item We will break up the $y$-interval $[0,2]$ into $n$ pieces.  This means our slabs will horizontal slabs, not vertical as before.  Draw a slab on the graph.

\item  We then will compute the width of the slab in terms of $y$.  Rewrite $y=2-\sqrt{x}$ to find the width.  
 \end{enumerate}

\end{minipage}

\vspace{.5in}
\begin{enumerate}
\item[(c)] We now will rotate the slab around the $x$-axis.  What will the radius be?  
\vspace{.5in}

\item[(d)] Using the Shell Method set up an integral that computes the volume.   
 \end{enumerate}




\index{Applications}
    
\begin{tags}
       Applications, Integration, DefiniteIntegral, Volume, Graph, Shell
\end{tags}
    
\begin{diary}
        
\end{diary}
	
\begin{solution}

\end{solution}
	
\end{question}

\end{tagblock}

%------------------------------------------------------------------------------------------------------------

\begin{tagblock}{Applications, Integration, DefiniteIntegral, Volume, Graph, DiskWasher, Shell}
\begin{question}
	


We've now learned two different methods for computing the volumes.   Sometimes it is easier to use one method over the other, in either case we need to first determine which variable to integrate in:

\begin{center}
\begin{tabular}{ |c| c | c|} \hline
& Integrate in $x$ & Integrate in $y$ \\ \hline
Rotate Horizontally & disk/washer & shell \\ \hline
Rotate Vertically & shell & disk/washer \\ \hline 
\end{tabular}
\end{center}

\textbf{Question: How do we know which method to work in?}

This is a matter of preference, but I tend to do the following:  I always like to work in $x$, so I use the Disk/Washer Method for horizontal rotations and the Shell Method for vertical rotations.

\bigskip

Let $R$ is the region bounded by the curves $f(x) =\sqrt{x}$ and $g(x) = 8x^2$. The curves intersect at the origin at $(\frac{1}{4}, \frac{1}{2})$.
\begin{enumerate}
\item Set up an integral to compute the volume of the solid obtained by rotating $R$ about the line $y=1$.  Did you use the \textbf{Disk/Washer Method} or the \textbf{Shell Method}?

\includegraphics[width=6cm]{Volume5a.jpg}

\vspace{1in}
\item Set up an integral to compute the volume of the solid obtained by rotating $R$ about the line $x=.5$.  Did you use the \textbf{Disk/Washer Method} or the \textbf{Shell Method}?

\includegraphics[width=6cm]{Volume5b.jpg}

\vspace{1in}
\end{enumerate}




\index{Applications}
    
\begin{tags}
       Applications, Integration, DefiniteIntegral, Volume, Graph, DiskWasher, Shell
\end{tags}
    
\begin{diary}
        
\end{diary}
	
\begin{solution}

\end{solution}
	
\end{question}

\end{tagblock}

%------------------------------------------------------------------------------------------------------------
\begin{tagblock}{Applications, Integration, DefiniteIntegral, Volume, Graph, DiskWasher, Shell}
\begin{question}
	


In the problems below $R$ is the region bounded by the curves $f(x) =\sqrt{x}$ and $g(x) = 8x^2$.  The curves intersect at the origin at $(\frac{1}{4}, \frac{1}{2})$.  

%\noindent In today's lab, we'll work with \emph{Mathematica} to get nice graphs of surfaces of revolutions.  You will then set up the integrals that compute the volumes of the various surfaces.  Start by opening up the \emph{Mathematica} notebook for the Volume Lab on Canvas.  Working through the \emph{Mathematica} notebook, you will create various surfaces of revolution.  At various points in the notebook, you will be directed to do a problem on this worksheet.  You will just need to turn in this worksheet, and show me that you did create the pictures on \emph{Mathematica}.  We will continue with this worksheet on Monday, but you should get through problems 1 and 2 on Friday.  

\begin{figure}[h]
\centering
\includegraphics[width=5cm]{volumelab1.jpg}
\end{figure}


%In the problems below $R$ is the region bounded by the curves $f(x) =\sqrt{x}$ and $g(x) = 8x^2$, as in the \emph{Mathematica} notebook.


\begin{enumerate}
\item  We first will rotate $R$ about the $x$-axis (a horizontal rotation)

\begin{enumerate}
\item Set up an integral to compute the volume of the solid obtained by rotating about the $x$-axis using the \textbf{disk/washer method}.
\vspace{1in}
\item Set up an integral to compute the volume of the solid obtained by rotating about the $x$-axis using the \textbf{shell method}.
\vspace{.8in}
\end{enumerate}

\item  Next we will rotate $R$ about the $y$-axis (a vertical rotation)

\begin{enumerate}
\item Set up an integral to compute the volume of the solid obtained by rotating about the $y$-axis using the \textbf{disk/washer method}.
\vspace{1in}
\item Set up an integral to compute the volume of the solid obtained by rotating about the $y$-axis using the \textbf{shell method}.
\vspace{.8in}
\end{enumerate}




\item Next we will  rotate $R$ about the horizontal line $y=1$. 

%\begin{enumerate}
 Set up an integral to compute the volume of the solid obtained by rotating about the line $y=1$ using either the \textbf{disk/washer method} or  \textbf{shell method} (\emph{State which method you are using})
\begin{figure}[h]
%\centering
\includegraphics[width=5cm]{volumelab2.jpg}
\end{figure}

\vspace{.8in}
%\item Set up an integral to compute the volume of the solid obtained by rotating about the line $y=1$ using the
%\end{enumerate}

\item Next we will rotate $R$ about the vertical line $x=.5$. 

%\begin{enumerate}
 Set up an integral to compute the volume of the solid obtained by rotating about the line $x=.5$ using either the \textbf{disk/washer method} or  \textbf{shell method} (\emph{State which method you are using})
\begin{figure}[h]
%\centering
\includegraphics[width=5cm]{volumelab3.jpg}
\end{figure}

Let $R$ be the region bounded by the curves $f(x) =\sqrt{x}$ and $g(x) = 8x^2$.  The curves intersect at the origin at $(\frac{1}{4}, \frac{1}{2})$.   

\begin{figure}[h]
\centering
\includegraphics[width=5cm]{volumelab1.jpg}
\end{figure}




\begin{enumerate}
\item  We first will rotate $R$ about the $x$-axis (a horizontal rotation)

\begin{enumerate}
\item Set up an integral to compute the volume of the solid obtained by rotating about the $x$-axis using the \textbf{disk/washer method}.
\vspace{1in}
\item Set up an integral to compute the volume of the solid obtained by rotating about the $x$-axis using the \textbf{shell method}.
\vspace{.8in}
\end{enumerate}

\item  Next we will rotate $R$ about the $y$-axis (a vertical rotation)

\begin{enumerate}
\item Set up an integral to compute the volume of the solid obtained by rotating about the $y$-axis using the \textbf{disk/washer method}.
\vspace{1in}
\item Set up an integral to compute the volume of the solid obtained by rotating about the $y$-axis using the \textbf{shell method}.
\vspace{.8in}
\end{enumerate}

\bigskip


Next we will  rotate $R$ about the horizontal line $y=1$. 

%\begin{enumerate}
 Set up an integral to compute the volume of the solid obtained by rotating about the line $y=1$ using either the \textbf{disk/washer method} or  \textbf{shell method} (\emph{State which method you are using})
\begin{figure}[h]
%\centering
\includegraphics[width=5cm]{volumelab2.jpg}
\end{figure}

\vspace{.8in}
%\item Set up an integral to compute the volume of the solid obtained by rotating about the line $y=1$ using the
%\end{enumerate}

\item Next we will rotate $R$ about the vertical line $x=.5$. 

%\begin{enumerate}
 Set up an integral to compute the volume of the solid obtained by rotating about the line $x=.5$ using either the \textbf{disk/washer method} or  \textbf{shell method} (\emph{State which method you are using})
\begin{figure}[h]
%\centering
\includegraphics[width=5cm]{volumelab3.jpg}
\end{figure}

%\item Set up an integral to compute the volume of the solid obtained by rotating about the line $x=.5$ using the \textbf{shell method}.


\end{enumerate}

\end{enumerate}


\index{Applications}
    
\begin{tags}
       Applications, Integration, DefiniteIntegral, Volume, Graph, DiskWasher, Shell
\end{tags}
    
\begin{diary}
        
\end{diary}
	
\begin{solution}

\end{solution}
	
\end{question}

\end{tagblock}

%------------------------------------------------------------------------------------------------------------
\begin{tagblock}{W1, Applications, Integration}
\begin{question}
	
 Dearth Vater and Double Zero come upon a recently deceased body in their search for Duke Cloudstrider. They take it to Dr. Ephra, who measures the temperature of the corpse and explains that the difference between the temperature $f(t)$ and the ambient temperature $T$ is proportional to its instantaneous rate of change of $f$. 

\bigskip

a) What does it mean for two functions to be proportional to eachother? 

\bigskip

b) Write down an equation that describes what it means for $f(t)-T$ to be proportional to the instantaneous rate of change of $f$. This is a \textit{differential equation}.

\bigskip

c) Suppose the ambient temperature in part b) is  $T=0^{\circ}$ Celsius. Do you know any functions that satisfy the equation this equation? If so, write them down and solve the equation formally. If not, that's what this class is for!

\index{Applications}
    
\begin{tags}
        Applications, Integration
\end{tags}
    
\begin{diary}
        %S2016-HW8-Q5
\end{diary}
	
\begin{solution}

\end{solution}
	
\end{question}

\end{tagblock}

%------------------------------------------------------------------------------------------------------------

\begin{tagblock}{W1, Applications, Integration}
\begin{question}
	
Now we get to what are called \emph{initial} or \emph{final} conditions. An initial condition just means that on a certain time interval, you know what happens at the beginning of the interval, and a final condition means you know something at the end of the interval. 

\bigskip

a) If Dr. Ephra measures the temperature at 88$^\circ$ F at 7:00 AM, how could you find out when time of death was? Is this an initial or final condition? 


\bigskip

b) Without knowing the constant of proportionality, suppose I tell you that a solution to the problem is of the form $Ce^{\alpha t}$ for some constants $\alpha$ and $C$. Solve for $f$ as far as you can. 

\bigskip

c) Suppose the constant of proportionality is $k=-.5$. Solve the problem.

\index{Applications}
    
\begin{tags}
        Applications, Integration
\end{tags}
    
\begin{diary}
        %S2016-HW8-Q5
\end{diary}
	
\begin{solution}

\end{solution}
	
\end{question}

\end{tagblock}

%------------------------------------------------------------------------------------------------------------

\begin{tagblock}{W3, Applications, Integration, Logarithms}
\begin{question}
	
The full differential equation for Newton's Law of Cooling is
\[
\frac {df}{dt}=-k(f(t)-T)
\]
where $f$ is the temperature at time $t$ and $T$ is the (assumed constant) ambient temperature. Let's say $t$ is measured in hours. Let's say $T$ is 68$^{\circ}$ farenheit. Dr. Ephra measures the temperature of the corpse at 80$^{\circ}$ farenheit at 5:00 in the morning and 79$^{\circ}$ farenheit at 7:00 in the morning.

\bigskip

a) Find $k$.

\bigskip

b) What was the time of death?

\index{Applications}
    
\begin{tags}
        Applications, Integration, Logarithms, W3
\end{tags}
    
\begin{diary}
        %S2016-HW8-Q5
\end{diary}
	
\begin{solution}

\end{solution}
	
\end{question}

\end{tagblock}

%------------------------------------------------------------------------------------------------------------

\begin{tagblock}{W4, Applications, Integration, Logarithms, Exponentials}
\begin{question}
	
 Small John's Funk Juice, containing .2kg of salt per liter, flows into a tank initially filled with 1000L of water containing 30 kg of salt. The Juice enters the tank at 6 L/min, flows out at the same rate, and the mixture is kept uniform by stirring. Call $s(t)$ the amount of salt in the tank at time $t$ 

\bigskip



\bigskip

a) At what rate does the salt enter the tank, in kg/min? At what rate does it exit the tank?

\bigskip

b) Use part a) to find a differential equation involving $s$, then solve the equation for $s(t)$. 

\bigskip

c) What should $s(0)$ be equal to?

\bigskip

d) How much salt is in the tank after 10 minutes? 

\index{Applications}
    
\begin{tags}
        W4, Applications, Integration, Logarithms, Exponentials
\end{tags}
    
\begin{diary}
        %S2016-HW8-Q5
\end{diary}
	
\begin{solution}

\end{solution}
	
\end{question}

\end{tagblock}

%------------------------------------------------------------------------------------------------------------

\begin{tagblock}{W4, Applications, Integration, Logarithms, Exponentials}
\begin{question}
	
 After 10 minutes, Small John decides his Juice could use a little more salt. He switches off the initial valve, which we will call valve A, and turns on valve B, which again delivers liquid at the rate of 6L/min, but now contains .4 kg/L of salt. Liquid still exits the tank at 6L/min. Only one of valves A and B can be open at any given time. 

At 6AM, valve A is open. At 6:10AM, valve A is closed and valve B is opened. 

\bigskip

Using the fact that $\dfrac{ds}{dt}=$(rate in)-(rate out), find an equation for $\dfrac{ds}{dt}$. You should get a piecewise-defined function. Don't solve for $s(t)$ unless you are a masochist.

\index{Applications}
    
\begin{tags}
        W4, Applications, Integration, Logarithms, Exponentials
\end{tags}
    
\begin{diary}
        %S2016-HW8-Q5
\end{diary}
	
\begin{solution}

\end{solution}
	
\end{question}

\end{tagblock}

%------------------------------------------------------------------------------------------------------------

\begin{tagblock}{W5, Applications, Integration, WarmUp}
\begin{question}
	
Recall the following problem: A tank containing 1000L of water, in which 30kg of salt delivering Small John's Funk Juice are dissolved, has two input valves, valve A and valve B. Only one valve can be open at any given time. Both valves deliver liquid at a rate of 6L/min, but the solution from valve A contains .2 kg of salt per liter and the solution in valve B contains .4 kg of salt per liter. Liquid exits the tank at 6L/min.

At 6AM, valve A is open. At 6:10AM, valve A is closed and valve B is opened. The problem is modeled by the differential equation
\begin{equation}\label{brine}
\frac{dx}{dt}=6h(t)-\frac{3x(t)}{500}
\end{equation}
where
\[
h(t)=\begin{cases} 1.2\textrm{kg} & 0\leq t<10 \\ 2.4\textrm{kg} & t\geq 10 \end{cases}
\]


\bigskip

The trick to ``simplify" this problem has three parts, and the first is absolute genius. What we want to do is to convert the differential equation we found in c) into an \textit{algebraic} equation, like a linear or quadratic equation, and then by solving the equation, somehow solve the differential equation. Sounds fun, right? It'll look like we're switching gears for a bit, but always have this problem in the back of your head. There's a TON of digression. 

\index{Applications}
    
\begin{tags}
        W5, Applications, Integration, WarmUp
\end{tags}
    
\begin{diary}
        %S2016-HW8-Q5
\end{diary}
	
\begin{solution}

\end{solution}
	
\end{question}

\end{tagblock}

%------------------------------------------------------------------------------------------------------------
\begin{tagblock}{Integration, W8, PartialFractions. ImproperIntegralInfinite, Applications}
\begin{question}
Check the following facts, then solve the two-valve tank problem given by the differential equation
\begin{equation}\label{brine}
\frac{dx}{dt}=6h(t)-\frac{3x(t)}{500}
\end{equation}
where
\[
h(t)=\begin{cases} 1.2\textrm{kg} & 0\leq t<10 \\ 2.4\textrm{kg} & t\geq 10 \end{cases}
\] 
by putting them all together. You can do this by assuming that, for each function $g$, there is a unique function $f$ on $[0,\infty)$ whose Laplace Transform is $g$. 

\bigskip

a) (Fact 1) If $s>-a$, show that $\displaystyle\eL\{e^{-at}\}(s)=\frac 1 {a+s}$. 

\bigskip

b) If we let
\[
h(t)=\begin{cases} 1 & t\geq 0 \\ 0 & t<0\end{cases},
\]
show $\displaystyle\eL\{h(t-a)\}(s)=\frac {e^{-as}} s$ for $s>a$. 

\bigskip

c) (Fact 2) If $h$ is the function from part b), check that 
\[
\eL\{f(t-a)h(t-a)\}(s)=e^{-as}\eL\{f\}(s)
\]
for $s>a$. 

\bigskip
	
\index{Integration}
	
\begin{tags}
	   Integration, W8, PartialFractions, Applications
\end{tags}
	
\begin{diary}
	   % S2016-HW1-Q1
\end{diary}
	
\begin{solution}
	   
	    \end{enumerate}
\end{solution}
	
\end{question}

\end{tagblock}

%-------------------------------------------------------------------------------------------------------------

\begin{tagblock}{W9, Applications, Integration, PartialFractions, WarmUp}
\begin{question}
	


Newton's Law of Cooling is inaccurate when applied to heat from thermal radiation. Stefan's law of radiation states that the change in temperature of a body at $T(t)$ kelvins in a medium $M(t)$ kelvins is proportional to $M^4-T^4$; that is,
\begin{equation}\label{Stef}
\frac{dT}{dt}=K(M(t)^4-T(t)^4)
\end{equation}
where $K$ is again a constant. Let $K=40^{-4}$ and assume that the median temperature is constant, $M(t)\equiv 70$ kelvins. Let $T(0)=100$ kelvins.

To solve equation \eqref{Stef}, you'd really like to just integrate both sides, but since $T$ is the unknown, we don't have any idea what its integral is. 

\bigskip

a) Divide both sides of equation \eqref{Stef} by $M(t)^4-T(t)^4$. 
 
\bigskip

b) Make the substitution $u=T(t)$ on the left-hand side of equation \eqref{Stef} and factor the denominator. 

\bigskip

c) Can you use partial fractions on part b)? Why or why not? 

\index{Applications}
    
\begin{tags}
       W9, Applications, Integration, PartialFractions, WarmUp
\end{tags}
    
\begin{diary}
        %S2016-HW8-Q5
\end{diary}
	
\begin{solution}

\end{solution}
	
\end{question}

\end{tagblock}

%------------------------------------------------------------------------------------------------------------
\begin{tagblock}{W9, Applications, Integration, PartialFractions, Example}
\begin{question}
	


Using the subsititution $u=T(t)$, the denominator in Stefan's Law of Cooling,
\begin{equation}\label{Stef}
\frac 1 {M(t)^4-T(t)^4}\frac{dT}{dt}=K
\end{equation}
factors as $(M^2+u^2)(M-u)(M+u)$, and we'd really like to use partial fractions to split this into manageable pieces. Unfortunately, this isn't as simple as the case where the denominator has distinct linear factors. Let's look at an example:

Given $\displaystyle\frac 1 {s^3+s}$, we may factor the denominator as $s(s^2+1)$. We can't factor $s^2+1$ any further using polynomials with real coefficients. In this version of partial fractions, there are numbers $A,B,$ and $C$ with
\[
\frac 1 {s(s^2+1)}=\frac A s + \frac {Cs+B}{s^2+1}.
\]
Again multiplying both sides by the denominator on the left-hand side of the equation, we get
\[
1=A(s^2+1)+(Cs+B)s.
\]
Following the second method from the last worksheet, which is more efficient since $s^2+1$ doesn't factor, we get
\[
1=As^2+A+Cs^2+Bs=(A+C)s^2+Bs+A,
\] 
and so $A=1$, $B=0$, and $A+C=0$ gives $C=-1$ (I tried so very hard not to put the letters ``B" and ``s" together...). We get
\begin{equation}\label{quadpart}
\frac 1 {s(s^2+1)}=\frac 1 s-\frac s {s^2+1}
\end{equation}

\bigskip

a) Check that the coefficients found in equation \eqref{quadpart} are correct by making a common denominator. 

\bigskip

b) Find the partial fraction decomposition for equation \eqref{Stef}.

\index{Applications}
    
\begin{tags}
       W9, Applications, Integration, PartialFractions, Example
\end{tags}
    
\begin{diary}
        %S2016-HW8-Q5
\end{diary}
	
\begin{solution}

\end{solution}
	
\end{question}

\end{tagblock}

%------------------------------------------------------------------------------------------------------------
\begin{tagblock}{W9, Applications, Integration, PartialFractions, TrigSubstitution, Example}
\begin{question}
	


Unlike tank problems, we need to integrate a partial fraction decomposition in order to find the solution to Stefan's Law of Cooling
\begin{equation}\label{Stef}
\frac 1 {M^4-u^4}=K
\end{equation}
Two of the summands aren't bad (why?), but we haven't seen anything like the integral of $\dfrac 1 {M^2+u^2}$. The desired technique is \textit{trig substitution}. It's really just substitution in disguise using ``inverse" functions, but we'll get to that. Let's start with the example
\begin{equation}\label{trigint}
\int \frac 1 {s^2+1} \ ds.
\end{equation}
Bowing to convention, let's let $s=\tan(\theta)$. Then $ds=\sec^2(\theta) \ d\theta$, and so substituting into equation \eqref{trigint}, we get
\[
\int \frac 1 {\tan^2(\theta)+1} \ \sec^2(\theta) \ d\theta.
\]
We can seize upon the appropriate trig identity $\tan^2(\theta)+1=\sec^2(\theta)$, the integral becomes
\[
\int \frac 1 {\sec^2(\theta)} \ \sec^2(\theta) \ d\theta=\int 1 \ d\theta.
\]
We just lucked into the easiest integral of all time, so integrating, we get 
\[
\int 1 \ d\theta=\theta +C
\]
All that remains is to convert $\theta$ into an $s$-valued function. We know $\tan(\theta)=s$. Now tangent is not invertible on its domain, but if we restrict its domain to $(-\pi/2,\pi,2)$, it is one-to-one and hence invertible on this restricted domain. The restricted inverse is called $\arctan(s)$, so that
\[
\int \frac 1 {s^2+1} \ ds=\arctan(s)+C.
\]

a) Carry out a similar argument to $\displaystyle\int \frac 1 {x^2+1} \ dx$ for the integral.
\[
\int \sqrt{1-x^2} \ dx
\]
using sine, restricted to $[-\pi/2,\pi/2]$.



\bigskip

b) Compute $\displaystyle\int_1^{\sqrt{2}}\frac 1 {2x\sqrt{x^2-1}} \ dx$ (use secant, restricted to $[0,\pi/2)\cup (\pi,3\pi/2]$)

\index{Applications}
    
\begin{tags}
       W9, Applications, Integration, PartialFractions, TrigSubstitution, Example
\end{tags}
    
\begin{diary}
        %S2016-HW8-Q5
\end{diary}
	
\begin{solution}

\end{solution}
	
\end{question}

\end{tagblock}

%------------------------------------------------------------------------------------------------------------
\begin{tagblock}{W23, Applications, WarmUp, SurfaceArea, Geometry, Theory}
\begin{question}

Now we move into the third dimension: surface area!

\bigskip

a) Write down every formula you know for the surface area of a 3-D object. If you don't know any, that's fine. 

\bigskip

b) Draw a sufficiently nice curve and approximate the length of the curve by drawing line segments between points on the curve, then revolve the line segment about the $x$-axis. What object do you get, and how would you calculate the surface area of such an object?

\bigskip

c) Suppose you are given that the surface area of a right circular cone with height $h$ and radius $r$ is $\pi r\sqrt{h^2+r^2}$. Does this help with part b)? Why or why not?

\index{Applications}
    
\begin{tags}
        W23, Applications, WarmUp, SurfaceArea, Geometry, Theory
\end{tags}
    
\begin{diary}
        %S2016-HW8-Q5
\end{diary}
	
\begin{solution}

\end{solution}
	
\end{question}

\end{tagblock}

%------------------------------------------------------------------------------------------------------------
\begin{tagblock}{W23, Applications, Definition, SurfaceArea, Parametric, Derivatives, Integration, Trigonometry}
\begin{question}
	
Using approximating objects, called \textit{frustrums}, we can derive a formula for the surface area of a surface obtained by revolving a parameterized curve around the $x$-axis from $t=a$ to $t=b$. The formula is
\[
SA=2\pi\int_a^b|y(t)|\sqrt{(x'(t))^2+(y'(t))^2} \ dt.
\]

Note the similarity to the formula for arc length!

\bigskip

Consider the curve given parametrically by $x(t)=t\sin(t)+\cos(t)$, $y(t)=t\cos(t)-\sin(t)$. 

\bigskip

a)  Set up an integral that represents the surface area obtained by revolving the curve about the $x$-axis from $t=0$ to $t=\pi/4$. 

\bigskip

b) Determine the surface area obtained by revolving the curve about the $x$-axis from $t=0$ to $t=\pi/4$. 

\index{Applications}
    
\begin{tags}
        W23, Applications, Definition, SurfaceArea, Parametric, Derivatives, Integration, Trigonometry
\end{tags}
    
\begin{diary}
        %S2016-HW8-Q5
\end{diary}
	
\begin{solution}

\end{solution}
	
\end{question}

\end{tagblock}

%------------------------------------------------------------------------------------------------------------
\begin{tagblock}{W23, Applications, Definition, SurfaceArea, Parametric, Derivatives, Integration}
\begin{question}
	
There is a similar surface area formula for revolving a parametric curve about the $y$-axis from $t=a$ to $t=b$ given by
\[
SA=2\pi\int_a^b|x(t)|\sqrt{(x'(t))^2+(y'(t))^2} \ dt.
\]

\bigskip

a) From the equation above, deduce a formula for the surface area obtained by revolving the graph of a Cartesian function $y=f(x)$ from $x=a$ to $x=b$. about the $y$-axis.

\bigskip

For the remainder of the problem, let $f(x)=x^2+12$.

\bigskip

b) Set up an integral that represents the surface area obtained by revolving the graph of $f$ from $x=1$ to $x=2$ about the $y$-axis. 

\bigskip

c) Determine the surface area obtained by revolving the graph of $f$ from $x=1$ to $x=2$ about the $y$-axis.

\index{Applications}
    
\begin{tags}
        W23, Applications, Definition, SurfaceArea, Parametric, Derivatives, Integration
\end{tags}
    
\begin{diary}
        %S2016-HW8-Q5
\end{diary}
	
\begin{solution}

\end{solution}
	
\end{question}

\end{tagblock}

%------------------------------------------------------------------------------------------------------------
\begin{tagblock}{W23, Applications, SurfaceArea, Derivatives, Integration, Geometry}
\begin{question}
	
If you knew the surface area for an object in 1a), express the surface as a surface of revolution (if possible) and compute the surface area. If you didn't know any surface areas, show that the surface area of a sphere of radius $r$ is $4\pi r^2$. 

\index{Applications}
    
\begin{tags}
        W23, Applications, Definition, SurfaceArea, Parametric, Derivatives, Integration, Geometry
\end{tags}
    
\begin{diary}
        %S2016-HW8-Q5
\end{diary}
	
\begin{solution}

\end{solution}
	
\end{question}

\end{tagblock}

%------------------------------------------------------------------------------------------------------------
\begin{tagblock}{W24, Applications, Curvature, Geometry, WarmUp}
\begin{question}
	

We will now explore curvature for planar curves! Let's start small:

\bigskip

a) What do you think the curvature of a line should be?  

\bigskip

b) Draw two circles with different radii. Which circle do you think has greater curvature?

\bigskip

c) Guess a formula for the curvature of a circle. 

\index{Applications}
    
\begin{tags}
       W24, Applications, Curvature, Geometry, WarmUp
\end{tags}
    
\begin{diary}
        %S2016-HW8-Q5
\end{diary}
	
\begin{solution}

\end{solution}
	
\end{question}

\end{tagblock}

%------------------------------------------------------------------------------------------------------------
\begin{tagblock}{W24, Applications, Curvature, Derivatives, ArcLength, TangentLine, Trigonometry, Theory}
\begin{question}
	


For a parametric curve $x=x(t)$, $y=y(t)$, remember we can define the arc length of the curve from $t=a$ to $t=b$ as 
\[
\int_a^b\sqrt{(x'(t))^2+(y'(t))^2} \ dt.
\]
Fixing a value of $a$ and making a change of the variable inside the integral to $s$, we can define the \textit{arc length function} as
\[
L(t)=\int_a^t\sqrt{(x'(s))^2+(y'(s))^2} \ ds.
\]



\bigskip


a) Calculate $L'(t)$. What is the smallest value you could get for $L'(t)$?

\bigskip

Taking the tangent line to the curve at a point and shifting the $y$-intercept to the origin, we can define the \textit{curvature} $\kappa(L)$ to be
\[
\kappa(L)=\frac{d\theta}{dL}
\]
where $\theta$ is the angle the tangent line makes with the horizontal line $y=y(t)$, measured counterclockwise, and $L$ is the arc length.

\bigskip

b) Draw $\theta$ for a given curve. Can you calculate $\tan(\theta)$ in terms of the derivatives of $x$ and $y$?

\index{Applications}
    
\begin{tags}
       W24, Applications, Curvature, Derivatives, ArcLength, TangentLine, Trigonometry, Theory
\end{tags}
    
\begin{diary}
        %S2016-HW8-Q5
\end{diary}
	
\begin{solution}

\end{solution}
	
\end{question}

\end{tagblock}

%------------------------------------------------------------------------------------------------------------
\begin{tagblock}{W24, Applications, Curvature, Derivatives, ArcLength, TangentLine, Trigonometry, Theory}
\begin{question}
	


By the chain rule and the fact that, if $L$ is the arclength of a curve, $dL/dt=\sqrt{(x'(t))^2+(y'(t))^2}$,
\[
\frac{d\theta}{dt}=\frac{d\theta}{dL}\frac{dL}{dt}=\frac{d\theta}{dL}\sqrt{(x'(t))^2+(y'(t))^2}=\kappa(L)\sqrt{(x'(t))^2+(y'(t))^2}.
\]
Solving for $\kappa$, we get
\begin{equation}\label{curv}
\kappa=\frac{1} {\sqrt{(x'(t))^2+(y'(t))^2}}\frac{d\theta}{dt},
\end{equation}
which is now a formula depending on $t$ instead of $L$.
\bigskip

a) Using the chain rule, differentiate the formula $\tan(\theta)=\dfrac{y'(t)}{x'(t)}$ with respect to $t$ and solve for $\dfrac{d\theta}{dt}$.

\bigskip

b) Substitute your answer from part a) into equation \eqref{curv} to obtain a formula for the curvature entirely in terms of the derivatives of $x$ and $y$.

\bigskip

c) Use the formula in b) to compute the curvature of a circle centered at the origin of radius $r$. 

\index{Applications}
    
\begin{tags}
       W24, Applications, Curvature, Derivatives, ArcLength, TangentLine, Trigonometry, Theory
\end{tags}
    
\begin{diary}
        %S2016-HW8-Q5
\end{diary}
	
\begin{solution}

\end{solution}
	
\end{question}

\end{tagblock}

%------------------------------------------------------------------------------------------------------------
\begin{tagblock}{W24, Applications, Curvature, Derivatives, Theory, Challenge}
\begin{question}
	


Choose your favorite curve and apply the formula
\[
\kappa(t)=\frac{x'(t)y''(t)-y'(t)x''(t)}{(\sqrt{(x'(t))^2+(y'(t))^2})^3}
\]
for the curvature to two different parameterizations of the curve. What do you get? 

\index{Applications}
    
\begin{tags}
       W24, Applications, Curvature, Derivatives, Theory, Challenge
\end{tags}
    
\begin{diary}
        %S2016-HW8-Q5
\end{diary}
	
\begin{solution}

\end{solution}
	
\end{question}

\end{tagblock}

%------------------------------------------------------------------------------------------------------------

 \section{Challenge Questions}\index{Challenge}
\fancyhead[R]{\large Challenge Questions}

\begin{tagblock}{Challenge}
\begin{question}[CHALLENGE!!!]
	This kind of problem gets asked at interviews for financial positions and (I know this for a fact) at apple. 

\bigskip

a) The  Assembler has a balancing scale and seven orbs that he can't open, one of which contains the Reality Bauble. He knows that the orb with the Reality Bauble in it must be heavier than the other six, but he can't perceive it. He could weigh them two at a time, but unfortunately, he only has time for two weighings before Thonas busts in and steals all the orbs from him, and he can only carry one orb with him. 

How does the  Assembler determine which orb contains the Reality Bauble?

\bigskip

b) Now suppose the  Assembler knows either six of the orbs have Infinity Baubles in them or only one does, but he doesn't know which. Can he figure out which case he is in with only two weighings, or is he doomed no matter what?
	
 \index{Challenge}
	
\begin{tags}
	    Challenge
\end{tags}
	
\begin{diary}
	%F2015-HW8-Q11, S2016-HW8-Q11
\end{diary}
		
\begin{solution}
       
\end{solution}

\end{question}

\end{tagblock}

%-------------------------------------------------------------------------------------------------------------

\begin{tagblock}{Challenge}
\begin{question}[CHALLENGE!!!]
	Dr. Weird is more that just the world's Magician Mostest; he is also a mathemagician! For his first amazing feat, he will add the first 100 counting numbers for you. Dr. Weird closes his eyes and mumbles a few cryptic phrases before confidently stating ``5050." The whole process takes him roughly 5 seconds. 

\bigskip

a) Are you astounded? Why or why not?

\bigskip

b) Dr. Weird takes pity on you and invites you to consider the sum of the first 5 counting numbers. He smiles mysteriously and says ``Rectangles." Try to draw your way to the solution.

\bigskip

c) Now see if you know how Dr. Weird came up with the answer 5050 for the sum of the first 100 counting numbers. If you do, you probably know how to get the sum of the first 250...

\bigskip

d) This one's a bit harder: what about the sum of the \textbf{squares} of the first 100 counting numbers?
	
 \index{Challenge}
	
\begin{tags}
	    Challenge
\end{tags}
	
\begin{diary}
	%F2015-HW8-Q11, S2016-HW8-Q11
\end{diary}
		
\begin{solution}
       
\end{solution}

\end{question}

\end{tagblock}

%-------------------------------------------------------------------------------------------------------------

\begin{tagblock}{Challenge}
\begin{question}[CHALLENGE!!!]

Walking Boss Godfrey is playing a game with three inmates in his prison. He puts blindfolds on all three prisoners and tells them he has five hats, three of which are red and two of which are blue. He puts a hat on each prisoner and hides the remainder. He then informs the inmates that he will remove their blindfolds one at a time, and if any one of them can deduce the color of their own hat by only looking at the other two prisoners, that person can go free. If they guess wrong or tell another prisoner the color of their hat, though, it's solitary confinement for a week! There is no penalty for not guessing besides staying in jail.

Boss Godfrey takes the bindfold off the first prisoner, who says he can't determine the color of his own hat. The second prisoner makes the same declaration after his blindfold is removed. The third prisoner, who is Cool Hand Luke, then tells Walking Boss Godfrey the color of his own hat without even taking off his blindfold.

What color is Luke's hat, and how does he know?
	
 \index{Challenge}
	
\begin{tags}
	    Challenge
\end{tags}
	
\begin{diary}
	%F2015-HW8-Q11, S2016-HW8-Q11
\end{diary}
		
\begin{solution}
       
\end{solution}

\end{question}

\end{tagblock}

%-------------------------------------------------------------------------------------------------------------

\begin{tagblock}{Challenge, Factorial}
\begin{question}[CHALLENGE!!!]
	

\bigskip

Overlord Tremortine has captured spies whom he wishes to interrogate. However, wanting to deliver his message of order through fear to the upstart alliance, he has decided to release one of the spies, but he will only choose the spy to be released via a rather odd method: he arranges the spies in a circle and, starting with a spy labelled \#1 and continuing in numerical order, recites, ``Skip, cell, skip, cell, skip, cell,...". Anyone who got ``cell" goes back to their cell. 

So if the Overlord has 4 prisoners, he says ``Skip, cell, skip, cell," and prisoners 2 and 4 go back to their cells, leaving 1 and 3. Then he starts where he left off (if he ended with ``skip", he starts with ``cell" now, and if he ended with ``cell," he starts with ``skip") and says ``Skip, cell," so prisoner 3 goes back to their cell, leaving only prisoner 1. When there's only one prisoner left, the Overlord releases them- but not before turning them over to Darth Vader for interrogation.  

\bigskip

a) Is there a difference between the Overlord starting with ``skip" or with ``cell"? Why or why not?

\bigskip

b) Is there a number of total spies between 10 and 20 where if Overlord Tremortine plays his game with that many spies, spy \#1 goes free? If so, what is the number? How about between 20 and 30? 30 and 40? 40 and 100?

\bigskip

c) Is there a number $n$ of total spies between 10 and 20 where if Overlord Tremortine plays his game with that many spies, spy \#n goes free? If so, what is the number? How about between 20 and 30? 30 and 40? 40 and 100?
	
 \index{Challenge}
	
\begin{tags}
	    Challenge
\end{tags}
	
\begin{diary}
	%F2015-HW8-Q11, S2016-HW8-Q11
\end{diary}
		
\begin{solution}
       
\end{solution}

\end{question}

\end{tagblock}

%-------------------------------------------------------------------------------------------------------------
\begin{tagblock}{Challenge}
\begin{question}[CHALLENGE!!!]
	

a) Michael Phelps, Ryan Lochte, and Ian Thorpe (all in comeback roles) are swimming the 200m freestyle. In how many different ways can they finish first, second, and third?

\bigskip

b) Juan Martin del Potro, Rafael Nadal, Andy Murray, and Kei Nishikori are all in the Olympic tennis singles semifinals. In how many different ways can they finish gold, silver, and bronze (theoretically)?

\bigskip

c) You're ordering toppings for your Turbines pizza. Each topping can be ordered repeatedly as an ``extra" option. If Turbines has 15 toppings and you want three of them (maybe with extras), how many choices of toppings do you have?

\bigskip

d) How many ways are there to order $n$ things in $n$ places if you can't choose the objects more than once? How many ways are there to choose $n$ objects for $n$ places if you can choose the objects more than once?

\bigskip

e) Now for the philosophical part. How many ways are there to put no things in no spots, either with repitition or without?
	
 \index{Challenge}
	
\begin{tags}
	    Challenge, Factorial
\end{tags}
	
\begin{diary}
	%F2015-HW8-Q11, S2016-HW8-Q11
\end{diary}
		
\begin{solution}
       
\end{solution}

\end{question}

\end{tagblock}

%-------------------------------------------------------------------------------------------------------------
\begin{tagblock}{Challenge}
\begin{question}[CHALLENGE!!!]
	
a) A frog is stuck in a well 19.75 feet high. Every 10 seconds, the frog can jump up 2 feet of the well, but then it slides half a foot down because it is tired and slimy. How long does it take the frog to get out of the well? 

\bigskip

b) A family of four wants to get through a tunnel. Dad can make it in 1 minute, Mom in 2 minutes, son in 4 minutes, and daughter in 5 minutes. Unfortunately, not more than two people can go through the narrow tunnel at one time, moving at the speed of the slower one. Can they all make it to the other side if they have a torch that lasts only 12 minutes and they are afraid of the dark? 
	
 \index{Challenge}
	
\begin{tags}
	    Challenge
\end{tags}
	
\begin{diary}
	%F2015-HW8-Q11, S2016-HW8-Q11
\end{diary}
		
\begin{solution}
       
\end{solution}

\end{question}

\end{tagblock}

%-------------------------------------------------------------------------------------------------------------
\begin{tagblock}{Challenge}
\begin{question}[CHALLENGE!!!]
	
You're standing at three light switches at the bottom of the stairs to the attic. Each one corresponds to one of three lights in the attic, but you cannot see the lights from where you stand. You can turn the switches on and off and leave them in any position. How can you identify which switch corresponds to which light bulb if you are only allowed one trip upstairs? 
	
 \index{Challenge}
	
\begin{tags}
	    Challenge
\end{tags}
	
\begin{diary}
	%F2015-HW8-Q11, S2016-HW8-Q11
\end{diary}
		
\begin{solution}
       
\end{solution}

\end{question}

\end{tagblock}

%-------------------------------------------------------------------------------------------------------------
\begin{tagblock}{Challenge}
\begin{question}[CHALLENGE!!!]
	
(From Raymond Smullyan) On the Isle of Glas, we have knights, knaves, and spies. Knights always tell the truth, knaves always lie, and spies can either lie or tell the truth. While visiting the Isle, you encounter three individuals: Able, Berthold, and Caan, one of which is a spy, one of which is a knight, and one of which is a knave.  Able says that Caan is a knave. Berthold says that Able is a knight. Caan says ``I am the spy." Which one is the spy, which one is the knight, and which one is the knave?
	
 \index{Challenge}
	
\begin{tags}
	    Challenge
\end{tags}
	
\begin{diary}
	%F2015-HW8-Q11, S2016-HW8-Q11
\end{diary}
		
\begin{solution}
       
\end{solution}

\end{question}

\end{tagblock}

%-------------------------------------------------------------------------------------------------------------
\begin{tagblock}{Challenge, InverseTrig}
\begin{question}[CHALLENGE!!!]

Restriction of domain seems kind of arbitrary, doesn't it? For example, why not restrict tangent to $(\pi/2,3\pi/2)$ instead of $(-\pi/2,\pi/2)$?

\bigskip

a) Does using $(\pi/2,3\pi/2)$ instead of $(-\pi/2,\pi/2)$ change your answer for $\displaystyle\int_0^1\frac 1 {1+x^2} \ dx$? Why or why not? 

\bigskip

b) Now try problem b) from \#3, except restrict secant to $[0,\pi/2)\cup (\pi/2,\pi]$. What do you get?

\bigskip

c) Make an argument using restriction of domain that will give you that ALL functions are invertible. 
	
 \index{Challenge}
	
\begin{tags}
	    Challenge, InverseTrig
\end{tags}
	
\begin{diary}
	%F2015-HW8-Q11, S2016-HW8-Q11
\end{diary}
		
\begin{solution}
       
\end{solution}

\end{question}

\end{tagblock}

%-------------------------------------------------------------------------------------------------------------
\begin{tagblock}{Challenge}
\begin{question}[CHALLENGE!!!]

Five pirates looted a chest full of 100 gold coins. Being a bunch of democratic pirates, they agree on the following method to divide the loot:

\bigskip

The most senior pirate will propose a distribution of the coins. All pirates, including the most senior pirate, will then vote. If at least 50\% of the pirates (3 pirates in this case) accept the proposal, the gold is divided as proposed. If not, the most senior pirate will be fed to the sharks and the process starts over with the next most senior pirate. The process is repeated until a plan is approved. You can assume that all pirates are perfectly rational: they want to stay alive first and to get as much gold as possible second. Finally, being blood-thirsty pirates, they want to have fewer pirates on the boat if given a choice between otherwise equal outcomes.

\bigskip

How will the gold coins be divided in the end?
	
 \index{Challenge}
	
\begin{tags}
	    Challenge
\end{tags}
	
\begin{diary}
	%F2015-HW8-Q11, S2016-HW8-Q11
\end{diary}
		
\begin{solution}
       
\end{solution}

\end{question}

\end{tagblock}

%-------------------------------------------------------------------------------------------------------------
 \section{Continuity}\index{Continuity}
\fancyhead[R]{\large Continuity}

\begin{tagblock}{Continuity, Definition, Graph}
\begin{question}
	In previous worksheets we have looked at limits of functions.  In this Worksheet we will introduce a ``nice'' class of functions.  



Consider the three functions $f, g$ and $h$ given below.  

\begin{figure}[h]
\includegraphics[width=5cm]{cont1f.png} \hspace{.05in} \includegraphics[width=5cm]{cont1g.png} \hspace{.05in} \includegraphics[width=5cm]{cont1h.png}
\end{figure}
\begin{enumerate}
\item Compute $f(1)$ and $\displaystyle \lim_{x \to 1} f(x)$.
\item Compute $g(1)$ and $\displaystyle \lim_{x \to 1} g(x)$.
\item Compute $h(1)$ and $\displaystyle \lim_{x \to 1} h(x)$.

\end{enumerate}


\bigskip
The above functions illustrate different behavior at $a=1$.  Intuitively, a function is continuous if we can draw it without ever lifting our pencil from the page. Alternatively, we might say that the graph of a continuous function has no jumps, holes or asymptotes in it.  Both $f$ and $g$ have a 	\emph{hole} at $a=1$ and we say that $f$ and $g$ are \emph{not continuous at $a=1$}.  $h$ does not have a hole at $a=1$, and we say that $h$ is \emph{continuous at $a=1$}.  More precisely:

\textbf{Definition:}  A function $f$ is \emph{continuous at $x=a$} if 
\begin{enumerate}
\item The limit $\displaystyle \lim_{x \to a} f(x)$ exists
\item $f$ is defined at $x =a$
\item $\displaystyle \lim_{x \to a} f(x)=f(a)$
\end{enumerate}

%Conditions (a) and (b) are technically contained implicitly in (c), but we state them explicitly to emphasize their individual importance. In words, (c) essentially says that a function is continuous at $x=a$ provided that its limit as $x\to a$ exists and equals its function value at $x=a$. 

If a function is continuous at every point in an interval $(a,b)$, we say the function is \emph{continuous on $(a,b)$.}  If a function is continuous at every point in its domain, we simply say the function is \emph{continuous.} 

\textbf{Thus, continuous functions are particularly nice: to evaluate the limit of a continuous function at a point, all we need to do is evaluate the function.}
	
	
\begin{tags}
	   Continuity, Definition, Graph
\end{tags}
	
\begin{diary}
	 
\end{diary}


	

	\end{question}
	
	\end{tagblock}




%-------------------------------------------------------------------------------------------------------------


\begin{tagblock}{Continuity, Definition, Graph}
\begin{question}
	
	Consider the graph $f$ given by the graph below
\begin{figure}[h]
\centering
\includegraphics[width=6cm]{cont2.png}
\end{figure}

\begin{enumerate}

\item For each of the values $a=-3,-2,-1,0,1,2,3,$ determine whether or not $\lim_{x \to a} f(x)$ exists.  If the function does not have a limit at a given point, write a sentence to explain why.
\begin{enumerate}
\item  $\displaystyle \lim_{x \to -3} f(x)=$ 

\bigskip

\item  $\displaystyle \lim_{x \to -2} f(x)=$ 

\bigskip


\item  $\displaystyle \lim_{x \to -1} f(x)=$ 

\bigskip

\item  $\displaystyle \lim_{x \to 0} f(x)=$ 

\bigskip

\item  $\displaystyle \lim_{x \to 1} f(x)=$ 

\bigskip

\item  $\displaystyle \lim_{x \to 2} f(x)=$ 

\bigskip

\item  $\displaystyle \lim_{x \to 3} f(x)=$ 

\bigskip

\end{enumerate}

\item For $a=-3,-2,-1,0,1,2,3$ determine $f(a)$.  
\vspace{.75in}

\item At which values of $a$ is $f(a)$ not defined?
\vspace{.75in}


\item At which values of $a$ does $f$ have a limit, but $\lim_{x \to a} f(x) \neq f(a) $?
\vspace{.75in}

\item State all values of $a$ for which $f$ is not continuous at $x=a$.
\vspace{.75in}

\item  Using interval notation, give the intervals in which $f(x)$ is continuous.
\vspace{.75in}

\item Which condition is stronger, and hence implies the other: $f$ has a limit at $x=a$ or $f$ is continuous at $x=a$? Explain, and hence complete the following sentence: 
\bigskip

``If $f$ \rule{4cm}{.1mm} at $x=a$, then $f$  \rule{4cm}{.1mm} at  $x=a$,'' 

where you complete the blanks with \emph{has a limit} and is \emph{continuous}, using each phrase once.
\end{enumerate}
	
\begin{tags}
	   Continuity, Definition, Graph
\end{tags}
	
\begin{diary}
	  
\end{diary}


	

	\end{question}
	
	\end{tagblock}




%-------------------------------------------------------------------------------------------------------------


\begin{tagblock}{Continuity, Definition, Theory}
\begin{question}
	
Next, we'll look more at what types of functions are continuous.  Thinking back to functions we've investigated: polynomials, rational functions, trig functions, piecewise functions, which ones do you think are continuous, and why? 
	
\begin{tags}
	   Continuity, Definition, Theory
\end{tags}
	
\begin{diary}
	  
\end{diary}


	

	\end{question}
	
	\end{tagblock}




%-------------------------------------------------------------------------------------------------------------


\begin{tagblock}{Continuity, Definition, Polynomial}
\begin{question}
	
\textbf{Definition:}  A function $f$ is \emph{continuous at $x=a$} if 
\begin{enumerate}
\item The limit $\displaystyle \lim_{x \to a} f(x)$ exists
\item $f$ is defined at $x =a$
\item $\displaystyle \lim_{x \to a} f(x)=f(a)$
\end{enumerate}

 Intuitively, a function is continuous if we can draw it without ever lifting our pencil from the page. Alternatively, we might say that the graph of a continuous function has no jumps, holes or asymptotes in it. \\
 
 \bigskip
  
  If a function is continuous at every point in an interval $(a,b)$, we say the function is \emph{continuous on $(a,b)$.}  If a function is continuous at every point in its domain, we simply say the function is \emph{continuous.}

\bigskip

In this Worksheet we'll determine which type of functions are continuous. 
\bigskip

 

 If $p(x)$ is a \textbf{polynomial} how do $\displaystyle \lim_{x \to a} p(x)$ and  $p(a)$ compare?  Is a polynomial a continuous function?  
	
\begin{tags}
	   Continuity, Definition, Polynomial
\end{tags}
	
\begin{diary}
	  
\end{diary}


	

	\end{question}
	
	\end{tagblock}




%-------------------------------------------------------------------------------------------------------------


\begin{tagblock}{Continuity, Definition, RationalFunctions}
\begin{question}
	
Consider the rational function $\displaystyle q(x)= \frac{x^2-3x+2}{2x+4}$
\begin{enumerate}
\item What is the domain of $q(x)$
\item Compute $\displaystyle \lim_{x \to 1} q(x)$ (Hint: Use the Limit Laws)
\item If $x=a$ is in the domain of $q(x)$, verify that  $\displaystyle \lim_{x \to a} q(x)=q(a)$
\end{enumerate}
	
\begin{tags}
	   Continuity, Definition, RationalFunctions
\end{tags}
	
\begin{diary}
	  
\end{diary}


	

	\end{question}
	
	\end{tagblock}




%-------------------------------------------------------------------------------------------------------------


\begin{tagblock}{Continuity, Piecewise, Graph}
\begin{question}
	
As a next example we'll look at piecewise functions.  Consider
\[ g(x) = \begin{cases} 
x^2 & \text{ for }  x< 1 \\
x+1 & \text{ for }  1 \leq x< 2 \\
1 &  \text{ for } x=2 \\
5-x & \text{ for } x>2 \end{cases}\]
Note that each piece is a polynomial so will be continuous, but we will need to check carefully where the function changes definition.

\begin{enumerate}
\item For each of the values $a= 1,2$ compute $g(a)$.
\vspace{.5in}

\item  For each of the values $a= 1,2$ determine $\lim_{x \to a^-} g(x)$ and  $\lim_{x \to a^+} g(x)$.  \emph{Make sure you use proper notation.}

\vspace{1.5in}


\item For each of the values $a= 1,2$ determine $\lim_{x \to a} g(x)$. If the limit fails to exist, explain why by discussing the left- and right-hand limits at the relevant $a$-value.

\vspace{1.5in}

\item For which values of $a$ is the following statement true?
\[\lim_{x \to a} g(x) \neq g(a)\]

\vspace{.5in}


\item For which intervals is $g(x)$ continuous?
\vspace{.5in}

\item On the axis above sketch an accurate, labeled graph of $y=g(x)$. Be sure to carefully use open circles and filled circles to represent key points on the graph, as dictated by the piecewise formula.
\end{enumerate}

\begin{figure}[h]
\centering
\includegraphics[width=8cm]{cont4.png}
\end{figure}
	
\begin{tags}
	   Continuity, Piecewise, Graph
\end{tags}
	
\begin{diary}
	  
\end{diary}


	

	\end{question}
	
	\end{tagblock}




%-------------------------------------------------------------------------------------------------------------

\begin{tagblock}{Continuity, Piecewise}
\begin{question}
	
What values of $c$ and $m$ do you need to make
\[ h(x) = \begin{cases} 
cx^2 & \text{ for }   x< 1 \\
4 & \text{ for }  x=1 \\
-x^3+mx & \text{ for } x>1 \end{cases}\]
continuous?  
	
\begin{tags}
	   Continuity, Piecewise
\end{tags}
	
\begin{diary}
	   
\end{diary}


	

	\end{question}
	
	\end{tagblock}




%-------------------------------------------------------------------------------------------------------------
 \section{Derivatives}\index{Derivatives}
\fancyhead[R]{\large Derivatives}



\begin{tagblock}{TangentLines, Derivatives, Graph, Definition}
\begin{question}
	

\bigskip

\textbf{The Tangent Problem}

Suppose we have a curve given by the function $f(x)$.  For a given value $x=a$, we have a point on the curve, namely $(a, f(a))$.  A line is \emph{tangent} to the curve, if it ``just touches'' the curve.  For example:
\begin{figure}[h]
\centering
\includegraphics[width=3cm,height=3cm]{tangent2.jpg}
\end{figure}
%graph of x^2 with tangent line at (1,1)

\textbf{Goal:} Given a curve and a point on the curve, find the equation of the tangent line to the curve at that point.

\bigskip
 Consider the function $f(x)=x^3$, and the point $P=(1,1)$ on the curve.  
\begin{figure}[h]
\centering
\includegraphics[width=8cm,height=8cm]{tangent3.jpg}
\end{figure}
%graph of x^3 with point (1,1)

a)  Try your best to draw a tangent line to the curve at the point $P=(1,1)$ 
 
 \bigskip
 
To find the slope of the line you just drew, you need two points on the line.  We have one point, namely $(1,1)$, and certainly we could guess another point from our drawing.  However, we would like to be more precise.  To do so, we will compute slopes of secant lines.  A \emph{secant line} of a curve is a line between any \emph{two} points on a curve.  

\bigskip
b) Plot and label the points on the graph of $f(x) = x^3$ (It will also help to compute the $y$-value)
\[ Q_0 = (0, f(0)), \,Q_{1/2} = (\frac{1}{2}, f(\frac{1}{2})) , \, Q_{3/4} = (\frac{3}{4}, f(\frac{3}{4}) ),  \, Q_{7/8} = (\frac{7}{8}, f(\frac{7}{8}))  \]

\vspace{.5in}
c) Draw a secant line to the curve at the points $P = (1,1)$ and $Q_0 = (0,f(0))$ (that is draw a line connecting $(1,1)$ and $(0,f(0)$ ). Note that we have two points on the line.... Compute the slope of this line.

\vspace{.75in}

d) Draw a secant line to the curve at the points $P$ and $Q_{1/2} =  (\frac{1}{2}, f(\frac{1}{2}))$.  Compute the slope of the line.  Notice that the point $Q_{1/2}$ is closer to $P$ than $Q_0$.

\vspace{.75in}

e)  Compute the slope of the secant line through the points $P$ and $Q_{3/4}$. 


\vspace{.75in}

f)  Compute the slope of the secant line through the points $P$ and $Q_{7/8}$. 

\vspace{.75in}

g) In our example $P=(1,1)$, and the point $(1 + h, f(1+ h))$ is ``close to'' $P$ (in part c) $h=-1$, in part d) $h=-0.5$, in part e) $h=-0.25$).   Compute the slope of the secant line between $P$ and $(1 + h, f(1+ h))$.   Be sure to fully simplify your answer.  (Note this is the \textbf{difference quotient} we looked at on the first day)
\vspace{1.5in}

 h) The tangent line to our curve at $(1,1)$ can be estimated by the secant line between the point $(1,1)$ and a point ``close to $(1,1)$,'' that is a point $Q=(1+h, f(1+h) )$, where $h$ is small, either positive or negative.    Using the language we have learned thus far, 
 \[ \text{slope of the tangent line to $f(x)$ at }(1,1) = \lim_{h \to 0} \{ \text{slope of a secant line between the points $P=(1,1)$ and $(1+h, f(1+h) )$} \} \]
 Using your answer to g), compute the slope of the tangent line to $f(x) = x^3$ at the point $P=(1,1)$.


 
 \vspace{1in}
 i) Now write an equation for the tangent line to $f(x)=x^3$ at the point $P=(1,1)$
  
 \vspace{.5in}

\bigskip


	
\index{Derivatives}
	
\begin{tags}
	    TangentLines, Derivatives, Graph, Definition
\end{tags}
	
\begin{diary}
	    %S2016-HW1-Q1
\end{diary}
	
\begin{solution}
	   
\end{solution}
	
\end{question}

\end{tagblock}

%-------------------------------------------------------------------------------------------------------------

\begin{tagblock}{TangentLines, Velocity, Derivatives, Graph}
\begin{question}
	

\bigskip

\textbf{Computing Velocity}

As a second application we will look at velocity.  We define the \emph{average velocity} of an object as
\[ \text{average velocity} = \frac{ \text{change in position}}{\text{time elapsed}} \]

Suppose an object moves so that it's position after $t$ seconds is $s(t)=3t^2$ feet from where it started.
\bigskip

a) Find the average velocity from $t=2$ to $t=3$ seconds.
\vspace{.5in}

b) Find the average velocity from $t=2$ to $t=2.5$ seconds.
\vspace{.5in}

c) Find the average velocity from $t=2$ to $t=2+h$ seconds (where $h>0$).  Simplify your answer.
\vspace{1in}

d) We define the \emph{instantaneous velocity} at $t=2$ to be the limit of the average velocities (from part c)) as $h$ approaches $0$.   Find the instantaneous velocity at $t=2$.

\vspace{.8in}

e)  Below is a graph of the particles distance $s(t)$.  How can you see the average velocity from parts a) and b) and the instantaneous velocity at $t=2$ from the graph. (Hint: what did we do in the first part?)
\begin{figure}[h]
\centering
\includegraphics[width=8cm,height=8cm]{velocity1.jpg}
\end{figure}


\bigskip


	
\index{Derivatives}
	
\begin{tags}
	    TangentLines, Velocity, Derivatives, Graph
\end{tags}
	
\begin{diary}
	    %S2016-HW1-Q1
\end{diary}
	
\begin{solution}
	   
\end{solution}
	
\end{question}

\end{tagblock}

%-------------------------------------------------------------------------------------------------------------

\begin{tagblock}{Derivatives, Definition}
\begin{question}
	

\bigskip

\textbf{Notation:}  If $y=f(x)$ we have many notations for the derivative:  $f'(x)$, $y'$, $\frac{dy}{dx}$ and $\frac{d}{dx}[f(x)]$ all mean the derivative of $f$.  



So given a function $f(x)$, to compute the derivative $f'(x)$, we form the difference quotient $\displaystyle \frac{f(x+h)-f(x)}{h}$ and then do some algebra (like we did earlier) until we take the limit as $h \to 0$.  





\begin{enumerate}

\item Let $f(x) = x^2$.  \textbf{Using the definition of the derivative} we will compute $f'(x)$.  I've started this off for you, finish up the computation.  

\[f'(x) = \lim_{h \to 0} \frac{f(x+h) - f(x)}{h} =  \lim_{h \to 0} \frac{(x+h)^2 - x^2}{h} \]


\vspace{5in}



\item Let $f(x) = \sqrt{3x+2}$.  Use the definition of the derivative to get an equation for $f'(x)$.  \emph{Hint: multiply and divide by the conjugate}

\vfill  As we see, using the definition of the derivative can be lengthy computation.  Our next goal will be to develop tools to compute derivatives efficiently.  

\end{enumerate}

\bigskip


	
\index{Derivatives}
	
\begin{tags}
	    Derivatives, Definition
\end{tags}
	
\begin{diary}
	    %S2016-HW1-Q1
\end{diary}
	
\begin{solution}
	   
\end{solution}
	
\end{question}

\end{tagblock}


%-------------------------------------------------------------------------------------------------------------

\begin{tagblock}{Derivatives, Definition, Graph}
\begin{question}
	

\bigskip







Let's look at two easy functions, the constant function, $f(x) = C$ and the linear function $g(x) =x$.  

\begin{figure}[h]
\centering
\includegraphics[width=5cm]{constant.png} \hspace{.05in} \includegraphics[width=5cm]{line.png}
\end{figure}
Both are lines, so if we draw a tangent line at any value $a$ we'll just get the same line. 
\begin{enumerate}
\item What is the slope of the constant line $f(x) = C$?  
\item What is the slope of $g(x) = x$?
\end{enumerate}

\bigskip


This tells us that $\frac{d}{dx}[C] = 0$ for any constant and $\frac{d}{dx}[x] = 1$.


	
\index{Derivatives}
	
\begin{tags}
	    Derivatives, Definition, Graph
\end{tags}
	
\begin{diary}
	    %S2016-HW1-Q1
\end{diary}
	
\begin{solution}
	   
\end{solution}
	
\end{question}

\end{tagblock}

%-------------------------------------------------------------------------------------------------------------

\begin{tagblock}{Derivatives, Definition, PowerRule}
\begin{question}
	

\bigskip



Next we'll consider functions of the form $f(x) = x^n$.  
\begin{enumerate}
\item From the previous problem we know $\frac{d}{dx}[x]=1$.
\item In Problem 1 you computed $\frac{d}{dx}[x^2] = $


\item Using the definition of the derivative, find $f'(x)$ for $f(x) = x^3$. \\
(Hint $(x+h)^3 = x^3+3x^2h+3xh^2 + h^3$).  

%\item Based on your work in (a), (b), and (c), (d) what do you conjecture is the derivative of $f(x) = x^{11}$ is?
\vfill

\item Conjecture a formula for the derivative of $f(x)=x^n$ that holds for any positive integer $n$. That is, given $f(x)=x^n$ where $n$ is a positive integer, what do you think is the formula for $f'(x)$?
\end{enumerate}
\newpage

Hopefully you conjectured the:

\begin{center}\textbf{Power Rule}: If $f(x) = x^n$, then $f'(x) = nx^{n-1}$. \end{center}

In fact, the Power Rule works for any $n$, not just positive integers:

\textbf{Example}: If $g(x) = x^{-3}$, then $g'(x) = -3x^{-3-1} = -3x^{-4}$ \\
\bigskip

\textbf{Example}: If $h(t) = t^{3/5}$, then $h'(t) = \frac{3}{5}t^{3/5-1} =  \frac{3}{5}t^{-2/5}$ \\






	
\index{Derivatives}
	
\begin{tags}
	    Derivatives, Definition, PowerRule
\end{tags}
	
\begin{diary}
	    %S2016-HW1-Q1
\end{diary}
	
\begin{solution}
	   
\end{solution}
	
\end{question}

\end{tagblock}

%-------------------------------------------------------------------------------------------------------------

\begin{tagblock}{Derivatives, Definition, PowerRule}
\begin{question}
	

\bigskip



Compute the following derivatives:
\begin{enumerate}
\item $f(x) = \sqrt{x}$ (Remember $\sqrt{x} = x^{1/2}$)
\vspace{.7in}

\item $g(w) = w^{4/3}$
\vspace{.7in}

\item $\displaystyle s(t) = \frac{1}{t^5}$
\vspace{.7in}

\item $h(x) = \sqrt{2}$
\end{enumerate}






	
\index{Derivatives}
	
\begin{tags}
	    Derivatives, Definition, PowerRule
\end{tags}
	
\begin{diary}
	    %S2016-HW1-Q1
\end{diary}
	
\begin{solution}
	   
\end{solution}
	
\end{question}

\end{tagblock}

%-------------------------------------------------------------------------------------------------------------

\begin{tagblock}{Derivatives, Example, PowerRule, SumRule}
\begin{question}
	

\bigskip



Most functions we encounter are a bit more complicated than simply a constant or a power.  For example, we might want to compute the derivative of the polynomial $p(x) = 5x^7 + 6x^3 -3x +2$, which is a function made up of constant multiples and sums of powers of $x$.  For this we'll need the \textbf{Constant Multiple Rule} and the \textbf{Sum Rule}.

%Suppose we have a function $y=f(x)$ whose derivative formula is known. How is the derivative of $y=cf(x)$ related to the derivative of the original function? Recall that when we multiply a function by a constant $c$, we vertically stretch the graph by a factor of $|c|$ (and reflect the graph across $y=0$ if $c<0$). This vertical stretch affects the slope of the graph, making the slope of the function $y=cf(x)$ be $c$ times as steep as the slope of $y=f(x)$. In terms of the derivative, this is essentially saying that when we multiply a function by a factor of $c$, we change the value of its derivative by a factor of $c$ as well.

\begin{center}\textbf{Constant Multiple Rule}: For any constant $c$, and any differentiable function $f(x)$, $$\frac{d}{dx}[cf(x)] = c \frac{d}{dx}[f(x)].$$\end{center}

In words, this rule says that ``the derivative of a constant times a function is the constant times the derivative of the function.''

\textbf{Example:} $\frac{d}{dx} [5x^4] = 5 \frac{d}{dx}[x^4] = 5 \cdot 4 x^3 = 20 x^3$.  

\newpage
Next we see what happens when we take a sum of two functions. %If we have $y=f(x)$ and $y=g(x)$, we can compute a new function $y=(f+g)(x)$ by adding the outputs of the two functions: $(f+g)(x)=f(x)+g(x)$. Not only does this result in the value of the new function being the sum of the values of the two known functions, but also the slope of the new function is the sum of the slopes of the known functions. Therefore, we arrive at the following Sum Rule for derivatives

\begin{center}\textbf{Sum Rule}: If $f(x)$ and $g(x)$ are differentiable functions with derivatives $f'(x)$ and $g'(x)$, then $\frac{d}{dx}[f(x)+g(x)] =f'(x)+g'(x)$. \end{center}

In words, the Sum Rule tells us that ``the derivative of a sum is the sum of the derivatives.''

We now can find the derivative of any polynomial.  \emph{In the below example I'll show every step; but as you get comfortable with the rules, you don't need to show this much detail}

\textbf{Example:} We'll find the derivative of $p(x) = 5x^7 + 6x^3 -3x +2$.  We'll start by using our Sum Rule
\[ p'(x) = \frac{d}{dx} [5x^7] +  \frac{d}{dx} [6x^3] +  \frac{d}{dx} [-3x] + \frac{d}{dx} [2] \]
Then apply the Constant Multiple Rule
\[p'(x) =5 \frac{d}{dx} [x^7] + 6 \frac{d}{dx} [x^3] + (-3) \frac{d}{dx} [x] + \frac{d}{dx} [2]\]
And finally the Power Rule and Constant Rule
\[p'(x) =5 \cdot 7x^6+ 6 \cdot 3x^2 + (-3)\cdot 1+ 0 = 35x^6 + 18x^2 -3 \]



\item Use only the rules for constant, and power functions, together with the Constant Multiple and Sum Rules, to compute the derivative of each function below with respect to the given independent variable. Note that we do not yet know any rules for how to differentiate the product or quotient of functions. This means that you may have to do some algebra first on the functions below before you can actually use existing rules to compute the desired derivative formula.
\begin{enumerate}
\item $f(x) = x^{5/3} + x^7 - 3$
\vspace{.7in}

\item $\displaystyle h(z) = \sqrt{z} + \frac{1}{z^4} + \sqrt{7}z^3$
\vspace{.7in}

\item $s(t) = (t^2+1)(t^2-1) $
\vspace{.7in}

\item $\displaystyle q(x) = \frac{x^3-x+1}{x}$.
\end{enumerate} 






	
\index{Derivatives}
	
\begin{tags}
	    Derivatives, Example, PowerRule, SumRule
\end{tags}
	
\begin{diary}
	    %S2016-HW1-Q1
\end{diary}
	
\begin{solution}
	   
\end{solution}
	
\end{question}

\end{tagblock}

%-------------------------------------------------------------------------------------------------------------

\begin{tagblock}{Derivatives, TangentLine, Graph}
\begin{question}
	

\bigskip



Let $\displaystyle f(x) = x^3 +2x^2-3x^{\frac{1}{3}}-2$.  The graph of $f(x)$ is given below:
\begin{figure}[h]
\centering
\includegraphics[width=8cm]{recaptangent.png}  \end{figure}

\begin{enumerate}
\item Draw the tangent line to $f(x)$ at $(1,-2)$.
\item Compute $f'(x)$ 

\vspace{1in}

\item Find the slope of the tangent line to $f(x)$ at $(1,-2)$, and then write the equation of the tangent line to $f(x)$ at $(1,-2)$.
\vspace{1in}

\item Draw the tangent line to $f(x)$ at $(0,-2)$.  What kind of line do you get?
\vspace{.5in}


\item Using the formula for $f'(x)$, can you calculate the slope of the tangent line to $f(x)$ at $(0,-2)$?  Explain.
\vspace{1in}
\end{enumerate}







	
\index{Derivatives}
	
\begin{tags}
	    Derivatives, TangentLine, Graph
\end{tags}
	
\begin{diary}
	    %S2016-HW1-Q1
\end{diary}
	
\begin{solution}
	   
\end{solution}
	
\end{question}

\end{tagblock}

%-------------------------------------------------------------------------------------------------------------

\begin{tagblock}{Derivatives, SecondDerivative, HigherDerivatives, Velocity}
\begin{question}
	




\textbf{Higher Derivatives}: Starting with an equation of a function $f(x)$, we now have tools to compute the derivative $f'(x)$.  We then could differentiate $f'(x)$ and get a new function $f''(x)$: the \emph{second derivative of $f(x)$}.  We could continue differentiating to get the third derivative $f'''(x)$, the fourth derivative $f'''(x)$, \ldots.  After a while we'd get tired of writing so many primes, and instead write $f^{[n]}(x)$ for the $n^{th}$ derivative  
\begin{enumerate}
\item  Find the first and second derivative of $f(x) = \sqrt{x} + 5e^x$ 
\vspace{1.5in}
\item A useful application of the second derivative is as follows: given a function $s(t)$ that gives the position of a function at time $t$, the velocity function, $v(t)$, is the derivative of position: $v(t) = s'(t)$.  If we differentiate again, we will get the acceleration of the object at time $t$, $a(t)$; that is $a(t) = v'(t) = s''(t)$.  \\
\smallskip

If the motion of a particle is given by $s(t) = t^4 - 2t^3 +t^2 -t$, where $s$ is in meters and $t$ is in seconds, find
\begin{enumerate}
\item the velocity and acceleration as functions of $t$
\vspace{2in}

\item the acceleration after $1$ sec (include units in your answer)
\end{enumerate}


\end{enumerate}








	
\index{Derivatives}
	
\begin{tags}
	    Derivatives, SecondDerivative, HigherDerivatives, Velocity
\end{tags}
	
\begin{diary}
	    %S2016-HW1-Q1
\end{diary}
	
\begin{solution}
	   
\end{solution}
	
\end{question}

\end{tagblock}

%-------------------------------------------------------------------------------------------------------------

\begin{tagblock}{Derivatives, HigherDerivatives, Polynomial}
\begin{question}
	




If $g(x) = 5x^4 + 3x^2 - 7x + \sqrt{2}$, how many derivatives would we need to take until we get to $0$? Explain. 











	
\index{Derivatives}
	
\begin{tags}
	    Derivatives, HigherDerivatives, Polynomial
\end{tags}
	
\begin{diary}
	    %S2016-HW1-Q1
\end{diary}
	
\begin{solution}
	   
\end{solution}
	
\end{question}

\end{tagblock}

%-------------------------------------------------------------------------------------------------------------

\begin{tagblock}{Derivatives, HigherDerivatives, Polynomial, Theory}
\begin{question}
	




If $p(x) = a_nx^n + a_{n-1}x^{n-1} + \cdots + a_1x + a_0$ is any polynomial, how many derivatives would we need to take until we get to $0$?  Explain.











	
\index{Derivatives}
	
\begin{tags}
	    Derivatives, HigherDerivatives, Polynomial, Theory
\end{tags}
	
\begin{diary}
	    %S2016-HW1-Q1
\end{diary}
	
\begin{solution}
	   
\end{solution}
	
\end{question}

\end{tagblock}

%-------------------------------------------------------------------------------------------------------------

\begin{tagblock}{Derivatives, TangentLine, ProductRule, Graph}
\begin{question}
	




Using the definition of the derivative, we just saw how to take the derivative of a product:  
\bigskip



\noindent\fbox{%
    \parbox{\textwidth}{
\textbf{Product Rule:} If $P(x)=f(x) \cdot g(x)$, then $P'(x) = f'(x)g(x) + f(x) g'(x)$.  




\bigskip


In words what the product rule says: if $P$ is the product of two functions $f$ (the first function) and $g$ (the second), then ``the derivative of $P$ is the first times the derivative of the second, plus the second times the derivative of the first.''}}


\bigskip

Let $P(x) = (x^5+3x^2 - \frac{1}{x}) (\sqrt{x} + \frac{x}{3})$, which is graphed on the right.

 
\begin{minipage}{.6\textwidth}
\begin{enumerate}
\item  Use the product rule to compute $P'(x)$.  (It is not necessary to algebraically simplify) %(Note that one could multiply out $P(x)$ and then use our previous rules to compute $P'(x)$. ) 
\item Draw the tangent line to $P(x)$ at $x=1$.  
\item    Find slope of the tangent line to $P(x)$ at $x=1$.  
\item Give the equation of the tangent line to $P(x)$ at $x=1$.  
%\item  Determine the slope of the tangent line to the curve $y=f(x)$ at the point where $a=\pi$ if $f$ is given by the rule $f(x)=x^3\sin(x)$.
\end{enumerate}
\end{minipage}% This must go next to `\end{minipage}`
\begin{minipage}{.4\textwidth}
\includegraphics[width=\textwidth]{product1.png}
\end{minipage}
%\begin{figure}[h]
%\centering
%\end{figure}











	
\index{Derivatives}
	
\begin{tags}
	    Derivatives, TangentLine, ProductRule, Graph
\end{tags}
	
\begin{diary}
	    %S2016-HW1-Q1
\end{diary}
	
\begin{solution}
	   
\end{solution}
	
\end{question}

\end{tagblock}

%-------------------------------------------------------------------------------------------------------------

\begin{tagblock}{Derivatives, TangentLine, QuotientRule}
\begin{question}
	









Similarly, we have a rule for differentiating quotients

\noindent\fbox{%
    \parbox{\textwidth}{
\textbf{Quotient Rule:} If $\displaystyle Q(x) = \frac{f(x)}{g(x)}$ then $\displaystyle Q'(x) = \frac{g(x)f'(x)-f(x)g'(x)}{(g(x))^2} $.  

\bigskip

In words, if a function $Q$ is the quotient of a top function $f$ and a bottom function $g$, then $Q'$ is given by ``the bottom times the derivative of the top, minus the top times the derivative of the bottom, all over the bottom squared.''  }}

\bigskip

Use the quotient rule to answer each of the questions below.  It is not necessary to algebraically simplify any of the derivatives you compute.
\begin{enumerate}
\item  Let $\displaystyle f(z)=\frac{z^3}{z^4+1}.$  Find $f'(z)$.
\vspace{1in}
%\item Let $\displaystyle v(t)=\frac{\sin(t)}{\cos(t)+t^2}$.  Find $v'(t).$
\item Determine the slope of the tangent line to the curve $\displaystyle R(x)=\frac{x^2-2x-8}{x^2-9}$ at the point where $x=0$.  
\end{enumerate}





	
\index{Derivatives}
	
\begin{tags}
	    Derivatives, TangentLine, QuotientRule
\end{tags}
	
\begin{diary}
	    %S2016-HW1-Q1
\end{diary}
	
\begin{solution}
	   
\end{solution}
	
\end{question}

\end{tagblock}

%-------------------------------------------------------------------------------------------------------------

\begin{tagblock}{Derivatives, Graphs, Differentiable}
\begin{question}
	


So far we have been given a function via a formula and found a formula for the derivative.  We next will take a different perspective and ask  given a graph of $y=f(x)$, how does this graph lead to the graph of the derivative function $y=f'(x)$? 

\bigskip

Below on the left is a graph of $g(x) = |x|$.  On the right sketch a graph of $g'(x)$. \emph{Remember that the derivative gives slopes of tangent lines}  


\begin{figure}[h]
\centering
\includegraphics[width=5cm]{absolutevalue.png} \hspace{.2in} \includegraphics[width=5cm]{derivblank.png} \end{figure}

Write 2-3 sentences explaining how you know that you have determined the correct graph. 

\vspace{2in}
Notice that something funny is happening at $x=0$.  At $x=0$ we can't draw the tangent line to $g(x)$.  We say that $g$ is \emph{not differentiable at $x=0$}.

In general:  we say that a function $f(x)$ is \emph{differentiable at $x=a$} if $f'(a)$ exists.

\index{Derivatives}
	
\begin{tags}
	    Derivatives, Graphs, Differentiable
\end{tags}
	
\begin{diary}
	    %S2016-HW1-Q1
\end{diary}
	
\begin{solution}
	   
\end{solution}
	
\end{question}

\end{tagblock}


%-------------------------------------------------------------------------------------------------------------

\begin{tagblock}{Derivatives, Graphs, Differentiable, Theory, Continuity}
\begin{question}
	
{How can a function fail to be differentiable at $a$?}

\begin{enumerate}
\item \textbf{If we can't draw a tangent line to $f(x)$ at $x=a$:}  
Consider the function $f(x)$ given by the graph below:
\begin{minipage}{.4\textwidth}
\includegraphics[width=5cm]{notcts.png} 
\end{minipage}% This must go next to `\end{minipage}`
\begin{minipage}{.6\textwidth}
Is $f$ continuous at $x=1$? \\ \vspace{.5in}

 Can you draw a tangent line at $x=1$?  
\end{minipage}


The sharp corner seen in the previous problem also prevented us from drawing a tangent line.  
\bigskip


\item \textbf{$f$ has a vertical tangent line at $x=a$}: Consider the function $g(x)$ given by the graph below:
\begin{minipage}{.4\textwidth}
\includegraphics[width=5cm]{vertical.png} 
\end{minipage}% This must go next to `\end{minipage}`
\begin{minipage}{.6\textwidth}
Draw the tangent line to $g(x)$ at $x=1$.  Can you compute it's slope?  Why or why not?  

\end{minipage}




\item  Determine if the following statements are TRUE or FALSE:

If a function $h(x)$ is \textbf{continuous} at $x=a$, then it is \textbf{differentiable} at $x=a$. \hfill TRUE \hspace{.2in} FALSE

\bigskip


If a function $h(x)$ is \textbf{differentiable} at $x=a$, then it is \textbf{continuous} at $x=a$. \hfill TRUE \hspace{.2in} FALSE



\end{enumerate}













	









	
\index{Derivatives}
	
\begin{tags}
	    Derivatives, Graphs, Differentiable, Theory, Continuity
\end{tags}
	
\begin{diary}
	    %S2016-HW1-Q1
\end{diary}
	
\begin{solution}
	   
\end{solution}
	
\end{question}

\end{tagblock}

%-------------------------------------------------------------------------------------------------------------

\begin{tagblock}{Derivatives, QuotientRule, Theory}
\begin{question}
	
One can prove the Quotient Rule, but using the product rule.  Let $\displaystyle Q(x) = \frac{f(x)}{g(x)}$, where $f$ and $g$ are differentiable functions.   We begin by observing that we can multiply by $g(x)$ to get
\[ f(x) = Q(x) \cdot g(x) \]
If we differentiate  both sides, using the product rule on the right hand side we get
\[ f'(x) =   \rule[-0.1cm]{8cm}{0.01cm} \] 
Now solve the above equation for $Q'(x)$ (your answer will involve $f'(x), g'(x)$ and $Q(x)$.)

\vspace{1.5in}
Replacing $\displaystyle Q(x) = \frac{f(x)}{g(x)}$ and multiplying by a clever choice of $\displaystyle 1 = \frac{g(x)}{g(x)}$, we get
\begin{eqnarray*}
Q'(x) &=& \frac{f'(x)-Q(x)g'(x)}{g(x)}\\
&=&  \frac{f'(x)-\frac{f(x)}{g(x)}g'(x)}{g(x)} \cdot \frac{g(x)}{g(x)} \\
&=&  \frac{g(x)f'(x)-f(x)g'(x)}{(g(x))^2}  \\
\end{eqnarray*}
















	










	
\index{Derivatives}
	
\begin{tags}
	    Derivatives, QuotientRule, Theory
\end{tags}
	
\begin{diary}
	    %S2016-HW1-Q1
\end{diary}
	
\begin{solution}
	   
\end{solution}
	
\end{question}

\end{tagblock}

%-------------------------------------------------------------------------------------------------------------

\begin{tagblock}{Derivatives, QuotientRule, ProductRule}
\begin{question}
	



Let $f(x)$ and $g(x)$ be continuous functions with $f(-3) = 4$, $g(-3)=2$, $f'(-3) = \frac{1}{3}$ and $g'(-3) =-4$. 
\begin{enumerate}
\item  Let $P(x) = f(x)g(x) + x^2f(x)$.  
\begin{enumerate} 
\item  Determine a formula for $P'(x)$ in terms of $f(x), g(x), f'(x)$ and $g'(x)$.

\vspace{2in}

\item Determine $P'(-3)$.

\end{enumerate} 

\vspace{1.5in}

\item  Let $\displaystyle Q(x) = \frac{3f(x) + e^x}{g(x)}$.  Determine $Q'(-3)$


\end{enumerate} 














	










	
\index{Derivatives}
	
\begin{tags}
	    Derivatives, QuotientRule, ProductRule
\end{tags}
	
\begin{diary}
	    %S2016-HW1-Q1
\end{diary}
	
\begin{solution}
	   
\end{solution}
	
\end{question}

\end{tagblock}

%-------------------------------------------------------------------------------------------------------------

\begin{tagblock}{Derivatives, Trigonometry, Graphs, TangentLines}
\begin{question}
	






 In this worksheet we'll look at trig functions and their derivatives.  

\begin{enumerate}

\item Consider the function $f(x)=\sin(x)$, which is graphed in below. Note carefully that the grid in the diagram does not have boxes that are $1\times 1$, but rather approximately $1.57 \times 1$, as the horizontal scale of the grid is $\pi/2$ units per box. 

\begin{figure}[h]
%\centering
\includegraphics[width=8cm]{sinxwithtangent.png} \hfill \includegraphics[width=8cm]{blank.png}

\end{figure}
\begin{enumerate}
\item At each of $x=-2 \pi, -\frac{3\pi}{2}, - \pi, -\frac{\pi}{2}, 0, \frac{\pi}{2}, \pi, \frac{3\pi}{2}, 2\pi$ use a straightedge to sketch an accurate tangent line to $y=f(x)$.  (I've already drawn in the tangent line at $x=0$) 

\item Use the provided grid to estimate the slope of the tangent line you drew at $x= 0$, $x=\frac{\pi}{2}$, $x=\pi$ and $x=\frac{3 \pi}{2}$. Pay careful attention to the scale of the grid.  Notice that the tangent lines at $x=0$ and $x=2\pi$ are parallel, and hence will have the same slope.  Fill in the table below.  

\bigskip

\begin{tabular}{|c|c|c|c|c|c|c|c|c|c|} \hline
$x$-value & $-2 \pi$ & $-\frac{3\pi}{2}$ & $- \pi$ &  $-\frac{\pi}{2}$ &$ 0$ & $\frac{\pi}{2}$ & $\pi$ & $\frac{3\pi}{2}$& $2\pi$ \\ 
& \hspace{.3in} &  \hspace{.3in} & \hspace{.3in} & \hspace{.3in} & \hspace{.3in} & \hspace{.3in} & \hspace{.3in} & \hspace{.3in} & \hspace{.3in} \\ \hline 
slope of tangent & \hspace{.3in} &  \hspace{.3in} & \hspace{.3in} & \hspace{.3in} & \hspace{.3in} & \hspace{.3in} & \hspace{.3in} & \hspace{.3in} & \hspace{.3in}  \\
& \hspace{.3in} &  \hspace{.3in} & \hspace{.3in} & \hspace{.3in} & \hspace{.3in} & \hspace{.3in} & \hspace{.3in} & \hspace{.3in} & \hspace{.3in}\\
& \hspace{.3in} &  \hspace{.3in} & \hspace{.3in} & \hspace{.3in} & \hspace{.3in} & \hspace{.3in} & \hspace{.3in} & \hspace{.3in} & \hspace{.3in}\\ \hline
\end{tabular}
\bigskip

%\item Use the limit definition of the derivative to estimate $f'(0)$ by using small values of $h$, and compare the result to your visual estimate for the slope of the tangent line to $y=f(x)$ at $x=0$ in (b). Using periodicity, what does this result suggest about $f'(2\pi)$? about $f'(-2\pi)$?
\item Based on your work in (a) and (b), sketch an accurate graph of $y=f'(x)$ on the axes adjacent to the graph of $y=f(x)$.  

\item What familiar function do you think is the derivative of $f(x)=\sin(x)$?

\end{enumerate}
\newpage

One can perform a similar process looking at the slopes of tangent lines to the graph of $f(x) = \cos(x)$.  On the left is a graph of $f(x) = \cos(x)$ and on the right is a graph of $f'(x)$.  
\begin{figure}[h]
%\centering
\includegraphics[width=8cm]{cosx.png} \hfill \includegraphics[width=8cm]{negsinx.png}
\[f(x) = \cos(x) \hspace{2.5in} f'(x)\]

\end{figure}


\item What familiar function do you think is the derivative of $f(x)=\cos(x)$?


\end{enumerate}








	









	
\index{Derivatives}
	
\begin{tags}
	    Derivatives, Trigonometry, Graphs, TangentLines
\end{tags}
	
\begin{diary}
	    %S2016-HW1-Q1
\end{diary}
	
\begin{solution}
	   
\end{solution}
	
\end{question}

\end{tagblock}

%-------------------------------------------------------------------------------------------------------------

\begin{tagblock}{Derivatives, Trigonometry, Exponentials, TangentLines}
\begin{question}
	




In this worksheet we'll look at trig functions and their derivatives.  On the Preview Worksheet, you hopefully  found that the sine and cosine functions are  even further linked through calculus, as the derivative of each involves the other. The following rules summarize our findings:
 
\[ \frac{d}{dx}[ \sin(x)] = \cos(x) \, \text{ and }\, \frac{d}{dx}[ \cos(x)] = -\sin(x)\]





%The results of the two preceding activities suggest that the sine and cosine functions not only have the beautiful interrelationships that are learned in a course in trigonometry -- connections such as the identities $\sin^2(x)+\cos^2(x)=1$ and $\cos(x-\frac{\pi}{ 2} )=\sin(x)$ -- but that they are even further linked through calculus, as the derivative of each involves the other. The following rules summarize the results of the above two problems:
%\[ \frac{d}{dx}[ \sin(x)] = \cos(x) \, \text{ and }\, \frac{d}{dx}[ \cos(x)] = -\sin(x)\]

One can formally show these by going back to the definition of the derivative (like we did with the product rule), and using some trig identities and limits.  


\bigskip

We can now add these two new rules to our previous rules, and compute more derivatives efficiently.  

\bigskip

Answer each of the following questions. Where a derivative is requested, be sure to label the derivative function with its name using proper notation.

\begin{enumerate}
\item    Determine the derivative of $h(t)=3\cos(t)-4\sin(t)$.  

\vspace{.5in}

%\item Find the exact slope of the tangent line to $y=f(x)=2x+\frac{\sin(x)}{2}$ at the point where $x=\frac{\pi}{6}$.
\item Determine the derivative of $f(x)=x^3e^x\sin(x)$

\vspace{1.3in}


\item Find the equation of the tangent line to $y=g(x)=x^2+2\cos(x)$ at the point $(\frac{\pi}{2}, (\frac{\pi}{2})^2)$.

\vspace{2in}


%\item Let $p(t)=(\sin(t)+\cos(t))(t^4+3t^2)$. Find $p'(t)$. (No need to simplify your final answer) 
 \end{enumerate}









	









	
\index{Derivatives}
	
\begin{tags}
	    Derivatives, Trigonometry, Exponentials, TangentLines
\end{tags}
	
\begin{diary}
	    %S2016-HW1-Q1
\end{diary}
	
\begin{solution}
	   
\end{solution}
	
\end{question}

\end{tagblock}

%-------------------------------------------------------------------------------------------------------------

\begin{tagblock}{Derivatives, Trigonometry, Exponentials, QuotientRule}
\begin{question}
	






Recall that the trig functions $\tan(x)$, $\sec(x)$, $\csc(x)$ and $\cot(x)$ can all be expressed in terms of $\sin(x)$ and $\cos(x)$:
\[ \tan(x) = \frac{\sin(x)}{\cos(x)}, \hspace{.1in} \sec (x) = \frac{1}{\cos(x)}, \, \csc (x) = \frac{1}{\sin(x)}, \, \cot (x) = \frac{\cos(x)}{\sin(x)}\]



   We'll first develop a formula for the derivative of $f(x) = \tan(x) = \frac{\sin(x)}{\cos(x)}$.
\begin{enumerate}
%\item  What is the domain of $f(x)$?

%\vspace{.5in}

\item Using the quotient rule, we have that
\[f'(x) = \frac{\cos(x) \cos(x) + \sin(x) \sin(x)}{\cos^2(x)}\]

\vspace{1in}

\item What is our favorite trig identity? How can this identity be used to simplify the above expression for $f'(x)$?

\vspace{1.2in} 
\item How can we express $f'(x)$ in terms of the secant function?
\end{enumerate}







	









	
\index{Derivatives}
	
\begin{tags}
	    Derivatives, Trigonometry, Exponentials, QuotientRule
\end{tags}
	
\begin{diary}
	    %S2016-HW1-Q1
\end{diary}
	
\begin{solution}
	   
\end{solution}
	
\end{question}

\end{tagblock}

%-------------------------------------------------------------------------------------------------------------

\begin{tagblock}{Derivatives, Trigonometry, QuotientRule}
\begin{question}
	






Let $g(x) = \cot(x) = \frac{\cos(x)}{\sin(x)}$.
\begin{enumerate}
%\item  What is the domain of $g(x)$?
%\vspace{.5in}


\item Use the quotient rule to develop a formula for $g'(x)$ that is expressed completely in terms of $\sin(x)$ and $\cos(x).$

\vspace{1.5in}


\item How can you use other relationships among trigonometric functions to write $g'(x)$ only in terms of  the cosecant function.

\vspace{1.2in}


%\item What is the domain of $g'$? How does this compare to the domain of $g$?
\end{enumerate}







	









	
\index{Derivatives}
	
\begin{tags}
	    Derivatives, Trigonometry, QuotientRule
\end{tags}
	
\begin{diary}
	    %S2016-HW1-Q1
\end{diary}
	
\begin{solution}
	   
\end{solution}
	
\end{question}

\end{tagblock}

%-------------------------------------------------------------------------------------------------------------

\begin{tagblock}{Derivatives, Trigonometry, QuotientRule}
\begin{question}
	






Let $h= \sec(x) =  \frac{1}{\cos(x)}$
\begin{enumerate}
%\item What is the domain of $h$?
%\vspace{.5in}


\item Use the quotient rule to develop a formula for $h'(x)$ that is expressed completely in terms of $\sin(x)$ and $\cos(x).$
\vspace{1.5in}


\item How can you use other relationships among trigonometric functions to write $h'(x)$ only in terms of $\tan(x)$ and $\sec(x)$?

\vspace{1.2in}


%\item What is the domain of $h'$? How does this compare to the domain of $h$?
\end{enumerate}







	









	
\index{Derivatives}
	
\begin{tags}
	    Derivatives, Trigonometry, QuotientRule
\end{tags}
	
\begin{diary}
	    %S2016-HW1-Q1
\end{diary}
	
\begin{solution}
	   
\end{solution}
	
\end{question}

\end{tagblock}

%-------------------------------------------------------------------------------------------------------------

\begin{tagblock}{Derivatives, Trigonometry}
\begin{question}
	






Similarly, one can determine the derivative of $\csc(x)$ (It's practice exercise \# 17 ).  The quotient rule has thus enabled us to determine the derivatives of the tangent, cotangent, and secant expanding our overall library of basic functions we can differentiate. Fill in the table to summarize our new rules:
\bigskip

\begin{tabular}{ll}
$\frac{d}{dx}[\sin(x)] = \rule[-0.1cm]{5cm}{0.01cm}$  & $\frac{d}{dx}[\cos(x)] = \rule[-0.1cm]{5cm}{0.01cm} $ \\ \\
$\frac{d}{dx}[\tan(x)] = \rule[-0.1cm]{5cm}{0.01cm}$  & $\frac{d}{dx}[\cot(x)] = \rule[-0.1cm]{5cm}{0.01cm} $ \\ \\
$\frac{d}{dx}[\sec(x)] =  \rule[-0.1cm]{5cm}{0.01cm}$  & $\frac{d}{dx}[\csc(x)] = -\csc(x) \cot(x) $ \\
\end{tabular}
%\newpage

\bigskip
\textbf{In practice, you should remember the derivative of $\sin(x)$, $\cos(x)$ and $\tan(x)$.  }








	









	
\index{Derivatives}
	
\begin{tags}
	    Derivatives, Trigonometry
\end{tags}
	
\begin{diary}
	    %S2016-HW1-Q1
\end{diary}
	
\begin{solution}
	   
\end{solution}
	
\end{question}

\end{tagblock}

%-------------------------------------------------------------------------------------------------------------

\begin{tagblock}{Derivatives, Trigonometry, HigherDerivatives}
\begin{question}
	






Let $g(x) = \cos(x)$, find the $33^{rd}$ derivative of $g(x)$, $g^{[33]}(x)$.   \\ \emph{Hint: You really don't want to compute all $33$ derivatives.  Differentiate a few times and see if you detect a pattern}



\index{Derivatives}
	
\begin{tags}
	    Derivatives, Trigonometry, HigherDerivatives
\end{tags}
	
\begin{diary}
	    %S2016-HW1-Q1
\end{diary}
	
\begin{solution}
	   
\end{solution}
	
\end{question}

\end{tagblock}

\begin{tagblock}{Derivatives, ChainRule, WarmUp}
\begin{question}
	






We've developed many rules for computing derivatives.  For example we can compute the derivative of $f(x)=\sin(x)$ and $g(x) = x^2$, as well as combinations of the two.

\begin{enumerate}
\item Warm-up: Compute the derivative of 
\begin{enumerate}
\item $s(x) = 3x^2 - 5\sin(x)$
\vspace{.5in}

\item $p(x) = x^2\sin(x)$
\vspace{.75in}
\item $\displaystyle q(x) = \frac{\sin(x)}{x^2}$
\end{enumerate}
\vspace{.75in}
\end{enumerate}

Recall another way of making functions is by composing them.  For example, consider $C(x) = \sin(x^2)$.  Here we see that $C(x)$ is a composition:
\[x \to x^2 \to \sin(x^2) \]
We can write $C$ as $C(x) = f(g(x))$, where $g(x) = x^2$ and $f(x) = \sin(x)$.  We will call  $g$, the function that is first applied to $x$,  the \emph{inner function}, and $f$, the function that is applied to the result, the \emph{outer function.}

\smallskip
\textbf{Our question in this worksheet is the following: given a composite function $C(x)=f(g(x))$, how can we compute the derivative of $C(x)$?  We expect that it will involve $f, g, f'$ and $g'$. }

\bigskip

 We first need to be comfortable with decomposing a function.
\begin{enumerate}
\item[2.] For each function, decompose it as $f(g(x))$, that is determine the inner function $g(x)$ and the outer function $f(x)$.

\begin{tabular}{|l l | c| c| } \hline
& &  inner function $g(x)$ \hspace{.2in} & outer function $f(x)$ \hspace{.2in} \\ \hline
&&&\\

(a) & $\cos(x^4)$ && \\ 
&&&\\
&&&\\ \hline 
&&&\\

(b) & $\cos^4(x)$ && \\
&remember $\cos^4(x) = (\cos (x))^4$ &&\\
&&&\\ \hline
&&&\\
(c) & $\sqrt{3x-1}$  &
& \\ 
&&&\\
&&&\\ \hline
&&&\\

(d) & $(\tan(x) + x^3)^5$ &&\\ 
&&&\\  
&&&\\ \hline
\end{tabular} 

\end{enumerate}


\index{Derivatives}
	
\begin{tags}
	    Derivatives, ChainRule, WarmUp
\end{tags}
	
\begin{diary}
	    %S2016-HW1-Q1
\end{diary}
	
\begin{solution}
	   
\end{solution}
	
\end{question}

\end{tagblock}

%-------------------------------------------------------------------------------------------------------------

\begin{tagblock}{Derivatives, Trigonometry, HigherDerivatives}
\begin{question}
	






Let $g(x) = \cos(x)$, find the $33^{rd}$ derivative of $g(x)$, $g^{[33]}(x)$.   \\ \emph{Hint: You really don't want to compute all $33$ derivatives.  Differentiate a few times and see if you detect a pattern}



\index{Derivatives}
	
\begin{tags}
	    Derivatives, Trigonometry, HigherDerivatives
\end{tags}
	
\begin{diary}
	    %S2016-HW1-Q1
\end{diary}
	
\begin{solution}
	   
\end{solution}
	
\end{question}

\end{tagblock}

%-------------------------------------------------------------------------------------------------------------


\begin{tagblock}{Derivatives, ChainRule, Example, Trigonometry}
\begin{question}
	






As a motivating example, consider $C(x) = \sin(2x)$.  Using the double-angle trig identity we can rewrite 
\[C(x) = \sin(2x) = 2\sin(x)\cos(x) \]
Note that we have now expressed $C(x)$ as a product. Using the product rule to compute the derivative of $C(x) = 2\sin(x) \cos(x)$ gives
\[C'(x) = (2\cos(x))(\cos(x)) + (2\sin(x))(-\sin(x)) = 2(\cos^2(x) - \sin^2(x))\]
Recalling the other double-angle trig identity $\cos(2x) = \cos^2(x) - \sin^2(x) $, we determine that 
\[C'(x) = 2\cos(2x)\]

Next let's approach $C(x) = \sin(2x)$ slightly differently.

\begin{enumerate}
\item  Decompose $C(x)$ as$f(g(x))$, that is determine the inner function $g(x)$ and the outer function $f(x)$.

\bigskip

$g(x) = $

\bigskip
$f(x) = $

\bigskip

\item Compute $g'(x)$ and $f'(x)$.

\vspace{.75in}

\item For this example, can you express $C'(x)$ in terms of $f, g, f',$ and $g'$?

\end{enumerate}
\newpage
It turns out this holds in general:

\textbf{Chain Rule}  If $g$ is differentiable at $x$ and $f$ is differentiable at $g(x)$, then the composite function $C(x)=f(g(x))$ is differentiable at $x$ and
\[C'(x)=f'(g(x))g'(x).\]

\emph{In words the Chain Rule says the derivative of a composition is the ''derivative of the outer function, evaluated at the inner function, times the derivative of the inner function.'' }


\index{Derivatives}
	
\begin{tags}
	    Derivatives, ChainRule, Example, Trigonometry
\end{tags}
	
\begin{diary}
	    %S2016-HW1-Q1
\end{diary}
	
\begin{solution}
	   
\end{solution}
	
\end{question}

\end{tagblock}

%-------------------------------------------------------------------------------------------------------------


	








\begin{tagblock}{Derivatives, ChainRule, Trigonometry}
\begin{question}
	






Returning to the functions you decomposed in Problem 1, determine $g'(x)$, $f'(x)$, and $f'(g(x))$, and finally apply the chain rule to determine the derivative of the given function. 
\bigskip


\begin{tabular}{| l | c| c| c| c| c| c| } \hline
 $C(x)$ &  inner $g(x)$  & outer $f(x)$  & $g'(x)$ \hspace{.2in} & $f'(x)$ \hspace{.2in} & $f'(g(x))$ \hspace{.5in} & $C'(x)= f'(g(x))g'(x)$ \hspace{.3in}  \\ \hline
 &&&&&&\\

 $\cos(x^4)$ &&&&&& \\ 
&&&&&&\\
&&&&&&\\
&&&&&&\\ \hline 
&&&&&&\\

 $\cos^4(x)$ &&&&&& \\
&&&&&&\\
&&&&&&\\
&&&&&&\\ \hline 
&&&&&&\\

 $\sqrt{3x-1}$  &&&&&& \\ 
&&&&&&\\
&&&&&&\\
&&&&&&\\ \hline 
&&&&&&\\

 $(\tan(x) + x^3)^5$ &&&&&&\\ 
&&&&&&\\
&&&&&&\\
&&&&&&\\ \hline 
\end{tabular} 


In the last two parts of the previous problem, our outer function was of the form $x^n$.  This special case of the chain rule is also called the \textbf{Generalized Power Rule:}  \[\frac{d}{dx}[(g(x))^n]= n(g(x))^{n-1} g'(x) \]


\index{Derivatives}
	
\begin{tags}
	    Derivatives, ChainRule, Trigonometry
\end{tags}
	
\begin{diary}
	    %S2016-HW1-Q1
\end{diary}
	
\begin{solution}
	   
\end{solution}
	
\end{question}

\end{tagblock}

%-------------------------------------------------------------------------------------------------------------


	
\begin{tagblock}{Derivatives, ChainRule, Trigonometry}
\begin{question}
	

In the previous worksheet  we computed the derivative of $\displaystyle \sec(x) = \frac{1}{\cos(x)}$ by using the quotient rule.  Note that we could re-write $\sec(x) = (\cos(x))^{-1}$.  Compute the derivative of  $\sec(x) = (\cos(x))^{-1}$ using the Chain Rule/Generalized Power Rule.  Which way do you find easier? 


\index{Derivatives}
	
\begin{tags}
	    Derivatives, ChainRule, Trigonometry
\end{tags}
	
\begin{diary}
	    %S2016-HW1-Q1
\end{diary}
	
\begin{solution}
	   
\end{solution}
	
\end{question}

\end{tagblock}

%-------------------------------------------------------------------------------------------------------------


	
\begin{tagblock}{Derivatives, ChainRule}
\begin{question}
	

Compute the derivative of $\displaystyle C(x) = \frac{4}{\sqrt{5x^3+7x}}$.  (Note: You can rewrite this first, like in the previous problem). 


\index{Derivatives}
	
\begin{tags}
	    Derivatives, ChainRule
\end{tags}
	
\begin{diary}
	    %S2016-HW1-Q1
\end{diary}
	
\begin{solution}
	   
\end{solution}
	
\end{question}

\end{tagblock}

%-------------------------------------------------------------------------------------------------------------


	
\begin{tagblock}{Derivatives, ChainRule, QuotientRule, Trigonometry}
\begin{question}
	

Compute the derivative of $\displaystyle H(x) =  \cos \left( \frac{1-x^2}{1+3x^2} \right)$.


\index{Derivatives}
	
\begin{tags}
	    Derivatives, ChainRule, QuotientRule, Trigonometry
\end{tags}
	
\begin{diary}
	    %S2016-HW1-Q1
\end{diary}
	
\begin{solution}
	   
\end{solution}
	
\end{question}

\end{tagblock}

%-------------------------------------------------------------------------------------------------------------


	
\begin{tagblock}{Derivatives, ChainRule, ProductRule, Trigonometry}
\begin{question}
	

Compute the derivative of $G(x) = \tan(5x)(6x+13)^{33}$  Note: we need to start with the product rule


\index{Derivatives}
	
\begin{tags}
	    Derivatives, ChainRule, ProductRule, Trigonometry
\end{tags}
	
\begin{diary}
	    %S2016-HW1-Q1
\end{diary}
	
\begin{solution}
	   
\end{solution}
	
\end{question}

\end{tagblock}

%-------------------------------------------------------------------------------------------------------------


	
\begin{tagblock}{Derivatives, ChainRule, Trigonometry}
\begin{question}
	

Compute the derivative of $F(x) = \sec^2(4x) = (\sec(4x))^2$.  Note: Here we need to use the Chain Rule twice! 


\index{Derivatives}
	
\begin{tags}
	    Derivatives, ChainRule, Trigonometry
\end{tags}
	
\begin{diary}
	    %S2016-HW1-Q1
\end{diary}
	
\begin{solution}
	   
\end{solution}
	
\end{question}

\end{tagblock}

%-------------------------------------------------------------------------------------------------------------


	
\begin{tagblock}{Derivatives, ChainRule, Exponentials}
\begin{question}
	

We have seen that $\frac{d}{dx}[e^x] = e^x$.  As an application of the Chain Rule we can now determine the derivative of $f(x) = a^x$, where $a>0$.  Recall that 
\[f(x) = a^x = e^{\ln(a^x)} = e^{x \ln(a)} \]

Now use the Chain Rule!


\index{Derivatives}
	
\begin{tags}
	    Derivatives, ChainRule, Exponentials
\end{tags}
	
\begin{diary}
	    %S2016-HW1-Q1
\end{diary}
	
\begin{solution}
	   
\end{solution}
	
\end{question}

\end{tagblock}

%-------------------------------------------------------------------------------------------------------------


	
\begin{tagblock}{Derivatives, ChainRule, Exponentials, ProductRule, Trigonometry}
\begin{question}
	

Compute the derivative of $G(x) = \tan(5x) e^{7x^2}$. 


\index{Derivatives}
	
\begin{tags}
	    Derivatives, ChainRule, Exponentials, ProductRule, Trigonometry
\end{tags}
	
\begin{diary}
	    %S2016-HW1-Q1
\end{diary}
	
\begin{solution}
	   
\end{solution}
	
\end{question}

\end{tagblock}

%-------------------------------------------------------------------------------------------------------------


	
\begin{tagblock}{Derivatives, ChainRule, Exponentials, ProductRule, Trigonometry, QuotientRule, Challenge}
\begin{question}
	

For each of the following functions, find the function's derivative. State the rule(s) you use, label relevant derivatives appropriately, and be sure to clearly identify your overall answer.
 \begin{enumerate}

\item $ f(x) = \sqrt{5^x + \sqrt{5x}}$

\vspace{1in}
\item $g(x) = \sin(x^2)\cos(x^3)$

\vspace{1in}

\item $ \displaystyle Q(x) = \frac{\sec(5x)}{3x^2 + \sqrt{7}}$
\vspace{1in}


\item $ P(x) = 3x^2 \tan^4(x)$
\vspace{1in}
\item $ y = \sin(\sin (\sin (x)))$
\vspace{1in}


\end{enumerate}


\index{Derivatives}
	
\begin{tags}
	    Derivatives, ChainRule, Exponentials, ProductRule, Trigonometry, QuotientRule, Challenge
\end{tags}
	
\begin{diary}
	    %S2016-HW1-Q1
\end{diary}
	
\begin{solution}
	   
\end{solution}
	
\end{question}

\end{tagblock}

%-------------------------------------------------------------------------------------------------------------


	
\begin{tagblock}{Derivatives, ChainRule, Trigonometry, Challenge, HigherDerivatives}
\begin{question}
	

 Let $g(x) = \cos(6x)$, determine $g^{[33]}(x)$.


\index{Derivatives}
	
\begin{tags}
	    Derivatives, ChainRule, Trigonometry, Challenge, HigherDerivatives
\end{tags}
	
\begin{diary}
	    %S2016-HW1-Q1
\end{diary}
	
\begin{solution}
	   
\end{solution}
	
\end{question}

\end{tagblock}

%-------------------------------------------------------------------------------------------------------------


	
\begin{tagblock}{Derivatives, ChainRule, TangentLines, ImplicitDifferentiation, Graph, Example}
\begin{question}
	

 In our study of derivatives, we've learned
\begin{itemize}
\item How to efficiently take derivatives of functions of the form $y=f(x)$, and
\item Given a function $y=f(x)$, the slope of the the tangent line of $f(x)$ at the point $(a,f(a))$ is given by $f'(a)$.
\end{itemize}


In this worksheet we'll look at other types of curves.  

\bigskip

Warm-up: Let $f(x)$ be a differentiable function.  Using our derivative rules, compute the derivative of the following functions, which are built from $f(x)$.  Your answer may involve $f(x)$ and it's derivative $f'(x)$.  
\begin{enumerate}
\item $\displaystyle \frac{d}{dx}[x^2 + f(x) ] = $
\vspace{.25in}

\item $\displaystyle \frac{d}{dx}[xf(x) ] = $
\vspace{.25in}

\item $\displaystyle \frac{d}{dx}[(f(x))^2 ] = $
\vspace{.25in}

\item $\displaystyle \frac{d}{dx}[\sin(x + f(x) )] = $
%\vspace{.25in}

%\item $\displaystyle \frac{d}{dx}[(x^2+f(x))^2]$
\end{enumerate}
\vspace{.3in}

\bigskip

\begin{enumerate}

\item Consider the following curves: the circle centered at the origin of radius $3$:  $x^2+y^2 = 9$; and the \emph{lemniscate} curve $x^3-y^3=6xy$

\begin{figure}[h]
%\centering
\includegraphics[width=6cm]{implicitcircle.png} \hfill \includegraphics[width=6cm]{implicitl.png}

\end{figure}
\begin{enumerate}
\item For either of these two curves, can you solve for $y$ as a function of $x$?
\vspace{.5in}

\item On the circle, draw a tangent line at the point $(\sqrt{5}, 2)$, and draw a tangent line at the point $(0,-3)$
\item On the lemniscate, draw a tangent line at the point $(-3,3)$.  
\end{enumerate}

\bigskip

Curves like the circle and the lemniscate that are \textbf{not} of the form $y=f(x)$ for a function $f$ are called \emph{implicit curves}.  

As you saw above, we can still draw tangent lines to implicit curves, so it makes sense to ask: 

\textbf{Question:} How can we find the slope of a tangent line to an implicit curve?  

\smallskip
\textbf{Answer:  Use Implicit Differentiation}  

We will  interested in finding an equation for $\frac{dy}{dx}$ that tells us the slope of the tangent line to the curve at a point $(x,y)$. To do so, it will be necessary for us to work with $y$ while thinking of $y$ as a function of $x$, but without being able to write an explicit formula for $y$ in terms of $x.$  This process is called \emph{implicit differentiation}.  

\newpage

\item Returning to the circle $x^2+y^2 = 9$.  We'll start by taking the derivative with respect to $x$ of both sides of the equation:
\[\frac{d}{dx} [x^2+y^2] = \frac{d}{dx}[9] \]
Using the Sum Rule and the fact that the derivative of a constant is $0$ gives
\[ \frac{d}{dx} [x^2]+\frac{d}{dx}[y^2] = 0\]
Since $x$ is the independent variable, it is the variable with respect to which we are differentiating, and thus $\frac{d}{dx} [x^2] = 2x$.  
But $y$ is the dependent variable and $y$ is an implicit function of $x$. Thus, to compute $\frac{d}{dx} [ y^2]$ we need the Chain Rule, just like you did in 1(c):  $\frac{d}{dx} [ y^2] = 2y^1 \frac{dy}{dx} = 2y \frac{dy}{dx}$.  Hence 
\[2x+2y \frac{dy}{dx} = 0 \]
Now solve the equation for $\frac{dy}{dx}$.
\vspace{1in}

Note that $\frac{dy}{dx}$ depends on both $x$ and $y$.  To determine the slope of the tangent line at a given point, we will need to evaluate $\frac{dy}{dx}$ at both the $x$ and $y$ coordinate of the point.
\begin{enumerate}
\item Find the slope of the tangent line to the circle at $(\sqrt{5},2)$, by evaluating $\frac{dy}{dx}$ at $x=\sqrt{5}$ and $y=2$.  (We often write this as $\frac{dy}{dx}|_{(\sqrt{5},2)}$).  Does this agree with the your line that you drew in 3(b)?
\vspace{.5in}


\item Find the slope of the tangent line to the circle at $(0,-3)$?  Does this agree with the your line that you drew in 3(b)?
\vspace{.5in}

\item Using the equation for $\frac{dy}{dx}$, find all points (both $x$ and $y$-coordinates) on the circle where the tangent line is a horizontal line.  
\vspace{1in}


\item Using the equation for $\frac{dy}{dx}$, find all points on the circle where the tangent line is a vertical line.
\end{enumerate}

\vfill
Note for (c) and (d) you can see this fairly clearly on the graph of the circle, but make sure you can also find the points using the equation for $\frac{dy}{dx}$.  

\end{enumerate}


\index{Derivatives}
	
\begin{tags}
	    Derivatives, ChainRule, TangentLines, ImplicitDifferentiation, Graph, Example
\end{tags}
	
\begin{diary}
	    %S2016-HW1-Q1
\end{diary}
	
\begin{solution}
	   
\end{solution}
	
\end{question}

\end{tagblock}

%-------------------------------------------------------------------------------------------------------------


	
\begin{tagblock}{Derivatives, ChainRule, TangentLines, ImplicitDifferentiation, Graph}
\begin{question}
	

Returning to the lemniscate curve $x^3-y^3=6xy$, find an equation for $\frac{dy}{dx}$.   Then find the slope of the tangent line at the point $(-3,3)$.  

\emph{Hints:}
\begin{itemize}
\item \emph{On the right hand side you will need to use the Product Rule like you did in 1(b).}
\item \emph{After differentiating you will have a $\frac{dy}{dx}$ term on both sides: get all the terms with a $\frac{dy}{dx}$ on one side of the equality and those without on the other side; then we can factor out the $\frac{dy}{dx}$ and divide.  This is the same process you did in 2. when you solved for $z$. }  
\end{itemize}



\index{Derivatives}
	
\begin{tags}
	    Derivatives, ChainRule, TangentLines, ImplicitDifferentiation, Graph
\end{tags}
	
\begin{diary}
	    %S2016-HW1-Q1
\end{diary}
	
\begin{solution}
	   
\end{solution}
	
\end{question}

\end{tagblock}

%-------------------------------------------------------------------------------------------------------------


	
\begin{tagblock}{Derivatives, ChainRule, TangentLines, ImplicitDifferentiation, Graph, Trigonometry}
\begin{question}
	

For the curve $\sin(x+y)+\cos(x-y) = 1$, use implicit differentiation to find an equation for $\frac{dy}{dx}$, and then find the slope of the tangent line to the curve  at $(\frac{\pi}{2},\frac{\pi}{2})$.  Draw the tangent line at $(\frac{\pi}{2},\frac{\pi}{2})$ and make sure it agrees with your answer.  

\begin{figure}[h]
\centering
\includegraphics[width=6cm]{implicitsin.png}
\end{figure}



\index{Derivatives}
	
\begin{tags}
	    Derivatives, ChainRule, TangentLines, ImplicitDifferentiation, Graph, Trigonometry
\end{tags}
	
\begin{diary}
	    %S2016-HW1-Q1
\end{diary}
	
\begin{solution}
	   
\end{solution}
	
\end{question}

\end{tagblock}

%-------------------------------------------------------------------------------------------------------------


	
\begin{tagblock}{Derivatives, ChainRule, InverseTrig}
\begin{question}
	

Compute the derivative of $g(x) = \sin^{-1}( x^7)$.



\index{Derivatives}
	
\begin{tags}
	    Derivatives, ChainRule, InverseTrig
\end{tags}
	
\begin{diary}
	    %S2016-HW1-Q1
\end{diary}
	
\begin{solution}
	   
\end{solution}
	
\end{question}

\end{tagblock}
%-------------------------------------------------------------------------------------------------------------


	
\begin{tagblock}{Derivatives, ChainRule, ProductRule, InverseTrig}
\begin{question}
	

Compute the derivative of $y=\sqrt{x} \cos^{-1}(\pi x)$. 



\index{Derivatives}
	
\begin{tags}
	    Derivatives, ChainRule, ProductRule, InverseTrig
\end{tags}
	
\begin{diary}
	    %S2016-HW1-Q1
\end{diary}
	
\begin{solution}
	   
\end{solution}
	
\end{question}

\end{tagblock}

%-------------------------------------------------------------------------------------------------------------


	
\begin{tagblock}{Derivatives, ChainRule, ProductRule, Exponentials}
\begin{question}
	

Compute the derivative of $g(x) = a^x e^{-bx+12}$, where $a$ and  $b$ are constants.



\index{Derivatives}
	
\begin{tags}
	    Derivatives, ChainRule, ProductRule, Exponentials
\end{tags}
	
\begin{diary}
	    %S2016-HW1-Q1
\end{diary}
	
\begin{solution}
	   
\end{solution}
	
\end{question}

\end{tagblock}

%-------------------------------------------------------------------------------------------------------------


	
\begin{tagblock}{Derivatives, ChainRule, ProductRule, ImplicitDifferentiation, TangentLines}
\begin{question}
	

Find $\frac{dy}{dx}$ for the implicit curve $e^y\sin(x)= y+xy$, and then find the slope of the tangent  line at the point $(0,0)$.



\index{Derivatives}
	
\begin{tags}
	    Derivatives, ChainRule, ProductRule, ImplicitDifferentiation, TangentLines
\end{tags}
	
\begin{diary}
	    %S2016-HW1-Q1
\end{diary}
	
\begin{solution}
	   
\end{solution}
	
\end{question}

\end{tagblock}

%-------------------------------------------------------------------------------------------------------------


	
\begin{tagblock}{Derivatives, ChainRule, ProductRule, Exponentials, QuotientRule, Theory}
\begin{question}
	

For each of the following, find a formula for the derivative of $f$, $f'(x)$, in terms of $g(x)$ and $g'(x)$.  If we additionally know that $g(2)=-1$ and $g'(2)=10$, determine $f'(2)$.  

\begin{enumerate}
\item $f(x) = x^2g(x)$ 

\vspace{1.25in}
\item $f(x) = (g(x))^2$

\vspace{1.25in}

%\item $f(x) = g(e^x)$

%\vspace{1.25in}

\item $f(x) = e^{g(x)}$ 

\vspace{1.25in}

\item $f(x) = g(g(x))$

\vspace{1.25in}

\item $\displaystyle f(x) = \frac{g(x)}{x^2+1}$
\end{enumerate}



\index{Derivatives}
	
\begin{tags}
	    Derivatives, ChainRule, ProductRule, Exponentials, QuotientRule, Theory
\end{tags}
	
\begin{diary}
	    %S2016-HW1-Q1
\end{diary}
	
\begin{solution}
	   
\end{solution}
	
\end{question}

\end{tagblock}

%-------------------------------------------------------------------------------------------------------------


	
\begin{tagblock}{Derivatives, RelatedRates, WarmUp}
\begin{question}
	

\textbf{Motivating Question:} If two quantities that are related, such as the radius and volume of a spherical balloon, are both changing as implicit functions of time, how are their rates of change related? That is, how does the relationship between the values of the quantities affect the relationship between their respective derivatives with respect to time?

\bigskip

In the last worksheet we introduced implicit differentiation and saw that for implicit curves $\frac{dy}{dx}$ evaluated at a point gave us the slope of the tangent line to the curve at that point.  

\bigskip

We are next going to consider situations where multiple quantities are related to one another and changing, but where each quantity can be considered an implicit function of the variable $t$, which represents time. Through knowing how the quantities are related, we will be interested in determining how their respective rates of change with respect to time are related. 

\bigskip

\textbf{Problem:}  Suppose that air is being pumped into a spherical balloon in such a way that its volume increases at a constant rate of $20$ cubic inches per second. It makes sense that since the balloon's volume and radius are related, by knowing how fast the volume is changing, we ought to be able to relate this rate to how fast the radius is changing. More specifically, can we find how fast the radius of the balloon is increasing at the moment the balloon's radius is 6 inches?

\bigskip
We'll do this in a number of steps and use this same strategy for other related rates problems.  Perhaps the challenge to these problems is translating the ``words'' into ``equations.''

\bigskip

\begin{enumerate}
\item {\bf Diagram: } Draw several spheres with different radii, and observe that as volume changes, the radius, diameter, and surface area of the balloon also change.  Label the radius, $r$, in each of your pictures.  

\vspace{2in}



\item {\bf Rates:} Identify rates given in the problem and the rate you need to compute. You may use units to help decide which numbers are rates.\\

\smallskip

Note that in the setting of this problem, both the volume, $V$ and the radius $r$ are changing as time $t$ changes, and thus both $V$ and $r$ may be viewed as implicit functions of $t$, with respective derivatives $\frac{dV}{dt}$ and $\frac{dr}{dt}$.  Recall that we are given in the problem that the balloon is being inflated at a constant rate of $20$ cubic inches per second. Is this rate the value of $\frac{dr}{dt}$ or $\frac{dV}{dt}$? Why?  What is the rate that we want to find?  

\vspace{1in}

\item {\bf Equation:} Write an equation relating the variables you identified. \\

\smallskip

Recall that the volume of a sphere of radius r is   $V=\frac{4}{3} \pi r^3$

\vspace{.5in}

\item {\bf Differentiate:} It wouldn't be calculus without this step! Go ahead and differentiate the equation that relates your variables.


Differentiate both sides of the equation $V=\frac{4}{3} \pi r^3$ with respect to $t$ (using implicit differentiation) to find a formula for $\frac{dV}{dt}$ that depends on both $r$ and $\frac{dr}{dt}$.

\vspace{2in}

At this point in the problem, by differentiating we have ``related the rates'' of change of $V$ and $r$, hence the name \emph{related rates}.   

\item {\bf Substitute and solve:} Plug in all known quantities into the equation from the last step. Solve for the desired rate and answer the question!  Don't forget your units on the final answer.  


\end{enumerate}
\vspace{3in}
We'll look at more complicated examples in the next worksheet, but the strategy we outlined will be the same.  



\index{Derivatives}
	
\begin{tags}
	    Derivatives, RelatedRates, WarmUp
\end{tags}
	
\begin{diary}
	    %S2016-HW1-Q1
\end{diary}
	
\begin{solution}
	   
\end{solution}
	
\end{question}

\end{tagblock}

%-------------------------------------------------------------------------------------------------------------


	
\begin{tagblock}{Derivatives, RelatedRates}
\begin{question}
	

Car A is traveling west at $50$ $mi/hr$ and car B is traveling north at $60$ $mi/hr$. Both are headed toward the intersection of the two roads. At what rate are the cars approaching each other when car A is $0.3$ $mi$ and car B is $0.4$ $mi$ from the intersection?
 \begin{enumerate}
 \item Diagram with labels:
 \vspace{1.5in}
 \item Rates: (Note that each car is getting closer to the intersection, so their rates will be negative)
 \vspace{1in}
 \item Equation:
 \vspace{1in}
 \item Differentiate:
 \vspace{1.5in}
 \item Substitute and solve:
 \vspace{2in}
\end{enumerate}




\index{Derivatives}
	
\begin{tags}
	    Derivatives, RelatedRates
\end{tags}
	
\begin{diary}
	    %S2016-HW1-Q1
\end{diary}
	
\begin{solution}
	   
\end{solution}
	
\end{question}

\end{tagblock}

%-------------------------------------------------------------------------------------------------------------


	
\begin{tagblock}{Derivatives, RelatedRates}
\begin{question}
	

Given an equation relating variables and knowing how one variable changes, we can use calculus to detect the rate of change of another variable. The problems in this worksheet ask you to compute rates of change by using the steps outlined in our Introduction to Related Rates Worksheet: 
\begin{enumerate}
\item {\bf Diagram: }Draw a picture of problem. Label important variables.
\item {\bf Rates:} Identify rates given in the problem and the rate you need to compute. You may use units to help decide which numbers are rates.
\item {\bf Equation:} Write an equation relating the variables you identified.
\item {\bf Differentiate:} It wouldn't be calculus without this step! Go ahead and differentiate the equation that relates your variables.
\item {\bf Substitute and solve:} Plug in all known quantities into the equation from the last step. Solve for the desired rate and answer the question.
\end{enumerate}

\bigskip



A $5$ foot ladder is leaning against a vertical wall and starts to fall.  If the bottom of the ladder moves away from the wall at $2$ ft/sec, how fast is the top of the ladder sliding when the bottom of the ladder is $4$ feet from the wall?
\begin{enumerate}
\item Diagram: Draw a picture of the ladder, wall and floor. Label all variables of interest: the distance from the floor to the top of the ladder and the distance from the wall to the bottom of the ladder.  
\vspace{1.5in}
\item Rates: Which rate of change is given? Which rate of change do you need?
\vspace{1in}
\item Equation: Note that the ladder, wall and floor make a right triangle.  Do you remember the Pythagorean Theorem?  
\vspace{1in}
\item Differentiate: Since the ladder is moving the distance from the floor to the top of the ladder and the distance from the wall to the bottom of the ladder are really functions of time. Differentiate your formula from the last part with respect to time, $t$.
\vspace{1.5in}
\item Substitute and solve: Substitute the numbers given in the problem into your last equation and solve for the desired rate.  Don't forget your units!
\vspace{3in}
\item Your final rate should be negative, why does that make sense?  
\end{enumerate}




\index{Derivatives}
	
\begin{tags}
	    Derivatives, RelatedRates
\end{tags}
	
\begin{diary}
	    %S2016-HW1-Q1
\end{diary}
	
\begin{solution}
	   
\end{solution}
	
\end{question}

\end{tagblock}

%-------------------------------------------------------------------------------------------------------------


	
\begin{tagblock}{Derivatives, RelatedRates}
\begin{question}
	

A water tank has the shape of an inverted right circular cone with height $4$ $m$. If water is being pumped into the tank at a rate of $2$ $m^3/min$, find the rate at which the water level is rising when the water is $3$ $m$ deep. (Recall the volume of a right cone is $V=\frac{1}{3}\pi r^2 h$, and the height will always be twice the radius.)
\begin{enumerate}
 \item Diagram with labels:
 \vspace{1.in}
 \item Rates:
 \vspace{1in}
 \item Equation:  
 \vspace{1in}
 \item Differentiate:
 \vspace{1.5in}
 \item Substitute and solve:
 \vspace{1in}
\end{enumerate}




\index{Derivatives}
	
\begin{tags}
	    Derivatives, RelatedRates
\end{tags}
	
\begin{diary}
	    %S2016-HW1-Q1
\end{diary}
	
\begin{solution}
	   
\end{solution}
	
\end{question}

\end{tagblock}

%-------------------------------------------------------------------------------------------------------------


	
\begin{tagblock}{Derivatives, RelatedRates, Challenge}
\begin{question}
	

A boy on a skateboard rolls away from a $15$ $ft$ lamppost at a speed of $3$ $ft/s$. The boy's height on the skateboard is $6$ feet. Find the rate at which his shadow is increasing in length. \\ See \url{http://webspace.ship.edu/msrenault/GeoGebraCalculus/derivative_app_rr_streetlamp.html}
 for some helpful animation.




\index{Derivatives}
	
\begin{tags}
	    Derivatives, RelatedRates, Challenge
\end{tags}
	
\begin{diary}
	    %S2016-HW1-Q1
\end{diary}
	
\begin{solution}
	   
\end{solution}
	
\end{question}

\end{tagblock}

%-------------------------------------------------------------------------------------------------------------


	
\begin{tagblock}{Derivatives, RelatedRates}
\begin{question}
	

The base of a triangle is shrinking at a rate of 1 cm/min and the height of the triangle is increasing at a rate of 5 cm/min. Find the rate at which the area of the triangle changes when the height is 22 cm and the base is 10 cm.



\index{Derivatives}
	
\begin{tags}
	    Derivatives, RelatedRates
\end{tags}
	
\begin{diary}
	    %S2016-HW1-Q1
\end{diary}
	
\begin{solution}
	   
\end{solution}
	
\end{question}

\end{tagblock}

%-------------------------------------------------------------------------------------------------------------


	
\begin{tagblock}{Derivatives, MaxMin, Graph, TangentLines, Differentiable, Absolute}
\begin{question}
	

Let $f(x)$ be given by the graph below, defined on the interval $[-3,3]$.  Use the graph to answer each of the following questions.
\begin{figure}[h]
\centering
\includegraphics[width=8cm]{minmax1.png} 
\end{figure}

\begin{enumerate}
\item Identify all $x$-value(s) $c$ at which $f(x)$ has an \textbf{absolute maximum}.

\vspace{.25in}
\item Identify all  $x$-value(s) $c$ at which $f(x)$ has an \textbf{absolute minimum}.
\vspace{.25in}

\item Identify all $x$-value(s) $c$ at which $f(x)$ has an \textbf{local maximum}.
\vspace{.25in}

\item Identify all  $x$-value(s) $c$ at which $f(x)$ has an \textbf{local minimum}.
\vspace{.25in}

\item Identify all values of $c$ for which $f'(c)$ does not exist.  (Remember the derivative won't exist at $c$ if we have a discontinuity at $c$, a corner at $c$, or a vertical tangent line at $c$).

\vspace{.25in}

\item Identify all values of $c$ for which $f'(c) = 0$.

\vspace{.25in}




\end{enumerate}


\index{Derivatives}
	
\begin{tags}
	   Derivatives, MaxMin, Graph, TangentLines, Differentiable, Absolute
\end{tags}
	
\begin{diary}
	    %S2016-HW1-Q1
\end{diary}
	
\begin{solution}
	   
\end{solution}
	
\end{question}

\end{tagblock}

%-------------------------------------------------------------------------------------------------------------


	
\begin{tagblock}{Derivatives, MaxMin, Graph, Continuity, Differentiable, Absolute}
\begin{question}
	

As you saw in the previous problem, determining absolute minimums and maximums from a \emph{graph} of a function isn't too difficult.  Our next goal is to use the \emph{equation} of a function to determine absolute and local minimums and maximums.  

\bigskip

In this problem we'll look at absolute minimum and maximums.  
\begin{enumerate}
\item Consider the function $f(x) = x^3$.  Does $f(x)$ have an absolute maximum?  Does $f(x)$ have an absolute minimum?  Why or why not?  (Hint: Think of the graph of $f(x)$)

\vspace{1.5in}
\item Now consider $f(x)=x^3$ on the closed interval $[-2,1]$.  Does $f(x)$ now have an absolute maximum on $[-2,1]$? Does $f(x)$ have an absolute minimum on $[-2,1]$?  If so, what is the absolute maximum and absolute minimum?  

\vspace{1in}
\item Consider the function $g(x)$ given by the graph below, which has a vertical asymptote at $x=3$.  

\begin{minipage}{.4\textwidth}
\includegraphics[width=6cm]{minmax2.png}\end{minipage}% This must go next to `\end{minipage}`
\begin{minipage}{.6\textwidth}
\begin{enumerate}
\item Is $g(x)$ continuous on $[1,3]$? \\ If not, where are the discontinuties?

\vspace{.5in}

\item Does $g(x)$ have an absolute minimum on $[1,3]$?  \\
Does $g(x)$ have an absolute maximum on $[1,3]$?  

\end{enumerate}

\end{minipage}

\end{enumerate}

\vfill

This problem gives us an example of the \\

\bigskip
\noindent\fbox{%
    \parbox{\textwidth}{
\textbf{Extreme Value Theorem}:  If $f$ is a continuous function on a closed interval $[a,b]$, then $f$ has both an absolute minimum and absolute maximum on $[a,b]$.  }}






\index{Derivatives}
	
\begin{tags}
	   Derivatives, MaxMin, Graph, Continuity, Differentiable, Absolute
\end{tags}
	
\begin{diary}
	    %S2016-HW1-Q1
\end{diary}
	
\begin{solution}
	   
\end{solution}
	
\end{question}

\end{tagblock}

%-------------------------------------------------------------------------------------------------------------


	
\begin{tagblock}{Derivatives, MaxMin, Graph, Differentiable, Definition, Critical}
\begin{question}
	

Returning to the function $f(x)$ given by

\bigskip

\begin{figure}[h]
\centering
\includegraphics[width=8cm]{minmax1.png} 
\end{figure}



\begin{enumerate}
\item True or false: every local maximum and minimum of $f$ occurs at a point where $f'(c)$ is either zero or does not exist.

\vspace{.25in}

\item True or false: at every point where $f'(c)$ is zero or does not exist, $f$ has a local maximum or minimum.

\vspace{.25in}


\end{enumerate}

In fact, the general statement is true and is known as \textbf{Fermat's Theorem}

\bigskip

\noindent\fbox{%
    \parbox{\textwidth}{
\textbf{Fermat's Theorem}  If $f(x)$ has a local minimum or maximum at $x=c$, and if $f'(c)$ exists, then $f'(c) =0$.}}

\bigskip

The upshot of Fermat's Theorem, is that we can make a list of candidate points where a local minimum or maximum may occur.  These points are so important, that we call them \emph{critical numbers}.

\bigskip

\textbf{Definition}  A \emph{critical number} of a function $f(x)$ is a number $x=c$ where $f'(c) =0$ or $f'(c)$ does not exist.




\index{Derivatives}
	
\begin{tags}
	   Derivatives, MaxMin, Graph, Differentiable, Definition, Critical
\end{tags}
	
\begin{diary}
	    %S2016-HW1-Q1
\end{diary}
	
\begin{solution}
	   
\end{solution}
	
\end{question}

\end{tagblock}

%-------------------------------------------------------------------------------------------------------------


	
\begin{tagblock}{Derivatives, MaxMin, Absolute}
\begin{question}
	


Consider the function $\displaystyle f(x) = \frac{1}{4} x^4 - \frac{2}{3} x^3 - 4x^2$.
\begin{enumerate}
\item Compute the derivative $f'(x)$.
\vspace{1in}
\item Are there any $x$ values for which $f'(x)$ is undefined?
\vspace{1in}
\item Find all the $x$ values where $f'(x) = 0$.
\vspace{1in}
\item What are the critical numbers of $f(x)$?
\end{enumerate}



\newpage
\textbf{How to Find Absolute Minimum and Maximum of a function $f$ on a closed interval $[a,b]$.}
\begin{itemize}
\item Find all the critical numbers $c$ of $f(x)$.
\item For each critical number $c$ in the interval $[a,b]$, compute the $y$-value $f(c)$.
\item Compute the $y$-value of the endpoints $f(a)$ and $f(b)$.
\item Compare all the $y$-values: the largest $y$-value will give you the absolute maximum, the smallest $y$-value will give you the absolute minimum.  
\end{itemize}

\bigskip

Returning to the function $\displaystyle f(x) = \frac{1}{4} x^4 - \frac{2}{3} x^3 - 4x^2$ from the previous problem.   Find the absolute maximum and absolute minimum on $[-6,6]$. 




\index{Derivatives}
	
\begin{tags}
	   Derivatives, MaxMin, Absolute
\end{tags}
	
\begin{diary}
	    %S2016-HW1-Q1
\end{diary}
	
\begin{solution}
	   
\end{solution}
	
\end{question}

\end{tagblock}

%-------------------------------------------------------------------------------------------------------------


	
\begin{tagblock}{Derivatives, MaxMin, Absolute}
\begin{question}
	


 Let $g(x) = x + \frac{1}{x}$.
\begin{enumerate}
\item Find all the critical numbers of $g(x)$ 

\vspace{1.5in}
\item Find the absolute maximum and absolute minimum on $[.2,4]$.  
\end{enumerate}




\index{Derivatives}
	
\begin{tags}
	   Derivatives, MaxMin, Absolute
\end{tags}
	
\begin{diary}
	    %S2016-HW1-Q1
\end{diary}
	
\begin{solution}
	   
\end{solution}
	
\end{question}

\end{tagblock}

%-------------------------------------------------------------------------------------------------------------


	
\begin{tagblock}{Derivatives, MaxMin, Absolute, Graph, Continuity}
\begin{question}
	


Sketch the graph of a function $f$ that is continuous on $[1, 6]$ and has the given properties:  Absolute maximum at $x=2$, absolute minimum at $x=6$, and $f'(4)=0$, but $x=4$ is neither a local minimum or maximum.




\index{Derivatives}
	
\begin{tags}
	   Derivatives, MaxMin, Absolute, Graph, Continuity
\end{tags}
	
\begin{diary}
	    %S2016-HW1-Q1
\end{diary}
	
\begin{solution}
	   
\end{solution}
	
\end{question}

\end{tagblock}

\begin{tagblock}{Derivatives, InverseTrig, ImplicitDifferentiation}
\begin{question}
	


As an application of implicit differentiation, we can find the derivatives of inverse trig functions.


\bigskip

  We'll start with the inverse sine function or arcsine function:  $y=\sin^{-1}(x)$ for $-1<x<-1$.  Recall that since $\sin(x)$ and $\sin^{-1}(x)$ are inverses of each other: 
\[\sin(\sin^{-1}(x)) = x\]
Our goal is to determine the derivative of $y=\sin^{-1}(x)$.  

\bigskip
We'll start by applying $\sin$ to both sides of the equation $y=\sin^{-1}(x)$:
\begin{eqnarray*} \sin(y) &= &\sin(\sin ^{-1}(x)) \\
 \sin(y) &= & x \\
 \end{eqnarray*}
 
 Then we'll use implicit differentiation to find $\frac{dy}{dx}$!
 

 \begin{eqnarray*}
  \frac{d}{dx} [\sin(y)] &= & \frac{d}{dx}[x] \\
  \cos(y) \frac{dy}{dx} & = & 1 \\
    \frac{dy}{dx} &=&  \frac{1}{\cos(y)}
 \end{eqnarray*}
 
 Recall that $y=\sin^{-1}(x)$, so we have found that the derivative of $\sin^{-1}(x)$ is $ \frac{dy}{dx} =  \frac{1}{\cos(\sin^{-1}(x))}$.  
 
\begin{minipage}{.7\textwidth}
This is not a very pretty expression, but we can simplify it using trigonometry:  Let $\theta = \sin^{-1}(x)$, so that $\theta$ is the angle whose sine is $x$.  We can picture $\theta$ as an angle in a right triangle with hypotenuse $1$ and opposite side of $x$.  By the Pythagorean Theorem, the third side has length $\sqrt{1-x^2}$.
\end{minipage}% This must go next to `\end{minipage}`
\begin{minipage}{.3\textwidth} 
\includegraphics[width=3cm]{triangle.png}
\end{minipage}

Using this triangle, find an expression for $\cos(\theta)$ (which will involve an $x$):

\[ \cos (\theta) = \rule[-0.1cm]{2.5cm}{0.01cm}\]
 
 
 \bigskip
 
 Putting this all together then tells us 
 \[ \frac{d}{dx} [\sin^{-1}(x)] = \frac{1}{\cos(\sin^{-1}(x))} = \frac{1}{\cos(\theta)} = \frac{1}{ \hspace{1in}} \]
 
 





\bigskip

Find the derivative of $y= \tan^{-1}(x)$, using a similar process as we did with the derivative of $\sin^{-1}(x)$.  




\index{Derivatives}
	
\begin{tags}
	   Derivatives, InverseTrig, ImplicitDifferentiation
\end{tags}
	
\begin{diary}
	    %S2016-HW1-Q1
\end{diary}
	
\begin{solution}
	   
\end{solution}
	
\end{question}

\end{tagblock}

%-------------------------------------------------------------------------------------------------------------


	
\begin{tagblock}{Derivatives, MaxMin, Absolute, Graph, Continuity}
\begin{question}
	


Sketch the graph of a function $f$ that is continuous on $[1, 6]$ and has the given properties:  Absolute maximum at $x=2$, absolute minimum at $x=6$, and $f'(4)=0$, but $x=4$ is neither a local minimum or maximum.



\index{Derivatives}
	
\begin{tags}
	   Derivatives, MaxMin, Absolute, Graph, Continuity
\end{tags}
	
\begin{diary}
	    %S2016-HW1-Q1
\end{diary}
	
\begin{solution}
	   
\end{solution}
	
\end{question}

\end{tagblock}

%-------------------------------------------------------------------------------------------------------------


\begin{tagblock}{Derivatives, Logarithm, WarmUp}
\begin{question}
	


Today we'll look at logarithmic functions.  A few reminders about logarithms (that we looked at in the first worksheet on Functions)



\begin{tcolorbox}

\textbf{Logarithmic Rules} 
    \begin{itemize}
    \item     $\log_b(b^x)=x$  (in particular, $\ln(e^x) = x$ )
    \item $b^{\log_b(x)} =x$  (in particular, $e^{\ln(x)} = x$ )
    \item  $\log_b(xy) = \log_b(x) +  \log_b(y)$
     \item  $\log_b(\frac{x}{y}) = \log_b(x) -  \log_b(y)$
   \item $\log_b(x^r) = r \log_b(x)$
   \end{itemize}    

\end{tcolorbox}

\textbf{Warning:} We don't have a rule for $\log_b(x+y)$.  


\bigskip


Using our log rules, we can rewrite a single logarithm $g(x) = \log_2((14x^9-10x)^4(x^5+x^3-1))$ into multiple ones:
\begin{eqnarray*} g(x) &= &\log_2((14x^9-10x)^4(x^5+x^3-1)) =   \log_2((14x^9-10x)^4) + \log_2(x^5+x^3-1) \\
&=& 4 \log_2(14x^9-10x)+ + \log_2(x^5+x^3-1) \end{eqnarray*}

\begin{enumerate}
\item \textbf{Warm Up}.  
\begin{enumerate}
\item Rewrite the following functions 
\begin{enumerate}

\item $f(x) = \log_5 (x(x^2+2)^2)$

\vspace{1in}
\item $\displaystyle h(x) = \ln \left(\frac{1+e^x}{1-e^x} \right)$

\vspace{1.5in}

\end{enumerate}  

\item Recall our derivatives of exponential functions:
\begin{enumerate}

\item $\frac{d}{dx} [e^x] = $ \rule[-0.1cm]{2.5cm}{0.01cm} 

\vspace{.5in}

\item $\frac{d}{dx} [b^x] =$ \rule[-0.1cm]{2.5cm}{0.01cm}


\end{enumerate}  

\end{enumerate}

\end{enumerate}



\index{Derivatives}
	
\begin{tags}
	   Derivatives, Logarithm, WarmUp
\end{tags}
	
\begin{diary}
	    %S2016-HW1-Q1
\end{diary}
	
\begin{solution}
	   
\end{solution}
	
\end{question}

\end{tagblock}

%-------------------------------------------------------------------------------------------------------------


	
\begin{tagblock}{Derivatives, Logarithm, Example, ImplicitDifferentiation}
\begin{question}
	


\textbf{Our First Goal:} Determine the derivative of $y = \ln(x)$.  
 
 Since $e^x$ is the inverse of $\ln(x)$, we have
 \begin{eqnarray*} y &=& \ln (x) \\
 e^y & = & e^(\ln (x)) \\
 e^y & = & x \end{eqnarray*}
 
 Now use implicit differentiation to find $\frac{dy}{dx}$
 
  \begin{eqnarray*} 
\frac{d}{dx}[ e^y ]& = & \frac{d}{dx}[x] \end{eqnarray*}
 
 \vspace{2in}
 
 This then tells us $\frac{d}{dx}[\ln(x)] = $ \rule[-0.1cm]{2.5cm}{0.01cm}

\index{Derivatives}
	
\begin{tags}
	   Derivatives, Logarithm, Example, ImplicitDifferentiation
\end{tags}
	
\begin{diary}
	    %S2016-HW1-Q1
\end{diary}
	
\begin{solution}
	   
\end{solution}
	
\end{question}

\end{tagblock}

%-------------------------------------------------------------------------------------------------------------

\begin{tagblock}{Derivatives, Logarithm, Example, ImplicitDifferentiation}
\begin{question}

 We can employ the same process to determine the derivative of  $y = \log_b(x)$.  
 
 Since $b^x$ and $\log_b(x)$ are inverses of each other we have:
 
 \begin{eqnarray*} y &=& \log_b(x) \\
 b^y & = & b^(\log_b(x)) \\
 b^y & = & x \end{eqnarray*}
 
 Now use implicit differentiation as in the previous problem.

 \vspace{2in}
 
 This then tells us $\frac{d}{dx}[\log_b(x)] = $ \rule[-0.1cm]{2.5cm}{0.01cm}


\index{Derivatives}
	
\begin{tags}
	   Derivatives, Logarithm, Example, ImplicitDifferentiation
\end{tags}
	
\begin{diary}
	    %S2016-HW1-Q1
\end{diary}
	
\begin{solution}
	   
\end{solution}
	
\end{question}

\end{tagblock}

%-------------------------------------------------------------------------------------------------------------

\begin{tagblock}{Derivatives, Logarithm, Example, ImplicitDifferentiation}
\begin{question}

 We can employ the same process to determine the derivative of  $y = \log_b(x)$.  
 
 Since $b^x$ and $\log_b(x)$ are inverses of each other we have:
 
 \begin{eqnarray*} y &=& \log_b(x) \\
 b^y & = & b^(\log_b(x)) \\
 b^y & = & x \end{eqnarray*}
 
 Now use implicit differentiation as in the previous problem.

 \vspace{2in}
 
 This then tells us $\frac{d}{dx}[\log_b(x)] = $ \rule[-0.1cm]{2.5cm}{0.01cm}


\index{Derivatives}
	
\begin{tags}
	   Derivatives, Logarithm, Example, ImplicitDifferentiation
\end{tags}
	
\begin{diary}
	    %S2016-HW1-Q1
\end{diary}
	
\begin{solution}
	   
\end{solution}
	
\end{question}

\end{tagblock}

%-------------------------------------------------------------------------------------------------------------

\begin{tagblock}{Derivatives, Logarithm, ChainRule}
\begin{question}

 Compute the derivative of $f(x) = \sqrt{\ln(x)}$


\index{Derivatives}
	
\begin{tags}
	   Derivatives, Logarithm, ChainRule
\end{tags}
	
\begin{diary}
	    %S2016-HW1-Q1
\end{diary}
	
\begin{solution}
	   
\end{solution}
	
\end{question}

\end{tagblock}

%-------------------------------------------------------------------------------------------------------------

\begin{tagblock}{Derivatives, Logarithm, ChainRule}
\begin{question}

 Compute the derivative of $g(x) = \ln(ax)$, where $a$ is any constant.


\index{Derivatives}
	
\begin{tags}
	   Derivatives, Logarithm, ChainRule
\end{tags}
	
\begin{diary}
	    %S2016-HW1-Q1
\end{diary}
	
\begin{solution}
	   
\end{solution}
	
\end{question}

\end{tagblock}

%-------------------------------------------------------------------------------------------------------------

\begin{tagblock}{Derivatives, Logarithm, ChainRule}
\begin{question}

 Compute the derivative of $G(x) = \ln(x^2+25x)$.  (\emph{Hint:  You will need the Chain Rule})


\index{Derivatives}
	
\begin{tags}
	   Derivatives, Logarithm, ChainRule
\end{tags}
	
\begin{diary}
	    %S2016-HW1-Q1
\end{diary}
	
\begin{solution}
	   
\end{solution}
	
\end{question}

\end{tagblock}

%-------------------------------------------------------------------------------------------------------------

\begin{tagblock}{Derivatives, IncreasingDecreasing, Graph, TangentLine}
\begin{question}

 Recall that we say a function is \emph{increasing} on an interval $(a,b)$ provided that for all $x_1$ and $x_2$ in the interval, if $x_1 < x_2$, then $f(x_1) < f(x_2)$.  A function is \emph{decreasing} on an interval $(a,b)$ provided that for all $x_1$ and $x_2$ in the interval, if $x_1 < x_2$, then $f(x_1) > f(x_2)$.

\bigskip

Simply put, an increasing function is one that is rising as we move from left to right along the graph, and a decreasing function is one that falls as the value of the input increases.


 

  
Let $f(x)$ be given by the graph below.  Use the graph to answer each of the following questions. (You may need to estimate answers based on the graph)
\begin{figure}[h]
\centering
\includegraphics[width=8cm]{incdec1.png} 
\end{figure}

\begin{enumerate}
\item Find the intervals where $f(x)$ is increasing.  

\vspace{.25in}
\item Find the intervals where $f(x)$ is decreasing.  
\vspace{.25in}

\item Draw the tangent line to $f(x)$ at $x=-3$.  Is $f'(-3)$ positive, negative or zero?  
\vspace{.25in}

\item Draw the tangent line to $f(x)$ at $x=0$.  Is $f'(0)$ positive, negative or zero?  
\vspace{.25in}

\item Draw the tangent line to $f(x)$ at $x=4$.  Is $f'(4)$ positive, negative or zero?  

\vspace{.25in}

\item Fill in the blank with ``increasing'' or ``decreasing":  If the derivative is positive on an interval, that is if $f'(x) >0$ for all $x$ in $(a,b)$, then $f(x)$ is \rule{4cm}{0.1mm}
\vspace{.25in}

\item Fill in the blank with ``increasing'' or ``decreasing":  If the derivative is negative on an interval, that is if $f'(x) <0$ for all $x$ in $(a,b)$, then $f(x)$ is \rule{4cm}{0.1mm}




\end{enumerate}


\bigskip

So the first derivative will allow us to determine if a function is increasing or decreasing! 


\index{Derivatives}
	
\begin{tags}
	   Derivatives, IncreasingDecreasing, Graph, TangentLine
\end{tags}
	
\begin{diary}
	    %S2016-HW1-Q1
\end{diary}
	
\begin{solution}
	   
\end{solution}
	
\end{question}

\end{tagblock}

%-------------------------------------------------------------------------------------------------------------

\begin{tagblock}{Derivatives, IncreasingDecreasing, Critical, MaxMin, FirstDerivativeTest}
\begin{question}

Our next goal is to use the first derivative to determine where we have local minimums and local maximums.  Recall from our previous worksheet the \emph{critical numbers} are our candidates for local minimums and local maximums.  

 Let $f(x)$ be given by the graph below.  Use the graph to answer each of the following questions.
\begin{figure}[h]
\centering
\includegraphics[width=7cm]{minmax1.png} 
\end{figure}

\begin{enumerate}

\item Find the critical numbers of $f(c)$: identify all values of $c$ for which $f'(c)$ does not exist; identify all values of $c$ for which $f'(c) = 0$.

\vspace{.4in}



\item Which critical numbers $c$ give local minimums? For each $c$, does the function change from increasing to decreasing at $c$, or decreasing to increasing at $c$?

\vspace{.4in}



\item Which critical numbers $c$ give local maximums? For each $c$, does the function change from increasing to decreasing at $c$, or decreasing to increasing at $c$?

\vspace{.4in}

\item Which critical numbers $c$ are neither local minimums nor local maximums?  What can you say about the increasing and decreasing?  







\end{enumerate}


\bigskip

The results of the previous problem is exactly the \textbf{First Derivative Test} for determining local minimums and maximums.

\bigskip

\textbf{First Derivative Test:}  Suppose $c$ is a critical number of a function $f(x)$
\begin{itemize}
\item If $f'(x)$ changes sign from \emph{positive to negative} at $c$ (that is $f$ changes from increasing to decreasing) then $f(x)$ has a \emph{local maximum} at $x=c$.
\item If $f'(x)$ changes sign from \emph{negative to positive} at $c$ (that is $f$ changes from decreasing to increasing ) then $f(x)$ has a \emph{local minimum} at $x=c$.
\item If $f'(x)$ doesn't change sign at $c$, then $c$ is neither a local min nor a local max.  
\end{itemize}

\vspace{.5in}






\index{Derivatives}
	
\begin{tags}
	   Derivatives, IncreasingDecreasing, Critical, MaxMin, FirstDerivativeTest
\end{tags}
	
\begin{diary}
	    %S2016-HW1-Q1
\end{diary}
	
\begin{solution}
	   
\end{solution}
	
\end{question}

\end{tagblock}

%-------------------------------------------------------------------------------------------------------------

\begin{tagblock}{Derivatives, Graph, MaxMin, FirstDerivativeTest}
\begin{question}

Returning to $\displaystyle g(x) = \frac{7}{4} x^4 + 7x^3 - 14x^2$,

\begin{enumerate}
\item Find the local minimum(s) and local maximum(s) of $g(x)$.  Explain how you know each is a local min or max.  

\vspace{2in}
\item For each $x$ value that gives you a local min or max, compute the $y$-value $g(x)$.  Use these points together with the fact that each is a local min or max to get a rough sketch of the graph of $g$. 
\vspace{.5in}


\end{enumerate}



\index{Derivatives}
	
\begin{tags}
	   Derivatives, Logarithm, ChainRule
\end{tags}
	
\begin{diary}
	    %S2016-HW1-Q1
\end{diary}
	
\begin{solution}
	   
\end{solution}
	
\end{question}

\end{tagblock}

%-------------------------------------------------------------------------------------------------------------

\begin{tagblock}{Derivatives, IncreasingDecreasing, Concavity, Definition, Graph, MaxMin, FirstDerivativeTest, InflectionPoints}
\begin{question}

Recap from the last worksheet:  Let $f(x)$ be a function 
\begin{enumerate}
\item $c$ is a critical number of $f(x)$ if $f '(c)$  \rule{8cm}{0.1mm}

\bigskip
\item If $f '(x) >0$ for all $x$ in the interval $(a,b)$, then $f$ is (circle one) INCREASING  or DECREASING on $(a,b)$.  

\bigskip
\item If $f '(x) <0$ for all $x$ in the interval $(a,b)$, then $f$ is (circle one) INCREASING  or DECREASING on $(a,b)$.  

\bigskip
\item If $c$ is a critical number of $f(x)$  and $f'(x)$ changes sign from positive to negative at $c$, then $c$ is (circle one) a LOCAL MIN of $f$, a LOCAL MAX or $f$, NEITHER A LOCAL MIN NOR LOCAL MAX.

\end{enumerate}


In this worksheet we'll see how the second derivative can give us information about the graph of $f$.  Applying what we did in the last worksheet to $f '(x)$ we get

\begin{itemize}
\item If $f''(x) >0$ on $(a,b)$, then $f'(x)$ is increasing on $(a,b)$, in which case we say $f(x)$ is \emph{concave up}


\item If $f''(x) <0$ on $(a,b)$, then $f'(x)$ is decreasing on $(a,b)$, in which case we say $f(x)$ is \emph{concave down}
\begin{figure}[h]
\centering
\includegraphics[width=3cm]{concaveup.png}  \hspace{1in}
\includegraphics[width=3cm]{concavedown.png} 
\end{figure}

\centering Concave up \hspace{1in} Concave down

\end{itemize}

Graphically, concave up means that $f(x)$ looks like a ``bowl'', and concave down means that $f(x)$ looks like a ``hill.''

We will call a point $c$ in the domain of $f$ a \emph{point of inflection} if $f''(x)$ changes from positive to negative at $c$ or negative to positive at $c$, that is $f$ changes concavity at $c$.  In the graph below, $f$ has an inflection point at $x=1$.  
\begin{figure}[h]
\centering
\includegraphics[width=4cm]{inflection.png}
\end{figure}





\index{Derivatives}
	
\begin{tags}
	   Derivatives, IncreasingDecreasing, Concavity, Definition, Graph, MaxMin, FirstDerivativeTest, InflectionPoints
\end{tags}
	
\begin{diary}
	    %S2016-HW1-Q1
\end{diary}
	
\begin{solution}
	   
\end{solution}
	
\end{question}

\end{tagblock}

%-------------------------------------------------------------------------------------------------------------

\begin{tagblock}{Derivatives, InflectionPoints, Concavity}
\begin{question}

Let $f(x)$ be given by the graph below.  Use the graph to answer each of the following questions. (You may need to estimate answers based on the graph)
\begin{figure}[h]
\centering
\includegraphics[width=8cm]{incdec1.png} 
\end{figure}

\begin{enumerate}
\item Find the intervals where $f(x)$ concave up.  

\vspace{.75in}
\item Find the intervals where $f(x)$ concave down.
\vspace{.75in}

\item Find all the inflection points of $f(x)$.   
\vspace{.75in}

\end{enumerate}



Given an equation of a function $f$, we can determine the intervals of concavity similarly to how we determined the intervals of increasing/decreasing, but using the \emph{second derivative} instead of the first derivative.



\index{Derivatives}
	
\begin{tags}
	   Derivatives, InflectionPoints, Concavity
\end{tags}
	
\begin{diary}
	    %S2016-HW1-Q1
\end{diary}
	
\begin{solution}
	   
\end{solution}
	
\end{question}

\end{tagblock}

%-------------------------------------------------------------------------------------------------------------

\begin{tagblock}{Derivatives, InflectionPoints, Concavity}
\begin{question}

Let $\displaystyle f(x) = \frac {1}{30}x^6 - \frac{1}{10}x^5 - \frac{1}{4}x^4 + 2x$.  We will determine the intervals where $f(x)$ is concave up and concave down and the inflection points.
\begin{enumerate}
\item Compute the second derivative of $f(x)$ and find all values $c$ where $f''(c)$ is undefined or $f''(c) = 0$.

\vspace{2in}
\item Plot your $c$ values from (a) on a number line.  How many intervals do you have?

\vspace{.7in}
\item For each interval, choose a test value and determine if the second derivative at that test value is positive or negative.  

\vspace{3in}
\item Summarizing your work from part (c):  \\
$f$ is concave up on the interval(s) :  \rule{8cm}{0.1mm} \\
\bigskip
$f$ is concave down on the interval(s) :  \rule{8cm}{0.1mm} \\

\bigskip
$f$ has inflection point(s) at $x = $   \rule{8cm}{0.1mm} \\

\bigskip
\item In fact the graph from Problem 2. is the graph of this function.  How do your intervals of concavity compare?  
\end{enumerate}




\index{Derivatives}
	
\begin{tags}
	   Derivatives, InflectionPoints, Concavity
\end{tags}
	
\begin{diary}
	    %S2016-HW1-Q1
\end{diary}
	
\begin{solution}
	   
\end{solution}
	
\end{question}

\end{tagblock}

%-------------------------------------------------------------------------------------------------------------

\begin{tagblock}{Derivatives, InflectionPoints, Concavity, MaxMin, SecondDerivativeTest }
\begin{question}

In the last worksheet we saw that the \textbf{First Derivative Test} could be used to determine if a critical number is a local min or local max.  Similarly we have a test involving the second derivative.

\bigskip

\begin{enumerate}
\item Consider the graph of $f(x)$ given below. 
\begin{minipage}{.4\textwidth}
\includegraphics[width=4cm]{concavemax.png}\end{minipage}% This must go next to `\end{minipage}`
\begin{minipage}{.6\textwidth}
\begin{enumerate}
\item Compute $f '(c) = $

\vspace{.5in}

\item Is $f ''(c)$ positive or negative?  

\vspace{.5in}
\item Is $c$ a local min or local max of $f$?
\vspace{.5in}

\end{enumerate}

\end{minipage}

\item Consider the graph of $f(x)$ given below. 
\begin{minipage}{.4\textwidth}
\includegraphics[width=4cm]{concavemin.png}\end{minipage}% This must go next to `\end{minipage}`
\begin{minipage}{.6\textwidth}
\begin{enumerate}
\item Compute $f '(c) = $

\vspace{.5in}

\item Is $f ''(c)$ positive or negative?  

\vspace{.5in}
\item Is $c$ a local min or local max of $f$?

\vspace{.5in}

\end{enumerate}

\end{minipage}


\end{enumerate}



This leads us to the \textbf{Second Derivative Test}:  Let $c$ be a critical number of $f(x)$
\begin{itemize}
\item If $f''(c) >0$, then $c$ is a local min;
\item If $f''(c) <0$, then $c$ is a local max; 
\item If $f''(c) =0$, we can't conclude anything, and we need to instead use the First Derivative Test.
\end{itemize}




\index{Derivatives}
	
\begin{tags}
	   Derivatives, InflectionPoints, Concavity, MaxMin, SecondDerivativeTest
\end{tags}
	
\begin{diary}
	    %S2016-HW1-Q1
\end{diary}
	
\begin{solution}
	   
\end{solution}
	
\end{question}

\end{tagblock}

%-------------------------------------------------------------------------------------------------------------

\begin{tagblock}{Derivatives, InflectionPoints, Concavity, MaxMin, SecondDerivativeTest }
\begin{question}

Let $\displaystyle f(x) = \frac{1}{4}x^4 + \frac{1}{3}x^3 - x^2 +3$.  Find the critical numbers of $f$ and then use the \textbf{Second Derivative Test} to determine if they are local mins or maxs.


\index{Derivatives}
	
\begin{tags}
	   Derivatives, InflectionPoints, Concavity, MaxMin, SecondDerivativeTest
\end{tags}
	
\begin{diary}
	    %S2016-HW1-Q1
\end{diary}
	
\begin{solution}
	   
\end{solution}
	
\end{question}

\end{tagblock}

%-------------------------------------------------------------------------------------------------------------

\begin{tagblock}{Derivatives, InflectionPoints, Concavity, MaxMin, InflectionPoints, IncreasingDecreasing, Graph}
\begin{question}

Summarizing what we learned in the last two worksheets.
\begin{enumerate}
\item If $f '(x) >0$ on $(a,b)$, then $f$ is  \rule{8cm}{0.1mm} 
\bigskip

\item If $f '(x) <0$ on $(a,b)$, then $f$ is  \rule{8cm}{0.1mm}
\bigskip
\item If $f ''(x) >0$ on $(a,b)$, then $f$ is  \rule{8cm}{0.1mm} 
\bigskip

\item If $f ''(x) <0$ on $(a,b)$, then $f$ is  \rule{8cm}{0.1mm} 
\bigskip 
\item This then gives us 4 different possibilities:  increasing and concave up, increasing and concave down, decreasing and concave up, decreasing and concave down.  On the axes below draw the general shape of the graph.
\end{enumerate}

\hangindent=-1in \includegraphics[width=3.6cm]{incup.png}\, \includegraphics[width=3.6cm]{incdown.png} \, \includegraphics[width=3.6cm]{decup.png} \, \includegraphics[width=3.6cm]{decdown.png} 
 
 \begin{enumerate}
 \item[(f)] Give a sketch of a graph $f$ that satisfies the following properties:  $f(0)=2$;  $f'(2) = 0 = f'(3)$; \\
 $f'(x) >0$ on the intervals $(-\infty, 2)$ and $(3, \infty)$, and $f'(x)<0$ on the interval $(2,3)$; \\
 $f''(x)>0$ on the intervals $(0,1)$ and $(2.5,\infty)$, $f''(x)<0$ on the intervals $(-\infty, 0)$ and $(1,2.5)$.
 

\end{enumerate}


\index{Derivatives}
	
\begin{tags}
	   Derivatives, InflectionPoints, Concavity, MaxMin, InflectionPoints, IncreasingDecreasing, Graph
\end{tags}
	
\begin{diary}
	    %S2016-HW1-Q1
\end{diary}
	
\begin{solution}
	   
\end{solution}
	
\end{question}

\end{tagblock}

%-------------------------------------------------------------------------------------------------------------

\begin{tagblock}{Derivatives, HorizontalAsymptotes, VerticalAsymptotes, Graph, WarmUp, RationalFunction  }
\begin{question}

\textbf{Goal:  Given an equation of a function $f(x)$, tie together our work on derivatives and limits to sketch a graph of $f(x)$.}

\bigskip

We will use following technique to sketch a curve given an equation
\begin{itemize}
\item Find the intercepts (both $x$ and $y$-intercepts)
\item Find any asymptotes (horizontal, vertical, slant)
\item Find the intervals where the function is increasing and decreasing
\item Find the intervals where the function is concave up and concave down
\item Find both the $x$ and $y$ coordinates of any local maximums, local minimums, and inflection points
\end{itemize}


\bigskip


\textbf{Quick Review of Intercepts and Asymptotes:}

\bigskip

\textbf{Intercepts:}
\begin{itemize}
\item Recall to find the $x$-intercepts, we want to find when the function crosses the $x$- axis.  This means we will set our equation equal to $0$ and solve for $x$.
\item To find the $y$-intercepts, we want to find when the function crosses the $y$- axis.  This means we will set $x=0$ and compute $f(0)$.
\end{itemize}

\begin{enumerate}  
\item Intercepts:
\begin{enumerate}
\item Find both the $x$ and $y$-intercepts of $g(x) = x^2-4x -21$.  
\begin{itemize}
\item $x$-intercept(s):  
\vspace{.5in}




\item $y$-intercepts(s):

\vspace{.5in}

\end{itemize}
\item Can a function have more than one $x$-intercept?  Can a function have more than one $y$-intercept?  

\vspace{.5in}

\item Given an example of a function $f$ that does not have a $y$-intercept.  (Either give a graph or an equation of  $f$ )


\vspace{.5in}


\end{enumerate}
\end{enumerate}

\newpage
\textbf{Asymptotes:}


 \textbf{Vertical Asymptotes}  Recall we get a vertical asymptote at $x=a$, when $\lim_{x \to a^+} f(x) = \pm \infty$ or $\lim_{x \to a^-} f(x) = \pm \infty$.  More informally, if $f(x) = \frac{g(x)}{h(x)}$, with $h(a)=0$, but $g(a) \neq 0$, then we will get a vertical asymptote at $x=a$.

\bigskip

Find the vertical asymptotes of $\displaystyle f(x) = \frac{x^2+3}{x+1}$,  $\displaystyle g(x) = \frac{x^2+3}{2x^2 - 1}$ and $\displaystyle h(x) = \frac{x^2+3}{x^3-1}$.  

\vspace{1in}

\bigskip

 \textbf{Horizontal Asymptotes}  Recall we get a horizontal asymptote at $y=M$, when $\lim_{x \to \infty} f(x) = M$ or $\lim_{x \to -\infty} f(x) = M$, and to compute limits as $x \to \infty$ we ``divided by the highest power of $x$''.  

We also developed a short cut for determining horizontal asymptotes of \emph{rational functions}:

\[\text{ Let } f(x) = \frac{p(x)}{q(x)} = \frac{a_nx^n + a_{n-1}x^{n-1} + \cdots + a_1x+a_0}{b_mx^m + b_{m-1}x^{m-1} + \cdots + b_1x+b_0}\]
\bigskip

\begin{itemize}
\item If $n=m$, then $f(x)$ has a horizontal asymptote at $y =$ \rule{5cm}{0.1mm} 

\bigskip

\item If $n<m$, then $f(x)$ has a horizontal asymptote at $y = $ \rule{5cm}{0.1mm} 

\bigskip
\item If $n>m$, then $f(x)$ has \rule{8cm}{0.1mm} 
\bigskip

\end{itemize}



Find the horizontal asymptotes (if any) of $\displaystyle f(x) = \frac{x^2+3}{x+1}$,   $\displaystyle g(x) = \frac{x^2+3}{2x^2 - 1}$ and $\displaystyle h(x) = \frac{x^2+3}{x^3-1}$.



\index{Derivatives}
	
\begin{tags}
	   Derivatives, HorizontalAsymptotes, VerticalAsymptotes, Graph, WarmUp, RationalFunction
\end{tags}
	
\begin{diary}
	    %S2016-HW1-Q1
\end{diary}
	
\begin{solution}
	   
\end{solution}
	
\end{question}

\end{tagblock}

%-------------------------------------------------------------------------------------------------------------

\begin{tagblock}{Derivatives, HorizontalAsymptotes, VerticalAsymptotes, SlantAsymptotes, Graph, RationalFunction  }
\begin{question}


Consider the function $f(x)$ given by the graph below. 
\begin{figure}[h]
\centering
\includegraphics[width=8cm]{slant.png} 
\end{figure}
\begin{enumerate}
\item Does $f(x)$ have any vertical and/or horizontal asymptotes?  If so what are they.  

\vspace{1in}
\item Graph the line $y=x$ on the graph of $f(x)$ above.  As $x$ gets large, what do you notice about the graph of $f$ compared to the graph of $y=x$?

\vspace{1in}
\end{enumerate}


We call the line $y=x$ in the above example a \emph{slant asymptote}.  

A rational function 
\[ f(x) = \frac{p(x)}{q(x)} = \frac{a_nx^n + a_{n-1}x^{n-1} + \cdots + a_1x+a_0}{b_mx^m + b_{m-1}x^{m-1} + \cdots + b_1x+b_0}\]
will have a slant asymptote when $n=m+1$, that is the highest power of $x$ in the numerator is one more than the highest power of $x$ in the denominator.  

\newpage

\textbf{Computing Slant Asymptotes}  To compute slant asymptotes, we will do polynomial long division, which will leave us with a linear equation and a remainder.


Let $\displaystyle f(x) = \frac{x^2+3}{x+1}$

\begin{enumerate}
\item Use polynomial long division to divide $x^2+3$ by $x+1$.

\vspace{.5in}

\begin{center}
\Mydiv{x+1} {x^2+3 \hspace{.5in}} 
\end{center}

\vspace{2in}
\item We have a remainder of $4$, or in other words a term $\frac{4}{x+1}$.  What happens to $\frac{4}{x+1}$ as $x \to \infty$?

\vspace{1in}

\item This means we can ignore the remainder, and so we have a slant asymptote at $y = $

\end{enumerate}

\index{Derivatives}
	
\begin{tags}
	   Derivatives, HorizontalAsymptotes, VerticalAsymptotes, SlantAsymptotes, Graph, RationalFunction
\end{tags}
	
\begin{diary}
	    %S2016-HW1-Q1
\end{diary}
	
\begin{solution}
	   
\end{solution}
	
\end{question}

\end{tagblock}

%-------------------------------------------------------------------------------------------------------------

\begin{tagblock}{Derivatives, HorizontalAsymptotes, VerticalAsymptotes, SlantAsymptotes, Graph, RationalFunction, Critical, MaxMin, Concavity, InflectionPoints, IncreasingDecreasing  }
\begin{question}


We'll now put this all together to get a good graph of $\displaystyle f(x) = \frac{x^2+3}{x+1}$
\begin{enumerate}
\item Find the $x$ and $y$-intercepts of $f(x)$.

\vspace{.75in}
\item Find all the asymptotes of $f(x)$
\begin{enumerate}
\item Vertical Asymptotes:
\item Horizontal Asymptotes:
\item Slant Asymptotes:
\end{enumerate}
At this point we might want to start plotting our asymptotes and our intercepts on our graph.

\begin{figure}[h]
\centering
\includegraphics[width=8cm]{blank.png} 
\end{figure}




\item Find the critical numbers of $f(x)$.





\newpage
\item Find the intervals where $f$ is increasing and decreasing.
\vspace{2in}

\item Using the quotient rule again, and simplifying we find that the second derivative is $f''(x) = \frac{8}{(x+1)^3}$.  Use this find when $f$ is concave up and concave down.  Does $f$ have any inflection points?

\vspace{1.5in}

\item Find both the $x$ and $y$ coordinate of all the local minimums, local maximums and inflection points.  

\vspace{1.5in}
\item Combining parts (d) and (e):  
Interval(s) where $f$ is \\
increasing and concave down:   \rule{2cm}{0.1mm} \, decreasing and concave down:   \rule{2cm}{0.1mm} \\
\bigskip

decreasing and concave up:   \rule{2cm}{0.1mm} \,   increasing and concave up:   \rule{2cm}{0.1mm}

\item Finish the sketch of the graph, labeling all asymptotes, local mins, local maxs and inflection points.  
\end{enumerate}


\index{Derivatives}
	
\begin{tags}
	   Derivatives, HorizontalAsymptotes, VerticalAsymptotes, SlantAsymptotes, Graph, RationalFunction, Critical, MaxMin, Concavity, InflectionPoints, IncreasingDecreasing
\end{tags}
	
\begin{diary}
	    %S2016-HW1-Q1
\end{diary}
	
\begin{solution}
	   
\end{solution}
	
\end{question}

\end{tagblock}

%-------------------------------------------------------------------------------------------------------------

\begin{tagblock}{Derivatives, HorizontalAsymptotes, VerticalAsymptotes, SlantAsymptotes, Graph, RationalFunction, Critical, MaxMin, Concavity, InflectionPoints, IncreasingDecreasing  }

\begin{question}

We have followed the subsequent technique to sketch a curve given an equation
\begin{itemize}
\item Find the intercepts (both $x$ and $y$-intercepts)
\item Find any asymptotes (horizontal, vertical, slant)
\item Find when the function is increasing or decreasing
\item Find when the function is concave up or concave down
\item Find both the $x$ and $y$ coordinates of any local maximums, local minimums, and inflection points
\end{itemize}

Below are four functions  $f(x)$, $g(x)$, $h(x)$ and $j(x)$ along with their first and second derivatives (so you do not have to compute them).  Using the technique we outlined above you should sketch each function.  \textbf{Be sure to include any asymptotes on your graph and label the local maximums, local minimums, and inflection points (both the $x$ and $y$ coordinates).}  



\begin{enumerate}  
\item $\displaystyle f(x) = \frac{2x}{x^2+1} \hspace{.5in} f'(x) = \frac{-2(x^2-1)}{(x^2+1)^2} \hspace{.5in} f''(x) = \frac{4x(x^2-3)}{(x^2+1)^3} $
\item  $\displaystyle  g(x) = \frac{x^3}{x^2-1} \hspace{.5in} g'(x) = \frac{x^2(x^2-3)}{(x^2-1)^2} \hspace{.5in} g''(x) = \frac{2x(x^2+3)}{(x^2-1)^3} $
\item $\displaystyle h(x) =  \frac{x^3}{x^2+2x+1} \hspace{.2in} h'(x) = \frac{x^2(x+3)}{(x+1)^3} \hspace{.5in}  h''(x) = \frac{6x}{(x+1)^4} $
\item $\displaystyle  j(x) = \frac{3x^2-3}{x^2-4} \hspace{.5in} j'(x) = \frac{-18x}{(x^2-4)^2} \hspace{.5in} j''(x) = \frac{18(3x^2+4)}{(x^2-4)^3} $


\end{enumerate}


\index{Derivatives}
	
\begin{tags}
	   Derivatives, HorizontalAsymptotes, VerticalAsymptotes, SlantAsymptotes, Graph, RationalFunction, Critical, MaxMin, Concavity, InflectionPoints, IncreasingDecreasing

\end{tags}
	
\begin{diary}
	    %S2016-HW1-Q1
\end{diary}
	
\begin{solution}
	   
\end{solution}
	
\end{question}

\end{tagblock}

%-------------------------------------------------------------------------------------------------------------

\begin{tagblock}{Derivatives, WarmUp, L'Hopital, Limits }

\begin{question}

\textbf{Warm Up}

\begin{enumerate}
\item Earlier in the semester we investigated the function $\displaystyle f(x) = \frac{x^2-4}{x-2}$ and the $\lim_{x \to 2} f(x)$.  
What happens when we try to evaluate $f(x)$ at $x=2$?  
\vspace{.5in}

 How can we use algebraic tools to determine  the $\displaystyle \lim_{x \to 2} f(x)$?  
 \vspace{1in}
 
 \item Consider the function $\displaystyle H(x) = \frac{f(x)}{g(x)}$, where $f(x) = {e^x}$ and $g(x) = {x}$.  Compute the $\displaystyle \lim_{x \to \infty} f(x)$ and $\displaystyle \lim_{x \to \infty} g(x)$.  Can we use this information to determine $\displaystyle \lim_{x \to \infty} H(x)$?  
 
 
 \vspace{1in}
 
 \end{enumerate}
 
 If we are in the situation as in the previous examples where we get either $\frac{0}{0}$ or $\frac{\pm \infty}{\pm \infty}$ we say that we have an \textbf{indeterminate form}.  In this worksheet we will learn a new technique for evaluating limits of this type.  
 
 
 


\begin{tcolorbox}
 \textbf{l'Hospital's Rule:} Suppose $f(x)$ and $g(x)$ and differentiable and $g'(x) \neq 0$ on an open interval containing $a$.  If 
\[\frac{\lim_{x \to a}f(x)} { \lim_{x \to a}g(x)} = \frac{0}{0} \text{ or } \frac{\lim_{x \to a}f(x)} { \lim_{x \to a}g(x)} = \frac{\infty}{\infty}, \]\text{ then} \[ \lim_{x \to a} \frac{f(x)}{g(x)} = \lim_{x \to a} \frac{f'(x)}{g'(x)}\]

\end{tcolorbox}

\bigskip

\textbf{Warning!} We can only use l'Hospital's Rule if we start with an indeterminate form, that is $\frac{0}{0}$ or $\frac{\infty}{\infty}$.


\bigskip


Returning to the function $\displaystyle H(x) = \frac{e^x}{x}$.  To compute the $\displaystyle \lim_{x \to \infty}  \frac{e^x}{x}$ we saw we have an indeterminate form of the type $\frac{\infty}{\infty}$.  Use  l'Hospital's Rule to determine the limit.


\index{Derivatives}
	
\begin{tags}
	   Derivatives, WarmUp, L'Hopital, Limits

\end{tags}
	
\begin{diary}
	    %S2016-HW1-Q1
\end{diary}
	
\begin{solution}
	   
\end{solution}
	
\end{question}

\end{tagblock}

%-------------------------------------------------------------------------------------------------------------

\begin{tagblock}{Derivatives, L'Hopital, Limits }

\begin{question}

Calculate $\displaystyle  \lim_{x \to 0} \frac{e^{2x}-1}{x}$ using l'Hospital's Rule.  Make sure you justify why you can apply  l'Hospital's Rule.


\index{Derivatives}
	
\begin{tags}
	   Derivatives, L'Hopital, Limits

\end{tags}
	
\begin{diary}
	    %S2016-HW1-Q1
\end{diary}
	
\begin{solution}
	   
\end{solution}
	
\end{question}

\end{tagblock}

%-------------------------------------------------------------------------------------------------------------

\begin{tagblock}{Derivatives, L'Hopital, Limits, HorizontalAsymptotes }

\begin{question}

Calculate  $\displaystyle  \lim_{x \to \infty} \frac{\ln(x)}{x}$ using l'Hospital's Rule.   Make sure you justify why you can apply  l'Hospital's Rule.   Determine the horizontal asymptotes (if any) of $ \displaystyle f(x) =   \frac{\ln(x)}{x}$.


\index{Derivatives}
	
\begin{tags}
	   Derivatives, L'Hopital, Limits, HorizontalAsymptotes 

\end{tags}
	
\begin{diary}
	    %S2016-HW1-Q1
\end{diary}
	
\begin{solution}
	   
\end{solution}
	
\end{question}

\end{tagblock}

%-------------------------------------------------------------------------------------------------------------

\begin{tagblock}{Derivatives, L'Hopital, Limits}

\begin{question}

Note that it is possible that after applying l'Hospital's Rule you still have an indeterminate form.  In that case you may need to use l'Hospital's rule a second time!  \\
 Calculate $\displaystyle \lim_{x\to 0}\frac{x-\sin(x)}{x^2}$ using l'Hospital's Rule.


\index{Derivatives}
	
\begin{tags}
	   Derivatives, L'Hopital, Limits

\end{tags}
	
\begin{diary}
	    %S2016-HW1-Q1
\end{diary}
	
\begin{solution}
	   
\end{solution}
	
\end{question}

\end{tagblock}

%-------------------------------------------------------------------------------------------------------------

\begin{tagblock}{Derivatives, L'Hopital, Limits, Example}

\begin{question}

If we are trying to compute a limit that is not already a quotient, we may be able to do some algebra first, so that we can apply l'Hospital's Rule. 
\bigskip

  Consider 
 $\displaystyle \lim_{x\to 0^+} {x^2} \ln(x)$ 
 
 Note that we can rewrite $x^2 \ln(x)$ as $\displaystyle \frac{\ln(x)}{x^{-2}}$; and further we know $\displaystyle \lim_{x \to 0^+} \ln(x) = - \infty$, and $\displaystyle \lim_{x \to 0^+} x^{-2} = \lim_{x \to 0^+} \frac{1}{x^2} = \infty$.  Putting this together, we have 
 \[  \lim_{x\to 0^+} {x^2} \ln(x) =  \lim_{x\to 0^+} \frac{\ln(x)}{x^{-2}} \]
 which is an indeterminate form.  Now finish computing the limit using  l'Hospital's Rule. 


\index{Derivatives}
	
\begin{tags}
	   Derivatives, L'Hopital, Limits, Example

\end{tags}
	
\begin{diary}
	    %S2016-HW1-Q1
\end{diary}
	
\begin{solution}
	   
\end{solution}
	
\end{question}

\end{tagblock}

%-------------------------------------------------------------------------------------------------------------

\begin{tagblock}{Derivatives, L'Hopital, Limits}

\begin{question}

Calculate $\displaystyle\lim_{x\to \infty} {x} \sin(\frac{1}{x})$.  Make sure you justify why you can apply  l'Hospital's Rule.  


\index{Derivatives}
	
\begin{tags}
	   Derivatives, L'Hopital, Limits

\end{tags}
	
\begin{diary}
	    %S2016-HW1-Q1
\end{diary}
	
\begin{solution}
	   
\end{solution}
	
\end{question}

\end{tagblock}

%-------------------------------------------------------------------------------------------------------------

\begin{tagblock}{Derivatives, L'Hopital, Limits, Theory}

\begin{question}

Let $n>0$ be any integer.  Determine  $\displaystyle\lim_{x\to\infty}\frac {x^n}{e^{x}}$.   


\index{Derivatives}
	
\begin{tags}
	   Derivatives, L'Hopital, Limits, Theory

\end{tags}
	
\begin{diary}
	    %S2016-HW1-Q1
\end{diary}
	
\begin{solution}
	   
\end{solution}
	
\end{question}

\end{tagblock}

%-------------------------------------------------------------------------------------------------------------

\begin{tagblock}{Derivatives, L'Hopital, Limits}

\begin{question}

Consider $\displaystyle \lim_{x\to 0^+} {x}^{x^2}$.  Note that we don't have a quotient, but we can manipulate our function so that l'Hospital's Rule can be applied.

In general we can re-write $f(x)^{g(x)}$ as follows:
\[f(x)^{g(x)} = e^{\ln (f(x)^{g(x)})} = e^{g(x) \ln (f(x))}  \]

So for our function $ {x}^{x^2}$, we have
\[  {x}^{x^2} = e^{x^2 \ln(x)} \]
Then to compute 
\[ \lim_{x\to 0^+} {x}^{x^2} =  \lim_{x\to 0^+}  e^{x^2 \ln(x)} \]
we simply need to determine $ \lim_{x\to 0^+} {x^2 \ln(x)}$.  Recall we did this in an earlier problem, and found that $ \lim_{x\to 0^+} {x^2 \ln(x)} = 0$.  This tells us then that
\[ \lim_{x\to 0^+} {x}^{x^2} =  \lim_{x\to 0^+}  e^{x^2 \ln(x)}   = e^0 = 1\]

\bigskip

Calculate $\displaystyle\lim_{x\to \infty} (1+ \frac{3}{x})^{5x}$.  


\index{Derivatives}
	
\begin{tags}
	   Derivatives, L'Hopital, Limits

\end{tags}
	
\begin{diary}
	    %S2016-HW1-Q1
\end{diary}
	
\begin{solution}
	   
\end{solution}
	
\end{question}

\end{tagblock}

%-------------------------------------------------------------------------------------------------------------

\begin{tagblock}{Derivatives, Optimization, WarmUp}

\begin{question}

\begin{enumerate}
\item \textbf{A first example:}  In this problem we will find two numbers whose difference is $20$ and whose product is as small as possible. 
\begin{enumerate}
\item Let $x$ and $y$ be two numbers, whose difference is $20$, and suppose $y>x$.    For example, when $x=5$, $y=\underline{\hspace{1in}}$ and when $x=-5$, $y=\underline{\hspace{1in}}$.  Give a general equation for $y$ in terms of $x$.
\vspace{.5in}

\item Write a function in terms of $x$ and $y$ that gives their product, $P$.  This is the function we want to minimize.  Note though that it has two variables.
\vspace{.5in}
\item  Using part (a),  rewrite your function in (b) so that it only involves the variable $x$.
\vspace{.5in}
\item Now we have a function, $P(x)$ that gives the product in terms of one variable $x$.  Now we can use either the First or Second Derivative Test to find the minimum of our function.  Find the minimum using our calculus techniques!
\vspace{2in}
\item Finally we want both $x$ and $y$.  What is $y$?


\end{enumerate}
\end{enumerate}

\newpage

Our general strategy will be as follows:
\begin{itemize}
\item Draw a useful picture. You may need to draw more than one to understand what is going on.
\item Use variables to name your unknowns and identify the variable you want to optimize. 
\item Write an equation for the variable you want to optimize. Sometimes this is done for you.
\item If your equation from the last step is not a function of one variable, you may need to find an extra constraint relating the variables.
\item Once your equation is a function of just one variable, go ahead and optimize it by finding critical points and using first or second derivative test. \emph{Make sure you check that your critical number is indeed a min or max}
\end{itemize}



\index{Derivatives}
	
\begin{tags}
	   Derivatives, Optimization, WarmUp

\end{tags}
	
\begin{diary}
	    %S2016-HW1-Q1
\end{diary}
	
\begin{solution}
	   
\end{solution}
	
\end{question}

\end{tagblock}

%-------------------------------------------------------------------------------------------------------------

\begin{tagblock}{Derivatives, Optimization}

\begin{question}

Find the dimensions of a rectangle with maximum area and perimeter equal to $800$.
\begin{enumerate}
\item Draw a picture of a rectangle and label the sides (here you will need to choose variables)

\vspace{1in}
\item The problem asks us to maximize their area, so the equation to be optimized is\\
\bigskip
 $$A=\underline{\hspace{2in}} .$$
 \item What constraints do we have?  How can we use this to write the area in terms of one variable?
 
 \vspace{1.5in} 
 \item Once your area is a function of one variable, find the maximum.  Make sure you check that your critical number indeed gives a max!
 
\vfill
 \item Going back to our original question, we want the dimensions of the rectangle.  You found one dimension in (d), now find the other one.
 
 \end{enumerate}



\index{Derivatives}
	
\begin{tags}
	   Derivatives, Optimization

\end{tags}
	
\begin{diary}
	    %S2016-HW1-Q1
\end{diary}
	
\begin{solution}
	   
\end{solution}
	
\end{question}

\end{tagblock}

%-------------------------------------------------------------------------------------------------------------

\begin{tagblock}{Derivatives, Optimization}

\begin{question}

A box will have a square base, an open top and a volume of 4000 $\text{cm}^2$.  What are the dimensions of the box to minimize the surface area?
 \begin{enumerate}
 \item Draw a picture of your box and label the sides.  Note that you will have two dimensions: one that gives the length of the square base and one that gives the height.
 
 \vspace{1.5in}
\item The problem asks us to minimize the surface area.  The surface area will consist of the area of the square base and the area of the four sides (no top!)  so the equation to be optimized is\\
\bigskip
 $$SA=\underline{\hspace{2in}} .$$
 
  \item What constraints do we have?  How can we use this to write the surface area in terms of one variable?
 
 \vspace{1in} 
  \item Once your area is a function of one variable, find the minimum.  Make sure you check that your critical number indeed gives a min!
 
\vfill
 \item Going back to our original question, we want the dimensions of the box.  You found one dimension in (d), now find the other one.
 
 \end{enumerate}



\index{Derivatives}
	
\begin{tags}
	   Derivatives, Optimization

\end{tags}
	
\begin{diary}
	    %S2016-HW1-Q1
\end{diary}
	
\begin{solution}
	   
\end{solution}
	
\end{question}

\end{tagblock}

%-------------------------------------------------------------------------------------------------------------

\begin{tagblock}{Derivatives, Optimization}

\begin{question}

We'll modify the previous problem.  We still want a box with a square base, an open top and a volume of 4000 $\text{cm}^2$.  Suppose material for the base costs $\$.10$ per $\text{cm}^2$ and the material for the sides cost $\$.02$  per $\text{cm}^2$.  Find the cheapest cost of such a box.  
\begin{enumerate}
 \item Notice we have the same picture as in 1.
\item The problem asks us to minimize the cost, so the equation to be optimized is \\
 \bigskip
 $$C=\underline{\hspace{3in}} .$$
 \item We have the same constraints as in 1.   How can we use this to write the cost area in terms of one variable?
 \vspace{1in} 
  \item Once your area is a function of one variable, find the minimum.  Make sure you check that your critical number indeed gives a min!
 
\vspace{3in}
 \item Going back to our original question, we want the cheapest cost.  Find the cheapest cost.
 \end{enumerate}



\index{Derivatives}
	
\begin{tags}
	   Derivatives, Optimization

\end{tags}
	
\begin{diary}
	    %S2016-HW1-Q1
\end{diary}
	
\begin{solution}
	   
\end{solution}
	
\end{question}

\end{tagblock}

%-------------------------------------------------------------------------------------------------------------

\begin{tagblock}{Derivatives, Optimization}

\begin{question}

A farmer wants to start raising cows, horses, goats, and sheep, and desires to have a rectangular pasture for the animals to graze in. However, no two different kinds of animals can graze together. In order to minimize the amount of fencing she will need, she has decided to enclose a large rectangular area and then divide it into four equally sized pens by adding three segments of fence inside the large rectangle that are parallel to two existing sides. She has decided to purchase 7500 ft of fencing. What is the maximum possible area that the four pens will enclose?


\index{Derivatives}
	
\begin{tags}
	   Derivatives, Optimization

\end{tags}
	
\begin{diary}
	    %S2016-HW1-Q1
\end{diary}
	
\begin{solution}
	   
\end{solution}
	
\end{question}

\end{tagblock}

%-------------------------------------------------------------------------------------------------------------

\begin{tagblock}{Derivatives, Optimization}

\begin{question}

A farmer wants to start raising cows, horses, goats, and sheep, and desires to have a rectangular pasture for the animals to graze in. However, no two different kinds of animals can graze together. In order to minimize the amount of fencing she will need, she has decided to enclose a large rectangular area and then divide it into four equally sized pens by adding three segments of fence inside the large rectangle that are parallel to two existing sides. She has decided to purchase 7500 ft of fencing. What is the maximum possible area that the four pens will enclose?


\index{Derivatives}
	
\begin{tags}
	   Derivatives, Optimization

\end{tags}
	
\begin{diary}
	    %S2016-HW1-Q1
\end{diary}
	
\begin{solution}
	   
\end{solution}
	
\end{question}

\end{tagblock}

%-------------------------------------------------------------------------------------------------------------

\begin{tagblock}{Derivatives, Optimization}

\begin{question}

Consider a rectangle with bottom left vertex at the origin $(0,0)$ and top right vertex on the graph of $y=-x^2 +4$.  What is the largest possible area of such a rectangle?  As always, justify your answer using calculus.  
\begin{figure}[h]
%\centering
\includegraphics[width=5cm]{optimization6.png} 
\end{figure}


\index{Derivatives}
	
\begin{tags}
	   Derivatives, Optimization

\end{tags}
	
\begin{diary}
	    %S2016-HW1-Q1
\end{diary}
	
\begin{solution}
	   
\end{solution}
	
\end{question}

\end{tagblock}

%-------------------------------------------------------------------------------------------------------------

\begin{tagblock}{Derivatives, Optimization, Challenge}

\begin{question}

A piece of wire $10$ cm long is cut into two pieces. One piece is bent into a square. The second piece is bent into a rectangle whose width is twice its length. Where should the wire be cut in order to \textbf{minimize} the total area of the square and the rectangle? Where should the wire be cut in order to \textbf{maximize} the total area of the square and the rectangle?


\index{Derivatives}
	
\begin{tags}
	   Derivatives, Optimization, Challenge

\end{tags}
	
\begin{diary}
	    %S2016-HW1-Q1
\end{diary}
	
\begin{solution}
	   
\end{solution}
	
\end{question}

\end{tagblock}
\begin{tagblock}{Differentiation, W1}
\begin{question}
	Find the first derivative of $f(x)=x^2\tan(3x+1)$.
	
\index{Derivatives}
	
\begin{tags}
	    Differentiation, ProductRule, ChainRule, Trig
\end{tags}
	
\begin{diary}
	    %S2016-HW1-Q1
\end{diary}
	
\begin{solution}
	   
\end{solution}
	
\end{question}

\end{tagblock}

%-------------------------------------------------------------------------------------------------------------

\begin{tagblock}{Differentiation, ChainRule, FundamentalTheoremI, SquareRoots, W1}
\begin{question}
	Compute $\displaystyle\frac {dg}{dx}$ where $\displaystyle g(x)=\int_3^{x^5}\sqrt{1+t^6} \ dt$.
	
\index{Derivatives}
	
\begin{tags}
	    Differentiation, FundamentalTheoremI, ChainRule, SquareRoots
\end{tags}
	
\begin{diary}
	    %S2016-HW1-Q1
\end{diary}
	
\begin{solution}
	   
\end{solution}
	
\end{question}

\end{tagblock}

%-------------------------------------------------------------------------------------------------------------

\begin{tagblock}{Differentiation, W3, Exponentials, Inverse, Theory}
\begin{question}
	Last class, we defined $\ln(x)=\displaystyle\int^x_1 \frac 1 t \ dt$. We defined $e$ as the number with $\ln(e)=1$. Our point all along has been to solve a certain differential equation, which we will do at the end of this worksheet. 

\bigskip
a) Remind yourselves what $\displaystyle\frac d {dx}\ln(x)$ is equal to for $x>0$. 

\bigskip

b) Remind yourselves that if a function is always increasing, it is invertible with respect to function composition. What condition on the derivative guarantees that a function is increasing?

\bigskip
                
c) Combine parts a) and b) to show that $\ln$ is invertible for $x>0$.


	
\index{Derivatives}
	
\begin{tags}
	    Differentiation, W3, Exponentials, Inverse, Theory
\end{tags}
	
\begin{diary}DFV            
	    %S2016-HW1-Q1
\end{diary}
	
\begin{solution}
	   
\end{solution}
	
\end{question}

\end{tagblock}

%-------------------------------------------------------------------------------------------------------------

\begin{tagblock}{Differentiation, W3, Exponentials, Inverse, Theory}
\begin{question}
	Now we will try to describe the inverse function to $\ln$ fully. 

\bigskip

a) Using the fact that $\ln(x^r)=r\ln(x)$ for all $x>0$ and all $r$, show that $f(x)=e^x$ is the inverse function to $\ln$.  

\bigskip


b) Write 
\[
x=\ln(e^x)=\ln(f(x))
\]
and differentiate both sides, using the chain rule on the right hand side, then solve for $f'(x)$.

\bigskip

c) Plug $e^x$ back in for $f(x)$ to find the derivative of $e^x$. 
	
\index{Derivatives}
	
\begin{tags}
	    Differentiation, W3, Exponentials, Inverse, Theory
\end{tags}
	
\begin{diary}
	    %S2016-HW1-Q1
\end{diary}
	
\begin{solution}
	   
\end{solution}
	
\end{question}

\end{tagblock}

%-------------------------------------------------------------------------------------------------------------

\begin{tagblock}{Differentiation, W1, TangentLines, PowerRule}
\begin{question}
	

\bigskip

Find the equation of the tangent line to $q(x)=x^{3/2}+4x^2+15$ at the point $a=1$.

\bigskip


	
\index{Derivatives}
	
\begin{tags}
	    Differentiation, W1, TangentLines, PowerRule
\end{tags}
	
\begin{diary}
	    %S2016-HW1-Q1
\end{diary}
	
\begin{solution}
	   
\end{solution}
	
\end{question}

\end{tagblock}

%-------------------------------------------------------------------------------------------------------------

\begin{tagblock}{Differentiation, W1, PowerRule, ProductRule, ChainRule, Trigonometry}
\begin{question}
	

\bigskip

Find the first derivative of $f(x)=x^2\tan(3x+1)$.

\bigskip


	
\index{Derivatives}
	
\begin{tags}
	    Differentiation, W1, PowerRule, ProductRule, ChainRule, Trigonometry
\end{tags}
	
\begin{diary}
	    %S2016-HW1-Q1
\end{diary}
	
\begin{solution}
	   
\end{solution}
	
\end{question}

\end{tagblock}

%-------------------------------------------------------------------------------------------------------------

\begin{tagblock}{W1, Integration, Substitution, Integration, Trigonometry, DefiniteIntegral}
\begin{question}
	

\bigskip

Determine the value of $\displaystyle\int_0^{\sqrt{\frac {\pi} 4}} 2x\cos(x^2) \ dx$.

\bigskip


	
\index{Derivatives}
	
\begin{tags}
	    W1, Integration, Substitution, Integration, Trigonometry, DefiniteIntegral
\end{tags}
	
\begin{diary}
	    %S2016-HW1-Q1
\end{diary}
	
\begin{solution}
	   
\end{solution}
	
\end{question}

\end{tagblock}
%-------------------------------------------------------------------------------------------------------------
 

\section{Exam Questions}\index{Exam}
\fancyhead[R]{\large Exam Questions}




 \section{Geometry}\index{Geometry}
\fancyhead[R]{\large Geometry}

\begin{tagblock}{TangentLines, W1, S2016, Geometry}
\begin{question}
	Find the equation of the tangent line to $q(x)=x^{3/2}+4x^2+15$ at the point $a=1$.
	
\index{Geometry}
	
\begin{tags}
	    Geometry, TangentLines, Differentiation
\end{tags}
	
\begin{diary}
	    S2016-HW1-Q1
\end{diary}
	
\begin{solution}
	   
\end{solution}
	
\end{question}

\end{tagblock}

%-------------------------------------------------------------------------------------------------------------
 
\section{Integration}\index{Integration}
\fancyhead[R]{\large Integration}



\begin{tagblock}{Integration, AntiDerivative, Definition, WarmUp, Polynomial}
\begin{question}

	So far, we have learned about derivatives and we know they represent rates of change. We next ask whether the process can be reversed. In other words, if you are given a rate of change, can you produce the function that was differentiated to get that rate? For example, given the velocity of an object, we want to find a function that gives the object's position.
\bigskip

 Here is our first key definition for this worksheet: \\
 \bigskip

A function $F(x)$ is called an \textbf{antiderivative} of $f(x)$ on an interval $I$ if:
$$F'(x) = f(x) \mbox{  for all  } x \mbox{  in  } I\,.$$


Let's look at some examples.


\begin{enumerate}
\item Show that $F_1(x) = \frac{x^3}{3} +2 x +2$ is an antiderivative of $f(x) = x^2 +2$. Hint: Look at $F_1'(x)$.
\vspace{1in}

\item Show that $F_2(x) = \frac{x^3}{3} +2 x +18\pi$ is also an antiderivative of $f(x) = x^2 +2$. 
\vspace{1in}
\end{enumerate}

Notice that the antiderivative is not unique and one function may have two (or more!) antiderivatives. Since the derivative of any constant is zero, $F(x) = \frac{x^3}{3} +2 x +C$ is an antiderivative of  $f(x) = x^2 +2$ for any constant $C$. This means we can shift the graph of $F(x)$ up or down by any amount without changing the values for $F'(x) = f(x)$.
\bigskip
Our second key definition is:\\
\bigskip

If $F(x)$ is an antiderivative of $f(x)$ on an interval $I$, then the \textbf{most general antiderivative} of $f$ is
$$F(x) + C$$
where $C$ is an arbitrary constant.
	
\index{Integration}
	
\begin{tags}
	    Integration, AntiDerivative, Definition, WarmUp, Polynomial
\end{tags}
	
\begin{diary}
	   
\end{diary}
	
\begin{solution}
	   
	    \end{enumerate}
\end{solution}
	
\end{question}

\end{tagblock}

%-------------------------------------------------------------------------------------------------------------

\begin{tagblock}{Integration, AntiDerivative}
\begin{question}

	Find the most general antiderivative for each function below. \textbf{Check your answer by differentiating it.}
\begin{enumerate}
\item $\displaystyle{f(x) = \sin x}$
\vspace{1in}
\item $\displaystyle{f(x) = x^n}$ for $n\geq 0$.
\vspace{1in}
\item $\displaystyle{f(x) = x^{1/2}}$
\vspace{1in}
\end{enumerate}

The table below lists a few handy anti-differentiation rules. We are assuming that $G$ is an antiderivative of $g$, $F$ is an antiderivative of $f$, and $a$ is a constant.\\
\begin{table}[h]

\begin{center}
\begin{tabular}{|c|c||c|c|}\hline
Function & Antiderivative & Function & Antiderivative \\ 
& & & \\ \hline
$af(x)$ & $aF(x) + C$ & $\sin x$ & $-\cos x+C$ \\
& & & \\
$f(x) + g(x)$ & $F(x)+G(x)+C$ & $\cos x$ & $\sin x+C$ \\
& & & \\
$x^n$ for $n\neq -1$ & $\frac{x^{n+1}}{n+1} + C$ & $\sec^2 x$ & $\tan x +C$ \\ 
& & & \\
$\frac{1}{x}$ for $x>0$ & $\ln(x)+ C$ & $ \sec x \tan x$ & $\sec x +C$ \\
 & & & \\  
 $b$, for $b$ a constant & $bx+C$ &  $e^x$ & $e^x+C$  \\
 &&& \\ \hline



\end{tabular}
\end{center}
\label{default}
\end{table}%

Take a moment to make sure you understand where each rule in the table comes from.
	
\index{Integration}
	
\begin{tags}
	    Integration, AntiDerivative
\end{tags}
	
\begin{diary}
	   
\end{diary}
	
\begin{solution}
	   
	    \end{enumerate}
\end{solution}
	
\end{question}

\end{tagblock}

%-------------------------------------------------------------------------------------------------------------

\begin{tagblock}{Integration, AntiDerivative}
\begin{question}

Find the most general antiderivative for each function below.  Note in some cases you may need to rewrite your function before anti-differentiating.  
\begin{enumerate}
\item $\displaystyle{f(x)= \frac{1}{2} x^2 - 2 x+\pi^2 +\frac{3}{x} + \sqrt{x}}$
\vspace{1in}
\item $\displaystyle{h(t)=2\sqrt{t} + \frac{\sec t}{\cot t} + 2 \cos t}$
\vspace{1in}
\end{enumerate}
	
\index{Integration}
	
\begin{tags}
	    Integration, AntiDerivative
\end{tags}
	
\begin{diary}
	   
\end{diary}
	
\begin{solution}
	   
	    \end{enumerate}
\end{solution}
	
\end{question}

\end{tagblock}

%-------------------------------------------------------------------------------------------------------------

\begin{tagblock}{Integration, AntiDerivative}
\begin{question}

Sometimes we are given more information and can find $C$.  

\bigskip

Find $f$ for each function below. Make sure your function satisfies all the conditions given.
\begin{enumerate}
\item $f'(x) = x^2 + \sin(x), \quad f(0)=4$.
\vspace{1.5in}

\item $\displaystyle{f'(x)=x (3-x)^2\,,\quad f(0)=1}$
\vspace{1.8in}
\item $\displaystyle{f''(x)=x \sqrt{x}\,,\quad f'(1)=2\,,\quad f(0)=0}$
\vspace{1.8in}
\item $\displaystyle{f''(x)=20x^3-4x^2\,,\quad f(0)=3\,,\quad f(1)=4}$

\end{enumerate}
	
\index{Integration}
	
\begin{tags}
	    Integration, AntiDerivative
\end{tags}
	
\begin{diary}
	   
\end{diary}
	
\begin{solution}
	   
	    \end{enumerate}
\end{solution}
	
\end{question}

\end{tagblock}

%-------------------------------------------------------------------------------------------------------------



\begin{tagblock}{Integration, AntiDerivative, Velocity}
\begin{question}

These problems involve applications of objects moving in a straight line. Given a function that measures an object's position At time $t$, $s(t)$, we know that $s'(t) = v(t)$ is the velocity of the object. This means that given a velocity function $v(t)$, we can find the function that represents the position by using the fact that $s(t)$ is the antiderivative of $v$. 

\bigskip

The velocity of a particle moving in a straight line is given by $v(t)= 6 t +4$. If the initial position of the object is $s(0)=-2$, find the position function.
\vspace{1in}

\item A ball is thrown upward with a speed of $48$  $ft/sec$ from the edge of a cliff, $432$ $ft$ above the ground. Assume the acceleration of the ball is constant $-32$ $ft/sec^2$.
\begin{enumerate}
\item Write an equation for the acceleration $a(t)$.
\vspace{.5in}
\item Find the velocity of the ball at any time $t$. Your answer should not involve the arbitrary constant $C$. Hint: $v'(t) = a(t)$.
\vspace{1in}
\item Find the position of the ball at any time $t$. Your answer should not involve the arbitrary constant $C$.
\vspace{1in}

\item At what time does the ball reach the ground?
\end{enumerate}


	
\index{Integration}
	
\begin{tags}
	    Integration, AntiDerivative, Velocity
\end{tags}
	
\begin{diary}
	   
\end{diary}
	
\begin{solution}
	   
	    \end{enumerate}
\end{solution}
	
\end{question}

\end{tagblock}

%-------------------------------------------------------------------------------------------------------------

\begin{tagblock}{Integration, Velocity, WarmUp, Area}
\begin{question}

Suppose that a person is taking a walk along a long straight path and walks at a constant rate of $3$ miles per hour. 
\begin{enumerate}
\item  On the graph below, sketch the velocity function $v(t)=3$.

\hspace{1.25in}\includegraphics[width=2.5in]{dist1.png}

\item How far did the person travel during the two hours? How is this distance related to the area of a certain region under the graph of $y=v(t)$?
\end{enumerate}


\vfill
If the velocity is not constant, the same idea will hold true, and the distance will be given by the \textbf{area under the velocity graph}.  So our next goal will be to find the area under the graph of a function.  

\textbf{Main idea: estimate the area with rectangles!}


	
\index{Integration}
	
\begin{tags}
	    Integration, Velocity, WarmUp, Area
\end{tags}
	
\begin{diary}
	   % COPYRIGHT PROBLEMS?
\end{diary}
	
\begin{solution}
	   
	    \end{enumerate}
\end{solution}
	
\end{question}

\end{tagblock}

%-------------------------------------------------------------------------------------------------------------

\begin{tagblock}{Integration, Velocity, RiemannSum, Area}
\begin{question}

A person walking along a straight path has her velocity in miles per hour at time t given by the function $v(t)=0.25t^3-1.5t^2+3t+0.25$, for times in the interval $0\leq t \leq 2$. The graph of this function is also given in each of the three diagrams below

\hspace{1.25in}\includegraphics[width=6in]{dist2.png}

Note that in each diagram, we use four rectangles to estimate the area under $y=v(t)$ on the interval $[0,2]$, but the method by which the four rectangles' respective heights are decided varies among the three individual graphs. 
\begin{enumerate}
\item What is the width of each rectangle?
\vspace{.5in}

\item How are the heights of rectangles in the left-most diagram being chosen? Explain, and hence determine the value of
\[S=A_1 + A_2 + A_3 + A_4\]
by evaluating the function $y=v(t)$  at appropriately chosen values and observing the width of each rectangle. Note, for example, that
$A_3=v(1)\cdot\frac{1}{2}=2\cdot\frac{1}{2}=1.$

\vspace{1.5in}
\item Explain how the heights of rectangles are being chosen in the middle diagram and find the value of
\[T=B_1 + B_2 + B_3 + B_4\]

\vspace{1in}
\item Likewise, determine the pattern of how heights of rectangles are chosen in the right-most diagram and determine
\[U=C_1+C_2+C_3+C_4.\]

\vspace{1.7in}

\item Of the estimates $S,T$, and $U$, which do you think is the best approximation of $D$, the total distance the person traveled on $[0,2]$? Why?

\vspace{1in}
\item If we used more rectangles would we get a better estimate?  Explain why or why not.  

\end{enumerate}


	
\index{Integration}
	
\begin{tags}
	    Integration, Velocity, RiemannSum, Area
\end{tags}
	
\begin{diary}
	   % COPYRIGHT PROBLEMS?
\end{diary}
	
\begin{solution}
	   
	    \end{enumerate}
\end{solution}
	
\end{question}

\end{tagblock}

%-------------------------------------------------------------------------------------------------------------

\begin{tagblock}{Integration, Definition, Example, Area, Graph}
\begin{question}

We have defined the \textbf{definite integral of $f(x)$ from $x=a$ to $x=b$} 
\[ \int_a^b f(x) \, dx = \lim_{n \to \infty} \sum_{i=1}^n f\left(a+ \left(\frac{b-a}{n}\right) i \right)\left(\frac{b-a}{n}\right) \]

Or more simply stated $\int_a^b f(x) \, dx$ is the ``net area'' under the graph of $f(x)$ from $x=a$ to $x=b$, that is any area below the $x$-axis will count as negative area.  

\bigskip

\textbf{Example:} To compute $\int_0^{2\pi} \sin(x) \,dx$, we are looking for the area shaded below
\[ \includegraphics[width=2in]{sinintegral.png}\]
Note that the area of $A$ is the same as the area of $B$, but $B$ is below the $x$-axis, so 
\[ \int_0^{2\pi} \sin(x) \,dx = \text{Area of A} - \text{Area of B} = 0 \]

A first strategy to compute definite integrals is as follows:  
\begin{itemize}
\item Graph the function
\item Shade the area we want to compute 
\item Hope we can break up shapes whose areas we know how to compute (like rectangles, triangles, circles). 
\item Compute the area of each piece and add up those that are above the $x$-axis and subtract those below the $x$-axis.
\end{itemize}

\bigskip

For each of the following definite integrals $\int _a^b f(x) \, dx$, graph the function $f(x)$, shade the area represented by the definite integrals and then compute the area.
\begin{enumerate}
\item $\displaystyle \int_0^3 2 \, dx$
\vspace{1in}
\item $\displaystyle \int _{-1}^3 4x \, dx$
\vspace{2in}
\item $\displaystyle \int_{0}^3 |x| -1 \, dx$
\vspace{2in}
\item $\displaystyle \int_{0}^3 \sqrt{9-x^2} \, dx$ \\ \emph{Hint: Remember that a circle with center at the origin and radius $r$ is given by the equation $x^2+y^2=r^2$}

\end{enumerate} 


	
\index{Integration}
	
\begin{tags}
	    Integration, Definition, Example, Area, Graph
\end{tags}
	
\begin{diary}
	   % COPYRIGHT PROBLEMS?
\end{diary}
	
\begin{solution}
	   
	    \end{enumerate}
\end{solution}
	
\end{question}

\end{tagblock}

%-------------------------------------------------------------------------------------------------------------


\begin{tagblock}{Integration, Area, Graph, DefiniteIntegral, Piecewise}
\begin{question}

Let $g(x)$ be the peicewise function given by the graph below; each piece of the function is part of a circle or part of a line.
\[ \includegraphics[width=4in]{integralgraph.png}\]

\begin{enumerate}
\item Compute $\displaystyle \int_{-3}^{-2} g(x) \, dx$
\vspace{1in}
\item Compute $\displaystyle \int_{-2}^{-1} g(x) \, dx$
\vspace{1in}
\item Compute $\displaystyle \int_{-3}^{-1} g(x) \, dx$.  How is this related to your answers from (a) and (b)?
\vspace{1in}
\item Compute $\displaystyle \int_{2}^{2} g(x) \, dx$
\vspace{.5in}
\item Compute $\displaystyle \int_{-3}^{4} g(x) \, dx$ 

\end{enumerate}




	
\index{Integration}
	
\begin{tags}
	   Integration, Area, Graph, DefiniteIntegral, Piecewise
\end{tags}
	
\begin{diary}
	   % COPYRIGHT PROBLEMS?
\end{diary}
	
\begin{solution}
	   
	    \end{enumerate}
\end{solution}
	
\end{question}

\end{tagblock}

%-------------------------------------------------------------------------------------------------------------

\begin{tagblock}{Integration, DefiniteIntegral, Theory}
\begin{question}

What follows are some \emph{Properties of the Definite Integral:} If $f$ and $g$ are continuous functions then 
\begin{itemize}
\item $\displaystyle \int_{a}^{a} f(x) \, dx = 0$
\item $\displaystyle \int_{a}^{b} f(x) \, dx +  \int_{b}^{c} f(x) \, dx =  \int_a^c f(x) \, dx$
\item $\displaystyle \int_{a}^{b} f(x) \, dx =  -\int_{b}^{a} f(x) \, dx$
\item $\displaystyle \int_{a}^{b} (f(x) \pm g(x)) \, dx = \int_{a}^{b} f(x) \, dx  \pm \int_{a}^{b} g(x)) \, dx$
\item $\displaystyle \int_{a}^{b} cf(x)  dx =c \int_{a}^{b} f(x) \, dx $,  where $c$ is a constant.  

\end{itemize}

Suppose we know $\displaystyle \int_0^2 f(x) \, dx =-3$, $\displaystyle \int_2^5 f(x) \, dx =2$, $\displaystyle \int_0^5 g(x) \, dx =4$ and $\displaystyle \int_2^5 g(x) \, dx =-1$.  Use the Properties above to evaluate the following definite integrals
\begin{enumerate}
\item $\displaystyle \int_5^2 f(x) \, dx $

\vspace{1in}
\item $\displaystyle \int_0^5 (3f(x) + g(x)) \, dx $
\vspace{1.5in}

\item $\displaystyle \int_0^2 g(x) \, dx $

\end{enumerate}


	
\index{Integration}
	
\begin{tags}
	   Integration, DefiniteIntegral, Theory
\end{tags}
	
\begin{diary}
	   % COPYRIGHT PROBLEMS?
\end{diary}
	
\begin{solution}
	   
	    \end{enumerate}
\end{solution}
	
\end{question}

\end{tagblock}

%-------------------------------------------------------------------------------------------------------------
\begin{tagblock}{Integration, FundamentalTheoremI, DefiniteIntegral, WarmUp, TangentLines, Theory, Graph}
\begin{question}

We have defined the \textbf{definite integral of $f(x)$ from $x=a$ to $x=b$} 
\[ \int_a^b f(x) \, dx = \lim_{n \to \infty}( \sum_{i=1}^n f(a+ (\frac{b-a}{n}) i )(\frac{b-a}{n})) \]

Or more simply stated $\int_a^b f(x) \, dx$ is the ``net area'' under the graph of $f(x)$ from $x=a$ to $x=b$, that is any area below the $x$-axis will count as negative area.  

\bigskip


Let $y=f(t)$ be the graph given below. 
\begin{enumerate}
\item Using the graph of $y=f(t)$, determine $f(1) =   \rule{1.5cm}{.1mm}$, $f(3) =   \rule{1.5cm}{.1mm}$, $f(4) =   \rule{1.5cm}{.1mm}$ and $f(5) =   \rule{1.5cm}{.1mm}$.  



 \item We will define a new function $g(x) = \int_0^x f(t) \, dt$, so that $g(x)$ simply computes the area under the graph of $f$ from $0$ to $x$.  \\ 

\begin{minipage}{.4\textwidth}
\includegraphics[width=6cm]{FTCg1.png}\end{minipage}% This must go next to `\end{minipage}`
\begin{minipage}{.6\textwidth}
For example $g(0) = \int_0^0 f(t) \, dt = 0$.  \\ \\ Compute the following:  \\ \\
\begin{tabular}{lll}
 $g(1) = \int_0^1 f(t) \, dt =   \rule{1.5cm}{.1mm}$ &\hspace{.2in} \\ \\
  $g(2) =  \int_0^2 f(t) \, dt =   \rule{1.5cm}{.1mm}$ \\ \\
$g(3) =  \rule{1.5cm}{.1mm}$ &\hspace{.2in} & $g(4) =  \rule{1.5cm}{.1mm}$ \\ \\ 
$g(5) =  \rule{1.5cm}{.1mm}$ &\hspace{.2in} & $g(6) =  \rule{1.5cm}{.1mm}$ \\ \\ 
 \end{tabular}
\end{minipage}

\bigskip

\item Below is a graph of $g(x)$, verify that the points you found above are on the graph of $g(x)$:

\begin{minipage}{.4\textwidth}
\includegraphics[width=6cm]{FTCg2.png}\end{minipage}% This must go next to `\end{minipage}`
\begin{minipage}{.6\textwidth}

Draw the tangent line to $g(x)$ at $x=1$, and estimate the slope of the tangent line, this then gives   $g'(1) =  \rule{1.5cm}{.1mm}$ \\
\bigskip

Draw tangent lines at $x=3, 4$ and $5$ to determine the values below.  

\bigskip

$g'(3) =  \rule{1.5cm}{.1mm}$ 
\bigskip

$g'(4) =  \rule{1.5cm}{.1mm}$ 

\bigskip

$g'(5) =  \rule{1.5cm}{.1mm}$ 
 \end{minipage}
 
 \bigskip
\item  Do you notice any relationship between $g'(1)$ and $f(1)$?  between $g'(3)$ and $f(3)$? between $g'(4)$ and $f(4)$?  between $g'(5)$ and $f(5)$?
 
\end{enumerate}


\bigskip

This relationship between derivatives and integrals holds in general and is the FUNdamental Theorem of Calculus, Part 1: 

\bigskip

\textbf{FUNdamental Theorem of Calculus, Part 1}:  If $f(x)$ is  continuous on $[a,b]$, and $g(x) =  \int_a^x f(t) \, dt$, then $g(x)$ is a continuous function on $[a,b]$, $g(x)$ is differentiable on $(a,b)$, \textbf{and}  $g'(x) = f(x). $

In other words:
\[ \frac{d}{dx} [ \int_a^x f(t) \, dt ] = f(x) \]
So that the derivative undoes the integral!

	
\index{Integration}
	
\begin{tags}
	   Integration, FundamentalTheoremI, DefiniteIntegral, WarmUp, TangentLines, Theory, Graph
\end{tags}
	
\begin{diary}
	   % COPYRIGHT PROBLEMS?
\end{diary}
	
\begin{solution}
	   
	    \end{enumerate}
\end{solution}
	
\end{question}

\end{tagblock}

%-------------------------------------------------------------------------------------------------------------
\begin{tagblock}{Integration, FundamentalTheoremI, Derivatives, ChainRule}
\begin{question}

\text{Let  }\[ g(x) =  \int_0^x \cos(t^3+1) \, dt, \,   h(x) =  \int_3^x \cos(t^3+1) \, dt,  \,
 j(x) =  \int_x^0 \cos(t^3+1) \, dt, \text{ and } \,  k(x) =  \int_3^{x^2} \cos(t^3+1) \, dt\]

\begin{enumerate}
\item Use the Fundamental Theorem of Calculus, Part 1 to compute the \textbf{derivative} of $g(x)$.

\vspace{.5in}

\item What is the difference between the function $g(x)$ and $h(x)$?  Use the Fundamental Theorem of Calculus, Part 1 to compute the \textbf{derivative} of $h(x)$.  How do $g'(x)$ and $h'(x)$ compare?
\vspace{1in}

\item What is the difference between the function $g(x)$ and $j(x)$?  Use the Fundamental Theorem of Calculus, Part 1 to compute the \textbf{derivative} of $j(x)$. \\
\emph{Hint:  Remember how $\int_a^b f(t) \, dt$ and $\int_b^a f(t) \, dt$ are related}

\vspace{1in}

\item What is the difference between the function $g(x)$ and $k(x)$?   Use the Chain Rule together with Fundamental Theorem of Calculus, Part 1 to compute the \textbf{derivative} of $k(x)$. \\
\emph{Hint: We can express k(x) as a composition with $g(x) =$ the outside function and $x^2 =$ the inside function}.

\end{enumerate}

	
\index{Integration}
	
\begin{tags}
	   Integration, FundamentalTheoremI, Derivatives, ChainRule
\end{tags}
	
\begin{diary}
	   % COPYRIGHT PROBLEMS?
\end{diary}
	
\begin{solution}
	   
	    \end{enumerate}
\end{solution}
	
\end{question}

\end{tagblock}

%-------------------------------------------------------------------------------------------------------------
\begin{tagblock}{Integration, FundamentalTheoremII, DefiniteIntegral, Graph, Area, Theory, AntiDerivative }
\begin{question}

As a useful application the Fundamental Theorem of Calculus, Part 1 can help us compute definite integrals!\\

\textbf{Fundamental Theorem of Calculus, Part 2}:  If $f(x)$ is continuous on $[a,b]$, then 
\[ \int_a^b f(x) \, dx = F(x) |_a^b = F(b) - F(a), \text{ where $F(x)$ is \emph{any} anti-derivative of $f(x)$} \]

The FTC, Part 2 in practice is the more useful part.  If we want to compute $\displaystyle  \int_a^b f(x) \, dx$ using the FTC, Part 2, we 
\begin{itemize}  
\item Find an anti-derivative $F(x)$ of $f(x)$,
\item Evaluate $F(x)$ at the endpoints $b$ and $a$,
\item Compute the difference $F(b) - F(a)$.
\end{itemize}


Set up an integral that computes the area under $f(x)=x^3$ from $x=0$ to $x=1$, then use the Fundamental Theorem of Calculus, Part 2 to find the exact area.  

\includegraphics[width=4cm]{FTCx3.png}


	
\index{Integration}
	
\begin{tags}
	   Integration, FundamentalTheoremII, DefiniteIntegral, Graph, Area, Theory, AntiDerivative
\end{tags}
	
\begin{diary}
	   % COPYRIGHT PROBLEMS?
\end{diary}
	
\begin{solution}
	   
	    \end{enumerate}
\end{solution}
	
\end{question}

\end{tagblock}

%-------------------------------------------------------------------------------------------------------------
\begin{tagblock}{Integration, FundamentalTheoremII, DefiniteIntegral, Graph, Area, AntiDerivative }
\begin{question}

Compute the definite integral $\displaystyle \int_0^\pi \sin(x) \, dx$ and sketch a graph of the area that the definite integral computes.


	
\index{Integration}
	
\begin{tags}
	   Integration, FundamentalTheoremII, DefiniteIntegral, Graph, Area, AntiDerivative
\end{tags}
	
\begin{diary}
	   % COPYRIGHT PROBLEMS?
\end{diary}
	
\begin{solution}
	   
	    \end{enumerate}
\end{solution}
	
\end{question}

\end{tagblock}

%-------------------------------------------------------------------------------------------------------------
\begin{tagblock}{Integration, FundamentalTheoremII, DefiniteIntegral, Graph, Area, AntiDerivative }
\begin{question}

Compute the definite integral $\displaystyle \int_4^9 \sqrt{x} \, dx$ and sketch a graph of the area that the definite integral computes.


	
\index{Integration}
	
\begin{tags}
	   Integration, FundamentalTheoremII, DefiniteIntegral, Graph, Area, AntiDerivative
\end{tags}
	
\begin{diary}
	   
\end{diary}
	
\begin{solution}
	   
	    \end{enumerate}
\end{solution}
	
\end{question}

\end{tagblock}

%-------------------------------------------------------------------------------------------------------------
\begin{tagblock}{Integration, FundamentalTheoremII, DefiniteIntegral, Graph, Area, AntiDerivative, Polynomial, Absolute }
\begin{question}

Below is a graph of $f(x) = x^2+2x-3$.  

 \includegraphics[width=4cm]{FTCabsval.png} \hspace{1in}  \includegraphics[width=4cm]{FTCblank.png}

 
 \begin{enumerate}

\item Shade the area given by $\displaystyle \int_{-4}^2 x^2+2x-3 \, dx$ and compute the area using the FTC, Part 2.

\vspace{1in}


\item  We'll next compute   $\displaystyle \int_{-4}^2 |x^2+2x-3| \, dx$.  

\begin{enumerate}
\item Start by finding where  $x^2+2x-3$ crosses the $x$-axis and sketch a graph of $|x^2+2x-3|$ on the blank axis above.  

\vspace{1in}

 
\item  Rewrite  $|x^2+2x-3|$ on the interval $-4 \leq x \leq 2$ as a piecewise function to eliminate the absolute value. 

\[ |x^2+2x-3| = \begin{cases}  \hspace{1.5in} & \text{ if }   -4 \leq x \leq -3\\ \\
 \hspace{1in} & \text{ if }   -3 \leq x \leq 1\\ \\

 \hspace{1in} & \text{ if }   1 \leq x \leq 2 \end{cases}\]

\bigskip

\item Next break up the integral  $\displaystyle \int_{-4}^2 |x^2+2x-3| \, dx$ into three pieces, based on the pieces you found  in ii., so that each piece no longer has an absolute value.  

\[  \int_{-4}^2 |x^2+2x-3| \, dx = \int_{-4}^{-3} \hspace{1.15in} dx +  \int_{-3}^1 \hspace{1.15in} dx +  \int_1^2 \hspace{1.15in} dx \]

Then finish up with the Fundamental Theorem of Calculus!  



\vfill
\item TRUE or FALSE:  $\displaystyle | \int_{-4}^2 x^2+2x-3 \, dx | =  \int_{-4}^2 |x^2+2x-3| \, dx$  



\end{enumerate} 
\end{enumerate} 


	
\index{Integration}
	
\begin{tags}
	   Integration, FundamentalTheoremII, DefiniteIntegral, Graph, Area, AntiDerivative, Polynomial, Absolute
\end{tags}
	
\begin{diary}
	   
\end{diary}
	
\begin{solution}
	   
	    \end{enumerate}
\end{solution}
	
\end{question}

\end{tagblock}

%-------------------------------------------------------------------------------------------------------------
\begin{tagblock}{Integration, AntiDerivative, Definition }
\begin{question}

Remember from our work on antiderivatives, given a function $f(x)$, the most general antiderivative was $F(x) + C$, where $C$ is an arbitrary constant and $F'(x) = f(x)$.  We will now denote the most general antiderivative of a function $f(x)$ using the \textbf{indefinite integral of $f(x)$} 
\[ \int f(x) \, dx \]
For example $\displaystyle \int x \, dx = \frac{x^2}{2} + C .$

\textbf{Note:} The answer to a definite integral is a \emph{number}, while the answer to an indefinite integral is a \emph{function}.  

Evaluate the following indefinite integrals.  \textbf{Note that you may need to rewrite your function before you can antidifferentiate.  }
\begin{enumerate}  


\item $\displaystyle \int 3x^2 +4x +5 \, dx $

\vspace{.75in}

\item $\displaystyle \int e^x \, dx $

\vspace{.5in}

\item $\displaystyle \int b^x \, dx $

\vspace{.75in}


\item $\displaystyle \int x(x^2 + \frac{7}{x^2} ) \, dx $

\vspace{1in}


\item $\displaystyle \int \frac{x^2-x+1}{\sqrt{x}}\, dx $


\end{enumerate}





	
\index{Integration}
	
\begin{tags}
	  Integration, AntiDerivative, Definition
\end{tags}
	
\begin{diary}
	   
\end{diary}
	
\begin{solution}
	   
	    \end{enumerate}
\end{solution}
	
\end{question}

\end{tagblock}

%-------------------------------------------------------------------------------------------------------------

\begin{tagblock}{Integration, AntiDerivative, DefiniteIntegral, WarmUp}
\begin{question}

We've looked at two different types of integrals: \textbf{definite integrals} $\int_a^b f(x) \, dx$ and \textbf{indefinite integrals} $\int f(x) \, dx$, and seen that in both cases to evaluate these integrals we need anti-differentiation!


Warm up:  Evaluate the integrals:
\begin{enumerate} 
\item $\displaystyle \int 2x(x^2+3) \, dx$
\vspace{1in}

\item $\displaystyle \int_1^4 2x(x^2+3) \, dx$
\vspace{1in}

\end{enumerate}






	
\index{Integration}
	
\begin{tags}
	  Integration, AntiDerivative, DefiniteIntegral, WarmUp
\end{tags}
	
\begin{diary}
	   
\end{diary}
	
\begin{solution}
	   
	    \end{enumerate}
\end{solution}
	
\end{question}

\end{tagblock}

%-------------------------------------------------------------------------------------------------------------
\begin{tagblock}{Derivative, Differential, Substitution, WarmUp}
\begin{question}

We will develop a tool, often called $u$-substitution, for computing more complicated integrals, in which we can't immediately apply our anti-derivative rules.  

\bigskip


Before diving into more complicated integrals, let's take a step back.  In both the definite integral and indefinite integral we have this slightly mysterious term $dx$.  So far we've treated it as follows:  it tells us which variable we are working in, and it acts like a period, ending our equation sentence.  
\bigskip


$dx$ is called the \textbf{differential with respect to $x$}.  If $y=f(x)$ is a differentiable function of $x$, then we can relate the differential with respect to $y$, $dy$, to the differential with respect to $x$, $dx$, as follows:   
\[\frac{dy}{dx} = f'(x), \text{so that the differential with respect to $y$ is just $dy = f'(x) \, dx$.}\]

For example if $y=x^3$ then $dy = 3x^2 \, dx$

 
\begin{enumerate}
\item If $y=\sin(x)$, find $dy$
\vspace{1in}
\item If $u=x^5+ 3x$, find $du$
\end{enumerate}






	
\index{Integration}
	
\begin{tags}
	  Derivative, Differential, Substitution, WarmUp
\end{tags}
	
\begin{diary}
	   
\end{diary}
	
\begin{solution}
	   
	    \end{enumerate}
\end{solution}
	
\end{question}

\end{tagblock}

%-------------------------------------------------------------------------------------------------------------
\begin{tagblock}{Integration, Substitution, Differential, Derivative, ChainRule }
\begin{question}

Consider the integral  $\displaystyle \int 2x(x^2+3)^{500} \, dx$.  We could multiply this all out and then use our basic anti-derivative rules, but I'd rather not do that.  So we will try to re-write this integral to make it more managable.
\begin{enumerate}
\item Let $u=x^2+3$, and compute $du$.  Do you ``see'' $du$ in our original integral $\displaystyle \int 2x(x^2+3)^{500} \, dx$?  

\vspace{.5in}
\item Rewrite $\displaystyle \int 2x(x^2+3)^{500} \, dx$ so that all the terms involving $x$'s have been replaced with terms involving $u$ and $du$.  Once you do this your integral will be simpler.

\vspace{1in}
\item Next use our anti-derivative rules to antidifferentiate and find the most general antiderivative.  At this point our variable will be a $u$.

\vspace{1in}
\item Now, replace all the $u$'s with $x^2+3$, so that we have the most general antiderivative of $2x(x^2+3)^{500}$.
\vspace{1in}
\item  Lastly, check your work.  That is take the derivative of your final answer and make sure you get back $2x(x^2+3)^{500}$.  Which derivative rules did you need to use?

\end{enumerate}


Notice that when we checked our work in the last problem we use the Generalized Power Rule/Chain Rule, and our method to evaluate the integral $\displaystyle \int 2x(x^2+3)^{500} \, dx$ was the \textbf{Chain Rule in reverse!}  Moreover our choice of $u$ was to choose the ``inside function,'' which gives us some indication of how we might want to choose $u$.   \bigskip

\noindent\fbox{%
    \parbox{\textwidth}{
\textbf{Strategy for applying $u$-substitution to integrals }
\begin{description}
\item[Step 1:] Define $u$ as a function of $x$ (often $u$ will be an ``inside function'')
\item[Step 2:]  Compute $du$.
\item[Step 3:]  Rewrite your entire integral in terms of $u$ (no $x$'s can remain).  Your resulting integral in terms of $u$ should be ``easier'' to work with.
\item[Step 4:]  Use our antiderivative rules to evaluate the integral in terms of $u$.
\item[Step 5:]  Replace all the $u$'s with our function of $x$ from step 1.
\end{description}}}





	
\index{Integration}
	
\begin{tags}
	  Integration, Substitution, Differential, Derivative, ChainRule
\end{tags}
	
\begin{diary}
	   
\end{diary}
	
\begin{solution}
	   
	    \end{enumerate}
\end{solution}
	
\end{question}

\end{tagblock}

%-------------------------------------------------------------------------------------------------------------
\begin{tagblock}{Integration, Substitution, Trigonometry}
\begin{question}

Evaluate the indefinite integral $\displaystyle \int  \sin^3(x) \cos(x) \, dx$.  \\
 \emph{Remember $\sin^3(x) = (\sin(x))^3$}



	
\index{Integration}
	
\begin{tags}
	  Integration, Substitution, Trigonometry
\end{tags}
	
\begin{diary}
	   
\end{diary}
	
\begin{solution}
	   
	    \end{enumerate}
\end{solution}
	
\end{question}

\end{tagblock}

%-------------------------------------------------------------------------------------------------------------
\begin{tagblock}{Integration, Substitution, Trigonometry, Logarithms}
\begin{question}

Evaluate the indefinite integral $\displaystyle \int  \sin^3(x) \cos(x) \, dx$.  \\
 \emph{Remember $\sin^3(x) = (\sin(x))^3$}



	
\index{Integration}
	
\begin{tags}
	  Integration, Substitution, Trigonometry, Logarithms
\end{tags}
	
\begin{diary}
	   
\end{diary}
	
\begin{solution}
	   
	    \end{enumerate}
\end{solution}
	
\end{question}

\end{tagblock}

%-------------------------------------------------------------------------------------------------------------
\begin{tagblock}{Integration, Substitution}
\begin{question}

Sometimes we might need to do a little algebra in addition to our basic strategy.  

Consider the indefinite integral $\displaystyle \int x\sqrt{16-x^2} \, dx$.
\begin{enumerate}
\item Choose $u=16-x^2$ and compute $du$.

\vspace{.5in}

\item  Note that we will have an $x$ and a $dx$ that we need to replace.  Do a little algebra, to express $x \, dx$ in terms of $u$ and/or $du$.

\vspace{.5in}
\item Rewrite the original integral $\displaystyle \int x\sqrt{16-x^2} \, dx$ in terms of $u$ and continue antidifferentiating.  

\vspace{1.5in}
\end{enumerate}



	
\index{Integration}
	
\begin{tags}
	  Integration, Substitution
\end{tags}
	
\begin{diary}
	   
\end{diary}
	
\begin{solution}
	   
	    \end{enumerate}
\end{solution}
	
\end{question}

\end{tagblock}

%-------------------------------------------------------------------------------------------------------------
\begin{tagblock}{Integration, Substitution, Trigonometry}
\begin{question}

Evaluate the indefinite integral $\displaystyle \int \cos(7x) \, dx$. 



	
\index{Integration}
	
\begin{tags}
	  Integration, Substitution, Trigonometry
\end{tags}
	
\begin{diary}
	   
\end{diary}
	
\begin{solution}
	   
	    \end{enumerate}
\end{solution}
	
\end{question}

\end{tagblock}

%-------------------------------------------------------------------------------------------------------------
\begin{tagblock}{Integration, Substitution}
\begin{question}

Evaluate the indefinite integral $\displaystyle \int (x+1)\sqrt[3]{x+5} \ dx$. 



	
\index{Integration}
	
\begin{tags}
	  Integration, Substitution
\end{tags}
	
\begin{diary}
	   
\end{diary}
	
\begin{solution}
	   
	    \end{enumerate}
\end{solution}
	
\end{question}

\end{tagblock}

%-------------------------------------------------------------------------------------------------------------
\begin{tagblock}{Integration, Substitution, FundamentalTheoremII, DefiniteIntegral, Trigonometry}
\begin{question}

So far we've looked only at indefinite integrals, but of course we can use $u$-substitution with definite integrals.  There are two slightly different methods:

 \textbf{Method 1:} Evaluate the indefinite integral first, then use FTC, Part 2 \\
\textbf{Method 2:}  Work in $u$ and change the endpoints

We'll compute $\displaystyle \int_0^{\sqrt{\pi}} x \cos(x^2) \, dx$ in two different ways.

\textbf{Method 1:}
\begin{enumerate}
\item Evaluate the indefinite integral  $\displaystyle \int x \cos(x^2) \, dx$ to find the most general anti-derivative.
\vspace{1in}
\item Use your answer to (a) along with FTC, Part 2 to evaluate  $\displaystyle \int_0^{\sqrt{\pi}} x \cos(x^2) \, dx$
\vspace{1in}
\end{enumerate}

\textbf{Method 2:}  

Using the same $u$-substitution as above, we'll change our endpoints from $x$-values to $u$-values:  \\

\[ \text{If $x=0$ then $u =  \rule{1.5cm}{.1mm} $ and if $x=\sqrt{\pi}$ then $u =  \rule{1.5cm}{.1mm} $. }\]

These then will be our endpoints for the definite integral after we rewrite it in terms of $u$.  Make the substitution and fill in the boxes below.  

\[ \int_0^{\sqrt{\pi}} x \cos(x^2) \, dx= \int_{\framebox[.25in]{\textcolor{white}{A}}}^{\framebox[.25in]{\textcolor{white}{A}}}  \framebox[.8in]{\textcolor{white}{A}} \, du \]

Then continue antidifferentiating.  Note with this method, you do not need to change back to the $x$-variable.



	
\index{Integration}
	
\begin{tags}
	  Integration, Substitution, FundamentalTheoremII, DefiniteIntegral, Trigonometry
\end{tags}
	
\begin{diary}
	   
\end{diary}
	
\begin{solution}
	   
	    \end{enumerate}
\end{solution}
	
\end{question}

\end{tagblock}

%-------------------------------------------------------------------------------------------------------------
\begin{tagblock}{Integration, Substitution, FundamentalTheoremII, DefiniteIntegral}
\begin{question}

Evaluate the definite integral using any method you want $\displaystyle \int_{-2}^3 \frac{x}{(x^2+1)^2} \, dx$.



	
\index{Integration}
	
\begin{tags}
	  Integration, Substitution, FundamentalTheoremII, DefiniteIntegral
\end{tags}
	
\begin{diary}
	   
\end{diary}
	
\begin{solution}
	   
	    \end{enumerate}
\end{solution}
	
\end{question}

\end{tagblock}

%-------------------------------------------------------------------------------------------------------------
\begin{tagblock}{Integration, Substitution, DefiniteIntegral}
\begin{question}

One question you could ask is ``Can every integral be solved with a $u$-substitution?"  What happens if you try to solve $\displaystyle \int_0^2 \sqrt{4-x^2} \, dx$ with the $u$-substitution $u=4-x^2$?

\vspace{.5in}
In general $u$-substitution is a very useful technique for solving integrals.  If you are sad that we can't solve \emph{every} integral this way, then take Calculus 2 and you'll learn lots of other techniques!!!!



	
\index{Integration}
	
\begin{tags}
	  Integration, Substitution, DefiniteIntegral
\end{tags}
	
\begin{diary}
	   
\end{diary}
	
\begin{solution}
	   
	    \end{enumerate}
\end{solution}
	
\end{question}

\end{tagblock}

%-------------------------------------------------------------------------------------------------------------
\begin{tagblock}{Integration, Substitution, Challenge}
\begin{question}

\begin{itemize}
\item$\displaystyle \int e^{ax} \, dx,$  where $a$ is any constant
\item $\displaystyle \int x^5 \sqrt{1+x^2} \, dx$ 
\item $\displaystyle \int x^2 \sin(x^3) \cos(x^3) \, dx$ 
\end{itemize} 



	
\index{Integration}
	
\begin{tags}
	  Integration, Substitution, Challenge
\end{tags}
	
\begin{diary}
	   
\end{diary}
	
\begin{solution}
	   
	    \end{enumerate}
\end{solution}
	
\end{question}

\end{tagblock}

%-------------------------------------------------------------------------------------------------------------
\begin{tagblock}{Integration, W1, Substitution, Trig}
\begin{question}
	Determine the value of $\displaystyle\int_0^{\sqrt{\frac {\pi} 4}} 2x\cos(x^2) \ dx$.
	
\index{Integration}
	
\begin{tags}
	    Integration, Substitution, Trig
\end{tags}
	
\begin{diary}
	   
\end{diary}
	
\begin{solution}
	   
	    \end{enumerate}
\end{solution}
	
\end{question}

\end{tagblock}

%-------------------------------------------------------------------------------------------------------------
\begin{tagblock}{Integration, W2, FundamentalTheoremI, Logarithms}
\begin{question}
	 a) State the Fundamental Theorem of Calculus. What kind of functions does it apply to? This is a bit of a trick question, since there is a complete answer but I don't expect you to know it! Give an incomplete answer in its stead if you don't know the complete answer. 


\bigskip

b) Use the Fundamental Theorem to find an antiderivative for the function $f(t)=t^{n}$ where $n$ is a real number. Does your solution hold for all values of $n$?

\bigskip

\bigskip

c) Does the expression
\[
\int_1^x \frac 1 t \ dt 
\]

always make sense? Why or why not? 

\bigskip

d) According to the fundamental theorem, what is the derivative of $f(x)=\displaystyle\int^x_1 \frac 1 t \ dt$?
	
\index{Integration}
	
\begin{tags}
	   Integration, W2, FundamentalTheoremI, Logarithms
\end{tags}
	
\begin{diary}
	   
\end{diary}
	
\begin{solution}
	   
	    \end{enumerate}
\end{solution}
	
\end{question}

\end{tagblock}

%-------------------------------------------------------------------------------------------------------------
\begin{tagblock}{Integration, W2, FundamentalTheoremI, Logarithms, Area}
\begin{question}
	Let's set $\ln(x)=\displaystyle\int_1^x\frac 1 t \ dt$. 

\bigskip

a) Why is $\ln(1)=0$?

\bigskip


b) Draw the graph of $1/t$ from $t=1$ to $t=2$. Which is bigger, $\ln(2)$ or $1/2$?

\bigskip

c) Repeat part b) from $t=1$ to $t=3$. Which is bigger, $\ln(3)$ or $1/3+1/2$?

\bigskip

d) Finally, repeat part b) from $t=1$ to $t=4$. Which is bigger, $\ln(4)$ or $1/4+1/3+1/2$?

\bigskip

e) Convince your group members (and more importantly, convince me) that there is a number $e$ with $1<e<4$ such that $\ln(e)=1$. You can happily use whatever technology you like.

\bigskip

f) Now the hard part: what fantastic theorem guarantees the existence of the number $e$ in part e) (I totally did not plan that)? Why does the theorem even apply?
	
\index{Integration}
	
\begin{tags}
	    Integration, W2, FundamentalTheoremI, Logarithms, Area
\end{tags}
	
\begin{diary}
	   
\end{diary}
	
\begin{solution}
	   
	    \end{enumerate}
\end{solution}
	
\end{question}

\end{tagblock}

%-------------------------------------------------------------------------------------------------------------
\begin{tagblock}{Integration, W2, Substitution, Trig, Logarithms}
\begin{question}
	Assuming $\ln(x)=\displaystyle\int^x_1 \frac 1 t \ dt$, compute the following derivatives and integrals.

\bigskip

a) $\dfrac d {dx}(\ln(x^3\tan(x^4)))$ 

\bigskip

b)  $\displaystyle\int_0^{\sqrt{e-1}} \frac t {t^2+1} \ dt$

\bigskip

c) $\displaystyle\int \frac {\sin(t)\cos(t)}{1+\cos(t)} \ dt$
	
\index{Integration}
	
\begin{tags}
	    Integration, W2, Substitution, Trig, Logarithms
\end{tags}
	
\begin{diary}
	   
\end{diary}
	
\begin{solution}
	   
	    \end{enumerate}
\end{solution}
	
\end{question}

\end{tagblock}

%-------------------------------------------------------------------------------------------------------------

\begin{tagblock}{Integration, W3, Substitution, Trig, Exponentials}
\begin{question}
	Compute the following derivatives and integrals.

\bigskip

a) $\dfrac d {dx} (e^{\cos^2(x)})$

\bigskip

b) $\displaystyle\int_1^3 e^{5x} \ dx$

\bigskip

c) $\displaystyle\int e^{\tan(x)}+\tan^2(x)e^{\tan(x)} \ dx$

\bigskip

d) $\dfrac d {dx} (x^{\sin(x)})$, $x>0$.
	
\index{Integration}
	
\begin{tags}
	    Integration, W3, Substitution, Trig, Exponentials
\end{tags}
	
\begin{diary}
	   
\end{diary}
	
\begin{solution}
	   
	    \end{enumerate}
\end{solution}
	
\end{question}

\end{tagblock}

%-------------------------------------------------------------------------------------------------------------
\begin{tagblock}{W5, ImproperIntegralInfinite, Integration, Definition, Exponentials, Trigonometry, Logarithms}
\begin{question}
	
We're going to use the following tried-and-true definition: if $f$ is continuous on $[a,\infty)$ or has finitely many discontinuities with no vertical asymptotes,
\[
\int_a^{\infty}f(t) \ dt=\lim_{x\to\infty} \int_a^xf(t) \ dt
\]
By the fundamental theorem of calculus, $\int_a^x f(t) \ dt$ always exists, but the limit may not! 

\bigskip



Compute the following improper integrals, if an answer exists. If there is no answer, say why.

\bigskip

a) $\displaystyle\int_0^\infty e^{-t} \ dt$. 

\bigskip

b) $\displaystyle\int_e^{\infty}\frac 1 x \ dx$ 

\bigskip

c) $\displaystyle\int_1^{\infty} \frac 1 {x^2} \ dx$

\bigskip

d) $\displaystyle\int_{\pi}^{\infty}\sin(x) \ dx$

\bigskip

e) $\displaystyle\int_1^{\infty}\frac x {2+x} \ dx$

\index{Integration}
    
\begin{tags}
        W5, ImproperIntegralInfinite, Integration, Definition, Exponentials, Trigonometry, Logarithms
\end{tags}
    
\begin{diary}
        
\end{diary}
	
\begin{solution}

\end{solution}
	
\end{question}

\end{tagblock}

%------------------------------------------------------------------------------------------------------------
\begin{tagblock}{W5, ImproperIntegralInfinite, Integration, Definition, WarmUp}
\begin{question}
	
We define the \textbf{Laplace Transform} $\mathcal{L}\{ f\}$ of $f$ by 
\[
\eL\{ f\}(s)=\int_0^{\infty} f(t)e^{-st} \ dt.
\]
provided the integral exists. 

\bigskip

a) If $\mathcal{L}\{ f\}$ exists, is it a number or a function?

\bigskip

b) Calculate the Laplace Transform of the function $f(t)=0$. 
 
\bigskip

c) Now try calculating the Laplace Transform of the function $f(t)=1$. 

\bigskip

d) If you got an answer for part c), did it depend on the value of $s$? Why or why not?  

\index{Integration}
    
\begin{tags}
        W5, ImproperIntegralInfinite, Integration, Definition, WarmUp
\end{tags}
    
\begin{diary}
        
\end{diary}
	
\begin{solution}

\end{solution}
	
\end{question}

\end{tagblock}

%------------------------------------------------------------------------------------------------------------
\begin{tagblock}{W5, ImproperIntegralInfinite, Integration, Challenge, Volume, SurfaceArea}
\begin{question}
	
Ringo Starr is painting his house (again). Ringo, however, is famous, and so doesn't want to use an ordinary paint can. He wants a can that is obtained by revolving the area in the first quadrant under the graph of the function $f(x)=1/x$ about the $x$-axis from $x=1$ to $x=2$, where the dimensions are in feet. 

\bigskip

a) If the can is full, find the amout of paint in the can. You should have seen this formula in Calc I, but it's pretty easy to deduce if you haven't.


\bigskip

Now Ringo gets really ambitious: he wants an infinite paint can, described by revolving the area in the first quadrant under the graph of the function $f(x)=1/x$ about the $x$-axis from $x=1$ to infinity. 

\bigskip

b) Figure out the volume of such a can.

\bigskip

Without justification, here is the formula for the surface area of such a can:
\[
SA=2\pi\int_a^bf(x)\sqrt{1+\{f'(x)\}^2} \ dx.
\]

\bigskip

c) Is there enough paint to cover the sides of Ringo's can? Why or why not? What does this mean?

\index{Integration}
    
\begin{tags}
        W5, ImproperIntegralInfinite, Integration, Challenge, Volume, SurfaceArea
\end{tags}
    
\begin{diary}
        
\end{diary}
	
\begin{solution}

\end{solution}
	
\end{question}

\end{tagblock}

%------------------------------------------------------------------------------------------------------------
\begin{tagblock}{W7, ImproperIntegralInfinite, Integration, IntegrationByParts, L'Hopital, Example, Definition}
\begin{question}
	
When we try to take the Laplace Transform of the equation
\begin{equation}\label{brine}
\frac{dx}{dt}=6h(t)-\frac{3x(t)}{500}
\end{equation}
where 
\[
h(t)=\begin{cases} 1.2\textrm{kg} & 0\leq t<10 \\ 2.4\textrm{kg} & t\geq 10 \end{cases}
\]
we run into a problem trying to evaluate the Laplace Transform of $\dfrac{dx}{dt}$. Today, we'll develop a technique that will solve the problem: integration by parts. We'll first look at this technique for simpler functions. 

\bigskip

1) We know how to take the Laplace Transform of constants, but what about $f(t)=t$? We end up with
\[
\int_0^{\infty} te^{-st} \ dt.
\]
This is precisely the kind of integral that integration by parts was born to defeat. The formula is
\begin{equation}\label{parts}
\int f(t)g'(t) \ dt=f(t)g(t)-\int g(t)f'(t) \ dt.
\end{equation}
The shorthand is
\[
\int u \ dv=uv-\int v \ du.
\]

This may look intimidating, but it's nothing other than the product rule in reverse. 

\bigskip

a) Verify this claim by adding $\displaystyle\int g(t)f'(t) \ dt$ to both sides and differentiating.

\newpage

We want to write this in the form
\[
\int u \ dv
\]
and then use integration by parts. But what is $u$ and what is $dv$? The idea is $u$ should be something you can differentiate that will (hopefully) simplify the expression and $dv$ should be something you can integrate. So let's choose
\begin{center}
$u=t$, \ $dv=e^{-t} \ dt$. 
\end{center}
ALWAYS choose $u$ first; $dv$ is then whatever is left over in the integral. 

Now, $du$ is the derivative of $t$ with respect to $t$, which is 1, and $v$ is the integral of $e^{-t}$ with respect to $t$, which is $-e^{-t}$. Using integration by parts,
\[
\int te^{-t} \ dt=uv-\int v \ du=-te^{-t}-\int -e^{-t} \ dt=-te^{-t}+\int e^{-t} \ dt.
\]
The remaining integral is simpler than what we started with; in fact, we've already done it! So

\bigskip

b) Compute the Laplace Transform of $f(t)=t$, up to the point where you have to evaluate the limit. 

\bigskip

c) Same question for the Laplace Transform of $g(t)=t^2$.

\index{Integration}
    
\begin{tags}
       W7, ImproperIntegralInfinite, Integration, IntegrationByParts, L'Hopital, Example, Definition
\end{tags}
    
\begin{diary}
        
\end{diary}
	
\begin{solution}

\end{solution}
	
\end{question}

\end{tagblock}

%------------------------------------------------------------------------------------------------------------
\begin{tagblock}{W7, ImproperIntegralInfinite, Integration, IntegrationByParts, Example, L'Hopital}
\begin{question}
	
To figure out the Laplace Transform of $f(t)=t^2$, you'd have to do integration by parts twice, which is a bit of a pain. Fortunately, there is an easier way to evaluate integrals of the form 
\[
\int t^n e^{\alpha t} \ dt
\]
where $n$ is a counting number. This is the so-called \textit{tabular method}. Make a table with two columns, with $u$ on the top of one and $dv$ on the top of the other. Write $t^n$ under $u$ and $e^{\alpha t}$ under $dv$. Differentiate $t^n$ until you get 0, then integrate $e^{\alpha t}$ that many times. Here's $t^2e^{-st}$ for $s\ne 0$:

\begin{center}
\begin{tabular}{| l | r |} \hline $u$ & $dv$ \\ \hline $t^2$ & $e^{-st}$ \\ \hline $2t$ & $-e^{-st}/s$ \\ \hline 2 & $e^{-st}/s^2$ \\ \hline 0 & $-e^{-st}/s^3$ \\ \hline
\end{tabular}
\end{center}

Multiply diagonally and add the products, alternating plus and minus signs starting with ``+". We get
\[
\int t^2e^{-st} \ dt=-\frac{t^2e^{-st}}s-\frac{2te^{-st}}{s^2}-\frac{2e^{-st}}{s^3}.
\]
This should look vaguely familiar! It also works for integrals of the form $t^n\cos(\alpha t)$ or $t^n\sin(\alpha t)$. 

\bigskip

a) Use the tabular method to compute $\displaystyle\int t^3\sin(2t) \ dt$. 

\index{Integration}
    
\begin{tags}
        W7, ImproperIntegralInfinite, Integration, IntegrationByParts, Example, L'Hopital
\end{tags}
    
\begin{diary}
        
\end{diary}
	
\begin{solution}

\end{solution}
	
\end{question}

\end{tagblock}

%------------------------------------------------------------------------------------------------------------
\begin{tagblock}{W7, ImproperIntegralInfinite, Integration, IntegrationByParts, SqueezeTheorem}
\begin{question}
	
Compute the Laplace Transform of $f(t)=\cos(t)$. (The Tabular Method is ineffective here!)

\index{Integration}
    
\begin{tags}
        W7, ImproperIntegralInfinite, Integration, IntegrationByParts, SqueezeTheorem
\end{tags}
    
\begin{diary}
        
\end{diary}
	
\begin{solution}

\end{solution}
	
\end{question}

\end{tagblock}

%------------------------------------------------------------------------------------------------------------
\begin{tagblock}{Integration, W8, PartialFractions, Example, WarmUp}
\begin{question}
We will now examine a technique called \textit{Partial Fractions}.

\bigskip

We'll first look at the example $\dfrac 1 {s^2+s}=\dfrac 1 {s (s+1)}$. 

\bigskip

a) Which, if any, values of $s$ cause a problem in the example when we multiply by $s(s+1)$? Does your answer make the method of finding $A$ and $B$ seem slightly less legitimate? Why or why not?

\bigskip

We want to find numbers $A$ and $B$ such that 
\begin{equation}\label{partial}
\frac 1 {s(s+1)}=\frac A {s+1}+\frac B s.
\end{equation}
There are two ways to approach this problem; we'll take the easier way. Multiply both sides of the equation by $s(s+1)$ to get
\begin{equation}\label{partial2}
1=As+B(s+1).
\end{equation}
Plugging in $s=0$ gives $B=1$, and plugging in $s=-1$ give $A=-1$, so that 
\[
\frac 1 {s(s+1)}=\frac {-1} {s+1}+\frac 1 s.
\]
\bigskip
This is called the \textit{partial fraction decomposition} of $\dfrac 1 {s(s+1)}$.

\bigskip


b) Find the partial fraction decomposition of $\dfrac 1 {(s+1)(s-3)(10s+7)}$.
	
\index{Integration}
	
\begin{tags}
	   Integration, W8, PartialFractions, Example, WarmUp
\end{tags}
	
\begin{diary}
	   
\end{diary}
	
\begin{solution}
	   
	    \end{enumerate}
\end{solution}
	
\end{question}

\end{tagblock}

%-------------------------------------------------------------------------------------------------------------
\begin{tagblock}{Integration, W8, PartialFractions, Example}
\begin{question}
There's an alternative way to find the partial fraction decomposition that is less suspect-looking than clearing denominators and plugging in well-chosen values; it also generalizes well to more complicated cases.  Let's go back to $\dfrac 1 {s(s+1)}$ and the equation

\begin{equation}\label{partial2}
1=As+B(s+1).
\end{equation}

Gather all the terms with the same degree together on the right-hand side of the equality to get

\begin{equation}\label{partial3}
1=(A+B)s+B
\end{equation}
and think of equation \eqref{partial3} as an equality between polynomials. Then rewriting,
\[
0s+1=(A+B)s+B,
\]
so $A+B=0$ and $B=1$, which immediately gives us $A=-1$. 

\bigskip

a) Use the alternative method to find the partial fraction decomposition of  $\dfrac 1 {(s+1)(s-3)(10s+7)}$. 

\bigskip


b) Now try this method for $\dfrac 5 {s(s-1)^2}$. What happens?

\bigskip

c) Find the partial fraction decomposition of $\dfrac {s^4+1}{(s+1)(s^2+4)}$.
	
\index{Integration}
	
\begin{tags}
	   Integration, W8, PartialFractions, Example
\end{tags}
	
\begin{diary}
	   
\end{diary}
	
\begin{solution}
	   
	    \end{enumerate}
\end{solution}
	
\end{question}

\end{tagblock}

%-------------------------------------------------------------------------------------------------------------
\begin{tagblock}{Integration, W8, PartialFractions, Example, Applications}
\begin{question}
Check the following facts, then solve the two-valve tank problem by putting them all together. You can do this by assuming that, for each function $g$, there is a unique function $f$ on $[0,\infty)$ whose Laplace Transform is $g$. 

\bigskip

a) (Fact 1) If $s>-a$, show that $\displaystyle\eL\{e^{-at}\}(s)=\frac 1 {a+s}$. 

\bigskip

b) If we let
\[
h(t)=\begin{cases} 1 & t\geq 0 \\ 0 & t<0\end{cases},
\]
show $\displaystyle\eL\{h(t-a)\}(s)=\frac {e^{-as}} s$ for $s>a$. 

\bigskip

c) (Fact 2) If $h$ is the function from part b), check that 
\[
\eL\{f(t-a)h(t-a)\}(s)=e^{-as}\eL\{f\}(s)
\]
for $s>a$. 

\bigskip
	
\index{Integration}
	
\begin{tags}
	   Integration, W8, PartialFractions, Example
\end{tags}
	
\begin{diary}
	   
\end{diary}
	
\begin{solution}
	   
	    \end{enumerate}
\end{solution}
	
\end{question}

\end{tagblock}

%-------------------------------------------------------------------------------------------------------------

 
\section{Limits}\index{Limits}
\fancyhead[R]{\large Limits}


\begin{tagblock}{Limits, VerticalAsymptotes, HorizontalAsymptotes, InfiniteDiscontinuity, Definition, Table}
\begin{question}
	In the previous worksheets we introduced limits: given a function $f$, a fixed input $x=a$, and a real number $L$, we say that $f$ has  \emph{limit $L$ as $x$ approaches $a$}, and write 
\[\lim_{x \to a}f(x) = L\]
provided that we can make $f(x)$ as close to $L$ as we like by taking $x$ sufficiently close (but not equal) to $a$. If we cannot make $f(x)$ as close to a single value as we would like as $x$ approaches $a$, then we say that \emph{$f$ does not have a limit as $x$ approaches $a$.}

\begin{enumerate}
\item Consider the function $\displaystyle f(x) = \frac{1}{x^2}$
\begin{enumerate}
\item Compute the following table of values:


\begin{tabular}{c | c c | c  }
$x$ & $f(x)$  \hspace{2in} & $x$ & $f(x)$ \\ \hline
.5 & \hspace{.5in} & -.5 &  \\ &&& \\
.25 & & -.25  & \\ &&&\\
.1 & & -.1& \\&&& \\
.01 && -.01 &\\ &&&\\
.001 && -.001 &\\ &&&\\
\end{tabular}
\item Does the limit as $x \to 0$ appear to exist?  Why or why not?
\end{enumerate}

\vspace{1in} 

\item Consider the function $\displaystyle g(x) = \frac{1}{x}$
\begin{enumerate}
\item Compute the following table of values:


\begin{tabular}{c | c c | c  }
$x$ & $g(x)$  \hspace{2in} & $x$ & $g(x)$ \\ \hline
.5 & \hspace{.5in} & -.5 &  \\ &&& \\
.25 & & -.25  & \\ &&&\\
.1 & & -.1& \\&&& \\
.01 && -.01 &\\ &&&\\
.001 && -.001 &\\ &&&\\
\end{tabular}
\item Does the limit as $x \to 0$ appear to exist?  Why or why not?
\end{enumerate}


\newpage
 Below are the graphs of $\displaystyle f(x) = \frac{1}{x^2}$ and $\displaystyle g(x) = \frac{1}{x}$

\begin{figure}[h]
\includegraphics[width=6cm]{limits1_x2.png} \hfill \includegraphics[width=6cm]{limits1_x.png}
\end{figure}
In Problem 1 you saw that as $x$ got closer to $0$, $f(x)$ got larger and larger, we write this as 
\[ \lim_{x \to 0} f(x) = \lim_{x \to 0} \frac{1}{x^2} = \infty \]

In Problem 2 you saw that if $x$ was a positive number getting closer to $0$, $g(x)$ got larger and larger, and $x$ was a negative number getting closer to $0$, $g(x)$ became a large negative number.  Using our left and right hand limits we have
\[\lim_{x \to 0^+} g(x) = \lim_{x \to 0^+} \frac{1}{x} = \infty \, \text{ and } \, \lim_{x \to 0^-} g(x) = \lim_{x \to 0^-} \frac{1}{x} = -\infty\]

In both of these examples we call the vertical line $x=0$ a \emph{vertical asymptote}.

In general $x=a$ is a \emph{vertical asymptote} for a function $f(x)$ if one of the following is true:

\[\lim_{x \to a} f(x) = \infty \hspace{.5in} \lim_{x \to a^-} f(x)  = \infty \hspace{.5in}  \lim_{x \to a^+} f(x)  = \infty\]
\[\lim_{x \to a} f(x) = -\infty \hspace{.5in} \lim_{x \to a^-} f(x)  = -\infty \hspace{.5in}  \lim_{x \to a^+} f(x)  = -\infty\]

\bigskip
\item The graph of $h(x)$ is given below, compute the following.  If a limit does not exist, explain why.  
\begin{minipage}{.6\textwidth}
\includegraphics[width=8cm]{limitsVA.jpg}\end{minipage}% This must go next to `\end{minipage}`
\begin{minipage}{.4\textwidth}

\bigskip
 $\displaystyle \lim_{x \to -3^-} h(x) =  \rule{1.5cm}{.1mm}$  
 \bigskip
 
 $\displaystyle \lim_{x \to -3^+} h(x) =  \rule{1.5cm}{.1mm}$ 
 \bigskip
 
  $\displaystyle \lim_{x \to -3} h(x) =  \rule{1.5cm}{.1mm}$ 
\bigskip
 
 $\displaystyle \lim_{x \to 2} h(x) =  \rule{1.5cm}{.1mm}$ 
 \bigskip
 
 $\displaystyle \lim_{x \to 5} h(x) =  \rule{1.5cm}{.1mm}$ 
 \bigskip
  \end{minipage}
\bigskip
  
 Vertical Asymptotes of $h(x)$:   \rule{6cm}{.1mm}

\newpage

\item Let $\displaystyle f(x) = \frac{1}{x-3}$.  Compute the following limits, and if a limit does not exist, explain why. 
\[ \lim_{x \to 3^-}  \frac{1}{x-3} = \hspace{1in}  \lim_{x \to 3^+}  \frac{1}{x-3} = \hspace{1in}  \lim_{x \to 3}  \frac{1}{x-3} = \]


\vspace{1in}


We could also ask the question what happens to a function as $x$ gets very large
\item Consider the function $\displaystyle g(x) = \frac{1}{x}$.   Evaluate $g(x)$ at $x=10, 100,$ and $1000$.  What do you notice?

\vspace{1in}
\item Consider the function $h(x)$ given by the graph below.  As $x$ gets larger what do you notice about $h(x)$?
\begin{figure}[h]
\centering
\includegraphics[width=6cm]{limitsHA.png}
\end{figure}

For a function $f(x)$ we will write $\lim_{x \to \infty} f(x) = L$  provided that we can make $f(x)$ as close to $L$ as we like by taking $x$ sufficiently large.  If $L$ is a finite number then we say that $f(x)$ has a \emph{horizontal asymptote} at $y=L$.  Similarly we can consider what happens as $x \to -\infty$.  

\item 
\begin{enumerate}
\item Returning to $\displaystyle g(x) = \frac{1}{x}$.   Compute 
\[ \lim_{x \to \infty} \frac{1}{x} = \hspace{1in} \text{ and }  \hspace{1in} \lim_{x \to -\infty} \frac{1}{x} = \]
What horizontal asymptote(s) does $g(x)$ have?
\[y = \rule{3cm}{.1mm} \]

\bigskip

\item Next consider $\displaystyle g(x) = \frac{1}{x^2}$.   Compute 
\[ \lim_{x \to \infty} \frac{1}{x^2} = \hspace{1in} \text{ and }  \hspace{1in} \lim_{x \to -\infty} \frac{1}{x^2} = \]
What horizontal asymptote(s) does $g(x)$ have?
\[y = \rule{3cm}{.1mm} \]


\item Generalize this to  $\displaystyle g(x) = \frac{1}{x^r}$, where $r>0$
\[ \lim_{x \to \infty} \frac{1}{x^r} = \hspace{1in} \text{ and }  \hspace{1in} \lim_{x \to -\infty} \frac{1}{x^r} = \]
\end{enumerate}

\bigskip
  
\item Let $p(x)=x^2$.  Sketch a graph of $p(x)$ and compute
\[ \lim_{x \to \infty} p(x) = \hspace{1in} \text{ and }  \hspace{1in} \lim_{x \to -\infty} p(x) = \]
Does $p(x)$ have any horizontal asymptotes?



\end{enumerate}

\vspace{1.5in}






	
\index{Limits}
	
\begin{tags}
	    Limits, VerticalAsymptotes, HorizontalAsymptotes, InfiniteDiscontinuity, Definition, Table
\end{tags}
	
\begin{diary}
	    
\end{diary}
	
\begin{solution}
	  
\end{solution}
	
\end{question}

\end{tagblock}

%-------------------------------------------------------------------------------------------------------------

\begin{tagblock}{Limits, VerticalAsymptotes, HorizontalAsymptotes, InfiniteDiscontinuity, Example, AlgebraicLimits, RationalFunctions}
\begin{question}
	
\bigskip

In the previous worksheets we introduced vertical and horizontal asymptotes.  In this Worksheet we want to determine if a function has vertical and/or horizontal asymptotes given the equation.  Recall that to find a horizontal asymptote, we compute the limit as $x \to \infty$ and as $x \to -\infty$.  More precisely, 
\[ \text{if } \lim_{x \to \infty} f(x) = L, \text{ where $L$ is a finite number, then $f(x)$ has a horizontal asymptote at $y=L$, and} \]
\[ \text{if } \lim_{x \to -\infty} f(x) = M, \text{ where $M$ is a finite number, then $f(x)$ has a horizontal asymptote at $y=M$.} \]

Today we'll focus on functions that are fractions $\displaystyle f(x) = \frac{N(x)}{D(x)}$, so we will have a vertical asymptotes at any $x=a$ which makes  the denominator $0$, i.e. $D(a)=0$ \textbf{and}  the numerator is not $0$, i.e. $N(a)\neq 0$.  

Given an equation of a function, to determine the horizontal asymptotes.  We will use the fact that for any $r>0$, we have
\[ \lim_{x \to \infty} \frac{1}{x^r} = 0 \text{ and }  \lim_{x \to -\infty} \frac{1}{x^r} = 0 \]

The main algebraic tool we will use to compute limits as $x \to \infty$ is to multiply by $1$ where $1=\frac{1/x^d}{1/x^d}$ and $d$ is the ``highest power of $x$''

 Consider $\displaystyle f(x) = \frac{2x^3-3}{x^3+1}$ .  The highest power of $x$ that we see is $3$, so we'll start by dividing both the numerator and denominator by $\frac{1}{x^3}$
 
 \[ \lim_{x \to \infty} f(x) =  \lim_{x \to \infty} \frac{2x^3-3}{x^3+1} =  \lim_{x \to \infty} \frac{2x^3-3}{x^3+1} \cdot \frac{\frac{1}{x^3}}{\frac{1}{x^3}} = \lim_{x \to \infty} \frac{2-\frac{3}{x^3}}{1+\frac{1}{x^3}}\]

Now, $\lim_{x \to \infty} \frac{1}{x^3} =0$, so  $\lim_{x \to \infty} \frac{3}{x^3}$ is also $0$.  Thus 
\[\lim_{x \to \infty} f(x) =  \lim_{x \to \infty} \frac{2-\frac{3}{x^3}}{1+\frac{1}{x^3}} =  \lim_{x \to \infty} \frac{2-0}{1+0} = 2 \]

So we have a horizontal asymptote at $y=2$ as $x \to \infty$

Similarly, $ \lim_{x \to -\infty} f(x) = 2$ and we  have a horizontal asymptote at $y=2$ as $x \to -\infty$.

\begin{enumerate}
\item Let $\displaystyle g(x) = \frac{2x^2-3}{x^3+1}$.
\begin{enumerate}
\item What vertical asymptotes does $g(x)$ have?
\item What is the highest power of $x$ in $g(x)$?
\item Compute $\lim_{x \to \infty} g(x)$ and $\lim_{x \to -\infty} g(x)$.
\item Does $g(x)$ have a horizontal asymptote?  If so what are they?
\end{enumerate}

\newpage

\item Let $\displaystyle h(x) = \frac{2x^3-3}{x^2+1}$.
\begin{enumerate}
\item What vertical asymptotes does $h(x)$ have?
\item What is the highest power of $x$ in $h(x)$?
\item Compute $\lim_{x \to \infty} h(x)$ and $\lim_{x \to -\infty} h(x)$.
\item Does $h(x)$ have a horizontal asymptote?  If so what are they?
\end{enumerate}

\vspace{4in}

\item The three previous functions we looked at were all rational functions:  $\displaystyle f(x) = \frac{2x^3-3}{x^3+1}$, $\displaystyle g(x) = \frac{2x^2-3}{x^3+1}$ and $\displaystyle h(x) = \frac{2x^3-3}{x^2+1}$.  What do you notice about their horizontal asymptotes?  Can you come up with a general statement that determines how to find a horizontal asymptote of a rational function?




\newpage
\item Let $\displaystyle f(x) = \frac{\sqrt{5x^2+3}}{x-2}$.  Recall if $x$ is positive then $\sqrt{x^2} = x$ and if $x$ is negative then $-\sqrt{x^2} = x$.  
\begin{enumerate}
\item What vertical asymptotes does $f(x)$ have?
\item What is the highest power of $x$ in $f(x)$? (Careful here)
\item Compute $\lim_{x \to \infty} f(x)$.
\item Compute $\lim_{x \to -\infty} f(x)$ (you should get a different answer than you got in (c)).
\item What horizontal asymptotes does $f(x)$ have?
\end{enumerate}



\end{enumerate}

	
\index{Limits}
	
\begin{tags}
Limits, VerticalAsymptotes, HorizontalAsymptotes, InfiniteDiscontinuity, Example, AlgebraicLimits, RationalFunctions

\end{tags}
	
\begin{diary}
	    
\end{diary}
	
\begin{solution}
	  
\end{solution}
	
\end{question}

\end{tagblock}

%-------------------------------------------------------------------------------------------------------------

\begin{tagblock}{Limits, Definition}
\begin{question}
	
Given a function $f(x)$ we can evaluate $f$ at an $x$-value $a$.  In this worksheet we will introduce the limit of $f(x)$ as $x$ approaches $a$, which may be different than evaluating at $a$.  


\begin{enumerate}
\item Let $f(x)$ be the function given by the graph below. Use the graph to answer each of the following questions.
\begin{figure}[h]
\centering
\includegraphics[width=8cm]{limits1.png}
\end{figure}

\begin{enumerate}
\item     Determine the values $f(-2), f(-1), f(0), f(1)$ and $f(2)$ if defined.   If the function value is not defined, explain what feature of the graph tells you this.

\vspace{1in}

\item For each of the values $a=-1$, $a=0$, and $a=2$, complete the following sentence: \\
 ``As $x$ gets closer and closer (but not equal) to $a=-1$, $f(x)$
gets as close as we want to  \rule{1.1cm}{0.1mm} .'' \\ \bigskip
``As $x$ gets closer and closer (but not equal) to $a=0$, $f(x)$
gets as close as we want to  \rule{1.1cm}{0.1mm} .'' \\ \bigskip
``As $x$ gets closer and closer (but not equal) to $a= 2$, $f(x)$
gets as close as we want to  \rule{1.1cm}{0.1mm} .'' \\ \bigskip




\item What happens as $x$ gets closer and closer (but not equal) to $a=1$? Does the function $f(x)$ get as close as we would like to a single value?

\end{enumerate}
\end{enumerate}


\newpage
In the previous problem we saw that for the given function $f$, as $x$ gets closer and closer (but not equal) to $0$, $f(x)$ gets as close as we want to the value $4$. At first, this may feel counterintuitive, because the value of $f(0)$ is $1$, not $4$. By their very definition, limits regard the behavior of a function arbitrarily close to a fixed input, but the value of the function at the fixed input does not matter. More formally, we say the following:



Given a function $f$, a fixed input $x=a$, and a real number $L$, we say that $f$ has  \emph{limit $L$ as $x$ approaches $a$}, and write 
\[\lim_{x \to a}f(x) = L\]
provided that we can make $f(x)$ as close to $L$ as we like by taking $x$ sufficiently close (but not equal) to $a$. If we cannot make $f(x)$ as close to a single value as we would like as $x$ approaches $a$, then we say that \emph{$f$ does not have a limit as $x$ approaches $a$.}

For the function $f$ from the previous problem we have
\[ \lim_{x \to -1} f(x) = 3, \hspace{.2in}  \lim_{x \to 0} f(x) = \rule{1.5cm}{.1mm}  \hspace{.2in} \lim_{x \to 2} f(x) =  \rule{1.5cm}{.1mm}\]



In a situation such as the jump in the graph of $f$ at $x=1$, the issue is that if we approach $x=1$ from the left, the function values tend to get as close to $3$ as we'd like, but if we approach $x=1$ from the right, the function values get as close to $2$ as we'd like.

We call these left and right hand limits, and denote them as follows:
\smallskip

$\displaystyle \lim_{x \to a^-} f(x)$ ``the limit as $x$ approaches $a$ from the left''  (think of the $-$ indicating that we are smaller than $a$)\\
\bigskip

$\displaystyle \lim_{x \to a^+} f(x)$ ``the limit as $x$ approaches $a$ from the right'' (think of the $+$ indicating that we are bigger than $a$) \\

\bigskip


For our function $f$, we have
\[ \lim_{x \to 1^-} f(x) = 3 \hspace{.2in}  \lim_{x \to 1^+} f(x) = 2 \]

Since $\displaystyle  \lim_{x \to 1^-} f(x) \neq  \lim_{x \to 1^+} f(x)$,  the limit of $f$ does not exist at $x=1$.
\bigskip
	
\index{Limits}
	
\begin{tags}
	    Limits, Definition
\end{tags}
	
\begin{diary}
	    
\end{diary}
	
\begin{solution}
	  
\end{solution}
	
\end{question}

\end{tagblock}

%-------------------------------------------------------------------------------------------------------------

\begin{tagblock}{Limits, Definition}
\begin{question}
	
  \setcounter{enumi}{1}
Let $h(x)$ be the function given by the graph below. Use the graph to answer each of the following questions.  \textbf{If the function value or limit does not exist, explain why.}

\begin{minipage}{.4\textwidth}
\includegraphics[width=6cm]{limits4.png}\end{minipage}% This must go next to `\end{minipage}`
\begin{minipage}{.6\textwidth}
\begin{tabular}{lll}
 $h(-1) =  \rule{1.5cm}{.1mm}$ &\hspace{.2in} & $\displaystyle \lim_{x \to -1} h(x) =  \rule{1.5cm}{.1mm}$ \\ \\
 $h(3) =  \rule{1.5cm}{.1mm}$ &\hspace{.2in}&  $\displaystyle \lim_{x \to 3^-} h(x) =  \rule{1.5cm}{.1mm}$ \\ \\
  $\displaystyle \lim_{x \to 3^+} h(x) =  \rule{1.5cm}{.1mm}$ & \hspace{.2in} &$\displaystyle \lim_{x \to 3} h(x) =  \rule{1.5cm}{.1mm}$ \\
  \end{tabular}
\end{minipage}	
\index{Limits}
	
\begin{tags}
	    Limits, Definition
\end{tags}
	
\begin{diary}
	    
\end{diary}
	
\begin{solution}
	  
\end{solution}
	
\end{question}

\end{tagblock}

%-------------------------------------------------------------------------------------------------------------
\begin{tagblock}{Limits, Definition}
\begin{question}
	Sketch the graph of a function $f$ that satisfies the following conditions: \\
\[  f(0)=-1,   \, f(3) =1 , \, \lim_{x \to 0} f(x) =1, \, \lim_{x \to 3^-} f(x) = -2, \text{ and } \lim_{x \to 3^+} f(x) = 2 \]

\emph{(Note: There are many possibilites for $f$)}

\begin{figure}[h]
\centering
\includegraphics[width=8cm]{limitsintroblank.png}
\end{figure}
	
\index{Limits}
	
\begin{tags}
	    Limits, Definition
\end{tags}
	
\begin{diary}
	    
\end{diary}
	
\begin{solution}
	  
\end{solution}
	
\end{question}

\end{tagblock}

%-------------------------------------------------------------------------------------------------------------

\begin{tagblock}{Limits, Definition, Table}
\begin{question}
	We will next investigate limits when we are given an equation of a function.


\item Consider the function $\displaystyle f(x) = \frac{x^2-4}{x-2}$.  
\begin{enumerate}
\item What is the domain of $f$?  Can we evaluate $f(2)$?  
\bigskip

\item Compute the following table of values:


\begin{tabular}{c | c c | c  }
$x$ & $f(x)$  \hspace{2in} & $x$ & $f(x)$ \\ \hline
1.5 & \hspace{.5in} & 2.5 &  \\ &&& \\
1.75 & & 2.25  & \\ &&&\\
1.9 & & 2.1& \\&&& \\
1.99 && 2.01 &\\ &&&\\
\end{tabular}
  \item Based on your values for $f(x)$, what do you guess the $\displaystyle \lim_{x \to 2} f(x)$ is?
  
\vspace{.2in}

\item We can also graph $f(x)$ to determine the limit.  Returning to the equation of $f(x)$:  $\displaystyle f(x) = \frac{x^2-4}{x-2}$ we see that we can factor the numerator, \[\displaystyle f(x) = \frac{x^2-4}{x-2} = \frac{(x-2)(x+2)}{x-2},\]
as long as $x \neq 2$  we then have $ \displaystyle \frac{(x-2)}{x-2} =1$, so that for $x \neq 2$, $\displaystyle f(x) = \frac{(x-2)(x+2)}{x-2} = x+2$.  

This means graph of $f(x)$ is the line $x+2$ but with a hole at $x=2$.  Draw the graph of $f(x)$ below.   

\begin{figure}[h]
\centering
\includegraphics[width=8cm]{limits2.png}
\end{figure}
\item Based on your graph, what is $\displaystyle \lim_{x \to 2} f(x)$?

\end{enumerate}
	
\index{Limits}
	
\begin{tags}
	    Limits, Definition, Table
\end{tags}
	
\begin{diary}
	    
\end{diary}
	
\begin{solution}
	  
\end{solution}
	
\end{question}

\end{tagblock}

%-------------------------------------------------------------------------------------------------------------


\begin{tagblock}{Limits, Definition, Table}
\begin{question}
	\item Consider the function $\displaystyle g(x) = \cos (\frac{1}{x})$. 
 \begin{enumerate}
\item What is the domain of $g$?  Can we evaluate $g(0)$?  
\bigskip

\item Compute the following table of values (Remember always work in radians in calculus)


\begin{tabular}{c | c c | c  }
$x$ & $g(x)$  \hspace{2in} & $x$ & $g(x)$ \\ \hline
.5 & \hspace{.5in} & -.5 &  \\ &&& \\
.25 & & -.25  & \\ &&&\\
.1 & & -.1& \\&&& \\
.01 && -.01 &\\ &&&\\
\end{tabular}
  \item Based on your values for $g(x)$, do you think the $\displaystyle \lim_{x \to 0} g(x)$ exists?  Explain why or why not.  
  
  \vspace{1in}
  \item We can also graph $g(x)$.  Based on the graph, do you think the $\displaystyle \lim_{x \to 0} g(x)$ exists?
  \begin{figure}[h]
\centering
\includegraphics[width=8cm]{limits3.png}
\end{figure}


\end{enumerate}




Using a table of values is not perfect, as we might have missed something.  We'll learn a better method for computing limits in the next Worksheet. 
	
\index{Limits}
	
\begin{tags}
	    Limits, Definition, Table
\end{tags}
	
\begin{diary}
	    
\end{diary}
	
\begin{solution}
	  
\end{solution}
	
\end{question}

\end{tagblock}

%-------------------------------------------------------------------------------------------------------------


\begin{tagblock}{Limits, Graph, WarmUp}
\begin{question}
	In the previous Worksheet we introduced limits: given a function $f$, a fixed input $x=a$, and a real number $L$, we say that $f$ has  \emph{limit $L$ as $x$ approaches $a$}, and write 
\[\lim_{x \to a}f(x) = L\]
provided that we can make $f(x)$ as close to $L$ as we like by taking $x$ sufficiently close (but not equal) to $a$. If we cannot make $f(x)$ as close to a single value as we would like as $x$ approaches $a$, then we say that \emph{$f$ does not have a limit as $x$ approaches $a$.}




As a warm up consider the function $h(x)$ given by the the graph below. Use the graph to answer each of the following questions.  If the function value or limit does not exist, explain why.


\begin{minipage}{.5\textwidth}
\includegraphics[width=6cm]{alglimitswarmup.png}\end{minipage}

\begin{minipage}{.5\textwidth}
\begin{tabular}{llll}
   $h(-1) =  \rule{1.5cm}{.1mm}$ & \hspace{.2in} & $h(1) =  \rule{1.5cm}{.1mm}$  & \hspace{1in}   \\ \\
  $\displaystyle \lim_{x \to -1^-} h(x) =  \rule{1.5cm}{.1mm}$ &\hspace{.2in} &  $\displaystyle \lim_{x \to 1^-} h(x) =  \rule{1.5cm}{.1mm}$ & \hspace{.2in} \\ \\ 
   $\displaystyle \lim_{x \to -1^+} h(x) =  \rule{1.5cm}{.1mm}$ &\hspace{.2in}  &  $\displaystyle \lim_{x \to 1^+} h(x) =  \rule{1.5cm}{.1mm}$ & \hspace{.2in} \\ \\
  $\displaystyle \lim_{x \to -1} h(x) =  \rule{1.5cm}{.1mm}$ &\hspace{.2in}  &  $\displaystyle \lim_{x \to 1} h(x) =  \rule{1.5cm}{.1mm}$ & \hspace{.2in}  \\
  \end{tabular}
\end{minipage}  


\emph{Explanation for those that don't exist:}
	
\index{Limits}
	
\begin{tags}
	    Limits, Graph, WarmUp
\end{tags}
	
\begin{diary}
	    
\end{diary}
	
\begin{solution}
	  
\end{solution}
	
\end{question}

\end{tagblock}

%-------------------------------------------------------------------------------------------------------------


\begin{tagblock}{Limits, AlgebraicLimits, Graph}
\begin{question}
	Given an equation for a function $f(x)$, we'd like to \emph{algebraically} determine the $\lim_{x \to a} f(x)$.  

Let's start with two easy functions $f(x) = C$ where $C$ is a constant and $g(x) =x$
\begin{enumerate}
\item On the left is the graph of $f(x) = C$, for some constant $C$.  From the graph compute $\lim_{x \to 1} f(x)$,   $\lim_{x \to 2} f(x)$ and then  $\lim_{x \to a} f(x)$ for any $a$.  
\item Graph $g(x) = x$ below on the right labeling your axes.  From the graph compute $\lim_{x \to 1} g(x)$,   $\lim_{x \to 2} g(x)$ and then  $\lim_{x \to a} g(x)$ for any $a$. 
\end{enumerate}

\includegraphics[width=6cm]{limitsconstant.png} \hspace{.5in} \includegraphics[width=6cm]{limitsblank.png} 
	
\index{Limits}
	
\begin{tags}
	    Limits, AlgebraicLimits, Graph
\end{tags}
	
\begin{diary}
	    
\end{diary}
	
\begin{solution}
	  
\end{solution}
	
\end{question}

\end{tagblock}

%-------------------------------------------------------------------------------------------------------------


\begin{tagblock}{Limits, AlgebraicLimits}
\begin{question}
	Computing limits behave nicely with algebra.  In particular we have the following
\bigskip

 \textbf{Limit Laws}:  if $\lim_{x \to a} f(x)$ and $\lim_{x \to a} g(x)$ both exist then
\begin{enumerate}
\item[LL 1.] Sum/Difference Law: $\displaystyle \lim_{x \to a} (f(x) \pm g(x) ) = \lim_{x \to a} f(x) \pm \lim_{x \to a} g(x) $
\item[LL 2.] Constant Multiple Law: $\displaystyle \lim_{x \to a} (c(f(x)) = c \lim_{x \to a} (f(x)$ for any constant $c$
\item[LL 3.]Product Law:  $\displaystyle \lim_{x \to a} (f(x) \cdot g(x) ) = \lim_{x \to a} f(x) \cdot \lim_{x \to a} g(x) $
\item[LL 4.]Quotient Law: $\displaystyle \lim_{x \to a} \frac{f(x)}{g(x) } =  \frac{\lim_{x \to a} f(x)}{\lim_{x \to a} g(x) }$ provided $\lim_{x \to a} g(x) \neq 0$
\end{enumerate}

\newpage

\item Let $h(x) = 3x^2 + 4$. 
\begin{enumerate}
\item Use the Limit Laws and what we discovered in Question 2. to compute $ \lim_{x \to 2} h(x)$ justifying each step (tell me which Limit Law you use at each step).  For example, we can start using LL 1.
\begin{eqnarray*}
 \lim_{x \to 2} h(x) &=&  \lim_{x \to 2} (3x^2 + 4) \\
 &=&  \lim_{x \to 2} (3x^2) +  \lim_{x \to 2} (4) \hspace{.5in} \text{by LL 1. Sum/Difference} \\ \\
 & = & 
 \end{eqnarray*}
\vspace{2.5in}

\item Compute $h(2)$.  How does this compare to $ \lim_{x \to 2} h(x)$?

\vspace{.5in}
\item If $p(x)$ is any \textbf{polynomial} what can you say about $\lim_{x \to a} p(x)$ and $p(a)$?  Explain your answer.  

\vspace{2in}

\item Use the Limit Laws to compute $\displaystyle \lim_{x \to 2} \frac{3x^2+4}{x+2}$  
\end{enumerate}
	
\index{Limits}
	
\begin{tags}
	    Limits, AlgebraicLimits
\end{tags}
	
\begin{diary}
	    
\end{diary}
	
\begin{solution}
	  
\end{solution}
	
\end{question}

\end{tagblock}

%-------------------------------------------------------------------------------------------------------------


\begin{tagblock}{Limits, AlgebraicLimits}
\begin{question}
	In the last worksheet we investigated the function $\displaystyle f(x) = \frac{x^2-4}{x-2}$ and the $\lim_{x \to 2} f(x)$.  
Why can't we use the Quotient Law to compute the  $\lim_{x \to 2} f(x)$?  

\vspace{1in}

 However, we did a little algebra and factored the numerator as $(x-2)(x+2)$ so that
 \[ \lim_{x \to 2} \frac{x^2-4}{x-2} =  \lim_{x \to 2} \frac{(x-2)(x+2)}{x-2} = \lim_{x \to 2} x+2 = 2+2 = 4 \]
	
\index{Limits}
	
\begin{tags}
	    Limits, AlgebraicLimits
\end{tags}
	
\begin{diary}
	    
\end{diary}
	
\begin{solution}
	  
\end{solution}
	
\end{question}

\end{tagblock}

%-------------------------------------------------------------------------------------------------------------


\begin{tagblock}{Limits, AlgebraicLimits}
\begin{question}
	\emph{Roughly the idea is to algebraically manipulate our function until we can evaluate at $a$.  }
 
 
 
 
  Let $\displaystyle g(x) =  \frac{(1+x)^2 - 1}{x}$.  What is the domain of $g(x)$?   Compute $\displaystyle \lim_{x \to 0} g(x)$. \\

	
\index{Limits}
	
\begin{tags}
	    Limits, AlgebraicLimits
\end{tags}
	
\begin{diary}
	    
\end{diary}
	
\begin{solution}
	  
\end{solution}
	
\end{question}

\end{tagblock}

%-------------------------------------------------------------------------------------------------------------



\begin{tagblock}{Limits, AlgebraicLimits}
\begin{question}
	 Let $\displaystyle h(t)= \frac{\sqrt{t^2+4}-2}{t^2}$.   What is the domain of $h(t)$?  Compute $\displaystyle  \lim_{t \to 0} h(t)$.\\
 \emph{Hint: Multiply by a clever choice of $\displaystyle 1 = \frac{\sqrt{t^2+4}+2}{\sqrt{t^2+4}+2}$, we call this the ``conjugate'' }
	
\index{Limits}
	
\begin{tags}
	    Limits, AlgebraicLimits
\end{tags}
	
\begin{diary}
	    
\end{diary}
	
\begin{solution}
	  
\end{solution}
	
\end{question}

\end{tagblock}

%-------------------------------------------------------------------------------------------------------------


\begin{tagblock}{Limits, AlgebraicLimits}
\begin{question}
	 Let $\displaystyle q(x) = \frac{\frac{1}{x} - \frac{1}{3}}{3-x}$.  What is the domain of $q(x)$?   Compute $\displaystyle \lim_{x \to 3} q(x)$.\\
% \emph{Hint: Rewrite the numerator as one fraction by finding a common denominator }
	
\index{Limits}
	
\begin{tags}
	    Limits, AlgebraicLimits
\end{tags}
	
\begin{diary}
	    
\end{diary}
	
\begin{solution}
	  
\end{solution}
	
\end{question}

\end{tagblock}

%-------------------------------------------------------------------------------------------------------------


\begin{tagblock}{Limits, AlgebraicLimits, Theory}
\begin{question}
	 \begin{enumerate}

\item Can you find two functions $f(x)$ and $g(x)$ such that $\displaystyle \lim_{x \to 5} f(x) = 0$ and $\displaystyle \lim_{x \to 5} g(x) = 0$ and 


$\displaystyle \lim_{x \to 5} ({f(x)} + {g(x)})  = 13$?  If so, give explicit formulas for $f(x)$ and $g(x)$; if not explain why. 


\vspace{3in}

\item Can you find two functions $f(x)$ and $g(x)$ such that $\displaystyle \lim_{x \to 5} f(x) = 0$ and $\displaystyle \lim_{x \to 5} g(x) = 0$ and 


$\displaystyle \lim_{x \to 5} \frac{f(x)}{g(x)}  = 13$?  If so, give explicit formulas for $f(x)$ and $g(x)$; if not explain why.  

\end{enumerate}
	
\index{Limits}
	
\begin{tags}
	    Limits, AlgebraicLimits, Theory
\end{tags}
	
\begin{diary}
	    
\end{diary}
	
\begin{solution}
	  
\end{solution}
	
\end{question}

\end{tagblock}

%-------------------------------------------------------------------------------------------------------------

\begin{tagblock}{Limits, W1}
\begin{question}
	Evaluate $\displaystyle\lim_{x\to 5}\frac{x^2-2x-15}{4x^2-20x}$
	
\index{Limits}
	
\begin{tags}
	    Limits, W1
\end{tags}
	
\begin{diary}
	    
\end{diary}
	
\begin{solution}
	  
\end{solution}
	
\end{question}

\end{tagblock}

%-------------------------------------------------------------------------------------------------------------
\begin{tagblock}{Limits, SquareRoots, W1}
\begin{question}
	Evaluate $\displaystyle\lim_{x\to\infty}(\sqrt{x+2}-\sqrt{x})$.
	
\index{Limits}
	
\begin{tags}
	    Limits, SquareRoots, W1
\end{tags}
	
\begin{diary}
	    
\end{diary}
	
\begin{solution}
	  
\end{solution}
	
\end{question}

\end{tagblock}

%-------------------------------------------------------------------------------------------------------------

\begin{tagblock}{Limits, W4, Exponentials, Logarithms}
\begin{question}
	Write down the values of the following limits, no rigorous justification is necessary.

\bigskip

a) $\displaystyle\lim_{x\to\infty}2^x$

\bigskip

b) $\displaystyle\lim_{x\to-\infty}e^{-x}$

\bigskip

c) $\displaystyle\lim_{x\to\infty}e^{-x}$

\bigskip

d) $\displaystyle\lim_{x\to\infty}\ln(\sqrt{x})$

\bigskip

e) $\displaystyle\lim_{x\to-\infty}\ln(x)$
	
\index{Limits}
	
\begin{tags}
	    Limits, W4, Exponentials, Logarithms
\end{tags}
	
\begin{diary}
	    
\end{diary}
	
\begin{solution}
	  
\end{solution}
	
\end{question}

\end{tagblock}

%-------------------------------------------------------------------------------------------------------------
\begin{tagblock}{Limits, W6, Exponentials, L'Hopital, ImproperIntegralInfinite}
\begin{question}
	To calculate the Laplace Transform of even a function as simple as $f(t)=t$, we arrive at a limit of the form
\[
\lim_{x\to\infty} \frac x {e^{sx}}.
\]

No algebraic technique can help you with this limit, so we need a new trick: L'H\^opital's Rule! The statement is:

\bigskip

\textbf{L'Hopital's Rule:} Let $f$ be differentiable on an open interval containing $x=c$, or on $(a,\infty)$ for some number $a$ if $c=\infty$. Suppose $g'(x)\ne 0$ on that same interval except possibly at $x=c$. Then if either $\displaystyle\lim_{x\to c}f(x)=\lim_{x\to c}g(x)=0$ or $\displaystyle\lim_{x\to c}f(x)=\lim_{x\to c}g(x)=\pm\infty$,
\[
\lim_{x\to a}\frac{f(x)}{g(x)}=\lim_{x\to a}\frac{f'(x)}{g'(x)}.
\]

a) Calculate $\displaystyle\lim_{x\to\infty}\frac{x}{e^{x}}$ using l'H\^opital's rule. 
 
\bigskip

b) Compute the Laplace Transform of $f(t)=t$.
	
\index{Limits}
	
\begin{tags}
	    Limits, W6, Exponentials, L'Hopital, ImproperIntegralInfinite
\end{tags}
	
\begin{diary}
	    
\end{diary}
	
\begin{solution}
	  
\end{solution}
	
\end{question}

\end{tagblock}

%-------------------------------------------------------------------------------------------------------------
\begin{tagblock}{Limits, W6, Exponentials, L'Hopital, ImproperIntegralInfinite, Example}
\begin{question}
	You can repeatedly use l'H\^opital's rule as long as you keep on getting $0/0$ or $\pm\infty/\pm\infty$ quotients. For example,
\[
\lim_{x\to 0}\frac{x-\sin(x)}{x^2}=\lim_{x\to 0}\frac{1-\cos(x)}{2x},
\]
which is still a $0/0$ quotient. We get to use l'H\^opital's rule again!
\[
\lim_{x\to 0}\frac{x-\sin(x)}{x^2}=\lim_{x\to 0}\frac{1-\cos(x)}{2x}=\lim_{x\to 0} \frac {\sin(x)}2=0.
\]

\bigskip

a) Compute the Laplace Transform of $g(t)=t^2$. 

\bigskip

b) Compute the Laplace Transform of $h(t)=t^3$. 

\bigskip

c) Can you guess what the Laplace Transform of $f_n(t)=t^n$ will be for $n$ a whole number?
	
\index{Limits}
	
\begin{tags}
	    Limits, W6, Exponentials, L'Hopital, ImproperIntegralInfinite, Example
\end{tags}
	
\begin{diary}
	    
\end{diary}
	
\begin{solution}
	  
\end{solution}
	
\end{question}

\end{tagblock}

%-------------------------------------------------------------------------------------------------------------
\begin{tagblock}{Limits, W6, Exponentials, L'Hopital}
\begin{question}
	You NEED to have a quotient when using l'Hopital's Rule. A limit is in \textit{indeterminate form} if it can be rearranged to a quotient in which l'Hopital's Rule applies. Here are examples of indeterminate forms, along with suggestions on how to calculate the limits. Remember, quotient!

\bigskip

a) $\infty-\infty$: $\displaystyle\lim_{x\to 0^+}\left(\cot(x)-\frac 1 x\right)$ (common denominator)

\bigskip

b) $\infty^0$: $\displaystyle\lim_{t\to\infty}t^{1/t}$ (apply $\ln$, move exponent down, compute limit, then exponentiate)

\bigskip

c) $1^{\infty}$: $\displaystyle\lim_{t\to\infty}\left(1+\frac 2 t\right)^t$ (same trick as part b))

\bigskip

d) $0^0$: $\displaystyle\lim_{x\to 1}(x-1)^{\ln(x)}$ (same trick as part b))
	
\index{Limits}
	
\begin{tags}
	   Limits, W6, Exponentials, L'Hopital
\end{tags}
	
\begin{diary}
	    
\end{diary}
	
\begin{solution}
	  
\end{solution}
	
\end{question}

\end{tagblock}

%-------------------------------------------------------------------------------------------------------------
\begin{tagblock}{Limits, W6, L'Hopital, Theory, Challenge}
\begin{question}

Here is an argument for how l'Hopital's Rule actually works. 
\begin{align*}
\lim_{x\to a}\frac {f(x)}{g(x)}&=\lim_{x\to a}\left(\frac {f(x)}{g(x)}\cdot\frac{x-a}{x-a}\right) \\
&=\lim_{x\to a}\left(\frac {f(x)}{x-a}\cdot\frac{x-a}{g(x)}\right)\\
&=\lim_{x\to a}\frac {f(x)}{x-a}\cdot\lim_{x\to a}\frac{x-a}{g(x)}\\
&=\frac {f'(a)}{g'(a)}\\
&=\lim_{x\to a}\frac {f'(x)}{g'(x)}
\end{align*}
Does this argument seem correct to you or does it take some liberties with the truth (if not outright lie to you)? If the latter, where?
	
\index{Limits}
	
\begin{tags}
	   Limits, W6, L'Hopital, Theory, Challenge
\end{tags}
	
\begin{diary}
	    
\end{diary}
	
\begin{solution}
	  
\end{solution}
	
\end{question}

\end{tagblock}

%-------------------------------------------------------------------------------------------------------------
 \section{Polar}\index{Polar}
\fancyhead[R]{\large Polar}

\begin{tagblock}{Polar, Geometry, WarmUp, W19}
\begin{question}
	Remember your first experience with vectors in physics (if you have never seen vectors, let me know). Polar coordinates are really the proper way to describe vectors: a magnitude and a direction. 

\bigskip

a) Draw a right triangle with legs of length $3$ and $4$- you may pick your own units. Find the length of the hypotenuse.

\bigskip

b) Again draw a right triangle with legs of length $3$ and $3\sqrt{3}$. Find the angle opposite either side. 
	
\index{Polar}
	
\begin{tags}
	    Polar, Geometry, WarmUp, W19
\end{tags}
	
\begin{diary}
	    %S2016-HW1-Q1
\end{diary}
	
\begin{solution}
	   
\end{solution}
	
\end{question}

\end{tagblock}

%-------------------------------------------------------------------------------------------------------------


\begin{tagblock}{Polar, Geometry, Definition, W19}
\begin{question}
	If you have a point $(x,y)$ in rectangular coordinates, the polar coordinates are (almost) given by 
\begin{equation}\label{cartpolar}
r=\sqrt{x^2+y^2}, \quad \theta=\arctan(y/x).
\end{equation}
The quantities $r$ and $\theta$ represent the distance to the origin and the angle with the $x$-axis, respectively, for the vector from $(0,0)$ to $(x,y)$. There isn't much of a problem with $r$ in general, but remember that the range of arctangent is the restricted domain of tangent, $(-\pi/2,\pi/2)$- not quite enough angle!

\bigskip

a) Mindlessly apply equation \eqref{cartpolar} to the point $(-2,3)$. Is what you get correct? How do you fix it?

\bigskip

b) Same question as a) for the point $(-2,-3)$.

\bigskip

c) Try to apply equation \eqref{cartpolar} to the point $(0,2)$. What goes wrong? What should the polar coordinates of $(0,2)$ be, interpreted geometrically?

\bigskip

d) Same question as c) for the point $(0,-2)$. You should now have covered all the bases. 
	
\index{Polar}
	
\begin{tags}
	    Polar, Geometry, Definition, W19
\end{tags}
	
\begin{diary}
	    %S2016-HW1-Q1
\end{diary}
	
\begin{solution}
	   
\end{solution}
	
\end{question}

\end{tagblock}

%-------------------------------------------------------------------------------------------------------------


\begin{tagblock}{Polar, Geometry, W19}
\begin{question}

Going from polar coordinates to rectangular coordinates is even easier. 

\bigskip

a) Draw a right triangle with base on the $x$-axis and hypotenuse of length $r$ making an angle $\theta$ with the $x$-axis in the first quadrant. What are the $x$ and $y$ coordinates of the highest point on the triangle?

\bigskip

b) Find the rectangular coordinates of the polar point $(16,-\pi/3)$ using the formulas $x=r\cos(\theta)$, $y=r\sin(\theta)$. 
	
\index{Polar}
	
\begin{tags}
	    Polar, Geometry, W19
\end{tags}
	
\begin{diary}
	    %S2016-HW1-Q1
\end{diary}
	
\begin{solution}
	   
\end{solution}
	
\end{question}

\end{tagblock}

%-------------------------------------------------------------------------------------------------------------


\begin{tagblock}{Polar, Geometry, Definition, Example, W20}
\begin{question}

The neat thing about points in polar coordinates is that there is more than one representation:
\[
(2,\pi/4)=(2,9\pi/4)=(2,17\pi/4),\cdots
\]

There are a couple of things to cover before we leave the basics. Since people really, really want freedom of expression when it comes to writing down points, we'd best be able to handle \textit{negative} radius and angle. Negative angles are easy: just rotate clockwise intead of counterclockwise.

\bigskip

a) Find two representations in polar coordinates for $(4,-7\pi/8)$ that have positive angle. 

\bigskip

Negative radius is handled by the idea that $-r$ should be in the opposite direction as $r$, so 
\[
(-r,\theta)=(r,\theta+\pi).
\]

b) Find two representations in polar coordinates, both with positive radius but one with positive angle and the other negative, for $(-7,13\pi/6)$. 

\bigskip

c) Find two representations of $(-4,-3\pi/17)$. b) Find the rectangular coordinates of the polar point $(16,-\pi/3)$ using the formulas $x=r\cos(\theta)$, $y=r\sin(\theta)$. 
	
\index{Polar}
	
\begin{tags}
	    Polar, Geometry, Definition, Example, W20
\end{tags}
	
\begin{diary}
	    %S2016-HW1-Q1
\end{diary}
	
\begin{solution}
	   
\end{solution}
	
\end{question}

\end{tagblock}

%-------------------------------------------------------------------------------------------------------------
\begin{tagblock}{Polar, Geometry, Graph, W20}
\begin{question}

Let's take a look at how to express curves given in Cartesian coordinates via polar. Try to find a polar equation for the following curves. Your equation should contain at least one of the variables $r$ and $\theta$, though not necessarily both.

\bigskip

a) $x^2+y^2=16$

\bigskip

b) $y=x$

\bigskip

c) $\dfrac{x^2}9+\dfrac{y^2}{4}=1$

\bigskip

d) $(x-1/2)^2+y^2=1/4$
	
\index{Polar}
	
\begin{tags}
	    Polar, Geometry, Graph, W20
\end{tags}
	
\begin{diary}
	    %S2016-HW1-Q1
\end{diary}
	
\begin{solution}
	   
\end{solution}
	
\end{question}

\end{tagblock}

%-------------------------------------------------------------------------------------------------------------
\begin{tagblock}{Polar, Geometry, Graph, W20}
\begin{question}

Now we'll try the opposite direction: try to find a rectangular equation for the given polar curve. All angle measures are in radians. 

\bigskip

a) $r=7$

\bigskip

b) $\theta=\pi/3$

\bigskip

c) $r=2\sin(\theta)+\cos(\theta)$

\bigskip

d) $r=\theta$ (try to graph this curve!)
	
\index{Polar}
	
\begin{tags}
	    Polar, Geometry, Graph, W20
\end{tags}
	
\begin{diary}
	    %S2016-HW1-Q1
\end{diary}
	
\begin{solution}
	   
\end{solution}
	
\end{question}

\end{tagblock}

%-------------------------------------------------------------------------------------------------------------

 \section{Parametric}\index{Parametric}
\fancyhead[R]{\large Parametric}

\begin{tagblock}{Parametric, Geometry, Graph, Trigonometry, Challenge, W21}
\begin{question}
Parameterization is a way of bringing graphs that don't play by the rules of regular functions into the fold. 

\bigskip

a) Draw a graph that is NOT the graph of a function $y=f(x)$. What important geometric test does the graph fail?

\bigskip

If you couldn't come up with an example for 1a), the easiest thing to draw is a circle. Let's take the unit circle defined by the equation $x^2+y^2=1$. 

A \textit{parameterization} of the circle is a choice of functions $x$ and $y$ depending on the real variable $t$ that together satisfy the equation $x^2+y^2=1$. 

\bigskip

b) Check that $x(t)=\cos(t)$, $y(t)=\sin(t)$ parameterizes the unit circle. 

\bigskip

c) Find another parameterization for the unit circle (so parameterizations aren't unique).

\bigskip

d) Find infinitely many parameterizations of the unit circle (so parameterizations REALLY aren't unique)!
	
\index{Parametric}
	
\begin{tags}
	    Parametric, Geometry, Graph, Trigonometry, Challenge, W21
\end{tags}
	
\begin{diary}
	   % S2016-HW1-Q1
\end{diary}
	
\begin{solution}
	   
\end{solution}
	
\end{question}

\end{tagblock}

%-------------------------------------------------------------------------------------------------------------
\begin{tagblock}{Parametric, Graph, Challenge, Definition, Example, Theory, W21}
\begin{question}
A \textit{parametric curve} is just the collection of all $(x,y)$ values that satisfy a given parameterization $x=x(t)$ and $y=y(t)$. The dependence on $t$ is (almost) completely suppressed, but when you draw the curve, you usually write an arrow in the direction the $(x,y)$ values move as $t$ increases.

\bigskip

a) Try this for the parameterizations of the unit circle from 1b) and 1c).

\bigskip

One of the great features of parameterizations is that they completely subsume regular Cartesian functions! For example, if $y=f(x)=x^2$, a parameterization can be almost cheatingly obtained by 
\[
x(t)=t \ \textrm{and} \ y(t)=t^2.
\] 

\bigskip
 
b) Find a parameterization for $y=\cos(x)$. Draw the direction you move as $t$ increases. 

\bigskip

c) Find a parameterization for an arbitrary Cartesian function $y=f(x)$.
	
\index{Parametric}
	
\begin{tags}
	    Parametric, Graph, Challenge, Definition, Example, Theory, W21
\end{tags}
	
\begin{diary}
	   % S2016-HW1-Q1
\end{diary}
	
\begin{solution}
	   
\end{solution}
	
\end{question}

\end{tagblock}

%-------------------------------------------------------------------------------------------------------------
\begin{tagblock}{Parametric, Graph, Challenge, Exponentials, Logarithms, Trigonometry, W21}
\begin{question}

Find Cartesian equations for the following parametric curves in a) through d). 

\bigskip

a) $x(t)= \ln(7t)$, $y(t)=\ln(5t)$

\bigskip

b) $x(t)= e^t$, $y(t)=e^{-t}$

\bigskip

c) $x(t)= 2\cos(t)$, $y(t)=3\sin(t)$.

\bigskip

d) $x(t)=\dfrac{\cos(t)}{4}$, $y(t)=\dfrac{\sin(t)}9$.
	
\index{Parametric}
	
\begin{tags}
	    Parametric, Graph, Challenge, Exponentials, Logarithms, Trigonometry, W21
\end{tags}
	
\begin{diary}
	   % S2016-HW1-Q1
\end{diary}
	
\begin{solution}
	   
\end{solution}
	
\end{question}

\end{tagblock}

%-------------------------------------------------------------------------------------------------------------
\begin{tagblock}{Parametric, Graph, Theory, TangentLines, Derivatives, W22}
\begin{question}

 Now that we have introduced parameterizations, what else would we do but figure out a way to do some calculus with them? The first thing on our list is to recover derivatives and tangent lines. 

\bigskip

a) If you parameterize $y=f(x)$ as $x(t)=t$, $y(t)=f(t)$, what is the slope of the tangent line to the graph of $f$ at the point $t=1$?

\bigskip

b) Look at the coordinates from the previous example and see if you can express the slope of the tangent line in terms of the $x$ and $y$ derivatives.

\bigskip

c) Generalize your answer in b) to arbitrary parametric curves. What happens when $y'(1)=0$?
	
\index{Parametric}
	
\begin{tags}
	    Parametric, Graph, Theory, TangentLines, Derivatives, W22
\end{tags}
	
\begin{diary}
	   % S2016-HW1-Q1
\end{diary}
	
\begin{solution}
	   
\end{solution}
	
\end{question}

\end{tagblock}

%-------------------------------------------------------------------------------------------------------------
\begin{tagblock}{Parametric, TangentLines, Derivatives, Exponentials, Logarithms, Trigonometry, W22}
\begin{question}

Find an equation for the tangent line for the following parametric curves at the given point, then parameterize the tangent lines.

\bigskip

a) $x(t)=\ln(t^2), \ y(t)=16^{\frac t 2}$, $t=1$

\bigskip

b) $x(t)=(e\cos(t))^{\sin(t)}, \ y(t)=\arctan(t^2+1)$, $(x,y)=(1,\pi/4)$
	
\index{Parametric}
	
\begin{tags}
	    Parametric, TangentLines, Derivatives, Exponentials, Logarithms, Trigonometry, W22
\end{tags}
	
\begin{diary}
	   % S2016-HW1-Q1
\end{diary}
	
\begin{solution}
	   
\end{solution}
	
\end{question}

\end{tagblock}

%-------------------------------------------------------------------------------------------------------------
\begin{tagblock}{Parametric, ArcLength, Derivatives, Integration, Definition, Trigonometry, Logarithms, Challenge, W22}
\begin{question}

Now for another ``geometric" application of parametric curves: arc length!

\bigskip

If you draw a sufficiently nice curve, you can approximate the length of the curve by drawing line segments between points on the curve.

\bigskip

a) Draw a curve and try doing this. 

\bigskip

The distance between points on the curve is given by the \textit{distance formula}, which will also give you the lengths of the line segments used to approximate the length of the curve. If we take the limit as the number of segments goes to infinity (and the length goes to zero), we arrive at the following formula for the arc length $L$ of a parametric curve from $t=a$ to $t=b$:

\bigskip
\[
L=\int_a^b\sqrt{(x'(t))^2+(y'(t))^2} \ dt.
\]

Calculate the arc length for the following curves: 

\bigskip

b) Take your favorite line, write down a parameterization, choose two $t$ values, and check that this formula actually works. 

\bigskip

c) $x(t)=\sin(\pi\sin(t))$, $y(t)=\cos(\pi\sin(t))$ from $t=0$ to $t=\pi/6$

\bigskip

d) $x(t)=\ln(t)$, $y(t)=t$ from $t=1$ to $t=e^2$

\bigskip

e) $y=x^2$ from $x=0$ to $x=1$ (this is actually horrendously hard- contrast it with part b)!)
	
\index{Parametric}
	
\begin{tags}
	    Parametric, ArcLength, Derivatives, Integration, Definition, Trigonometry, Logarithms, Challenge, W22
\end{tags}
	
\begin{diary}
	   % S2016-HW1-Q1
\end{diary}
	
\begin{solution}
	   
\end{solution}
	
\end{question}

\end{tagblock}

%-------------------------------------------------------------------------------------------------------------
\begin{tagblock}{Parametric, ArcLength, Graph, Derivatives, Integration, Theory, Challenge, W22}
\begin{question}

Let $C$ be the circle $x^2+y^2=1$.  

\bigskip

a) Find a parameterization for $C$. 

\bigskip

b) Let $0\leq \theta_0<2\pi$ be any angle in \textit{radians}. Show that the length of the portion of the graph of $C$ from the angle zero to the angle $\theta_0$ is precisely equal to $\theta_0$. 

\bigskip

c) Does the result from part b) help you understand what radian measure actually means (if you didn't already know it)?
	
\index{Parametric}
	
\begin{tags}
	    Parametric, ArcLength, Derivatives, Integration, Theory, Challenge, W22
\end{tags}
	
\begin{diary}
	   % S2016-HW1-Q1
\end{diary}
	
\begin{solution}
	   
\end{solution}
	
\end{question}

\end{tagblock}

%-------------------------------------------------------------------------------------------------------------
 \section{Review} \index{Review}
\fancyhead[R]{\large Review}

\begin{tagblock}{Review, PreCalc}
\begin{question}

In this worksheet we will have a quick review of functions. Functions are at the heart of mathematics: a function is a process or rule that associates each individual input to exactly one corresponding output. Students learn in courses prior to calculus that there are many different ways to represent functions, including through formulas, graphs, tables, and even words. %For example, the squaring function can be thought of in any of these ways. In words, the squaring function takes any real number $x$ and computes its square. The formulaic and graphical representations go hand in hand, as $y=f(x)=x^2$ is one of the simplest curves to graph. 
Today, we'll focus on functions via formulas and via graphs.  Recall that, the \emph{domain} of a function is the set of all possible input values and the \emph{range} of a function is the set of all possible output values.

\bigskip

Below are some examples of functions given by formulas: 

\begin{itemize}
\item \textbf{Polynomials:} in general a polynomial is of the form $p(x) = a_0 + a_1x + a_2x^2 + \cdots + a_nx^n$, where the coefficients $a_i$ are any real numbers.  
\item \textbf{Rational functions:} A rational function is a quotient of two polynomials 
\item \textbf{Root functions:} Functions involving a square-root, cube-root, etc.  for example, $g(x) = \sqrt{x+2}$ and $h(x)=\sqrt[3]{5x^2+1} = (5x^2+1)^{1/3}$
\item \textbf{Exponential functions:} For example $e^x$ or more generally $a^x$, where $a>0$ is some constant.
\item \textbf{Trigonometric functions:}  For example, $\sin(x)$, $\cos(x)$, $\tan(x) = \frac{\sin(x)}{\cos(x)}$ 
\item \textbf{Piece-wise functions:} 
$f(x) = \begin{cases} x^2 & \text{ if } x<1 \\ x+2 & \text{ if } x \geq 1 \end{cases} $ \, 
or  $|x|= \begin{cases} x & \text{ if } x\geq 0 \\ -x & \text{ if } x \leq 0 \end{cases} $, the absolute value function.  
\end{itemize} 

 Give the domain of the following functions:
\begin{enumerate}
\item  $p(x) = 53x^{40}-9.7x^2 + 88$
\vspace{.2in}


\item  $\displaystyle Q(x) = \frac{x^4+6}{x^2-6}$

\vspace{.2in}

\item  $\displaystyle R(x) = \frac{x - 1}{x^2+1}$
\vspace{.4in}

\item $\displaystyle F(x) = \frac{x-3}{x^2 -x -6}$
\vspace{.4in}

\item  A general rational function $\displaystyle Q(x) = \frac{p(x)}{q(x)}$, where $p(x)$ and $q(x)$ are polynomials.  
\vspace{.4in}

\item $g(x) = \sqrt{x+2}$

\vspace{.4in}
\item $f(x) = e^x$


\end{enumerate}
	
 

\begin{tags}
	    Review, PreCalc
\end{tags}
	
\begin{diary}
	    
\end{diary}
	
\begin{solution}
		
\end{solution}
	
\end{question}

\end{tagblock}

%-------------------------------------------------------------------------------------------------------------
 
\begin{tagblock}{Review, PreCalc}
\begin{question}

If we have a formula for a function we can evaluate the function at a specific number.

Let $f(x) = x^2 + 3$
\begin{enumerate}
\item Compute $f(2)$ and $f(-1)$
\item We can also evaluate at unknowns:  Compute $f(a)$, where $a$ is arbitrary
\item Compute $f(2+h)$ and simplify your answer
\item Compute $\displaystyle \frac{f(2+h) - f(2)} {h}$ and simplify your answer.  
\item Compute $\displaystyle \frac{f(a+h) - f(a)} {h}$ and simplify your answer.  
\end{enumerate}

This quantity $\displaystyle \frac{f(a+h) - f(a)} {h}$ is called the \emph{difference quotient} and will be important to us later.



	
 

\begin{tags}
	    Review, PreCalc
\end{tags}
	
\begin{diary}
	    
\end{diary}
	
\begin{solution}
		
\end{solution}
	
\end{question}

\end{tagblock}

%-------------------------------------------------------------------------------------------------------------
 
\begin{tagblock}{Review, PreCalc, Graph}
\begin{question}

We may also be given a function via a graph. Let $f(x)$ be the function given by the graph below. Using the graph, determine the values $f(-2), f(-1), f(0), f(1)$ and $f(2)$ if defined.   If the function value is not defined, explain what feature of the graph tells you this.  
\begin{figure}[h]
\centering
\includegraphics[width=6cm]{limits1.png}
\end{figure}
	
 

\begin{tags}
	    Review, PreCalc, Graph
\end{tags}
	
\begin{diary}
	    
\end{diary}
	
\begin{solution}
		
\end{solution}
	
\end{question}

\end{tagblock}

%-------------------------------------------------------------------------------------------------------------


\begin{tagblock}{Review, PreCalc, Graph}
\begin{question}

One can ask if every curve is the graph of a function.  Which of the following curves are graphs of functions, and why?  (Do you remember the \emph{Vertical Line Test}?)

\begin{figure}[h]\includegraphics[width=5cm]{function2.png} \hspace{.2in}  \includegraphics[width=5cm]{function3.png} \hspace{.2in} \includegraphics[width=5cm]{function4.png} \end{figure}
	
 

\begin{tags}
	    Review, PreCalc, Graph
\end{tags}
	
\begin{diary}
	    
\end{diary}
	
\begin{solution}
		
\end{solution}
	
\end{question}

\end{tagblock}

%-------------------------------------------------------------------------------------------------------------


\begin{tagblock}{Review, PreCalc, Graph}
\begin{question}

One of the easiest functions is a \emph{linear function}.  Recall a linear function has a graph that is a line can be expressed in 
\begin{itemize}
\item ``slope-intercept form:'' $y=mx+b$, where $m$ is the slope and $b$ is the $y$-intercept, or in 
\item``point-slope form:'' $y-y_1 = m(x-x_1)$, where $m$ is the slope and $(x_1,y_1)$ is a point on the line.  
\end{itemize}
Find the equation of the line connecting the points $(1,2)$ and $(3,-5)$, and graph the line below.
\begin{figure}[h]\includegraphics[width=7cm]{functionblank.png} \end{figure}

Even though linear functions are simple, they will be of great importance to us later.  

	
 

\begin{tags}
	    Review, PreCalc, Graph
\end{tags}
	
\begin{diary}
	    
\end{diary}
	
\begin{solution}
		
\end{solution}
	
\end{question}

\end{tagblock}

%-------------------------------------------------------------------------------------------------------------


\begin{tagblock}{Review, PreCalc}
\begin{question}

\textbf{Combinations of Functions}


 Given two functions $f$ and $g$ we can build a new function by adding to get $f+g$, where $(f+g)(x) = f(x) + g(x)$.  Similarly we can build new functions subtracting, multiplying or dividing.  
 
 Another way to build new functions from old ones is \emph{composition}: Given functions $f$ and $g$, the \emph{composition} of $f$ and $g$ is
\[(f \circ g) (x) = f(g(x)) \]
We think of this as a chain, where we first apply the function $g$ and then the function $f$:
\[x \to g(x) \to f(g(x))) \]

\bigskip

Let $f(x) = \sin(x)$ and $g(x) = x^3+1$ and $h(x)=\sqrt{x-5}$, find
\begin{enumerate}
\item $f+g$
\item $gh$
\item $f \circ g$ and  $g \circ f$.   Is $f \circ g = g \circ f$?
\item $g \circ h$ and  $h \circ g$.  Is $g \circ h = h \circ g$?
\end{enumerate} 

	
 

\begin{tags}
	    Review, PreCalc
\end{tags}
	
\begin{diary}
	    
\end{diary}
	
\begin{solution}
		
\end{solution}
	
\end{question}

\end{tagblock}

%-------------------------------------------------------------------------------------------------------------


\begin{tagblock}{Review, PreCalc, Exponential}
\begin{question}

\textbf{Exponential Functions}

An exponential function is of the form $f(x) = a^x$ where $a>0$ is a constant and $x$ is a variable. 


On the graph determine which function is 

\begin{minipage}{.4\textwidth}
\includegraphics[width=6cm]{exponential.png}\end{minipage}% This must go next to `\end{minipage}`
\begin{minipage}{.6\textwidth}
\begin{enumerate}
\item  $e^x$
\item $5^x$ 
\item $(\frac{1}{2})^x$
\end{enumerate}

\end{minipage}

A reminder of our \textbf{Exponential Rules} 



\begin{tcolorbox}

    \begin{tabular}{l l l l }
     $a^0=1$ & $a^{-n} = \frac{1}{a^n}$ & $a^{p/q} = \sqrt[q]{a^p}$ &     $a^{x+y} = a^x a^y$ \\ 
     $a^{x-y} = \frac{a^x}{a^y}$ & $(a^x)^y = a^{xy}$ &     $(ab)^x = a^xb^x$ & \\ 
\end{tabular}
\end{tcolorbox}




	
 

\begin{tags}
	    Review, PreCalc, Exponential
\end{tags}
	
\begin{diary}
	    
\end{diary}
	
\begin{solution}
		
\end{solution}
	
\end{question}

\end{tagblock}

%-------------------------------------------------------------------------------------------------------------


\begin{tagblock}{Review, PreCalc, Logarithms, Inverse}
\begin{question}

\textbf{Inverses}


Recall that a function is \emph{one-to-one} if it passes the horizontal line test, i.e. any horizontal line passes through the graph at most once.  

\begin{enumerate}
\item Is $f(x) = x^2$ one-to-one?  Is $g(x) = x^3$ one-to-one? Why or why not?


\vspace{1in} 

Given a one-one-function $f(x)$, the \emph{inverse function}, denoted by $f^{-1}(x)$ is the function with the property 
\[f^{-1}(y) = x \iff f(x) = y\]

This means 
\[f (f^{-1}(x)) = x \text{ and } f^{-1}(f(x)) = x \]
and
\[ \text{domain of } f^{-1}(x) = \text{ range of } f(x) \]

\[ \text{domain of } f(x) = \text{ range of } f^{-1}(x) \]


\bigskip

If we start with the function $f(x) = a^x$, where $a>0$ is a constant, then the inverse is the \emph{logarithmic function with base $a$}, denote $\log_a(x)$.  If we are start with $a=e$, i.e. $f(x) = e^x$, we usually write $\ln(x)$ instead of $\log_e(x)$.  

\bigskip

A reminder of our \textbf{Logarithmic Rules} 
\begin{tcolorbox}


    \begin{itemize}
    \item     $\log_a(a^x)=x$  (in particular, $\ln(e^x) = x$ )
    \item $a^{\log_a(x)} =x$  (in particular, $e^{\ln(x)} = x$ )
    \item  $\log_a(xy) = \log_a(x) +  \log_a(y)$
     \item  $\log_a(\frac{x}{y}) = \log_a(x) -  \log_a(y)$
   \item $\log_a(x^r) = r \log_a(x)$
   \end{itemize}    

\end{tcolorbox}


\item Write $\ln(1+x^2) + \frac{1}{x} \ln(x) - \ln(\sin(x))$ as a single logarithm.  

\vspace{1in}
\item Solve $\ln(5-2x) = -3$

\end{enumerate}




	
 

\begin{tags}
	    Review, PreCalc, Logarithms, Inverse
\end{tags}
	
\begin{diary}
	    
\end{diary}
	
\begin{solution}
		
\end{solution}
	
\end{question}

\end{tagblock}

%-------------------------------------------------------------------------------------------------------------


\begin{tagblock}{Review, PreCalc, Area, Geometry}
\begin{question}

Lastly, we would like to apply the math concepts we are studying to real-life applications.  In this case our problem is given in words, and we need to translate those words into a function (either equation or graph).  As we will see, it is often useful to draw pictures!  For example, let's look at a square:
\begin{enumerate}
\item Let $x$ be the length of the side of a square.  Draw a picture of a square, and label the sides.
\item Write a formula for $P$ which gives the perimeter of the square as function of $x$.  
\item Write a formula for $A$ which gives the area of the square as function of $x$. 
\item Write a formula for  $A$ which gives the area of the square as function of the perimeter $P$.  
\end{enumerate}



	
 

\begin{tags}
	    Review, PreCalc, Area, Geometry
\end{tags}
	
\begin{diary}
	    
\end{diary}
	
\begin{solution}
		
\end{solution}
	
\end{question}

\end{tagblock}

%-------------------------------------------------------------------------------------------------------------

 \section{Sequences}\index{Sequences}
\fancyhead[R]{\large Sequences}

\begin{tagblock}{Sequences, Challenge, WarmUp, W11}
\begin{question}

A \textit{sequence} is an ordered list of terms. Here are some examples with just numbers, see if you can find all the terms in the pattern:

\bigskip

a) $1,2,5,10,20,\dots$ (7 terms)
 
\bigskip

b) $31,28,31,30,31,30,31,\dots$ (12 terms)

\bigskip

c) $1,1,1,3,1,\dots$ (26 terms)

	
\index{Sequences}
	
\begin{tags}
	    Sequences, Challenge, WarmUp, W11
\end{tags}
	
\begin{diary}
	   
\end{diary}
	
\begin{solution}	

\end{solution}
	
\end{question}

\end{tagblock}

%-------------------------------------------------------------------------------------------------------------



\begin{tagblock}{Sequences, Challenge, WarmUp, W11}
\begin{question}

A sequence doesn't just have to be numbers. You can use pretty much anything that you can dream up an order on. The second example is finite and the other two are infinite. 

\bigskip

a) $\triangle,\square,\dots$

\bigskip

b) $AZ,GT,MN,\dots$ (26 terms)

\bigskip

c) $O,T,T,F,F,S,S,\dots$

	
\index{Sequences}
	
\begin{tags}
	    Sequences, Challenge, WarmUp, W11
\end{tags}
	
\begin{diary}
	   
\end{diary}
	
\begin{solution}	

\end{solution}
	
\end{question}

\end{tagblock}

%-------------------------------------------------------------------------------------------------------------



\begin{tagblock}{Sequences, Challenge, WarmUp, W11}
\begin{question}

Our focus in this class will be on infinite sequences of numbers. See if you can find the pattern:

\bigskip

a) $1,0,1,0,1,\dots$

\bigskip

b) $1,3,7,15,31,\dots$

\bigskip

c) $3,8,13,18,\dots$ 

\bigskip 

d) $1/4,-1/9,1/16,-1/25,\dots$

\bigskip

e) $0,6,24,60,120,\dots$

\bigskip

f) $1,1,2,3,5,8,13,\dots$

\bigskip

g) $2,3,5,7,11,13,\dots$

\bigskip

h) $23,21,24,19,26,15,\dots$ 

\bigskip

i) $1,11,21,1211,111221,312211,\dots$
	
\index{Sequences}
	
\begin{tags}
	    Sequences, Challenge, WarmUp, W11
\end{tags}
	
\begin{diary}
	   
\end{diary}
	
\begin{solution}	

\end{solution}
	
\end{question}

\end{tagblock}

%-------------------------------------------------------------------------------------------------------------


\begin{tagblock}{Sequences, Challenge, W11}
\begin{question}

Find the ones digit of the 63rd term of $7,7^2,7^3,7^4,7^5,\dots$
	
\index{Sequences}
	
\begin{tags}
	    Sequences, Challenge, W11
\end{tags}
	
\begin{diary}
	   
\end{diary}
	
\begin{solution}	

\end{solution}
	
\end{question}

\end{tagblock}

%-------------------------------------------------------------------------------------------------------------


\begin{tagblock}{Sequences, GeometricSequences, Definition, W12}
\begin{question}

Let's start with a very basic kind of sequence called a \textit{geometric sequence}. For such a sequence, any given term is a fixed multiple of the previous term; so if the $n^{th}$ term is $a_n$ and the $(n+1)^{st}$ term is $a_{n+1}$, there is a number $x$ called the \textit{ratio} with
\[
\frac{a_{n+1}}{a_n}=x.
\]
So for example, the sequence $(-1)^n$ that we saw in the last worksheet has
\[
\frac{a_{n+1}}{a_n}=-1
\]
and so is a geometric sequence.

This may look pretty simple, but it will eventually be the basis by which we can ensure that functions constructed with sequences make sense. 

Check whether the following sequences are geometric or not, and if so, find the ratio $x$.

\bigskip

a) $\left(4^{2n}\right)$
 
\bigskip

b) $\left(\dfrac{(-2)^n} n\right)$

\bigskip

c) $\left(\dfrac{(-5)^{n+2}} {12^n}\right)$

	
\index{Sequences}
	
\begin{tags}
	    Sequences, GeometricSequences, Definition, W12
\end{tags}
	
\begin{diary}
	   
\end{diary}

\begin{solution}	

\end{solution}
	
\end{question}

\end{tagblock}

%-------------------------------------------------------------------------------------------------------------



\begin{tagblock}{Sequences, GeometricSequences, L'Hopital, W12}
\begin{question}

Recall the following result:

\bigskip

\begin{itemize} 



\item (Representation Theorem) If $f$ is a real valued function of a real variable $x$ and $f(n)=a_n$ for some sequence $\{a_n\}_{n=1}^{\infty}$, then if $\displaystyle\lim_{x\to\infty}f(x)=L$, then $\displaystyle\lim_{n\to\infty}a_n=L$.

\end{itemize}

This theorem opens up l'Hopital's Rule for you if you need it! Use the Representation Theorem to check whether the following geometric sequences converge or diverge, and if they converge, find the limit.

\bigskip

a) $a_n=1$

\bigskip

b) $b_n=\left(\dfrac{-1}{3}\right)^n$

\bigskip

c) $d_n=\left(\dfrac{15}{14}\right)^n$

	
\index{Sequences}
	
\begin{tags}
	    Sequences, GeometricSequences, L'Hopital, W12
\end{tags}
	
\begin{diary}
	   
\end{diary}
	
\begin{solution}	

\end{solution}
	
\end{question}

\end{tagblock}

%-------------------------------------------------------------------------------------------------------------



\begin{tagblock}{Sequences, Factorial, SqueezeTheorem, W12}
\begin{question}

Recall the Representation Theorem, which we used to examine geometric sequences:

\bigskip

\begin{itemize} 



\item (Representation Theorem) If $f$ is a real valued function of a real variable $x$ and $f(n)=a_n$ for some sequence $\{a_n\}_{n=1}^{\infty}$, then if $\displaystyle\lim_{x\to\infty}f(x)=L$, then $\displaystyle\lim_{n\to\infty}a_n=L$.

\end{itemize} 



Not every sequence is geometric, and for those, there are no hard and fast rules, so we need more tricks. Another consequence of the definition is 

\begin{itemize} 

\item (Squeeze Theorem for Sequences:) If there is an $N$ with $a_n\leq b_n\leq c_n$ for all $n\geq N$, then $\displaystyle\lim_{n\to\infty}a_n=\displaystyle\lim_{n\to\infty}c_n=L$, implies $\displaystyle\lim_{n\to\infty}b_n=L$. 

\end{itemize}

Use either the Squeeze or Representation Theorem to check whether the following sequences converge or diverge, and if they converge, find the limit.

\bigskip

a) $a_n=\dfrac {(-1)^n}{n+1}$

\bigskip

b) $b_n=\dfrac{2n+1}{n+3}$

\bigskip

c) $d_n=\dfrac{3^n}{n!}$ (this one takes some thought.)

	
\index{Sequences}
	
\begin{tags}
	    Sequences, Factorial, SqueezeTheorem, W12
\end{tags}
	
\begin{diary}
	   
\end{diary}
	
\begin{solution}	

\end{solution}
	
\end{question}

\end{tagblock}

%-------------------------------------------------------------------------------------------------------------



\begin{tagblock}{Sequences, Theory, Challenge, W12}
\begin{question}

Finally, we look at some special sequences, called \textit{continued fractions}. Every real number $x$ has a continued fraction expansion. Such an expansion is just a sequence that converges to $x$. What follows are some continued fraction expansions; try to find the number $x$ that the sequence converges to (you don't have to worry about whether there is actually a limit, you may assume there is). The terms are purposefully written in a rather silly way. Parts a) and b) are infinite sequences. 

\bigskip 

a) $1, 1+\frac 1{1+1},1+\frac 1 {1+\frac 1 {1+1}},1+\frac 1 {1+\frac 1 {1+\frac 1 {1+1}}},\dots$ 

\bigskip

b) $2, 2+\frac 1 {2+2},2+\frac 1 {2+\frac 1 {2+2}},2+\frac 1 {2+\frac 1 {2+\frac 1 {2+2}}},\dots$ (If you've gotten a), this should be easy.)

\bigskip

c) Notice what kind of numbers you're getting in a) and b). If you have a \textit{finite} continued fraction expansion, what kind of number results? Can you determine a continued fraction expansion for such a number?

	
\index{Sequences}
	
\begin{tags}
	    Sequences, Theory, Challenge, W12
\end{tags}
	
\begin{diary}
	   
\end{diary}
	
\begin{solution}	

\end{solution}
	
\end{question}

\end{tagblock}

%-------------------------------------------------------------------------------------------------------------


 \section{Series}\index{Series}
\fancyhead[R]{\large Series} 

\begin{tagblock}{Series, WarmUp, W13}
\begin{question}

The only reason we need sequences is that sequences allow us to define \textit{series}. Knowing the difference between these concepts is probably the most difficult part of this course because...a series is a sequence, too!

From Calc I (or maybe earlier) you saw the notation $\Sigma$ for a sum of numbers when it came to defining a definite integral. Just to refresh your memory:

\bigskip
\[
\sum_{n=1}^51=1+1+1+1+1=5
\]
\bigskip
\[
\sum_{n=2}^4n=2+3+4=9
\]

Try a few! Find the sum.

\bigskip

a) $\displaystyle\sum_{n=1}^3n^2$
 
\bigskip

b) $\displaystyle\sum_{n=1}^6\left(\frac 1 2\right)^n$

\bigskip

c) $\displaystyle\sum_{n=3}^6\frac 1 n$
	
\index{Series}
	
\begin{tags}
	    Series, WarmUp, W13
\end{tags}
	
\begin{diary}
	    
\end{diary}
	
\begin{solution}
	   
\end{solution}
	
\end{question}

\end{tagblock}

%-------------------------------------------------------------------------------------------------------------

 

\begin{tagblock}{Series, Theory, GeometricSeries, W13}
\begin{question}

We're now almost ready to define an infinite sum of numbers, but before we do, we should ruminate a bit on why we even need a proper definition! Here is the best example I know: look at the sum
\[
1+(-1)+1+(-1)+1+(-1)+\cdots
\]
And now with some clever use of parentheses...
\begin{align*}
0&=(1+(-1))+(1+(-1))+(1+(-1))+\cdots\\
&=1+((-1)+1)+((-1)+1)+((-1)+1)+\cdots\\
&=1
\end{align*}
So $0=1$ and we can all give up on math now. Discuss this with your group.
	
\index{Series}
	
\begin{tags}
	    Series, Theory, GeometricSeries, W13
\end{tags}
	
\begin{diary}
	    
\end{diary}
	
\begin{solution}
	   
\end{solution}
	
\end{question}

\end{tagblock}

%-------------------------------------------------------------------------------------------------------------


 \begin{tagblock}{Series, Definition, PartialSums, W13}
\begin{question}

Previously, we alluded to how a series was really just a sequence. It's now time to make sense of that assertion. Suppose you have a sequence $\displaystyle(a_n)_{n=1}^{\infty}$ of real numbers. Let
\begin{equation}\label{partial}
S_k=\sum_{n=1}^ka_n
\end{equation}

So for example, if $a_n=\dfrac 1 {2^n}$, 
\[
S_1=a_1=\frac 1 2
\]
\smallskip
\[
S_2=a_1+a_2=\frac 1 2 +\frac 1 4=\frac 3 4
\]
\smallskip
\[
S_3=a_1+a_2+a_3=\frac 1 2 +\frac 1 4 +\frac 1 8=\frac 7 8
\]


\bigskip

If there is a number $L$ such that the sequence $(S_k)$ converges to $L$, we write
\[
\sum_{n=1}^{\infty}a_n=L.
\]
If no such number $L$ exists, we say the series \textit{diverges}. So an ``infinite sum" of real numbers is really just a sequence of (finite sums of) real numbers. The numbers $S_k$ are called the \textit{partial sums} of the infinite series.  

\bigskip

a) Now let $a_n=(-1)^{n+1}$. Find a formula for the partial sums $(S_k)$. Does $\displaystyle\lim_{k\to\infty}S_k$ exist? Why or why not?

\bigskip

b) Let $b_n=1^n$. Find a formula for the partial sums $(S_k)$. Does $\displaystyle\lim_{k\to\infty}S_k$ exist? Why or why not?


\bigskip

c) Finally, let 
\[
c_n=\dfrac 1 n-\frac 1 {n+1}.
\] 
Find a formula for the partial sums $(S_k)$. Does $\displaystyle\lim_{k\to\infty}S_k$ exist? Why or why not?
	
\index{Series}
	
\begin{tags}
	    Series, Definition, PartialSums, W13
\end{tags}
	
\begin{diary}
	    
\end{diary}
	
\begin{solution}
	   
\end{solution}
	
\end{question}

\end{tagblock}

%-------------------------------------------------------------------------------------------------------------
 

\begin{tagblock}{Series, GeometricSeries, Challenge, W13}
\begin{question}

Two trains are on the same track a distance 60 miles apart heading towards one another, each at a speed of 30 mph. A fly starting out at the front of one train flies towards the other at a speed of 35 mph. Upon reaching the other train, the fly turns around and continues towards the first train. When it reaches the first train, it turns around and flies towards the other train, and so on. How many miles does the fly travel before getting squashed in the collision of the two trains?
	
\index{Series}
	
\begin{tags}
	    Series, GeometricSeries, Challenge, W13
\end{tags}
	
\begin{diary}
	    
\end{diary}
	
\begin{solution}
	   
\end{solution}
	
\end{question}

\end{tagblock}


%-------------------------------------------------------------------------------------------------------------
 

\begin{tagblock}{Series, GeometricSeries, Theory, W14}
\begin{question}

Now we will look at \textit{geometric series}, infinite series whose terms are a geometric sequence. Parts a) and b) of problem \#2 are geometric series. For the remainder of this problem, let's have $a_n=x^n$ for some real number $x$ where $x\ne 1$. We took care of $x=1$ in \#2c).

\bigskip

a) Write out the partial sum $S_3$. Then write out $xS_3$. If you subtract $xS_3$ from $S_4$, what is left over? 

\bigskip

b) Same question as a), but for $S_5$ and $xS_4$. 

\bigskip

c) Same question as a), but for $S_{k+1}$ and $xS_{k}$. 

\bigskip

d) Observe that $S_{k+1}=S_k+x^{k+1}$. Use this observation and your answer from part c) to come up with a formula for $S_k$ that doesn't involve any summations. 

\bigskip

e) Take the limit as $k\to\infty$ in your formula for $S_k$ from part d). When does the limit exist? 

\bigskip

f) When the limit in part e) does exist, what is the value?
	
\index{Series}
	
\begin{tags}
	    Series, GeometricSeries, Theory, W14
\end{tags}
	
\begin{diary}
	    
\end{diary}
	
\begin{solution}
	   
\end{solution}
	
\end{question}

\end{tagblock}

%-------------------------------------------------------------------------------------------------------------
 

\begin{tagblock}{Series, GeometricSeries, Defnition, W14}
\begin{question}

You should have obtained that
\[
\sum_{n=1}^{\infty}x^n=\frac x {1-x}
\]
when $|x|<1$ and diverges when $|x|\geq 1$. In fact, there is a more general formula: the sum of a geometric series is
\[
\frac a {1-x}
\]
where $a$ is the first term of the series and $x$ is the ratio. Use this formula with abandon on the following problems: does the infinite series converge, and if so, what is the value? 

\bigskip 

a) $\displaystyle\sum_{n=0}^{\infty}\frac 1 {4^n}$ 

\bigskip

b) $\displaystyle\sum_{n=4}^{\infty}\left(\frac 2 3\right)^{-n}$ 

\bigskip

c) $\displaystyle\sum_{n=2}^{\infty}\frac {(-5)^n}{17^{n+1}}$ 

\bigskip

d) $\displaystyle\sum_{n=m}^{\infty}\frac {r^{n+k}}{s^{n+l}}$ 
	
\index{Series}
	
\begin{tags}
	    Series, GeometricSeries, Definition, W14
\end{tags}
	
\begin{diary}
	    
\end{diary}
	
\begin{solution}
	   
\end{solution}
	
\end{question}

\end{tagblock}

%-------------------------------------------------------------------------------------------------------------
 

\begin{tagblock}{Series, PowerSeries, Defnition, W14, WarmUp}
\begin{question}

We start with a sequence of real numbers $(a_n)_{n=0}^{\infty}$. A \textit{power series} centered at zero is a function 
\[
f(x)=\sum_{n=0}^{\infty}a_nx^n.
\]

\bigskip

We call $(a_n)$ the \textit{coefficients} of the power series. The difference between a power series and a regular series is the variable $x$. 

Our first example of a power series came from geometric series; namely,
\begin{equation}\label{geo}
g(x)=\sum_{n=1}^{\infty}x^n=\frac x {1-x}.
\end{equation}


\bigskip

a) Find the coefficients $(a_n)$ for equation \eqref{geo}.
 
\bigskip

b) For the general sum $\displaystyle\sum_{n=0}^{\infty}a_nx^n$, what is a (and perhaps the only) value of $x$ that you can show the series converges for?
	
\index{Series}
	
\begin{tags}
	    Series, PowerSeries, Definition, W14, WarmUp
\end{tags}
	
\begin{diary}
	    
\end{diary}
	
\begin{solution}
	   
\end{solution}
	
\end{question}

\end{tagblock}

%-------------------------------------------------------------------------------------------------------------
 

\begin{tagblock}{Series, GeometricSeries, Challenge, W14}
\begin{question}

Watch this \href{https://www.youtube.com/watch?v=E-d9mgo8FGk}{video}.

\bigskip

Putting aside the sheer lunacy of what the gentlemen in the video is claiming to show, find all the points in the video where he lies to you. What are the lies?
	
\index{Series}
	
\begin{tags}
	    Series, GeometricSeries, Challenge, W14
\end{tags}
	
\begin{diary}
	    
\end{diary}
	
\begin{solution}
	   
\end{solution}
	
\end{question}

\end{tagblock}

%-------------------------------------------------------------------------------------------------------------
 

\begin{tagblock}{Series, PowerSeries, Example, Definition, RatioTest, RadiusOfConvergence, W15}
\begin{question}

The power series $\displaystyle\sum_{n=0}^{\infty}a_nx^n$ always converges for $x=0$. How do we tell if there are other values? The answer actually comes from geometric series! If we look at the $(n+1)^{st}$ term in the geometric series divided by the $n^{th}$ term in the series, for all nonzero numbers $x$, we get 
\[
\frac{x^{n+1}}{x^n}=x.
\]
We know the series converges when $|x|<1$, so another way of phrasing this is that the ratio of the $(n+1)^{st}$ term in the geometric series divided by the $n^{th}$ term must be less than 1 in absolute value. This is the basis of the \textit{ratio test}, almost the only test you need to know for series.

\bigskip

\noindent\textbf{Ratio Test, c=0:} A power series $\displaystyle\sum_{n=0}^{\infty}a_nx^n$, centered at zero,
\begin{itemize}
\item converges when $\displaystyle\lim_{n\to\infty}\left|\frac{a_{n+1}x^{n+1}}{a_nx^n}\right|=|x|\lim_{n\to\infty}\left|\frac{a_{n+1}}{a_n}\right|<1$,
\item diverges when $\displaystyle\lim_{n\to\infty}\left|\frac{a_{n+1}x^{n+1}}{a_nx^n}\right|=|x|\lim_{n\to\infty}\left|\frac{a_{n+1}}{a_n}\right|>1$, and 

\item when $\displaystyle\lim_{n\to\infty}\left|\frac{a_{n+1}x^{n+1}}{a_nx^n}\right|=|x|\lim_{n\to\infty}\left|\frac{a_{n+1}}{a_n}\right|=1$, the test fails and YOU KNOW NOTHING. 
\end{itemize}
\bigskip

The values of $x$ for which the power series converges is known as the \textit{interval of convergence}. The \textit{radius of convergence} is half the length of the interval and can be found from the ratio test by computing the number $r$ with $|x|<r$ giving convergence and $|x|>r$ giving divergence for the power series. It's also possible to have degenerate intervals where only $x=0$ gives convergence and infinite intervals where the series converges for all real numbers. In the first case, we say the radius is 0, and in the second, we say it is infinite.

\bigskip

Here's an example for the power series $\displaystyle\sum_{n=1}^{\infty}\frac{2^nx^n}n$ (it's easier to do the algebra without limits).
\begin{align*}
\left|\frac{a_{n+1}x^{n+1}}{a_nx^n}\right|&=\left|\frac{2^{n+1}x^{n+1}/(n+1)}{2^nx^n/n}\right|\\
&=\left|\frac{2^{n+1}}{2^n}\frac {x^{n+1}}{x^n}\frac n {n+1}\right|\\
&=\left| 2x\right|\frac n {n+1}\\
&=2|x|\frac n {n+1}
\end{align*}
For the calculus part, we take the limit as $n\to\infty$. The 2 and $|x|$ don't have any $n$'s, so they just pop out of the limit:
\[
\lim_{n\to\infty}2|x|\frac n {n+1}=2|x|\lim_{n\to\infty}\frac n {n+1}=2|x|
\]
since $\displaystyle\lim_{n\to\infty}\dfrac n {n+1}=1$.

\bigskip

a) If it wasn't obvious why $\displaystyle\lim_{n\to\infty}\dfrac n {n+1}=1$, try using l'Hopital's rule. 

\bigskip

To find the radius of convergence, set $2|x|<1$ to get $|x|<1/2$, which means the radius of convergence is equal to 1/2.

\bigskip

Find the radius of convergence for the following power series. 

\bigskip

b) $\displaystyle\sum_{n=0}^{\infty}\frac{x^{2n+1}(-1)^n}{2n+1}$

\bigskip

c) $\displaystyle\sum_{n=0}^{\infty}\frac{x^n}{n!}$ (the convention is that $0!=1$.)
	
\index{Series}
	
\begin{tags}
	    Series, PowerSeries, Example, Definition, RatioTest, RadiusOfConvergence, W15
\end{tags}
	
\begin{diary}
	    
\end{diary}
	
\begin{solution}
	   
\end{solution}
	
\end{question}

\end{tagblock}

%-------------------------------------------------------------------------------------------------------------
 

\begin{tagblock}{Series, PowerSeries, Definition, RatioTest, RadiusOfConvergence, W15}
\begin{question}

You may have been wondering why we used ``centered at 0" for power series. In special cases (like when zero is not in the domain), we'll need to shift our power series around. The most general formula for a power series is 
\[
\displaystyle\sum_{n=0}^{\infty}a_n(x-c)^n.
\]
The point $x=c$ is called the \textit{center} of the series. 

\bigskip

a) Now for what value of $x$ do you know the series converges?

\bigskip

For power series not centered at zero, the ratio test becomes

\bigskip

\noindent\textbf{Ratio Test:} A power series $\displaystyle\sum_{n=0}^{\infty}a_n(x-c)^n$, centered at zero,
\begin{itemize}
\item converges when $\displaystyle\lim_{n\to\infty}\left|\frac{a_{n+1}(x-c)^{n+1}}{a_n(x-c)^n}\right|=|x-c|\lim_{n\to\infty}\left|\frac{a_{n+1}}{a_n}\right|<1$,
\item diverges when $\displaystyle\lim_{n\to\infty}\left|\frac{a_{n+1}(x-c)^{n+1}}{a_n(x-c)^n}\right|=|x-c|\lim_{n\to\infty}\left|\frac{a_{n+1}}{a_n}\right|>1$, and
\item when $\displaystyle\lim_{n\to\infty}\left|\frac{a_{n+1}(x-c)^{n+1}}{a_n(x-c)^n}\right|=|x-c|\lim_{n\to\infty}\left|\frac{a_{n+1}}{a_n}\right|\cdot |x-c|=1$, the test fails and YOU KNOW NOTHING. 
\end{itemize}

\bigskip

b) Find the radius of convergence of $\displaystyle\sum_{n=1}^{\infty}\frac{(x-1)^n(-1)^{n+1}}{n}$
	
\index{Series}
	
\begin{tags}
	    Series, PowerSeries, Definition, RatioTest, RadiusOfConvergence, W15
\end{tags}
	
\begin{diary}
	    
\end{diary}
	
\begin{solution}
	   
\end{solution}
	
\end{question}

\end{tagblock}

%-------------------------------------------------------------------------------------------------------------

 

\begin{tagblock}{Series, PowerSeries, RatioTest, RadiusOfConvergence, Derivatives, W15}
\begin{question}

For the differential equation
\begin{equation}\label{series}
f''(r)+\frac 1 rf'(r)+\alpha f(r)=0,
\end{equation}
the solution is a power series of the form $\displaystyle\sum_{n=0}^{\infty}a_nx^n$, but to check that this is a solution, we'd have to know how to differentiate a power series! Fortunately, what we'd like to be true is actually true. Let's start off with 
 \begin{equation}\label{geo}
g(x)=\sum_{n=1}^{\infty}x^n.
\end{equation}

If we write this out longhand, we get $x+x^2+x^3+x^4+\cdots$. Remember that $|x|<1$!

\bigskip

a) How do you think you'd take a derivative of equation \eqref{geo}?

\bigskip

b) Remember that for $|x|<1$, $\displaystyle\sum_{n=0}^{\infty}x^n=\frac 1 {1-x}$. Take the derivative of $\dfrac 1 {1-x}$. What is the domain of the derivative?

\bigskip

c) What do you think the radius of convergence should be for the power series you found in part a)?

\bigskip

d) Using the ratio test, confirm the radius of convergence of your answer from a). 
	
\index{Series}
	
\begin{tags}
	    Series, PowerSeries, RatioTest, RadiusOfConvergence, Derivatives, W15
\end{tags}
	
\begin{diary}
	    
\end{diary}
	
\begin{solution}
	   
\end{solution}
	
\end{question}

\end{tagblock}

%-------------------------------------------------------------------------------------------------------------
 

\begin{tagblock}{Series, PowerSeries, RatioTest, RadiusOfConvergence, Derivatives, Example, W16}
\begin{question}

Let $\displaystyle f(x)=\sum_{n=0}^{\infty} a_n(x-c)^n$ be a power series with a nonzero radius of convergence $R$. Then 
\[
f'(x)=\sum_{n=1}^{\infty}na_n(x-c)^{n-1}=a_1+2a_2(x-c)+3a_3(x-c)^2+\cdots
\]
The radius of convergence of $f'$ is again $R$ (this can be confirmed via the ratio test).

\bigskip

For an example, let's again use
\[
g(x)=\sum_{n=1}^{\infty}x^n=1+x+x^2+x^3+x^4\cdots
\]
for $|x|<1$.
Then
\begin{align*}
g'(x)&=1+2x+3x^2+4x^3\cdots\\
&=\sum_{n=0}^{\infty}(n+1)x^n
\end{align*}

with radius of convergence equal to one. 

\bigskip

a) Let $e(x)=\displaystyle\sum_{n=0}^{\infty}\frac{x^n}{n!}$. Find $e'(x)$. How does this answer compare with $e(x)$?

\bigskip

b) Now let $f(x)=g(1-x)=\displaystyle\sum_{n=1}^{\infty}(1-x)^n$. Find $f'(x)$. What do you see? 

\bigskip

c) For a big step up, try differentiating $\displaystyle\sum_{n=0}^{\infty}(-1)^n\dfrac {x^{2n}}{(2n)!}$ twice. If you can do this, then I pronounce you ready.
	
\index{Series}
	
\begin{tags}
	    Series, PowerSeries, RatioTest, RadiusOfConvergence, Derivatives, Example, W16
\end{tags}
	
\begin{diary}
	    
\end{diary}
	
\begin{solution}
	   
\end{solution}
	
\end{question}

\end{tagblock}

%-------------------------------------------------------------------------------------------------------------
 

\begin{tagblock}{Series, PowerSeries, RatioTest, RadiusOfConvergence, Integration, W16}
\begin{question}

Let's take a moment to examine integration of power series. Happily, it works just like differention: if $\displaystyle f(x)=\sum_{n=0}^{\infty} a_n(x-c)^n$ is a power series with a nonzero radius of convergence $R$, then 
\[
\int f(x) \ dx=C+\sum_{n=0}^{\infty}a_n\frac{(x-c)^{n+1}}{n+1}=C+a_0(x-c)+a_1\frac{(x-c)^2} 2+\cdots
\]
Again you can show via the ratio test that the radius of convergence of $\displaystyle\int f(x) \ dx$ is $R$. \bigskip

a) Again consider the geometric series $g(x)=\displaystyle\sum_{n=1}^{\infty}x^n$ for $|x|<1$, and let $f(x)=g(x)+1=\displaystyle\sum_{n=0}^{\infty}x^n$. What familiar function does $f$ represent?

\bigskip

b) Let $h(x)=f(1-x)$. Integrate $h$. What familiar function should this represent?
	
\index{Series}
	
\begin{tags}
	    Series, PowerSeries, RatioTest, RadiusOfConvergence, Integration, W16
\end{tags}
	
\begin{diary}
	    
\end{diary}
	
\begin{solution}
	   
\end{solution}
	
\end{question}

\end{tagblock}

%-------------------------------------------------------------------------------------------------------------
 

\begin{tagblock}{Series, PowerSeries, RatioTest, RadiusOfConvergence, Derivatives, Challenge, W16}
\begin{question}

Let 
\[
J_0(x)=1+\sum_{n=1}^{\infty}\frac {x^{2n}(-1)^n}{(n!)^24^n}.
\]

\bigskip

a) What do you think the radius of convergence of $J_0$ is? 

\bigskip

b) Use the ratio test to find the radius of convergence for $J_0$. Is this what you guessed?

\bigskip

c) Show that, with $\alpha=1$, $J_0$ is a solution to 
\[
f''(r)+\frac 1 rf'(r)+\alpha f(r)=0.
\]
	
\index{Series}
	
\begin{tags}
	    Series, PowerSeries, RatioTest, RadiusOfConvergence, Derivatives, Challenge, W16
\end{tags}
	
\begin{diary}
	    
\end{diary}
	
\begin{solution}
	   
\end{solution}
	
\end{question}

\end{tagblock}

%-------------------------------------------------------------------------------------------------------------
 

\begin{tagblock}{Series, PowerSeries, RatioTest, RadiusOfConvergence, Derivatives, Integration, Challenge, W16}
\begin{question}

a) Suppose that the radius of convergence of $\displaystyle f(x)=\sum_{n=0}^{\infty} a_n(x-c)^n$ is $R>0$. Show that the radius of convergence of $f'$ is also $R$.

\bigskip

b) Suppose that the radius of convergence of $\displaystyle f(x)=\sum_{n=0}^{\infty} a_n(x-c)^n$ is $R>0$. Show that the radius of convergence of $\int f(x) \ dx$ is also $R$.
	
\index{Series}
	
\begin{tags}
	    Series, PowerSeries, RatioTest, RadiusOfConvergence, Derivatives, Integration, Challenge, W16
\end{tags}
	
\begin{diary}
	    
\end{diary}
	
\begin{solution}
	   
\end{solution}
	
\end{question}

\end{tagblock}

%-------------------------------------------------------------------------------------------------------------
 

\begin{tagblock}{Series, PowerSeries, Derivatives, Factorials, Challenge, W17}
\begin{question}

Remember the geometric series
 \begin{equation}\label{geo}
g(x)=\sum_{n=1}^{\infty}x^n.
\end{equation}

We know $g(x)=\dfrac x {1-x}$ when $|x|<1$ and that $\displaystyle\sum_{n=0}^{\infty}x^n=\dfrac 1 {1-x}$. 

You might have discovered, or at least had the sneaking suspicion, that 
\begin{equation}\label{exp}
e^x=\sum_{n=0}^{\infty}\frac{x^n}{n!}.
\end{equation}

\textbf{a)} Quote the Fundamental Theorem of Calculus, applied to the differential equation $\dfrac{dy}{dx}=y$, to verify equation \eqref{exp}
	
\index{Series}
	
\begin{tags}
	    Series, PowerSeries, Derivatives, Factorials, Challenge, W17
\end{tags}
	
\begin{diary}
	    
\end{diary}
	
\begin{solution}
	   
\end{solution}
	
\end{question}

\end{tagblock}

%-------------------------------------------------------------------------------------------------------------
 

\begin{tagblock}{Series, PowerSeries, Derivatives, Integration, MacLaurinSeries, TaylorSeries, W17}
\begin{question}

Most of the power series we've seen are examples of \textit{MacLaurin series}; given a function $f$ that is defined at $x=0$, its MacLaurin series is the power series
\begin{equation}\label{mac}
\sum_{n=0}^{\infty}f^{(n)}(0)\frac{x^n}{n!},
\end{equation}
where $f^{(n)}(0)$ is the $n^{th}$ derivative of $f$ evaluated at zero, with the convention $f^{(0)}(x)=f(x)$. 

\bigskip

\textbf{a)} Find the MacLaurin series for $\ln(1-x)$. Please don't do this using the definition of a MacLaurin series. What is the radius of convergence? 

\bigskip

\textbf{b)} Find the MacLaurin series for $\arctan(x)$. Same request as for part c), also find the radius of convergence.
	
\index{Series}
	
\begin{tags}
	    Series, PowerSeries, Derivatives, Integration, MacLaurinSeries, TaylorSeries, W17
\end{tags}
	
\begin{diary}
	    
\end{diary}
	
\begin{solution}
	   
\end{solution}
	
\end{question}

\end{tagblock}

%-------------------------------------------------------------------------------------------------------------
 

\begin{tagblock}{Series, PowerSeries, MacLaurinSeries, TaylorSeries, W17}
\begin{question}

If you can find a power series centered at $c=0$ with nonzero radius of convergence that equals a function $f$, then the power series MUST be the MacLaurin series of $f$. Use this fact to find a power series representation for the following functions. Be sure to say something about (if not explicitly compute) the radius of convergence!

\bigskip

\textbf{a)} $f(x)=\dfrac {x}{1-3x}$

\bigskip

\textbf{b)} $g(x)=3^x$

\bigskip

\textbf{c)} $h(x)=\ln(4x)$
	
\index{Series}
	
\begin{tags}
	    Series, PowerSeries, MacLaurinSeries, TaylorSeries, W17
\end{tags}
	
\begin{diary}
	    
\end{diary}
	
\begin{solution}
	   
\end{solution}
	
\end{question}

\end{tagblock}

%-------------------------------------------------------------------------------------------------------------
 

\begin{tagblock}{Series, PowerSeries, MacLaurinSeries, TaylorSeries, Challenge, W17}
\begin{question}

If $f(x)=e^{x^5}$, find the value of $f^{(50)}(0)$. 
	
\index{Series}
	
\begin{tags}
	    Series, PowerSeries, MacLaurinSeries, TaylorSeries, Challenge, W17
\end{tags}
	
\begin{diary}
	    
\end{diary}
	
\begin{solution}
	   
\end{solution}
	
\end{question}

\end{tagblock}

%-------------------------------------------------------------------------------------------------------------
 

\begin{tagblock}{Series, PowerSeries, MacLaurinSeries, TaylorSeries, Derivatives, Challenge, W18}
\begin{question}

Remember that the MacLaurin series for a function $f$ is the power series 
\begin{equation}\label{mac}
\sum_{n=0}^{\infty}f^{(n)}(0)\frac{x^n}{n!},
\end{equation}
where $f^{(n)}(0)$ is the $n^{th}$ derivative of $f$ evaluated at zero, with the convention $f^{(0)}(x)=f(x)$. 

A trite comment about MacLaurin series is that $f$ needs to be infinitely differentiable in order to obtain its MacLaurin series. A not-so-trite question is, how do we know the MacLaurin series has any meaningful relationship to $f$? In the last worksheet, we used integration, differentiation, and minor tricks to show equality.

\bigskip

a) Compute the MacLaurin series for $f(x)=\sin(x)$.

\bigskip

If we knew that sine was equal to its MacLaurin series, we could differentiate to get the MacLaurin series for cosine. Unfortunately, our tricks have run out, so we need to stretch our hand into the future and steal some information from differential equations. Remember that $f(x)=e^x$ satisfies the differential equation $y'=y$? 
\bigskip

b) Check that both sine and cosine satisfy the differential equation 
\begin{equation}\label{dd}
y''+y=0.
\end{equation}

\bigskip

c) Check that the MacLaurin series for sine satisfies equation \eqref{dd}. 

\bigskip

From differential equations, any solution $f$ to equation \eqref{dd} must be of the form $f(x)=c_1\sin(x)+c_2\cos(x)$ for some constants $c_1$ and $c_2$. 

\bigskip

d) Use this fact to show that if $f$ is the MacLaurin series for sine, $c_1=1$ and $c_2=0$. 

\bigskip

e) Show that cosine equals its MacLaurin series for all values of $x$. 
	
\index{Series}
	
\begin{tags}
	    Series, PowerSeries, MacLaurinSeries, TaylorSeries, Derivatives, Challenge, W18
\end{tags}
	
\begin{diary}
	    
\end{diary}
	
\begin{solution}
	   
\end{solution}
	
\end{question}

\end{tagblock}

%-------------------------------------------------------------------------------------------------------------
 

\begin{tagblock}{Series, PowerSeries, TaylorSeries, Derivatives, Challenge, W18}
\begin{question}

It isn't always possible to express a function as a MacLaurin series. Fortunately, there is a generalization to any center at all, not just $c=0$, called a \textit{Taylor Series}. The Taylor series of $f$ centered at $c$ is
\[
\sum_{n=0}^{\infty}f^{(n)}(c)\frac{(x-c)^{n}}{n!}.
\]

\bigskip

Note that every MacLaurin series is a Taylor series, but the reverse is not true.  

\bigskip

a) With $c=1$, find the Taylor series for $\ln(x)$. Please don't use the definition. What is the radius of convergence?

\bigskip

b) Again with $c=1$, find the Taylor series for $\sqrt{x}$. This has to be done via the definition and is mind-boggling from a notational standpoint.
	
\index{Series}
	
\begin{tags}
	    Series, PowerSeries, TaylorSeries, Derivatives, Challenge, W18
\end{tags}
	
\begin{diary}
	    
\end{diary}
	
\begin{solution}
	   
\end{solution}
	
\end{question}

\end{tagblock}

%-------------------------------------------------------------------------------------------------------------
 

\begin{tagblock}{Series, PowerSeries, TaylorSeries, Derivatives, Challenge, W18}
\begin{question}

Not every function is equal to its MacLaurin series, provided the series has a nonzero radius of convergence. Let 
\[
f(x)=\begin{cases} e^{-\frac 1 {x^2}} & x\ne 0 \\ 0 & x=0\end{cases}.
\]
Compute the MacLaurin series for $f$ using the definition.
	
\index{Series}
	
\begin{tags}
	    Series, PowerSeries, TaylorSeries, Derivatives, Challenge, W18
\end{tags}
	
\begin{diary}
	    
\end{diary}
	
\begin{solution}
	   
\end{solution}
	
\end{question}

\end{tagblock}

%-------------------------------------------------------------------------------------------------------------
 
 

\printindex

\end{document}



